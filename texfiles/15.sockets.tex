%%%  Chapter 15: Sockets and Networking
%%%  3rd Edition

\setcounter{SSTUDYcount}{1}
\setcounter{chapter}{14}

\chapter{Sockets and Networking}
\label{chapter-sockets}
\label{pg-chapter-sockets}

\CObegin
\secCOBH{Objectives}
\noindent After studying this chapter, you will

\begin{COBL}
\item  Understand some basics about networks.
\item  Know how to use Java's {\tt URL} class
to download network resources from an applet or application.
\item  Be able to design networking applications,
using the client/server model.
\item  Understand how to use Java's {\tt Socket}
and {\tt ServerSocket} classes.
\end{COBL}

\secCOLH{Outline}
\begin{COL}
\item {Introduction}
\item {An Overview of Networks}
\item {Using Network Resources from an Applet}
\item {From the Java Library: {\tt java.net.URL}}
\item {The Slide Show Applet}
\item {Using Network Resources from an Application}
\item {Client/Server Communication via {\tt Socket}s}
\item {Case Study: Generic Client/Server Classes}
\item {Playing One Row Nim Over the Network}
\item {Java Network Security Restrictions}
\item[] {{\color{cyan}Special Topic:} Privacy and the Internet}
\item {Java Servlets and Server Pages}
\par\small\item[] {Chapter Summary}
\par\small\item[] {Solutions to Self-Study Exercises}
\par\small\item[] {Exercises}
\end{COL}
\COend

\section{Introduction}

One of the key strengths of Java is the support it provides for the
Internet and client/server programming.  In the previous chapter, we saw
how to make Java programs transfer information to and from external
files.  Although files are external to the programs that process them,
they are still located on the same computer.  In this chapter, we learn
how to transfer information to and from files that reside on a
network.  This enables programs to communicate with programs running on
other computers.   With networking, we can communicate with computers
anywhere in the world.  

\section{An Overview of Networks}
\noindent Networking is a broad and complex topic.  In a typical computer science
curriculum, it is covered in one or more upper-level
courses.  Nevertheless, in this chapter you can learn enough about
networking to be able to use network resources and to design simple
Java networking applications.

\subsection{Network Size and Topology}
\noindent Computer networks come in a variety of sizes and shapes.  A {\it local
area network (LAN)} is usually a privately owned network located
within a single office or a single organization.  Your campus
network would be an example of a LAN.~A {\it wide area
network (WAN)} spans a wide geographical distance like a country
or a continent.  It may use a combination of public, private,
and leased communication devices.  Some of the large commercial
networks, such as MCI and Sprint, are examples of WANs.

The computers that make up a network can be arranged in a variety of
{\it topologies}, or shapes, some of the most common of which are
shown in Figures~\ref{fig-topology1} and \ref{fig-topology2}. 
As you
would expect, different topologies use different techniques for
%\begin{marginalnote}\it Network topology\end{marginalnote}
\marginnote{Network topology}
transmitting information from computer to computer.
%\begin{figure}
\begin{figure}[h]
\figaleft{chptr15/15f1.eps}{Star, bus, and ring topologies.
\label{fig-topology1}\label{pg-fig-topology1}}
\end{figure}
%\end{figure}

In a star network (Fig.~\ref{fig-topology1}), a central computer
functions as a hub, with every other computer in the network connected
to the hub.  Each computer can communicate with the others but only
through the hub.  The bus topology doesn't have a hub computer.
Instead, each node looks at each message sent on the  bus to find
those that are addressed to it. In sending a message, a node waits
until the bus is free and then transmits the message.

A ring network (Fig.~\ref{fig-topology1}) also has no host, and the
computers are connected in a loop, through which they exchange
information.  The tree topology (Fig.~\ref{fig-topology2}) is
organized into a hierarchy, with each level (trunk of the tree, major
branch of the tree) controlled by a hub.  The fully connected mesh
network directly connects all points to all points, eliminating the
``middleman.''  Here there is no need to go through one or more other
computers in order to communicate with a particular computer in the
network.

Network topologies differ quite a bit in the expense of the wiring
they require, their efficiency, their susceptibility to failure, and
the types of protocols they use.   These
differences are beyond the scope of this chapter.

\begin{figure}[tb]
\figa{chptr15/15f2.eps}{Tree and fully connected mesh topologies.
\label{fig-topology2}\label{pg-fig-topology2}}
\end{figure}

\subsection{Internets}
\noindent An {\bf internet}\index{internet} (lowercase {\it i}) is a collection of two or more
distinct networks, joined by devices called {\bf routers}\index{routers}
(Fig.~\ref{fig-internet}).  An internet is like a meeting of the
%\begin{marginalnote}\it An internet vs.  the Internet\end{marginalnote}
\marginnote{An internet vs.  the Internet}
United Nations.  Each country sends a delegation, all of whose members
speak that country's language.  A national delegation is like a single
computer network.  Language interpreters take on the task of
translating one language to another so that any two delegations, say,
the United States and China, can communicate.  The routers play a
similar translation role within an internet.  The UN conference,
composed of communicating delegations from all the different
countries of the world, is like a worldwide internet.

\begin{figure}[thb]
\figaleft{chptr15/15f3.eps}{An internet is a collection of distinct
networks joined together by routers.
\label{fig-internet}\label{pg-fig-internet}}
\end{figure}


The United Nations is an apt analogy for the {\bf Internet}\index{Internet}
(uppercase {\it I}), which is an example of a  particular  worldwide
internet.  Internets, in the generic sense, shouldn't be confused
with the Internet.   It's quite likely that your campus LAN is
itself composed of several, smaller networks, each of
which uses its own ``language.''

\pagebreak
\secEXRHone{Self-Study Exercises}
\begin{SSTUDY}
\item  In a network of ten computers, which topology would require the
most cables?

\item  Which topology would be most resistant to having one of
its computers crash?

\item  Which topology would be least resistant to having one of
its computers crash?
\end{SSTUDY}


\subsection{Network Protocols}
%<sec_protocols:\thisBheadno><pg_sec_protocols:\thispageno>
\noindent A {\bf protocol}\index{protocol} is a set of rules that governs the communication of
information.  For example, the {\bf World Wide Web }is based on the {\bf
HyperText} \mbox{\bf Transfer Protocol (HTTP)}. HTTP describes how information is
to be exchanged between a Web browser, such as Internet Explorer or
%\begin{marginalnote}\it Network protocols\end{marginalnote}
\marginnote{Network protocols}
Netscape Navigator, and a Web server, which stores an individual's or
company's Web pages.  Web pages are encoded in the {\it HyperText
Markup Language (HTML)}. Among other things, the HTTP protocol can
interpret HTML pages.

Similarly, the {\bf Simple Mail Transfer Protocol (SMTP)} is a set of
rules that governs the transfer of {\it e-mail}. And the {\bf File
Transfer Protocol (FTP)} is the protocol that governs the transfer of
files across the Internet.

\subsubsection*{Application Protocols}
\noindent These three examples---HTTP, SMTP, and FTP---are examples of
application protocols.  They are relatively high-level protocols that
support and govern a particular network application, such as e-mail or
WWW access.  Among the things they determine how we address
different computers on the network.  For example, the HTTP protocol
specifies Web addresses by using a {\bf Uniform Resource Locator
(URL)}\index{Uniform Resource Locator (URL)}. A URL specifies three
necessary bits of information: The method used to transfer information
(e.g., HTTP or FTP), the address of the host computer (e.g., {\tt
www.prenhall.com}), and the path describing where the file is located
on the host ( {\tt /morelli/index.html}):

\begin{jjjlisting}
\begin{lstlisting}[commentstyle=\color{black}]
METHOD://HOST/PATH
HTTP://www.prenhall.com/morelli/index.html
\end{lstlisting}
\end{jjjlisting}

\noindent Similarly, an e-mail address is specified by the SMTP protocol
to consist of a local mailbox address ({\tt George.W.Bush}) followed by
the address of the computer ({\tt mail.whitehouse.gov}):

\begin{jjjlisting}
\begin{lstlisting}
LOCAL_MAILBOX@COMPUTER
George.W.Bush@mail.whitehouse.gov
\end{lstlisting}
\end{jjjlisting}

\noindent Another good example of an application protocol is the Internet's
{\it Domain Name System (DNS)}, which is the system that governs how
names, such as {\tt whitehouse.gov} and {\tt java.trincoll.edu}, can
%\begin{marginalnote}\bf Internet domain names\end{marginalnote}
\marginnote{Internet domain names}
be translated into numeric addresses.  In the DNS, each host computer
on the Internet is identified with a unique host name---for example,
{\tt mail}, {\tt java}---which is usually made up by the network
administrator whose job it is to manage an organization's network.
The DNS divides the entire Internet into a hierarchy of {\it domains}
and {\it subdomains}. The generic domains are names like {\tt com},
{\tt edu}, and {\tt mil}, which refer to the type of organization---
commercial, educational, and military, respectively.  In addition to
these there are country domains, such as {\tt fr}, {\tt au}, and {\tt nz},
for France, Australia, and New Zealand.  Finally, individuals and
organizations can buy their own {\bf domain names}\index{domain names}, such as {\tt
whitehouse}, {\tt microsoft}, and {\tt trincoll}.


What makes the whole system work is that certain computers within the
network are designated as DNS servers.  It is their role to translate
names such as {\tt java.trincoll.edu} to numeric addresses whenever
they are requested to do so by clients such as the SMTP or the HTTP
server.  Also, the DNS servers must communicate among themselves to
make sure that their databases of names and addresses are up-to-date.

\secEXRHone{Self-Study Exercise}
\begin{SSTUDY}

\item  What's the URL of the Web server at Prentice Hall?  Identify its
component parts---host name, domain name, Internet domain.
\end{SSTUDY}

\subsection{Client/Server Applications}
\noindent The HTTP, FTP, SMTP, and DNS protocols are examples of {\bf
client/server protocols}\index{client/server protocols}, and the applications they support are
examples of client/server applications.  In general, a client/server
application is one in which the task at hand has been divided into two
subtasks, one performed by the {\bf client}\index{client} and one performed by the
{\bf server}\index{server} (Fig.~15.4).
%\begin{figure}
%\begin{figure}[tb]
\marginfig{chptr15/15f4.eps}{Client/server application.
\label{fig-clientserverapp}\label{pg-fig-clientserverapp}}
%\end{figure}
%\end{figure}


For example, in the HTTP case, the Web browser plays the role of a
client by requesting a Web page from a Web (HTTP) server.  A Web server
is just a computer that runs HTTP software---a program that
%\begin{marginalnote}\it e-mail client/server\end{marginalnote}
%%\marginnote{E-mail client/server}
implements the HTTP protocol.   For e-mail, the program you use to read
your e-mail---Eudora, Pine, or Outlook---is an e-mail client.  It
requests certain services, such as send mail or get mail, from an
e-mail (SMTP) server, which is simply a computer that runs SMTP
software.  In the FTP case, to transfer a program from one computer to
another, you would use an FTP client, such as Fetch.  Finally, in the
DNS case, the DNS servers handle requests for name to address
translations that come from HTTP, FTP, and SMTP servers, acting in this
case like clients.

So we can say that a client/server application is one that observes
the following protocol:

\begin{jjjlistingleft}[28pc]{-2pc}
\begin{lstlisting}
Server: Set up a service on a particular host computer.
Client: Contact the server and request the service.
Server: Accept a request from a client and provide the service.
\end{lstlisting}
\end{jjjlistingleft}


\noindent As these examples illustrate, many Internet applications are 
designed as client/server applications.

\JavaTIP{EFFECTIVE DESIGN}{Divide and Conquer.}{The client/server protocol
is an example of the effective use of the divide-and-conquer strategy.}


\secEXRHone{Self-Study Exercise}
\begin{SSTUDY}

\item  Lots of our everyday interactions fit into the cli\-ent/ser\-ver
model.  Suppose you are the client in the following services:

%%\begin{EXRBL}
\begin{itemize}
\item  Buying a piece of software at a bookstore.
\item  Buying a piece of software over the phone.
\item  Buying a piece of software over the Internet.
%%\end{EXRBL}
\end{itemize}

\noindent Identify the server and then describe the basic protocol.
\end{SSTUDY}

\subsection{Lower Level Network Protocols}
\noindent Modern computer networks, such as the Internet, are organized into a
number of levels of software and hardware.  Each level has its own
collection of protocols (Fig.~15.5).
%\begin{figure}
%\begin{graphic}
\marginfig{chptr15/15f5.eps}{Levels of network protocols.
\label{fig-levels}\label{pg-fig-levels}}
%\end{graphic}
%\end{figure}


The application level, which contains the HTTP, FTP, SMTP, and DNS
protocols, is the highest level.  Underlying the application-level
protocols are various {\it transmission protocols}, such as the {\it
Transfer Control Protocol (TCP)} and the {\it User Datagram Protocol
(UDP)}.  These protocols govern the transfer of large blocks of
information, or {\bf packets}\index{packets}, between networked computers.  All of
the applications we mentioned---WWW, e-mail, and file transfer---
involve
%\begin{marginalnote}\it Packet transfer\end{marginalnote}
\marginnote{Packet transfer}
data transmission and, therefore, rely on one or more of the
transmission protocols.

At the very lowest end of this hierarchy of protocols are those that
govern the transmission of bits or electronic pulses over wires and
those that govern the delivery of data from node to node.  Most of
these protocols are built right into the hardware---the wires,
connectors, transmission devices---that networks use.   On top of
these are protocols, such as the \mbox{\bf ethernet protocol}\index{ethernet protocol} and {\it
token ring protocol}, that govern the delivery of packets of
information on a local area network.  These too may be built right into
the network hardware.

As you might expect, these lower level protocols are vastly different
%\begin{marginalnote}\it Disparate protocols\end{marginalnote}
\marginnote{Disparate protocols}
from each other.  An ethernet network cannot talk directly to a
token ring network.  How can we connect such disparate networks
together?  Think again of our United Nations analogy.  How do we get
French-speaking networks to communicate with English-speaking networks?
The answer supplied by the Internet is to use the {\bf Internetworking
Protocol (IP)}\index{Internetworking Protocol (IP)}, which governs the task of translating one network
protocol to a common format (Fig.~\ref{fig-ipprotocol}).


To push the UN analogy a bit further, the Internet's IP is like a
%\begin{marginalnote}\it The Internet protocol\end{marginalnote}
\marginnote{The Internet protocol}
universal language built into the routers that transmit data between
disparate networks.   On one end of a transmission, a router takes a
French packet of information received from one of the delegates in its
network.  The router translates the French packet into an IP packet, which
it then sends on through the network to its destination.   When the IP
packet gets close to its destination, another router takes it and
translates it into an English packet before sending it on to its
destination on its network.

\subsection{The {\tt java.net} Package}

%\begin{figure}
\begin{figure}[tb]
\figa{chptr15/15f6.eps}{Routers between individual networks use the
IP protocol to translate one network protocol to another.
\label{fig-ipprotocol}\label{pg-fig-ipprotocol}}
\end{figure}
%\end{figure}

\noindent As we have seen, networks are glued together by a vast array of
protocols.   Most of these protocols are implemented in software that
runs on general-purpose computers.   You can install software on your
personal computer to turn it into a Web server, an FTP server, or an
e-mail server.  Some of the lower level protocols are implemented in
software that runs on special-purpose computers, the routers.  Still
other protocols, such as the ethernet protocol, are implemented
directly in hardware.

Fortunately, we don't have to worry about the details of even the
highest level protocols in order to write client/server applications
in Java.   The {\tt java.net} (Fig.~\ref{fig-nethier}) package
supplies a powerful and easy-to-use set of classes that supports
network programming.

%\begin{figure}
\begin{figure}[h]
\figaleft{chptr15/p825f1.eps}{The {\tt java.net} package.
\label{fig-nethier}\label{pg-fig-nethier}}
\end{figure}
%\end{figure}

The {\tt java.net.URL} class provides a representation of the
%\begin{marginalnote}\tt java.net.*\end{marginalnote}
\marginnote{\tt java.net.*}
Internet's Uniform Resource Locator that we described earlier.  We'll
show how to use its methods to download WWW pages.  We'll also
look at an example that uses a URL and an input stream so that
files stored on the Web can be used as input files to a Java applet
or application program.

The {\tt Socket} and {\tt ServerSocket} classes provide methods that
let us develop our own networking applications.  They enable us to
make a direct connection to an Internet host, and read and write data
through \mbox{\tt InputStream}s and {\tt OutputStream}s.  As we will see,
this is no more difficult than reading and writing data to and from
files.  The {\tt DatagramPacket} and {\tt DatagramSocket} classes
provide support for even lower-level networking applications, based on
Internet packets.

\marginfig{chptr15/15f8.eps}{An applet that continuously displays
slides downloaded from the Web.
\label{fig-slideshow}\label{pg-fig-slideshow}}
\section{Using Network Resources from an Applet}
\noindent Suppose you want to write an applet that will automatically display a
slide show consisting of images or documents that you've prepared and
stored on your Web site.   Perhaps you can use such an applet to give
people who visit your site a tour of your campus
(Fig.~15.8). Or perhaps a company might use such an
applet to advertise its products.  In addition to making the slide show
available through its main Web site, you can imagine it running
\marginnote{Problem statement}
continuously on a computer kiosk in the company's lobby.


In order to solve this problem we have to be able to download and
display Web resources.   As you know, Web resources are
multimedia.  That is, they could be documents, images, sounds, video
clips, and so on.  All Web resources are specified in terms of their
%\begin{marginalnote}\it Specifying Web resources\end{marginalnote}
\marginnote{Specifying Web resources}
Uniform Resource Locators (URLs).  Thus, to download an image (or
an HTML file or audio clip), we usually type its URL into a Web
browser.  We want our program to know beforehand the URLs of the
images it will display, so there won't be any need for inputting
the URL.~We want to implement something like the following algorithm:

\begin{jjjlisting}
\begin{lstlisting}
repeat forever
    Generate the URL for the next slide.
    Use the URL to download the image or document.
    Display the image or document.
\end{lstlisting}
\end{jjjlisting}

\noindent A URL specification is just a {\tt String}, such as, 

\begin{jjjlisting}[28pc]
\begin{lstlisting}[commentstyle=\color{black}]
http://www.cs.trincoll.edu:80/~ram/jjj/slideshow/slide1.gif
\end{lstlisting}
\end{jjjlisting}

\noindent which describes how to retrieve the resource.   First, it
specifies the protocol or method that should be used to download the
resource ({\tt http}). Then, it provides the domain name of the server
that runs the protocol and the port number where the service is
running ({\tt www.cs.trincoll.edu:80}).  Next, the URL specifies the
resource's file name ({\tt \verb|~|ram/jjj/slideshow/slide1.gif}).

\section{From the Java Library: {\tt java.net.URL}}

%\addcontentsline{toc}{section}{\S~~~ From the Java Library: {\tt java.net.URL} }


\tBOXseven{Given such} a URL specification, how can we download its associated
resource?  Are there Java classes that can help us solve this problem?
Fortunately, there are.  First, the {\tt java.net.URL} class contains
methods to help retrieve the resource associated with a particular URL
(Fig.~15.9).  The {\tt URL} class represents a Uniform
Resource Locator.  The {\tt URL()} constructor shown here (there are
\marginfig{chptr15/p827f1.eps}{The {\tt java.net.URL} class.
\label{fig-p827f1}\label{pg-fig-p827f1}}
others) takes a URL specification as a {\tt String} and, assuming it
specifies a valid URL, it creates a {\tt URL} object.  If the URL
\WWWjava
specification is invalid, a {\tt MalformedURLException} is thrown.  A
URL might be invalid if the protocol were left off or if it is not a
known protocol.  The following simple code creates a URL for the home
page of our companion Web site:

\begin{jjjlisting}
\begin{lstlisting}
URL url;
try {
 url = 
  new URL("http://www.prenhall.com:80/morelli/index.html");
} catch (MalformedURLException e) {
 System.out.println("Malformed URL: " + url.toString());
}
\end{lstlisting}
\end{jjjlisting}

\noindent Note how we catch the {\tt MalformedURLException}
when we create a new {\tt URL}.

Once we have a valid {\tt URL} instance, it can be used to download
the data or object associated with it.  There are different ways to do
this.  The {\tt openConnection()} method creates a {\tt
URLConnection}, which can then be used to download the resource.  You
would only use this method if your application required extensive
control over the download process.  A much simpler approach would use
the {\tt openStream()} method.  This method will open an {\tt
%\begin{marginalnote}\it URLs and streams\end{marginalnote}
InputStream}, which you can then use to read the associated URL data
the same way you would read a file.  This method is especially useful
for writing Java applications (rather than applets).  As you might
guess, downloading Web resources is particularly easy from a Java
\marginnote{URLs and streams}
applet.  Now let's search around for other methods that we can use.

\subsection{Code Reuse: The {\tt java.applet.Applet} Class}
%\addcontentsline{toc}{subsection}{~~~\S~~~ Code Reuse: The {\tt java.applet.Applet} Class}
\noindent The {\tt java.applet.Applet} class itself contains several useful
methods for downloading and displaying Web resources.  These
methods are inherited by {\tt javax.swing.JApplet}:

\begin{jjjlisting}
\begin{lstlisting}
public class Applet extends Panel {
    public AppletContext getAppletContext();
    public AudioClip getAudioClip(URL url);
    public Image getImage(URL url);
    public void play(URL url);
    public void showStatus(String msg);
}
\end{lstlisting}
\end{jjjlisting}

\noindent As you see, both the {\tt getImage()} and {\tt getAudioClip()}
methods use a URL to download a resource.   An {\tt AudioClip} is a
sound file encoded in {\tt AU} format, a special type of encoding for
sound files.  The {\tt getImage()} method can return files in either
{\tt GIF} or {\tt JPEG} format, two popular image file formats.  The
{\tt play()} method downloads and plays an audio file in
one easy step.  For example, to download and play an audio clip within
an applet requires just two lines of code:

\begin{jjjlistingleft}[30pc]{-4pc}
\begin{lstlisting}
URL url;
try {
  url = new 
   URL("http://www.cs.trincoll.edu/~ram/jjj/slideshow/sound.au");
  play(url);
} catch (MalformedURLException e) {
    System.out.println("Malformed URL: " + url.toString()) ;
}
\end{lstlisting}
\end{jjjlistingleft}

\noindent Similarly, to download (and store a reference to)
an image is just as simple:

\begin{jjjlistingleft}[30.5pc]{-4.5pc}
\begin{lstlisting}
URL url;
try {
  url = new 
   URL("http://www.cs.trincoll.edu/~ram/jjj/slideshow/slide0.gif") ;
  imgRef = getImage(url);
} catch (MalformedURLException e) {
    System.out.println( "Malformed URL: " + url.toString()) ;
}
\end{lstlisting}
\end{jjjlistingleft}

\noindent So, it looks as if we've found the methods we need to
implement our slide show applet.   We'll use the {\tt URL()}
%\begin{marginalnote}\it What methods can we use?\end{marginalnote}
\marginnote{What methods can we use?}
constructor to create a URL from a {\tt String}, and we'll use
the {\tt Applet.getImage(URL)} method to retrieve the images
from the Web.
%%%\end{javalibrary}
%%%\begin{textblock}{1}(-0.75,4)
%%%\noindent\epsfig{file=e2kl-D1:PH-e2kl:Morelli:0333700MOREL:CommonArt:BX7_Texture/cyan_03.eps,%
%%%   height=20pc,width=3.25pc,clip=}
%%%\end{textblock}


\section{The Slide Show Applet}
\subsection*{Problem Specification}
\noindent Let's suppose our slide show will repeatedly display a set of images
named ``slide0.gif,'' ``slide1.gif,'' and ``slide2.gif.'' Suppose these
images are stored on a Web site on {\tt www.cs.trincoll.edu} and are
stored in a directory named {\tt /\verb!~!ram/jjj/slideshow}. This means
our program will have to load the following three URLs:

\begin{jjjlisting}[28pc]
\begin{lstlisting}[commentstyle=\color{black}]
http://www.cs.trincoll.edu/~ram/jjj/slideshow/slide0.gif
http://www.cs.trincoll.edu/~ram/jjj/slideshow/slide1.gif
http://www.cs.trincoll.edu/~ramjjj/slideshow/slide2.gif
\end{lstlisting}
\end{jjjlisting}

\noindent We want our show to cycle endlessly through these
images, leaving about 5 seconds between each slide.

\subsection*{User Interface Design}
\noindent The user interface for this applet doesn't contain any GUI
components.  It just needs to display an image every 5 seconds.   It can
use a simple {\tt paint()} method to display an image each time it is
repainted:

\begin{jjjlisting}
\begin{lstlisting}
public void paint(Graphics g) {
    if (currentImage != null)
        g.drawImage(currentImage, 10, 10, this);
}
\end{lstlisting}
\end{jjjlisting}

\noindent The assumption here is that the {\tt currentImage} instance
variable will be set initially to {\tt null}. Each time an image is
downloaded, it will be set to refer to that image.   Because
{\tt paint()} is called before the applet starts downloading the images, it
\marginfig{chptr15/p830f1.eps}{The {\tt SlideShow- Applet} downloads and displays the images.
\label{fig-p830f1}\label{pg-fig-p830f1}}
is necessary to guard against attempting to draw a {\tt null} image,
which would lead to an exception.

\subsection*{Problem Decomposition}
\noindent One problem we face with this applet is getting it to pause between
\marginfigvspace{15pt}{chptr15/p830f2.eps}{The {\tt Timer} class delays the applet thread between each slide.
\label{fig-p830f2}\label{pg-fig-p830f2}}
each slide.  One way to do this is to set up a loop that does nothing
for about 5 seconds:

\begin{jjjlisting}
\begin{lstlisting}
for (int k = 0; k < 1000000; k++ ) ;// Busy waiting
\end{lstlisting}
\end{jjjlisting}

\noindent However, this isn't a very good solution.  As we saw in
Chapter~14, this is a form of {\em busy waiting}\index{busy waiting}
that monopolizes the CPU, making it very difficult to break out of the
loop.  Another problem with this loop is that we don't really know how many
iterations to do to approximate 5 seconds of idleness.

A much better design would be to use a separate timer thread, which
can {\tt sleep()} for 5 seconds between each slide.  So our program
will have two classes: one to download and display the slides and one
to serve as a timer (Figs.~15.10 and 15.11).

\begin{BL}
\item  {\tt SlideShowApplet}---This {\tt JApplet}  subclass
will take care of downloading and displaying the images and starting
the timer thread.

\item  {\tt Timer}---This class will implement the
{\tt Runnable} interface so that it can run as a separate thread.  It will
repeatedly sleep for 5 seconds and then tell the applet to display the
next side.
\end{BL}

\JavaTIP{EFFECTIVE DESIGN}{Busy Waiting.}{Instead of busy waiting, a
thread that sleeps for a brief period on each iteration is a better
way to introduce a delay into an algorithm.}

\subsection{The {\tt SlideShowApplet} class}
\noindent What should we do with the images we download? Should we repeatedly
download and display them, or should we just download them once and
store them in memory? The second of these alternatives seems more
efficient.  If an image has already been downloaded, it would be
wasteful to download it again.

\JavaTIP{EFFECTIVE DESIGN}{Network Traffic.}{In general, a design
that minimizes network traffic is preferable.}

So we'll need an array to store the images.  Our slide show will then
%\begin{marginalnote}\it What data do we need?\end{marginalnote}
\marginnote{What data do we need?}
consist of retrieving the next image from the array and displaying
it.  To help us with this task, let's use a {\tt nextImg} variable as
an array index to keep track of the next image.  Even though it isn't
absolutely necessary, we could use a third variable here,
{\tt currentImage}, to keep track of the current image being
displayed.  Thus, our applet needs the following instance variables:

\begin{jjjlisting}
\begin{lstlisting}
private static final int NIMGS = 3;
private Image[] slide = new Image[NIMGS];
private Image currentImage = null;
private int nextImg = 0;
\end{lstlisting}
\end{jjjlisting}

Given these variables, let's now write a method to take care of
%\begin{marginalnote}\it Method design\end{marginalnote}
\marginnote{Method design}
choosing the next slide.  Recall that the {\tt paint()} method is
responsible for displaying {\tt currentImage}, so all this method needs to
do is to update both {\tt currentImage} and {\tt nextImg}.  This
method should be designed so that it can be called by the {\tt Timer}
thread whenever it is time to display the next slide, so it should be
a {\tt public} method.   It can be a {\tt void} method with no
parameters, because the applet already contains all the necessary information
to display the next slide.  Thus, there's no need for information
to be passed back and forth between {\tt Timer} and this method:


\begin{jjjlisting}
\begin{lstlisting}
public void nextSlide() {
    currentImage = slide[nextImg];
    nextImg = (nextImg + 1) % NIMGS;
    repaint();
}// nextSlide()
\end{lstlisting}
\end{jjjlisting}

\noindent The method's algorithm is very simple.  It sets
{\tt currentImage} to whatever {\tt slide} is designated by
{\tt nextImg} and it then updates {\tt nextImg}'s value.   Note here the use
of modular arithmetic to compute the value of {\tt nextImg}. Given
that {\tt NIMGS} is 3, this algorithm will cause {\tt nextImg} to take
on the repeating sequence of values 0, 1, 2, 0, 1, 2, and so forth.
Finally, the method calls {\tt repaint()} to display the image.

\JavaTIP{PROGRAMMING TIP}{Modular Arithmetic.}{Modular arithmetic \break
({\tt x \% N}) is useful for cycling repeatedly through the values
$0, 1, \ldots, N-1$.}

\noindent The applet's {\tt init()} method will have two tasks:

\begin{BL}
\item  Download and store the images in {\tt slide[]}.
\item  Start the {\tt Timer} thread.
\end{BL}

\noindent As we discussed, downloading Web resources requires
the use of the {\tt getImage()} method. Here we just place these
method calls in a loop:

\begin{jjjlisting}
\begin{lstlisting}
for (int k=0; k < NIMGS; k++) 
    slide[k] = getImage(getCodeBase(), 
                 "gifs/demo" + k + ".gif");
\end{lstlisting}
\end{jjjlisting}

\noindent Note here how we convert the loop variable
{\it k} into a {\tt String} and concatenate it right into the URL
specification.  This allows us to have URLs containing ``slide0.gif,''
``slide1.gif,'' and ``slide2.gif.'' This makes our program easily extensible
should we later decide to add more slides to the show.   Note also the
use of the class constant {\tt NIMGS} as the loop bound.  This too adds
to the program's extensibility.

\JavaTIP{PROGRAMMING TIP}{Concatenation.}{Concatenating an
integer value ({\tt k}) with a string lets you create
file names of the form {\tt file1.gif}, {\tt file2.gif},
and so on.}

The task of starting the {\tt Timer} thread involves creating an
instance of the {\tt Timer} class and calling its {\tt start()}
method:

\begin{jjjlisting}
\begin{lstlisting}
Thread timer = new Thread(new Timer(this));
timer.start();
\end{lstlisting}
\end{jjjlisting}

\noindent Note that {\tt Timer} is passed a reference to
{\tt this} applet.  This enables {\tt Timer} to call the applet's
{\tt nextSlide()} method every 5 seconds.   This programming technique is
known as {\it callback} and the {\tt nextSlide()} method is an example
of a {\bf callback method}\index{callback method} (Fig.~15.12).

\JavaTIP[false]{PROGRAMMING TIP}{Callback.}{Communication between two objects
can often be handled using a callback technique.  One object is passed
a reference to the other object.  The first object uses the reference
to call one of the public methods of the other object.}

\noindent This completes our design and development
of {\tt SlideShowApplet}, which is shown in Figure~\ref{fig-slideapplet}.
\marginfig{chptr15/p832f1.eps}{{\tt Timer} uses the {\tt nextSlide()}
method to {\it call back} the applet to remind it to switch to the
next slide.\label{fig-p832f1}\label{pg-fig-p832f1}}

\begin{figure}[h]
\jjjprogstart
\begin{jjjlisting}
\begin{lstlisting}
import java.awt.*;
import javax.swing.*;
import java.net.*;

public class SlideShowApplet extends JApplet  {
    public static final int WIDTH=300, HEIGHT=200;
    private static final int NIMGS = 3;
    private Image[] slide = new Image[NIMGS];
    private Image currentImage = null;
    private int nextImg = 0;

    public void paint(Graphics g) {
        g.setColor(getBackground());
        g.fillRect(0, 0, WIDTH, HEIGHT);
        if (currentImage != null)
            g.drawImage(currentImage, 10, 10, this);
    }//paint()

    public void nextSlide() {
        currentImage = slide[nextImg];
        nextImg = (nextImg + 1) % NIMGS;
        repaint();
    }// nextSlide()

    public void init() {
        for (int k=0; k < NIMGS; k++) 
            slide[k] = getImage(getCodeBase(), 
                             "gifs/demo" + k + ".gif");
        Thread timer = new Thread(new Timer(this));
        timer.start();
        setSize( WIDTH, HEIGHT );
    }// init()
}// SlideShowApplet
\end{lstlisting}
\end{jjjlisting}
\jjjprogstop{The {\tt SlideShowApplet} class.
\label{fig-slideapplet}\label{pg-fig-slideapplet}}
\end{figure}

\subsection{The {\tt Timer} Class}
\noindent The {\tt Timer} class is a subclass of {\tt Thread}, which means it
must implement the {\tt run()} method.   Recall that we never directly
%\begin{marginalnote}\it The timer thread\end{marginalnote}
\marginnote{The timer thread}
call a thread's {\tt run()} method.  Instead, we call its {\tt start()}
method, which automatically calls {\tt run()}.  This particular thread
has a very simple and singular function.  It should call the
{\tt SlideShowApplet.nextSlide()} method and then sleep for 5 seconds.
So its main algorithm will be:

\begin{jjjlisting}
\begin{lstlisting}
while (true) {
    applet.nextSlide();
    sleep( 5000 );
}
\end{lstlisting}
\end{jjjlisting}

\noindent However, recall that {\tt Thread.sleep()} throws the
{\tt Interrupted\-Excep\-tion}. This means that we'll have to
embed this while loop in a {\tt try/catch} block.

To call the applet's {\tt nextSlide()} method, we also need a
reference to the {\tt SlideShowApplet}, so we need to give it a
reference, such as an instance variable, as well as a constructor that
allows the applet to pass {\tt Timer} a reference to itself.

Given these design decisions, the complete implementation of
{\tt Timer} is shown in Figure~\ref{fig-timerthread}. To see how it works,
download it from the {\it Java, Java, Java} Web site and run it.

\begin{figure}[h]
\jjjprogstart
\begin{jjjlisting}
\begin{lstlisting}
public class Timer implements Runnable {
    private SlideShowApplet applet;

    public Timer( SlideShowApplet app ) {
        applet = app;
    }

    public void run() {
        try {
            while ( true ) {
                applet.nextSlide();
                Thread.sleep( 5000 );
            }
        } catch (InterruptedException e) {
            System.out.println(e.getMessage());
        }
    }// run()
}// Timer
\end{lstlisting}
\end{jjjlisting}
\jjjprogstop{The {\tt Timer} class.
\label{fig-timerthread}\label{pg-fig-timerthread}}
\end{figure}

\secEXRHone{Self-Study Exercise}
\begin{SSTUDY}

\item  Describe the design changes you would make to 
{\tt SlideShow\-Applet} if you wanted to play a soundtrack along with
your slides.  Assume that the sounds are stored in a sequence of
files, ``sound0.au,'' sound1.au,'' and so forth, on your Web site.
\end{SSTUDY}

\section{Using Network Resources from an \\Application}
\noindent The {\tt SlideShowApplet} illustrates the ease of downloading Web
resources from an applet.  However, applets have limited use in this
\mbox{regard,} because they come with rather severe security restrictions that
would make them a poor choice for most networking applications (see
%\begin{marginalnote}\it Applet restrictions\end{marginalnote}
\marginnote{Applet restrictions}
Section~\ref{sec-security}). For example, applets cannot save
files that they download, because they cannot access the host computer's
file system.  Similarly, an applet can only download files from the
same host from which it was downloaded.  This wouldn't be a problem for
the slide show applet, since we can simply store the slides in the
same directory as the applet itself.  So, we want to learn how to
download Web resources from a Java application.  The next example
illustrates a solution to this problem.

\subsection*{Problem Specification}
\noindent Suppose a realtor asks you to write a Java application that will
allow customers to view pictures and descriptions of homes from
%\begin{marginalnote}\it Problem statement\end{marginalnote}
\marginnote{Problem statement}
an online database.   The application should allow the customer to
select a home and should then display both an image of the home and a
text description of its features, such as square footage, asking
price, and so on.

Suppose that the database of image and text files is kept at a fixed
location on the Web, but the names of the files themselves may change.
This will enable the company to change the database as it sells the
homes.  The company will provide a text file that contains the names
of the files for the current selection of homes to input into the
program.  To simplify matters, both image and text files have the same
name but different extensions---for example, {\tt ranch.txt} and {\tt
ranch.gif}.  The data file will store just the names of the files, one
per line, giving it the following format:

\begin{jjjlisting}
\begin{lstlisting}
beautifulCape
handsomeRanch
lovelyColonial
\end{lstlisting}
\end{jjjlisting}

\subsection{Downloading a Text File from the Web}
\noindent This application requires us to solve three new problems:

\begin{NL}
\item  How do we download a text file of names that we want to use as
menu items?
\item  How do we download a text file and display it in a {\tt JTextArea}?
\item  How do we download and display an image file?
\end{NL}

\noindent The third problem is very similar to the problem we solved
in {\tt SlideShowApplet}, but here we can't use the {\tt Applet.getImage()}
method.   However, as we shall see, we can find a Java library method
to perform this task for us. Therefore, the most challenging part of this program is the task of
\marginnote{Understanding the problem} 
downloading a Web file and using its data in the program.

For this program we must make use of two types of data downloaded from
the Web.  The first will be the names of the image and document files.
We'll want to read these names and use them as menu items that the
user can select.  Second, once the user has selected a house to view,
we must download and display an image and a text description of the
house.  Downloading the text is basically the same as downloading the
file of names.  The only difference is that we need to display this
text in a {\tt JTextArea}.  Downloading the image file can be handled
in more or less the same way that it was handled in the {\tt
SlideShowApplet}--- by using a special Java method to download and
display the image file.

Clearly, the problems of downloading a file from the Web and reading a
file from the disk are quite similar.  Recall that we used streams to
handle the I/O operation when reading disk files.  The various {\tt
InputStream} and {\tt OutputStream} classes contained the {\tt read()}
and {\tt write()} methods needed for I/O.~The situation is the same
for downloading Web files.

Recall that the {\tt URL} class contains the {\tt openStream()}
method, which opens an {\tt InputStream} to the resource associated
with the {\tt URL}. Once the stream has been opened, you can read data
from the stream just as if it were coming from a file.  The program
%\begin{marginalnote}\it File download algorithm\end{marginalnote}
\marginnote{File download algorithm}
doesn't care whether the data are coming from a file on the Internet or
a file on the disk.  It just reads data from the stream.  So, to
download a data file from the Internet, regardless of whether it's a
text file, image file, audio file, or whatever, you would use the
following general algorithm:

\begin{jjjlisting}
\begin{lstlisting}
URL url;
InputStream data;
try {
    url = new URL(fileURL);          // Create a URL
    data = url.openStream(); // Open a stream to URL
  // READ THE FILE INTO MEMORY} // Read dat
    data.close();                     // Close the stream
} catch (MalformedURLException e) { // Thrown by URL()
    System.out.println(e.getMessage());
} catch( IOException e ) { 
    System.out.println(e.getMessage());
}
\end{lstlisting}
\end{jjjlisting}

\noindent  The algorithm consists of four basic steps:

\begin{BL}
\item  Create a {\tt URL} instance.
\item  Open an {\tt InputStream} to it.
\item  Read the data.
\item  Close the stream.
\end{BL}

\noindent Step 3 of this algorithm---read the data---involves many
lines of code and has, therefore, been left as a subtask suitable for
encapsulation within a method.


\subsection*{Reading the Data}
\noindent As we saw in the previous chapter, the algorithm for step 3 will
%\begin{marginalnote}\it Text or binary data?\end{marginalnote}
\marginnote{Text or binary data?}
depend on the file's data.  If it's a text file, we would like to read
one line at a time, storing the input in a {\tt String}.  If it's an
image or an audio file, we would read one {\tt byte} at a time.

Because our data are contained in a text file, we want to read one
%\begin{marginalnote}\it What library methods can we use?\end{marginalnote}
\marginnote{\raggedright\it What library methods can we use?}
line at a time.  The {\tt BufferedReader} class contains a
{\tt readLine()} method that returns either a {\tt String} storing
the line or the value {\tt null} when it reaches the end of file.
The following method shows how you would read a text file into
the program's {\tt JTextArea}, which is named {\tt display}:

\begin{jjjlisting}
\begin{lstlisting}
private void readTextIntoDisplay(URL url) 
                                     throws IOException {
  BufferedReader data
    = new BufferedReader(
               new InputStreamReader(url.openStream()));
  display.setText("");            // Reset the text area
  String line = data.readLine();
  while (line != null)  {       // Read each line
    display.append(line + "\n");// Add to display
    line = data.readLine();
  }
  data.close();
}// readTextIntoDisplay()
\end{lstlisting}
\end{jjjlisting}

\noindent The method is passed the file's URL and it uses the
{\tt URL.openStream()} method to open the input stream.  Note
that the method throws \mbox{\tt IOException}, which means that
%\begin{marginalnote}\it I/O exceptions\end{marginalnote}
\marginnote{I/O exceptions}
any I/O exceptions that get raised will be handled by the calling
method.

In this example, the input algorithm reads each line of the file and
adds it to the {\tt display}.  For our real estate application, the
same basic algorithm can be used to read the names of the data files
and store them in a menu from which a user makes selections.  For
example, if we use a {\tt JComboBox} menu named {\tt homeChoice}, we
would simply add each line to it:

\begin{jjjlisting}
\begin{lstlisting}
String line = data.readLine();
while (line != null) {
    homeChoice.addItem(line);
    line = data.readLine();
}
\end{lstlisting}
\end{jjjlisting}

\subsection*{Interface Design}

\begin{figure}[tb]
\figaright{chptr15/15f11.eps}{User interface design for the real
estate application.
\label{fig-realestategui}\label{pg-fig-realestategui}}
\end{figure}

\noindent The interface for this application is very important.  It
should provide some means to display a text file and an image.  The
text file can be displayed in a {\tt JTextArea}, and the image  can
be drawn on a {\tt JPanel}.

Next, let's consider the types of controls a user might need. The
customer should be allowed to select a home to view from a menu of
options.  Because the program will have the list of available homes,
it can provide the options in a {\tt JComboBox} pull-down menu.

To create an appropriate layout, we want to make sure that the
controls, the image, and {\tt JTextArea} all have their own region of
%\begin{marginalnote}\it Interface layout\end{marginalnote}
%\marginnote{Interface layout}
the application's window.  This suggests a {\tt BorderLayout}, which is
the default layout for a {\tt JFrame}.  We can put the {\tt JComboBox}
menu at the ``North'' border, and the image and text on the ``West''
and ``East'' borders, respectively.  Figure~\ref{fig-realestategui}
illustrates these various design decisions.
\marginfig{chptr15/p838f1.eps}{The {\tt RealEstate- Viewer} class defines 
the user interface.
\label{fig-p838f1}\label{pg-fig-p838f1}}

\subsection*{Problem Decomposition: {\tt RealEstateViewer}}
\noindent The task of downloading and displaying information from the Internet
%\begin{marginalnote}\it What classes do we need?\end{marginalnote}
%%\marginnote{What classes do we need?}
is best handled by two separate classes: One to perform the
downloading and user interface tasks and the other to take care of
displaying the image.

The task of downloading the image and text files from the Web can be
handled by the program's main class, the {\tt RealEstateViewer}, which
will also handle the user interface (Fig.~15.16).  As the
application's top-level window, {\tt RealEstateViewer} will is
subclass of {\tt JFrame}.  Because its controls will include a {\tt
JComboBox}, it must implement the {\tt itemStateChanged()} method of
the {\tt ItemListener} interface.

What components and other instance variables will we need for this
class? According to our interface design, it will need a {\tt JComboBox},
a {\tt JTextArea}, and the {\tt ImagePanel}. Because it will be
downloading images, it will need an {\tt Image} variable.


\begin{figure}[tb]
\jjjprogstart
\begin{jjjlisting}
\begin{lstlisting}
import java.awt.*;
import java.awt.event.*;
import java.net.*;
import java.io.*;
import javax.swing.*;

public class RealEstateViewer extends JFrame 
                                implements ItemListener {
  public static final int WIDTH=400,HEIGHT=200;
  private final String dataFileURL =
    "http://java.trincoll.edu/~jjjava/homes/homes.txt";
  private final String baseURL = 
    "http://java.trincoll.edu/~jjjava/homes/";
  private JTextArea display = new JTextArea(20,20);
  private JComboBox homeChoice = new JComboBox();
  private ImagePanel imagePanel = new ImagePanel(this);
  public Image currentImage = null;

  public RealEstateViewer () { }      // Stub Constructor
                            // ItemListener interface
  public void itemStateChanged(ItemEvent evt) { } // Stub

  public static void main(String args[]) {
    RealEstateViewer viewer = new RealEstateViewer();
    viewer.setSize(viewer.WIDTH,viewer.HEIGHT);
    viewer.setVisible(true);
    viewer.addWindowListener(new WindowAdapter() {     
      public void windowClosing(WindowEvent e) {
        System.exit(0);  // Quit the application
      }
    });
  }// main()
}// RealEstateViewer
\end{lstlisting}
\end{jjjlisting}
\jjjprogstop{The {\tt RealEstateViewer}, Version 1.
\label{fig-realviewer1}\label{pg-fig-realviewer1}}
\end{figure}


The constants used by this application include the URL string for the
%\begin{marginalnote}\it What data do we need?\end{marginalnote}
%%\marginnote{What data do we need?}
data file. Also, because all the
images and data files will start with the same prefix,

\begin{jjjlisting}
\begin{lstlisting}[commentstyle=\color{black}]
http://java.trincoll.edu/~jjjava/homes/
\end{lstlisting}
\end{jjjlisting}

\noindent we should make this a constant in the program.  These preliminary
decisions lead to the initial version of {\tt RealEstateViewer} shown
in Figure~\ref{fig-realviewer1}. Note that the {\tt main()} method
merely creates an instance of the application and shows it.  Note also
that the {\tt currentImage} variable is declared {\tt public}. This
will let the {\tt ImagePanel} have direct access to {\tt currentImage}
whenever it needs to display a new image.

\pagebreak
\subsection*{The {\tt ImagePanel} Class}
\noindent We'll use a second class, the {\tt ImagePanel}, to handle 
displaying  the image (Figs.~15.18
%\begin{figure}
%\begin{graphic}
\marginfig{chptr15/p839f1.eps}{An overview of the {\tt ImagePanel} class.
\label{fig-p839f1}\label{pg-fig-p839f1}}
%\end{graphic}
%\end{figure}
and~\ref{fig-imagepanel}). The reason we use a separate class for this
task is that we want the image to appear in its own panel (which
appears on the West border of the main window). In addition to its
constructor, the only method needed in this class
%\begin{marginalnote}\it Overriding {\tt paintComponent()}\end{marginalnote}
%\marginnote{Overriding {\tt paintComponent()}}
is the {\tt paintComponent()} method.  This method will be called
automatically whenever the main window is repainted.  Its task is
simply to get the current image from its parent frame and display
it.  Note that a reference to the parent frame is passed to the object
in its constructor.

\subsection*{Method Decomposition}
\noindent The stub methods listed in the initial version of
{\tt RealEstateViewer} (Fig.~\ref{fig-realviewer1}) outline the main
tasks required by the application.  Some of these methods are very
simple and even trivial to implement.  Others should be broken up into
subtasks.


\begin{figure}[tb]
\jjjprogstart
\begin{jjjlisting}
\begin{lstlisting}
import javax.swing.*;
import java.awt.*;

public class ImagePanel extends JPanel {

    private RealEstateViewer frame;

    public ImagePanel(RealEstateViewer parent) {
        frame = parent;
    }

    public void paintComponent(Graphics g) {
        if (frame.currentImage != null)
            g.drawImage(frame.currentImage, 0, 0, this);
    }
}// ImagePanel
\end{lstlisting}
\end{jjjlisting}
\jjjprogstop{The {\tt ImagePanel} class.
\label{fig-imagepanel}\label{pg-fig-imagepanel}}
\end{figure}

The constructor method should be responsible for creating the user
interface, most of which will involve the routine tasks of registering
a listener for the {\tt homeChoice} menu and setting up an appropriate
layout that implements the design we developed for the user interface:

\begin{jjjlisting}
\begin{lstlisting}
public RealEstateViewer () {
    super("Home Viewer Application");// Set window title
    homeChoice.addItemListener( this);
    this.getContentPane().add("North",homeChoice);
    this.getContentPane().add("East",display);
    this.getContentPane().add("Center",imagePanel);
    display.setLineWrap(true);
    initHomeChoices();      // Set up choice box
    showCurrentSelection(); // Display current home
}
\end{lstlisting}
\end{jjjlisting}

\noindent Note the last two statements of the method.
The first sets up the {\tt JComboBox} by reading its contents from a file
stored in the company's database.  Because that task will require
several statements, we define it as a separate method,
{\tt initHomeChoices()}, and defer its development for now.  Similarly,
the task of displaying the current menu choice has been organized
into the {\tt showCurrentSelection()} method, whose development
we also defer for now.

The {\tt itemStateChanged()} method is called automatically when the
%\begin{marginalnote}\it ItemListener\end{marginalnote}
\marginnote{ItemListener}
user selects a home from the {\tt JComboBox} menu.  Its task is to
download and display information about the current menu selection.
To do this, it can simply call the {\tt showCurrentSelection()} method:

\begin{jjjlisting}
\begin{lstlisting}
public void itemStateChanged(ItemEvent evt) {
    showCurrentSelection();
}
\end{lstlisting}
\end{jjjlisting}


\subsection*{Downloading the Menu Items}
\noindent Recall that according to our specification, the real estate firm
stores its current listing of homes in a text file, one home per line.
The {\tt initHomeChoices()} method downloads the text and uses its
contents to set up the items in the {\tt homeChoice} {\tt JComboBox}
menu:

\begin{jjjlisting}
\begin{lstlisting}
private void initHomeChoices() {
  try {
    URL url = new URL(dataFileURL);
    BufferedReader data = new BufferedReader(
        new InputStreamReader(url.openStream()));
    String line = data.readLine();
    while (line != null) {
      homeChoice.addItem(line);
      line = data.readLine();
    }
    data.close();
  } catch (MalformedURLException e) {
    System.out.println( "ERROR: " + e.getMessage());
  } catch (IOException e) {
    System.out.println( "ERROR: " + e.getMessage());
  }
}// initHomeChoices()
\end{lstlisting}
\end{jjjlisting}

\noindent It uses the algorithm we developed earlier for downloading
a text file.  Each line of the text file represents a menu item, so, as
each line is read by {\tt readLine(data)}, it is added to the
{\tt JComboBox} menu.

\subsection*{Downloading and Displaying Home Information}
\noindent The {\tt showCurrentSelection()} method is responsible for downloading and
displaying images and text files whenever the user selects a home
to view.  Recall that our specification called for using the name of
the menu item as a basis for constructing the name of its corresponding
text file and image file.   Therefore, the basic algorithm we need is

\begin{BL}
\item  Get the user's home choice.
\item  Create a URL for the associated text file.
\item  Download and display the associated text file.
\item  Create a URL for the associated GIF file.
\item  Download and display the image.
\end{BL}

\noindent Because downloading a text document requires stream
processing, we should handle that in a separate method.  The task of
downloading an image file is also a good candidate for a separate
%\begin{marginalnote}\it Method decomposition\end{marginalnote}
\marginnote{Method decomposition}
method.  Both of these methods will use a URL, so we can leave that
task up to {\tt showCurrent\-Selection()} itself.  The
{\tt showCurrentSelection()} method will create the URLs and then invoke
the appropriate methods to download and display the resources:

\begin{jjjlisting}[28pc]
\begin{lstlisting}
private void showCurrentSelection() {
  URL url = null;
                              // Get user's choice
  String choice = homeChoice.getSelectedItem().toString();     
  try {                   // Create url and download file         
    url = new URL(baseURL + choice + ".txt");
    readTextIntoDisplay(url);          
                         // Create url and download image
    url = new URL(baseURL + choice + ".gif"); 
    currentImage = Toolkit.getDefaultToolkit().getImage(url);
    Toolkit.getDefaultToolkit().beep(); // Beep user
    repaint();
  } catch (MalformedURLException e) {
    System.out.println( "ERROR: " + e.getMessage()) ;
  } catch (IOException e) {
    System.out.println("ERROR: " + e.getMessage()) ;
  } // Try/catch block
} // showCurrentSelection()
\end{lstlisting}
\end{jjjlisting}

\noindent Note that we have also elected to handle both the
{\tt MalformedURLException} and {\tt IOException} in this method.  The
advantage of this design is that it separates exception handling from the
normal algorithm and organizes it into one method.  Finally, note how
string concatenation is used to build the URL specifications,
each of which consists of three parts: the {\tt baseURL}, the
user's {\tt choice}, and the file extension.

The task of reading the text file and displaying its contents has been
encapsulated into the {\tt readTextIntoDisplay()} method.  This
{\tt private} utility method performs a standard file-reading algorithm
using the {\tt readLine()} method that we developed
earlier.  Figure~\ref{fig-realestatescreen}
provides a view of the program's appearance as it is displaying
information to a user. Figure~\ref{fig-realestateclass} provides the complete
implementation of this program.

\begin{figure}[hbt]
\figa{chptr15/15f14.eps}{The {\tt RealEstate- Viewer} program
downloads images and documents over the Web.
\label{fig-realestatescreen}\label{pg-fig-realestatescreen}}
\end{figure}

\begin{figure}[p]
\jjjprogstart
\begin{jjjlisting}
\begin{lstlisting}
import java.awt.*;
import java.awt.event.*;
import java.net.*;
import java.io.*;
import javax.swing.*;

public class RealEstateViewer extends JFrame 
                                implements ItemListener {
  public static final int WIDTH=400,HEIGHT=200;
  private final String dataFileURL =
    "http://java.trincoll.edu/~jjjava/homes/homes.txt";
  private final String baseURL = 
    "http://java.trincoll.edu/~jjjava/homes/";
  private JTextArea display = new JTextArea(20,20);
  private JComboBox homeChoice = new JComboBox();
  private ImagePanel imagePanel = new ImagePanel(this);
  public Image currentImage = null;

  public RealEstateViewer () {
    super("Home Viewer Application"); // Set window title
    homeChoice.addItemListener( this);
    this.getContentPane().add("North",homeChoice);
    this.getContentPane().add("East",display);
    this.getContentPane().add("Center",imagePanel);
    display.setLineWrap(true);
    initHomeChoices();        // Set up the choice box
    showCurrentSelection();   // Display the current home
  } // RealEstateViewer()

  private void initHomeChoices() {
    try {
      URL url = new URL(dataFileURL);
      BufferedReader data = new BufferedReader(
               new InputStreamReader(url.openStream()));
      String line = data.readLine();
      while (line != null) {
        homeChoice.addItem(line);
        line = data.readLine();
      } data.close();
    } catch (MalformedURLException e) {
        System.out.println("ERROR: " + e.getMessage()) ;
    } catch (IOException e) {
        System.out.println("ERROR: " + e.getMessage()) ;
    }
  }// initHomeChoices()
\end{lstlisting}
\end{jjjlisting}
\jjjprogstop{The {\tt RealEstateViewer} class, Part I.
\label{fig-realestateclass}\label{pg-fig-realestateclass}}
\end{figure}


\begin{figure}[p]
\addtocounter{figure}{-1}
\jjjprogstart
\begin{jjjlisting}[30pc]
\begin{lstlisting}
  private void readTextIntoDisplay(URL url) throws IOException {
    BufferedReader data
      = new BufferedReader(
                      new InputStreamReader(url.openStream()));
    display.setText("");            // Reset the text area
    String line = data.readLine();
    while (line != null)  {         // Read each line
      display.append(line + "\n");  // And add it to the display
      line = data.readLine();
    } data.close();
  }// readTextIntoDisplay()

  private void showCurrentSelection() {
    URL url = null;                          // Get user's choice
    String choice = homeChoice.getSelectedItem().toString();     
    try {
      url = new URL(baseURL + choice + ".txt") ;    // Create URL
      readTextIntoDisplay(url); // Download and display text file
      url = new URL(baseURL + choice + ".gif");     // Create URL
                                                // Download image
      currentImage = Toolkit.getDefaultToolkit().getImage(url);
      Toolkit.getDefaultToolkit().beep();       // Alert the user
      repaint();
    } catch (MalformedURLException e) {
        System.out.println( "ERROR: " + e.getMessage()) ;
    } catch (IOException e) {
        System.out.println("ERROR: " + e.getMessage()) ;
    }
  }// showCurrentSelection()

  public void itemStateChanged(ItemEvent evt) {
    showCurrentSelection();
  } // itemStateChanged()

  public static void main(String args[]) {
    RealEstateViewer viewer = new RealEstateViewer();
    viewer.setSize(viewer.WIDTH,viewer.HEIGHT);
    viewer.setVisible(true);
    viewer.addWindowListener(new WindowAdapter() {     
      public void windowClosing(WindowEvent e) {
        System.exit(0);    // Quit the application
      }
    });
  }// main()
}// RealEstateViewer
\end{lstlisting}
\end{jjjlisting}
\jjjprogstop{{\it (continued)} {\tt RealEstateViewer}, Part II.}
\end{figure}

\subsection{Reusing Code}
\noindent As in other examples we have developed, our discovery and use of the
{\tt Toolkit.getImage()} method and other classes from the Java class
library illustrate an important principle of object-oriented
programming.

\JavaTIP[false]{EFFECTIVE DESIGN}{Code Reuse.}{Before writing code to
perform a particular task, search the available libraries to see if
there is already code that performs that task.}

\noindent An important step in designing object-oriented programs
is making appropriate use of existing classes and methods.   In some
cases, you want to directly instantiate a class and use its methods to
perform the desired tasks.  In other cases, it is necessary to create a
subclass (inheritance) or implement an interface (inheritance) in
order to gain access to the methods you need.

Of course, knowing what classes exist in the libraries is something
that comes with experience.  There's no way that a novice Java
programmer would know about, say, the {\tt Toolkit.getImage()} method.
However, one skill or habit that you should try to develop is
always to ask yourself the question: ``Is there a method that will do
what I'm trying to do here?''  That question should be the first
question on your search through the libraries and reference books.

\begin{jjjlisting}
\begin{lstlisting}[commentstyle=\color{black}]
http://java.sun.com/j2se/1.5.0/docs/api/
\end{lstlisting}
\end{jjjlisting}

\RHrightsec{15.6}{Client/Server Communication via {\tt Socket}s}
\section{Client/Server Communication via {\tt Socket}s}
\RHright{Client/Server Communication via {\tt Socket}s}
\noindent As we said earlier, many networking applications are based on the
client/ server model.  According to this model, a task is viewed as a
service that can be requested by clients and handled by servers.  In
this section, we develop a simple client/server framework based on a
{\em socket} connection between the client and the server.

A {\bf socket}\index{socket} is a simple communication channel through which two
programs communicate over a network.  A socket supports two-way
communication between a client and a server, using a well-established
protocol.  The protocol simply prescribes rules and behavior that both
the server and client must follow in order to establish two-way
communication.

According to this protocol, a server program creates a socket at a
\marginnote{Sockets and ports}
certain port and waits until a client requests a connection.  A {\bf
port}\index{port} is a particular address or entry point on the host
computer, which typically has hundreds of potential ports.  It is
usually represented as a simple integer value.  For example, the
standard port for an HTTP (Web) server is 80. Once the connection is
established, the server creates input and output streams to the socket
and begins sending messages to and receiving messages from the client.
Either the client or the server can close the connection, but it's
usually done by the client.

\JavaTIP[false]{DEBUGGING TIP}{Reserved Port Numbers.}{Port numbers below 1024
are reserved for system use and should not be used by an application
program.}

To help clarify this protocol, think of some service performed by a
human using a telephone connection.   The ``server'' waits for the
%\begin{marginalnote}\it Client/server protocol\end{marginalnote}
\marginnote{Client/server protocol}
phone to ring.  When it rings, the server picks it up and begins
communicating with the client.  A socket, combined with input
and output streams, is something like a two-way phone connection.

From the client's side, the protocol goes as follows:  The client
creates a socket and attempts to make a connection to the server.  The
client has to know the server's URL and the port at which the service
exists.  Once a connection has been established, the client creates
input and output streams to the socket and begins exchanging messages
with the server.  The client can close the connection when the service
is completed.

Think again of the telephone analogy.  A human client picks up the
phone and dials the number of a particular service.  This is analogous
to the client program creating a socket and making a connection to a
server.  Once the service agent answers the phone, two-way
communication between the client and the server can begin.

Figure~\ref{fig-clientserverpict} provides a view of the client/server
connection.   Note that a socket has two channels.  Once a connection
%\begin{marginalnote}\it Sockets and channels\end{marginalnote}
\marginnote{Sockets and channels}
has been established between a client and a server, a single two-way
channel exists between them.  The client's output stream is connected
to the server's input stream.  The server's output stream is connected
to the client's input stream.


\JavaTIP{PROGRAMMING TIP}{Socket Streams.}{Each socket has two
streams, one for input and one for output.}

\begin{figure}[hbt]
\figa{chptr15/15f16.eps}{A socket is a two-channel communication link.
\label{fig-clientserverpict}\label{pg-fig-clientserverpict}}
\end{figure}

\subsection{The Server Protocol}
\noindent Let's now see how a client/server application would be coded in Java.
The template in Figure~\ref{fig-serverpro} shows the code that is
necessary on the server side.   The first step the server takes is to
create a {\tt ServerSocket}.  The first argument to the
{\tt ServerSocket()} method is the port at which the service will
reside.  The second argument specifies the number of clients that can
be backlogged, waiting on the server, before a client will be refused
service.   If more than one client at a time should request service,
Java would establish and manage a waiting list, turning away clients
when the list is full.

The next step is to wait for a client request.  The {\tt accept()}
%\begin{marginalnote}\it Waiting for client requests\end{marginalnote}
\marginnote{Waiting for client requests}
method will {\it block} until a connection is established.   The Java
system is responsible for waking the server when a client request
is received.

\begin{figure}[thb]
\jjjprogstart
\begin{jjjlisting}
\begin{lstlisting}
Socket socket;    // Reference to the socket
ServerSocket port;// The port where the server will listen
try {
    port = new ServerSocket(10001, 5); // Create a port
    socket = port.accept();  // Wait for client to call

 // Communicate with the client

    socket.close();
} catch (IOException e) {
    e.printStackTrace();
}
\end{lstlisting}
\end{jjjlisting}
\jjjprogstop{Template for the server protocol.
\label{fig-serverpro}\label{pg-fig-serverpro}}
\end{figure}

Once a connection is established, the server can begin communicating
with the client.  As we have suggested, a socket connection is like
a two-way telephone conversation.  Both the client and server can
``talk'' back and forth to each other.  The details of this step
are not shown here.  As we will see, the two-way conversation
is managed by connecting both an input and an output stream to
the socket.

Once the conversation between client and server is finished---once
the server has delivered the requested service---the server can close
the connection by calling {\tt close()}. Thus, there are four steps
involved on the server side:

\begin{BL}
\item  Create a {\tt ServerSocket} and establish a port number.
\item  Listen for and accept a connection from a client.
\item  Converse with the client.
\item  Close the socket.
\end{BL}

\noindent What distinguishes the server from the client is that
the server establishes the port and accepts the connection.



\subsection{The Client Protocol}
\noindent The client protocol (Fig.~\ref{fig-clientpro}) is just as easy to
implement.   Indeed, on the client side there are only three steps involved.
The first step is to request a connection to the server.  This is done
%\begin{marginalnote}\it Initiating a request\end{marginalnote}
\marginnote{Initiating a request}
in the {\tt Socket()} constructor by supplying the server's URL and
port number.  Once the connection is established, the client can carry
out two-way communication with the server.  This step is not shown
here.  Finally, when the client is finished, it can simply {\tt close()}
the connection.  Thus, from the client side, the protocol involves just
three steps:

\begin{BL}
\item  Open a socket connection to the server, given its address.
\item  Converse with the server.
\item  Close the connection.
\end{BL}

\noindent What distinguishes the client from the server is that the
client initiates the two-way connection by requesting the service.

\begin{figure}[thb]
\jjjprogstart
\begin{jjjlisting}
\begin{lstlisting}
Socket connection;     // Reference to the socket
try {                  // Request a connection
  connection = new Socket("java.cs.trincoll.edu", 10001);

   // Carry on a two-way communication

    connection.close();   // Close the socket
} catch (IOException e ) {
    e.printStackTrace();
}
\end{lstlisting}
\end{jjjlisting}
\jjjprogstop{Template for the client protocol.
\label{fig-clientpro}\label{pg-fig-clientpro}}
\end{figure}

\subsection{A Two-Way Stream Connection}
\noindent Now that we have seen how to establish a socket connection between a
client and server, let's look at the actual two-way communication that
takes place.  Because this part of the process will be exactly the same
for both client and server, we develop a single set of methods,
{\tt writeToSocket()} and {\tt readFromSocket()}, that may be called by
either.

The {\tt writeToSocket()} method takes two parameters, the
%\begin{marginalnote}\it Output routine\end{marginalnote}
\marginnote{Output routine}
{\tt Socket} and a {\tt String}, which will be sent to the
process on the other end of the socket:

\begin{jjjlisting}
\begin{lstlisting}
protected void writeToSocket(Socket sock, String str) 
                             throws IOException {
    oStream = sock.getOutputStream();
    for (int k = 0; k < str.length(); k++)
        oStream.write(str.charAt(k));
}// writeToSocket()
\end{lstlisting}
\end{jjjlisting}


\noindent If {\tt writeToSocket()} is called by the server, then
the string will be sent to the client.  If it is called by the client,
the string will be sent to the server.

The method is declared {\tt protected} because we will define it in a
%\begin{marginalnote}\it Protected methods\end{marginalnote}
\marginnote{Protected methods}
super\-class so that it can be inherited and used by both the client and
server classes.  Note also that the method declares that it throws an
{\tt IOException}. Because there's no way to fix an {\tt IOException},
we'll just let this exception be handled elsewhere, rather than
handling it within the method.

In order to write to a socket we need only get the socket's
{\tt OutputStream} and then write to it.  For this example, {\tt oStream} is
an instance variable of the client/server superclass.   We use the
{\tt Socket.getOutputStream()} method to get a reference to the socket's
output stream.  Note that we are not creating a new output stream
here.  We are just getting a reference to an existing stream, which was
created when the socket connection was accepted.   Note also that we do
not close the output stream before exiting the method.   This is
important.  If you close the stream, you will lose the ability to
communicate through the socket.

\JavaRule{Socket Streams.}{When a socket is created,
it automatically creates its own streams.  To use one you just need
to get a reference to it.}

\JavaTIP{DEBUGGING TIP}{Socket Streams.}{After writing to or reading
from a socket I/O stream, do not close the stream.  That would make the
socket unusable for subsequent I/O.}

Given the reference to the socket's output stream, we simply write
each character of the string using the {\tt OutputStream.write()}
method.  This method writes a single {\tt byte}. Therefore, the input
stream on the other side of the socket must read bytes and convert
them back into characters.

\JavaTIP{EFFECTIVE DESIGN}{Designing a Protocol.}{In designing two-way
communication between a client and a server, you are designing a
protocol that each side must use.  Failure to design and implement a
clear protocol will cause the communication to break down.}

\noindent The {\tt readFromSocket()} method takes a {\tt Socket} parameter
%\begin{marginalnote}\it Input routine\end{marginalnote}
\marginnote{Input routine}
and returns a {\tt String}:

\begin{jjjlisting}
\begin{lstlisting}
protected String readFromSocket(Socket sock) 
                                throws IOException {
    iStream = sock.getInputStream();
    String str="";
    char c;
    while (  ( c = (char) iStream.read() ) != '\n')
        str = str + c + "";
    return str;
}
\end{lstlisting}
\end{jjjlisting}


\noindent It uses the {\tt Socket.getInputStream()} method to obtain a
reference to the socket's input stream, which has already been
created.   So here again it is important that you don't close the
stream in this method.  A socket's input and output streams will be
closed automatically when the socket connection itself is closed.

The {\tt InputStream.read()} method reads a single byte at a time from
the input stream until an end-of-line character is received.  For this
particular application, the client and server will both read and write
one line of characters at a time.  Note the use of the cast operator
{\tt (char)} in the {\tt read()} statement.   Because {\tt byte}s are
being read, they must be converted to {\tt char} before they can be
compared to the end-of-line character or concatenated to the
{\tt String}. When the read loop encounters an end-of-line character, it
terminates and returns the {\tt String} that was input.

\JavaTIP{DEBUGGING TIP}{Bytes and Chars.}{It is a syntax error to compare
a {\tt byte} and a {\tt char}. One must be converted to the other using
an explicit cast operator.}

\section{CASE STUDY: Generic Client/Server Classes}
%%%\begin{casestudy}{Generic Client/Server Classes}
%%%\secCSFH{CASE STUDY: Generic Client/Server Classes}
\label{sec-clientserver}\label{pg-sec-clientserver}
\noindent Suppose your boss asks you to set up generic client/server classes
%\begin{marginalnote}\it Problem statement\end{marginalnote}
\marginnote{Problem statement}
that can be used to implement a number of related client/server
applications.  One application that the company has in mind is a query
service, in which the client would send a query string to the server,
and the server would interpret the string and return a string that
provides the answer.  For example, the client might send the query,
``Hours of service,'' and the client would respond with the company's
business hours.

Another application the company wants will enable the client to fill out an
order form and transmit it as a string to the server.  The server will
interpret the order, fill it, and return a receipt, including
instructions as to when the customer will receive the order.

All of the applications to be supported by this generic client/server
will communicate via strings, so something very much like the
{\tt readFromSocket()} and {\tt writeToSocket()} methods can be used for
their communication.  Of course, you want to design classes so they
can be easily extended to support byte-oriented, two-way
communications, should that type of service become needed.

In order to test the generic models, we will subclass them to create a
%\begin{marginalnote}\it The echo service\end{marginalnote}
\marginnote{The echo service}
simple echo service.  This service will echo back to the client any
message that the server receives.  For example, we'll have the client
accept keyboard input from the user and then send the user's input to
the server and simply report what the server returns.   The following
shows the output generated by a typical client session:

\begin{jjjlisting}
\begin{lstlisting}
CLIENT: connected to 'java.cs.trincoll.edu'
SERVER: Hello, how may I help you?
CLIENT: type a line or 'goodbye' to quit
INPUT: hello
SERVER: You said 'hello'
INPUT: this is fun
SERVER: You said 'this is fun'
INPUT: java java java
SERVER: You said 'java java java'
INPUT: goodbye
SERVER: Goodbye
CLIENT: connection closed
\end{lstlisting}
\end{jjjlisting}

\noindent On the server side, the client's
message will be read from the input stream and then simply echoed back
(with  some additional characters attached) through the output
stream.  The server doesn't display a trace of its activity other than
to report when connections are established and closed.  We will code
the server in an infinite loop so that it will accept connections
from a (potentially) endless stream of clients.  In fact, most servers
are coded in this way.  They are designed to run forever and must be
restarted whenever the host that they are running needs to be rebooted.
The output from a typical server session is as follows:

\begin{jjjlistingleft}[33pc]{-7pc}
\begin{lstlisting}
Echo server at java.cs.trincoll.edu/157.252.16.21 waiting for connections
Accepted a connection from java.cs.trincoll.edu/157.252.16.21
Closed the connection

Accepted a connection from java.cs.trincoll.edu/157.252.16.21
Closed the connection
\end{lstlisting}
\end{jjjlistingleft}

\JavaTIP{EFFECTIVE DESIGN}{Infinite Loop.}{A server is an
application that's designed to run in an infinite loop.  The loop
should be exited only when some kind of exception occurs.}

\subsection{Object-Oriented Design}
\noindent A suitable solution for this project will make extensive use of
object-oriented design principles.  We want {\tt Server} and
{\tt Client} classes that can easily be subclassed to support a wide
variety of services.  The solution should make appropriate use of {\it
inheritance} and {\it polymorphism} in its design.  Perhaps the best
way to develop our generic class is first to design the echo service,
as a typical example, and then generalize it.

\subsection*{The Threaded Root Subclass: {\tt ClientServer} }
\noindent One lesson we can draw at the outset is that both clients and servers
\marginfig{chptr15/p851f1.eps}{Overall design of a client/server application.
\label{fig-p851f1}\label{pg-fig-p851f1}}
use basically the same socket I/O methods.  Thus, as we've seen, the
{\tt readFromSocket()} and {\tt writeToSocket()} methods could be used
by both clients and servers.  Because we want all clients and servers to
inherit these methods, they must be placed in a common
superclass.  Let's name this the {\tt ClientServer} class.


Where should we place this class in the Java hierarchy? Should it be a
%\begin{marginalnote}\it The class hierarchy\end{marginalnote}
direct subclass of {\tt Object}, or should it extend some other class
that would give it appropriate functionality?  One feature that would
make our clients and servers more useful is if they were independent
%���\marginnote{The class hierarchy}
threads.  That way they could be instantiated as part of another object
and given the subtask of communicating on behalf of that object. 

Therefore, let's define the {\tt ClientServer} class as a subclass of
%\begin{marginalnote}\it Inheritance\end{marginalnote}
%���\marginnote{Inheritance}
{\tt Thread} (Fig.~15.25). Recall from
Chapter~14 that the typical way to derive
functionality from a {\tt Thread} subclass is to override the {\tt
run()} method.  The {\tt run()} method will be a good place to
implement the client and server protocols.  Because they are
different, we'll define {\tt run()} in both the {\tt Client} and {\tt
Server} subclasses.

For now, the only methods contained in {\tt ClientServer}
(Fig.~\ref{fig-clientserver}) are the two I/O methods we designed.
The only modification we have made to the methods occurs in
the {\tt writeToSocket()} method, where we have added code to make
sure that any strings written to the socket are terminated with an
end-of-line character.

This is an important enhancement, because the read loop in the
{\tt read\-From\-Socket()} method expects to receive an end-of-line
character.  Rather than rely on specific clients to
guarantee that their strings end with \verb|\n|, our design takes care of
\begin{figure}[h]
\jjjprogstart
\begin{jjjlisting}
\begin{lstlisting}
import java.io.*;
import java.net.*;

public class ClientServer extends Thread {

  protected InputStream iStream;  // Instance variables
  protected OutputStream oStream;

  protected String readFromSocket(Socket sock) 
                                     throws IOException {
    iStream = sock.getInputStream();
    String str="";
    char c;
    while (  ( c = (char) iStream.read() ) != '\n')
      str = str + c + "";
    return str;
  }

  protected void writeToSocket(Socket sock, String str) 
                                     throws IOException {
    oStream = sock.getOutputStream();
    if (str.charAt( str.length() - 1 ) != '\n')
      str = str + '\n';
    for (int k = 0; k < str.length() ; k++)
      oStream.write(str.charAt(k));
  } // writeToSocket()
}// ClientServer
\end{lstlisting}
\end{jjjlisting}
\jjjprogstop{The {\tt ClientServer} class serves as the superclass for client/server applications.
\label{fig-clientserver}\label{pg-fig-clientserver}}
\end{figure}
this problem for them.   This ensures that every communication that
takes place between one of our clients and servers will be
line oriented.

\JavaTIP{EFFECTIVE DESIGN}{Defensive Design.}{Code that performs I/O,
whether across a network or otherwise, should be designed to
anticipate and remedy common errors.  This will lead to more robust
programs.}

\subsection{The {\tt EchoServer} Class}

\begin{figure}[ht]
\figaleft{chptr15/p852f1.eps}{Design of the {\tt EchoServer} class.
\label{fig-p852f1}\label{pg-fig-p852f1}}
\end{figure}

%%\marginfig{chptr15/p852f1.eps}{Design of the {\tt EchoServer} class.
%%\label{fig-p852f1}\label{pg-fig-p852f1}}

\noindent Let's now develop a design for the echo server.  This class will be a
subclass of {\tt ClientServer} (Fig.~15.27). As we saw in
discussing the server protocol, one task that echo server will do is
create a {\tt ServerSocket} and establish a port number for its
service.  Then it will wait for a {\tt Socket} connection, and once a
connection is accepted, the echo server will then communicate with the client.
This suggests that
%\begin{marginalnote}\it What data do we need?\end{marginalnote}
\marginnote{What data do we need?}
our server needs at least two instance variables.  It also suggests
that the task of creating a {\tt ServerSocket} would be an appropriate
action for its constructor method.  This leads to the following initial
definition:

\begin{jjjlisting}
\begin{lstlisting}
import java.net.*;
import java.io.*;

public class EchoServer extends ClientServer {

    private ServerSocket port;
    private Socket socket;

    public EchoServer(int portNum, int nBacklog)  {
      try {
        port = new ServerSocket (portNum, nBacklog);
      } catch (IOException e) {
        e.printStackTrace();
      }
    }

    public void run() { }  // Stub method
}// EchoServer
\end{lstlisting}
\end{jjjlisting}

\noindent Note that the constructor method catches the {\tt IOException}.
Note also that we have included a stub version of {\tt run()}, which
we want to define in this class.

Once {\tt EchoServer} has set up a port, it should issue the
{\tt port.accept()} method and wait for a client to connect.  This part
of the server protocol belongs in the {\tt run()} method.   As we have said, most servers
%\begin{marginalnote}\it The server algorithm\end{marginalnote}
\marginnote{The server algorithm}
are designed to run in an infinite loop.  That is, they don't just
handle one request and then quit.  Instead, once started (usually by
the system), they repeatedly handle requests until deliberately
stopped by the system.  This leads to the following run algorithm:

\begin{jjjlisting}
\begin{lstlisting}
public void run() {
  try {
    System.out.println("Echo server at "
                   + InetAddress.getLocalHost() 
                   + " waiting for connections ");
    while(true) {
      socket = port.accept();
      System.out.println("Accepted a connection from " 
                           + socket.getInetAddress());
      provideService(socket);
      socket.close();
      System.out.println("Closed the connection\n");
    }
  } catch (IOException e) {
     e.printStackTrace();
  }
}// run()
\end{lstlisting}
\end{jjjlisting}

\noindent For simplicity, we are printing
the server's status messages on {\tt System.out}. Ordinarily these
should go to a log file.   Note also that the details of the actual
service algorithm are hidden in the {\tt provideService()} method.

As described earlier, the {\tt provideService()} method consists of
writing a greeting to the client and then repeatedly reading a string
from the input stream and echoing it back to the client via the output
stream.  This is easily done using the {\tt writeToSocket()} and {\tt
readFromSocket()} methods we developed.  The implementation of this
method is shown, along with the complete implementation of {\tt
EchoServer}, in Figure~\ref{fig-echoserver}.


The protocol used by {\tt EchoServer.provideService()} starts by
saying ``hello'' and loops until the client says ``goodbye.''  When
the client says ``goodbye,'' the server responds with ``goodbye.''  In
all other cases it responds with ``You said X,'' where X is the string
that was received from the client.   Note the use of the
{\tt toLowerCase()} method to convert client messages to lowercase.   This
simplifies the task of checking for ``goodbye'' by removing the necessity of
checking for different spellings of ``Goodbye.''

\JavaTIP{EFFECTIVE DESIGN}{Defensive Design.}{Converting I/O to lowercase
helps to minimize miscommunication between a client and server and
leads to a more robust protocol.}

This completes the design of the {\tt EchoServer}. We have deliberately
designed it in a way that will make it easy to convert into a generic
server.  Hence, we have the motivation for using {\tt provideService()} as the
name of the method that provides the echo service.  In order to turn
{\tt EchoServer} into a generic {\tt Server} class, we can simply make
{\tt provideService()} an abstract method, leaving its implementation to
%\begin{marginalnote}\it Designing for extensibility\end{marginalnote}
\marginnote{Designing for extensibility}
the {\tt Server} subclasses.  We'll discuss the details of this change
later.

\begin{figure}[p]
\jjjprogstart
\begin{jjjlistingleft}[30pc]{-4pc}
\begin{lstlisting}
import java.net.*;
import java.io.*;
public class EchoServer extends ClientServer {
  private ServerSocket port;
  private Socket socket;
  public EchoServer( int portNum, int nBacklog)  {
    try {
      port = new ServerSocket (portNum, nBacklog);
    } catch (IOException e) {
      e.printStackTrace();
    }
  } // EchoServer()
  public void run() {
    try {
      System.out.println("Echo server at " + 
        InetAddress.getLocalHost() + " waiting for connections ");
      while(true) {
        socket = port.accept();
        System.out.println("Accepted a connection from " + 
                                        socket.getInetAddress());
        provideService(socket);
        socket.close();
        System.out.println("Closed the connection\n");
      } // while
    } catch (IOException e) {
       e.printStackTrace();
    } // try/catch
  }// run()
  protected void provideService (Socket socket) {
    String str="";
    try {
      writeToSocket(socket, "Hello, how may I help you?\n");
      do {
        str = readFromSocket(socket);
        if (str.toLowerCase().equals("goodbye"))
          writeToSocket(socket, "Goodbye\n");
        else
          writeToSocket( socket, "You said '" + str + "'\n");
      }  while (!str.toLowerCase().equals("goodbye"));
    } catch (IOException e) {
      e.printStackTrace();
    } // try/catch
  }// provideServer()
  public static void main(String args[]) {
      EchoServer server = new EchoServer(10001,3);
      server.start();
  }// main()
}// EchoServer
\end{lstlisting}
\end{jjjlistingleft}
\jjjprogstop{{\tt EchoServer} simply echoes the client's message.
\label{fig-echoserver}\label{pg-fig-echoserver}}
\end{figure}

\pagebreak
\subsection{The {\tt EchoClient} Class}

%%%%\begin{figure}[ht]
%%%%\figaleft{chptr15/p854f1.eps}{Design of the {\tt EchoClient} class.
%%%%\label{fig-p854f1}\label{pg-fig-p854f1}}
%%%%\end{figure}

\marginfig{chptr15/p854f1.eps}{Design of the {\tt EchoClient} class.
\label{fig-p854f1}\label{pg-fig-p854f1}}

\noindent The {\tt EchoClient} class is just as easy to design
(Fig.~15.29).  It, too, will be a subclass of {\tt
ClientServer}. It needs an instance variable for the {\tt Socket} that
it will use, and its constructor should be responsible for opening a
socket connection to a particular server and port.  The main part of
its protocol should be placed in the {\tt run()} method.  The initial
definition is as follows:


\begin{jjjlisting}
\begin{lstlisting}
import java.net.*;
import java.io.*;

public class EchoClient extends ClientServer {

     protected Socket socket;

     public EchoClient(String url, int port) {
       try {
         socket = new Socket(url, port);
         System.out.println("CLIENT: connected to " 
                             + url + ":" + port);
       } catch (Exception e) {
         e.printStackTrace();
         System.exit(1);
       }
     }// EchoClient()

    public void run() { }// Stub method
}// EchoClient
\end{lstlisting}
\end{jjjlisting}

\noindent The constructor method takes two
parameters that specify the URL and port number of the echo server.  By
making these parameters, rather than hard coding them within the
method, we give the client the flexibility to connect to servers on a
variety of hosts.

As with other clients, {\tt EchoClient}'s {\tt run()} method will
consist of requesting some kind of service from the server.  Our
initial design called for {\tt EchoClient} to repeatedly input a line
%\begin{marginalnote}\it The client algorithm\end{marginalnote}
\marginnote{The client algorithm}
from the user, send the line to the server, and then display the
server's response.   Thus, for this particular client, the
service requested consists of the following algorithm:

\begin{jjjlisting}
\begin{lstlisting}
Wait for the server to say "hello".
Repeat
    Prompt and get and line of input from the user.
    Send the user's line to the server.
    Read the server's response.
    Display the response to the user.
until the user types "goodbye"
\end{lstlisting}
\end{jjjlisting}

\noindent With an eye toward eventually turning {\tt EchoClient} into
a generic client, let's encapsulate this procedure into a {\tt
requestService()} method that we can simply call from the {\tt run()}
method.  Like for the {\tt provideService()} method, this design is
another example of the encapsulation principle:

\JavaTIP[false]{EFFECTIVE DESIGN}{Encapsulation.}{Encapsulating a
portion of the algorithm into a separate method makes it easy to
change the algorithm by overriding the method.}

\noindent The {\tt requestService()} method will take a {\tt Socket}
parameter and perform all the I/O for this particular client:

\begin{jjjlistingleft}[35pc]{-9pc}
\begin{lstlisting}
protected void requestService(Socket socket) throws IOException {
  String servStr = readFromSocket(socket);                // Check for "Hello"
  System.out.println("SERVER: " + servStr);    // Report the server's response
  System.out.println("CLIENT: type a line or 'goodbye' to quit");    // Prompt
  if (servStr.substring(0,5).equals("Hello")) {
    String userStr = "";
    do {
      userStr = readFromKeyboard();                              // Get input
      writeToSocket(socket, userStr + "\n");             // Send it to server
      servStr = readFromSocket(socket);         // Read the server's response
      System.out.println("SERVER: " + servStr);   // Report server's response
    } while (!userStr.toLowerCase().equals("goodbye"));    // Until 'goodbye'
  }
} // requestService()
\end{lstlisting}
\end{jjjlistingleft}

\noindent Although this method involves several lines, they should
all be familiar to you.  Each time the client reads a message from the
socket, it prints it on {\tt System.out}. The first message it reads
should start with the substring ``Hello''. This is part of its
protocol with the client.  Note how the {\tt substring()} method is
used to test for this.  After the initial greeting from the server, the
client begins reading user input from the keyboard, writing it to the
socket, then reading the server's response, and displaying it on
{\tt System.out}.

Note that the task of reading user input from the keyboard has been
made into a separate method, which is one we've used before:

\begin{jjjlisting}
\begin{lstlisting}
protected String readFromKeyboard() throws IOException {
  BufferedReader input = new BufferedReader(
                   new InputStreamReader(System.in));
  System.out.print("INPUT: ");
  String line = input.readLine();
  return line;
}// readFromKeyboard()
\end{lstlisting}
\end{jjjlisting}

The only method remaining to be defined is the {\tt run()}, which is
shown with the complete definition of {\tt EchoClient} in
Figure~\ref{fig-echoclient}.  The {\tt run()} method can simply call
the {\tt requestService()} method.  When control returns from the
{\tt requestService()} method, {\tt run()} closes the socket connection.
Because {\tt requestService()} might throw an {\tt IOException}, the entire
method must be embedded within a {\tt try/catch} block that catches
that exception.

\begin{figure}[p]
\jjjprogstart
\begin{jjjlisting}[37pc]
\begin{lstlisting}
import java.net.*;
import java.io.*;

public class EchoClient extends ClientServer {

  protected Socket socket;

  public EchoClient(String url, int port) {
     try {
        socket = new Socket(url, port);
        System.out.println("CLIENT: connected to " + url + ":" + port);
      } catch (Exception e) {
        e.printStackTrace();
        System.exit(1);
      }
   }// EchoClient()

  public void run() {
    try {
        requestService(socket);
        socket.close();
        System.out.println("CLIENT: connection closed");
    } catch (IOException e) {
        System.out.println(e.getMessage());
        e.printStackTrace();
    }
  }// run()

  protected void requestService(Socket socket) throws IOException {
    String servStr = readFromSocket(socket);                 // Check for "Hello"
    System.out.println("SERVER: " + servStr);     // Report the server's response
    System.out.println("CLIENT: type a line or 'goodbye' to quit");// Prompt user
    if (servStr.substring(0,5).equals("Hello")) {
       String userStr = "";
       do {
         userStr = readFromKeyboard();                 // Get input from user
         writeToSocket(socket, userStr + "\n");   // Send it to server
         servStr = readFromSocket(socket);          // Read server's response
         System.out.println("SERVER: " + servStr);  // Report server's response
       } while (!userStr.toLowerCase().equals("goodbye"));// Until 'goodbye'
    }
  }// requestService()

  protected String readFromKeyboard( ) throws IOException {
    BufferedReader input = new BufferedReader(new InputStreamReader(System.in));
    System.out.print("INPUT: ");
    String line = input.readLine();
    return line;
  }// readFromKeyboard()

  public static void main(String args[]) {
    EchoClient client = new EchoClient("java.trincoll.edu",10001);
    client.start();
  }// main()
}// EchoClient
\end{lstlisting}
\end{jjjlisting}
\jjjprogstop{The {\tt EchoClient} class prompts the user for a string
and sends it to the {\tt EchoServer}, which simply
echoes it back.
\label{fig-echoclient}\label{pg-fig-echoclient}}
\end{figure}


\subsection*{Testing the Echo Service}
\noindent Both {\tt EchoServer} and {\tt EchoClient} contain {\tt main()}
methods (Figs.~\ref{fig-echoserver} and \ref{fig-echoclient}). In
order to test the programs, you would run the server on one computer
and the client on another computer.  (Actually they can both be run on the
same computer, although they wouldn't know this and would still access
each other through a socket connection.)



The {\tt EchoServer} must be started first, so that its service will
be available when the client starts running.  It also must pick a port
number.   In this case it picks 10001. The only constraint on its
choice is that it cannot use one of the privileged port numbers---those 
below 1024---and it cannot use a port that's already in use.

\begin{jjjlisting}
\begin{lstlisting}
public static void main(String args[]) {
    EchoServer server = new EchoServer(10001,3);
    server.start();
}// main()
\end{lstlisting}
\end{jjjlisting}

\noindent When an {\tt EchoClient} is created, it must be given the server's
URL ({\tt java.\-trincoll.edu}) and the port that the service is using:

\begin{jjjlisting}
\begin{lstlisting}
public static void main(String args[]) {
    EchoClient client = 
            new EchoClient("java.trincoll.edu",10001);
    client.start();
}// main()
\end{lstlisting}
\end{jjjlisting}

\noindent As they are presently coded, you will have to modify both
{\tt EchoServer} and {\tt EchoClient} to provide the correct URL and port
for your environment.  In testing this program, you might wish to
experiment by trying to introduce various errors into the code and
observing the results.  When you run the service, you should observe
something like the following output on the client side:

\begin{jjjlisting}
\begin{lstlisting}
CLIENT: connected to java.trincoll.edu:10001
SERVER: Hello, how may I help you?
CLIENT: type a line or 'goodbye' to quit
INPUT: this is a test
SERVER: You said 'this is a test'
INPUT: goodbye
SERVER: Goodbye
CLIENT: connection closed
\end{lstlisting}
\end{jjjlisting}

\section{Playing One Row Nim Over the Network}

\noindent In the previous section we developed and tested a generic echo service.  
It is based on a common root class, {\tt ClientServer}, which is a subclass of
{\tt Thread}.  Both {\tt EchoServer} and {\tt EchoClient} extend the
root class, and each implements its own version of {\tt run()}. In
this section, we will generalize this design so that it can support a wide
range of services.  To illustrate the effectiveness of the design, we
will use it as the basis for a program that plays One Row Nim over the
Internet.

In order to generalize our design, we begin by identifying
those elements that are common to all
%\begin{marginalnote}\it Designing for extensibility\end{marginalnote}
\marginnote{Designing for extensibility}
servers and clients and what is particular to the echo service and
client.  Clearly, the general server and client protocols that are defined
here in their respective {\tt run()} methods, are something that all
servers and clients have in common.  What differs from one application
to another is the particular service provided and requested, as
detailed in their respective {\tt provideService()} and {\tt
requestService()} methods. In this example, the service that is
provided will be One Row Nim.  The clients that use this service will
be (human) players of the game.

%\begin{figure}
\begin{figure}[bt]
\figa{chptr15/p858f1.eps}{Class hierarchy for a generic client/server application.
\label{fig-p858f1}\label{pg-fig-p858f1}}
\end{figure}
%\end{figure}

Therefore, the way to generalize this application is to define the
%\begin{marginalnote}\it Abstract service methods\end{marginalnote}
\marginnote{Abstract service methods}
{\tt run()} method in the generic {\tt Server} and {\tt Client}
classes. The overall design of the One Row Nim service will now consist of five classes
organized into the hierarchy shown in Figure~\ref{fig-p858f1}.  At
the root of the hierarchy is the {\tt ClientServer} class, which
contains nothing but I/O methods used by both clients and servers.
The abstract {\tt Server} and {\tt Client} classes contain
implementations of the {\tt Thread.run()} method, which defines the
basic protocols for servers and clients, respectively.   The details of the
particular service are encoded in the {\tt provideService()} and
{\tt requestService()} methods. Because the {\tt run()} methods defined 
in {\tt Client} and {\tt Server} call {\tt
provideService()} and {\tt requestService()}, respectively,  these
methods must be declared as {\tt abstract} methods in the {\tt Server}
and {\tt Client} classes.  Any class that contains an
\mbox{\tt abstract} method must itself be declared {\tt abstract}. 

Note that we have left the {\tt readFromSocket()} and {\tt
writeToSocket()} methods in the {\tt ClientServer} class. These
methods are written in a general way and can be used, without change,
by a wide range of clients and servers.  If necessary, they can also
be overridden by a client or server. In fact, as we will see, the
{\tt NimServer} class does override the {\tt writeToSocket()} method.
Similarly, note that the {\tt readFromKeyboard()} method is defined in
the {\tt Client} superclass.  This is a general method that
can be used by a large variety of clients, so it is best if they don't
have to redefine it themselves.

These design decisions lead to the definitions of {\tt Server} and
{\tt Client} shown in Figures~\ref{fig-serverclass} and
~\ref{fig-clientclass}, respectively. Note that {\tt provideService()}
and {\tt requestService()} are left unimplemented.  Subclasses of {\tt
Server}, such as {\tt NimServer}, and subclasses of {\tt Client}, such
as {\tt NimClient}, can implement {\tt provideService()} and {\tt
requestService()} in a way that is appropriate for the particular
service they are providing.

\JavaTIP{EFFECTIVE DESIGN}{Polymorphism.}{Defining a method as
{\tt abstract} within a superclass, and implementing it in various ways in
subclasses, is an example of polymorphism.  Polymorphism is a powerful
object-oriented design technique.}

\begin{figure}[htb]
\jjjprogstart
\begin{jjjlisting}[27pc]
\begin{lstlisting}
import java.net.*;
import java.io.*;

public abstract class Server extends ClientServer {
  protected ServerSocket port;
  protected Socket socket;

  public Server(int portNum, int nBacklog)  {
    try {
      port = new ServerSocket (portNum, nBacklog);
    } catch (IOException e) {
      e.printStackTrace();
    } // try/catch
  }// Server()

  public void run() {
    try {
      System.out.println("Server at " + 
                       InetAddress.getLocalHost() +
                       " waiting for connections ");
      while (true) {
        socket = port.accept();
        System.out.println("Accepted a connection from " 
                                + socket.getInetAddress());
        provideService(socket);
        socket.close();
        System.out.println("Closed the connection\n");
      }// while
    } catch (IOException e) {
      e.printStackTrace();
    } // try/catch 
  }// run()

 // Implemented in subclass
 protected abstract void provideService(Socket socket);  
}// Server
\end{lstlisting}
\end{jjjlisting}
\jjjprogstop{The abstract {\tt Server} class.
\label{fig-serverclass}\label{pg-fig-serverclass}}
\end{figure}


\begin{figure}[htb]
\jjjprogstart
\begin{jjjlistingleft}[27pc]{-1pc}
\begin{lstlisting}
import java.net.*;
import java.io.*;

public abstract class Client extends ClientServer {
  protected Socket socket;

  public Client(String url, int port) {
    try {
      socket = new Socket(url,port);
      System.out.println("CLIENT: connected to " + url + 
                                              ":" + port);
    } catch (Exception e) {
      e.printStackTrace();
      System.exit(1);
    } // try/catch block
  }// Client()

  public void run() {
    try {
      requestService(socket);
      socket.close();
      System.out.println("CLIENT: connection closed");
    } catch (IOException e) {
      System.out.println(e.getMessage());
      e.printStackTrace();
    } // try/catch block
  }// run()

  // Implemented in subclass
  protected abstract void requestService(Socket socket) 
                                      throws IOException;

  protected String readFromKeyboard() throws IOException {
    BufferedReader input = new BufferedReader 
                     (new InputStreamReader(System.in));
    System.out.print("INPUT: ");
    String line = input.readLine();
    return line;
  }// readFromKeyboard()
}// Client
\end{lstlisting}
\end{jjjlistingleft}
\jjjprogstop{The abstract {\tt Client} class.
\label{fig-clientclass}\label{pg-fig-clientclass}}
\end{figure}

\subsection{The {\tt NimServer} Class}

Given the abstract definition of the {\tt Server} class, defining a
new service is simply a matter of extending {\tt Server} and
%\begin{marginalnote}\it Extensibility\end{marginalnote}
\marginnote{Extensibility} implementing the {\tt provideService()}
method in the new subclass. We will name the subclass {\tt NimServer}.

\clearpage
\begin{figure}[h!]
\jjjprogstart
\begin{jjjlisting}[29pc]
\begin{lstlisting}[basicstyle=\scriptsize]
import java.net.*;
import java.io.*;

public class NimServer extends Server {
  public NimServer(int port, int backlog) {
    super(port, backlog);
  }
  protected void provideService (Socket socket) {
    OneRowNim nim = new OneRowNim();
    try {
      writeToSocket(socket, "Hi Nim player. You're Player 1 and I'm Player 2. " +
           nim.reportGameState()  + " " + nim.getGamePrompt() + "\n");
      play(nim, socket);
    } catch (IOException e) {
      e.printStackTrace();
    } // try/catch
  } // provideService()

  private void play(OneRowNim nim, Socket socket) throws IOException {
    NimPlayer computerPlayer = new NimPlayer(nim);
    nim.addComputerPlayer(computerPlayer);
    String str="", response="";
    int userTakes = 0, computerTakes = 0;
    do {     
      str = readFromSocket(socket);
      boolean legalMove = false;
      do {
        userTakes = Integer.parseInt(str);
        if (nim.takeSticks(userTakes)) {
          legalMove = true;
          nim.changePlayer();
          response = nim.reportGameState() + " ";
          if (!nim.gameOver()) {
            computerTakes = Integer.parseInt(computerPlayer.makeAMove(""));    
            response = response + " My turn. I take " + computerTakes + " sticks. ";
            nim.takeSticks(computerTakes);
            nim.changePlayer();
            response = response + nim.reportGameState() + " ";
            if (!nim.gameOver())
               response = response + nim.getGamePrompt();
          } // if not game over
          writeToSocket(socket, response);
        } else {
           writeToSocket(socket, "That's an illegal move. Try again.\n");
           str = readFromSocket(socket);
        } // if user takes
      } while (!legalMove);
    }  while (!nim.gameOver());
  } // play

  // Overriding writeToSocket to remove \n from str
  protected void writeToSocket(Socket soc, String str) throws IOException {
    StringBuffer sb = new StringBuffer();
    for (int k = 0; k < str.length(); k++)
      if (str.charAt(k) != '\n')
        sb.append(str.charAt(k));
    super.writeToSocket(soc, sb.toString() + "\n");
  }
  public static void main(String args[]) {
    NimServer server = new NimServer(10001,5);
    server.start();
  } // main()
} // NimServer
\end{lstlisting}
\end{jjjlisting}
\jjjprogstop{The  {\tt NimServer} class.
\label{fig-nimserverclass}\label{pg-fig-nimserverclass}}
\end{figure}

Figure~\ref{fig-nimserverclass} provides a definition of the {\tt
NimServer} subclass. Note how its implementation of {\tt
provideService()} uses an instance of the {\tt OneRowNim} class from
Chapter~8. Also, the {\tt play()} method, which encapsulates the
game-playing algorithm, uses an instance of {\tt NimPlayer}, also from
Chapter~8. You might recall that {\tt OneRowNim} is a {\tt
TwoPlayerGame} and {\tt NimPlayer} defines methods that allow a
computer to play an optimal game of One Row Nim.  In this example, the
server acts as one of the players and use a {\tt NimPlayer} to manage
its playing strategy.  Thus, clients that connect to {\tt NimServer}
will be faced by a computer that plays an optimal game.

If you compare the details of the {\tt NimServer}'s {\tt play()}
method with the {\tt play()} method from the implementation of {\tt
OneRowNim}, you will see that they are very similar.  In this
implementation, we use public methods of the {\tt OneRowNim} object to
manage the playing of the game. Thus, {\tt addComputerPlayer()} adds
an instance of {\tt NimPlayer} to the game. The {\tt takeSticks()},
{\tt changePlayer()}, and {\tt gameOver()} methods are used to manage
the moves made by both itself and the client.  And the {\tt
getGamePrompt()} and {\tt reportGameState()} methods are used to
manage the interaction and communication with the client.  Note also
that whenever it is the server's turn to move, it uses the {\tt
NimPlayer}'s {\tt makeAMove()} method to determine the optimal move to
make.

Although the programming logic employed in the {\tt play()}
method looks somewhat complex, it is very similar to the logic
employed in the Chapter~8 version of the game. The main difference
here is that the server uses the {\tt writeToSocket()} and {\tt
readFromSocket()} methods to manage the communication with the client.
In this regard, this instance of {\tt provideService()} is no
different than the {\tt provideService()} method we used in the {\tt
EchoServer} class.

Finally, note that {\tt NimServer} provides an implementation of the
\marginnote{Overriding a method}
{\tt writeToSocket()} method. This method is implemented in the {\tt
ClientServer()} class and is inherited by {\tt NimServer}.  However,
the default implementation assumes that the client will use the a
carriage return ({\tt \verb|\n|}) to determine the end of a particular
message in the socket.  Because {\tt OneRowNim}'s methods, {\tt
getGamePrompt()} and {\tt reportGameState()}, contain embedded
carriage returns, it is necessary to filter these. 
The new version of {\tt writeToSocket()} performs this filtering and calls the
default method ({\tt super.writeToSocket()}, after it has finished its
filtering task.

\subsection{The {\tt NimClient} Class}

The {\tt NimClient} class is even easier to implement. As its task is
simply to manage the communication between the human user and the {\tt
NimServer}, it is very similar to the {\tt requestService()} method we
used in {\tt EchoClient}.  The relationship between the abstract {\tt
Client} class (Fig.~\ref{fig-clientclass}) and its extension in {\tt
NimClient} (Fig.~\ref{fig-nimclientclass}) is very similar to the
relationship between {\tt Server} and {\tt NimServer}.  The~{\tt
request\-Service()} method is called by {\tt Client.run()}.  It is
implemented in {\tt NimClient}. In this way, the {\tt Client} class
can serve as a superclass for any number of clients.
%\begin{marginalnote}\it Creating new clients\end{marginalnote}
\marginnote{Creating new clients}
New clients for new services can be derived from {\tt Client} by
simply implementing their own {\tt requestService()} method.  


\begin{figure}[tb]
\jjjprogstart
\begin{jjjlisting}[34pc]
\begin{lstlisting}
import java.net.*;
import java.io.*;

public class NimClient extends Client {
  private KeyboardReader kb = new KeyboardReader();
  public NimClient( String url, int port) {
    super(url,port);
  }

  protected void requestService(Socket socket) throws IOException {
    String servStr = readFromSocket(socket);    // Get server's response
    kb.prompt("NIM SERVER: " + servStr +"\n");  // Report server's response
    if ( servStr.substring(0,6).equals("Hi Nim") ) {
      String userStr = "";
      do {
        userStr = kb.getKeyboardInput();        // Get user's move
        writeToSocket(socket, userStr + "\n");  // Send it to server
        servStr = readFromSocket(socket);       // Read server's response
        kb.prompt("NIM SERVER: " + servStr + "\n");  // Report response
      } while(servStr.indexOf("Game over!") == -1); // Until game over
    }
  }// requestService()

  public static void main(String args[]) {
    NimClient client = new NimClient("localhost", 10001);
    client.start();
  } // main()
} // NimClient
\end{lstlisting}
\end{jjjlisting}
\jjjprogstop{The derived {\tt NimClient} class.
\label{fig-nimclientclass}\label{pg-fig-nimclientclass}}
\end{figure}

\JavaTIP[false]{EFFECTIVE DESIGN}{Inheritance.}{By placing as much functionality
as possible into a generic client/server superclass, you can simplify
the creation of new services.  This is an effective use of Java's
inheritance mechanism.}

\subsection{Testing the Nim Service}
\noindent Testing the One Row Nim service will be no different
than testing the Echo service.  To test the service, you want to run
both {\tt NimServer} and {\tt NimClient} at the same time and
preferably on different computers.  As they are presently coded, you
will have to modify the {\tt main()} methods of both {\tt NimServer}
and {\tt NimClient} to provide the correct URL and port for your
environment.

\secEXRHone{Self-Study Exercise}

\begin{SSTUDY}

\item  The design of the client/server hierarchy makes it 
easy to create a new service by extending both the {\tt Server} and
{\tt Client} classes.  Describe how you would implement a scramble
service with this model.  A scramble service can be used by people
trying to solve the daily scramble puzzles found in many newspapers.
Given a string of letters, the scramble service will return a string
containing all possible letter combinations.  For example,
given ``cat,'' the scramble service will return ``act atc cat cta tac
tca.''

\item  Describe what happens when each of the following errors is introduced
into the {\tt EchoClient} or {\tt EchoServer} programs:

%%\begin{EXRBL}
\begin{itemize}
\item  Specify the wrong host name when running {\tt EchoClient}.
\item  Specify the wrong port number when running {\tt EchoClient}.
\item  Remove the reference to \verb|\n| in the {\tt writeToSocket()}
call in {\tt request\-Service()}.
\end{itemize}
%%%\end{EXRBL}
\end{SSTUDY}
%%%%\endcasestudy



\section{Java Network Security Restrictions}
\label{sec-security}\label{pg-sec-security}
\noindent One of the most attractive features of Java is that extensive effort
has been made to make it a {\it secure} language.  This is especially
important for a language that makes it so easy to implement networking
applications.   After all, nobody wants to download a Java applet that
proceeds to erase the hard disk.  Such an applet might be written
by a cyber terrorist, deliberately aiming to cause severe damage,
or it might be written by a cyber doofus, who inadvertently writes
code that does severe damage.

What are some of Java's techniques for guarding against either
deliberate or inadvertent insecure code?  One level of security is
%\begin{marginalnote}\it Code verification\end{marginalnote}
\marginnote{Code verification}
Java's {\it bytecode verification} process, which the Java Virtual
Machine performs on any ``untrusted'' code that it receives.  Java
checks every class that it loads into memory to make sure it doesn't
contain illegal or insecure code.  Another line of defense is the
so-called {\bf sandbox security model}\index{sandbox security model}, which refers to the practice
of restricting the kinds of things that certain programs can do.  For
%\begin{marginalnote}\it Limited privileges\end{marginalnote}
\marginnote{Limited privileges}
example, the ``sandbox'' environment for Java applets
restricts them from having any access whatsoever to the local file
system.

Another restriction imposed on applets is to limit their networking
capabilities.  For example, a Java applet cannot create a network
connection to any computer except the one from which its code was
downloaded.  Also, a Java applet cannot listen for, or accept,
%\begin{marginalnote}\it Limited network access\end{marginalnote}
\marginnote{Limited network access}
connections on privileged ports---those numbered 1024 or
lower.  Together, these two restrictions severely limit the kinds of
client/server programs that can be built as applets.

Java sets aside certain locations as repositories for {\bf trusted
code}\index{trusted code}. For example, the Java class libraries would be placed in such a
location, as would the directories where your Java programs are
stored.  Any class loaded from some other directory is considered {\it
untrusted}. By this definition, applets downloaded over the Internet
would be considered untrusted code.

In addition to the restrictions for applets, which
%\begin{marginalnote}\it Trusted code\end{marginalnote}
\marginnote{Trusted code}
apply to all untrusted code, Java defines a number of
other limitations:

\begin{BL}
\item  Untrusted code cannot make use of certain system
facilities, such as {\tt System.exit()} and classes
in the {\tt java.security} package.

\item  Untrusted code cannot make use of certain AWT
methods, such as methods that access the system clipboard.   Another
AWT restriction is that any window created by untrusted code must
display a message informing the user that it is untrusted.  You might
have seen such messages on windows opened from applets.

\item  Untrusted code is limited in the kinds of threads
it can create.
\end{BL}

Security enhancements introduced in JDK 1.2 are based on the
concepts of ``permission'' and ``policy.'' Code is assigned
``permissions'' based on the security policy currently in effect.  Each
permission specifies the type of access allowed for a particular
resource (such as ``read'' and ``write'' access to a specified file or
directory, or ``connect'' access to a given host and port). The
policy that controls permissions can be initialized from an external
configurable policy file.  Unless a permission is explicitly granted to
code, it cannot access the resource that is guarded by that
permission.   These new enhancements offer a more fine-grained and
extensible approach to security for both applets and applications.

As this brief overview illustrates, the Java Virtual Machine is
designed with security as one of its primary issues.  This
doesn't guarantee 100 percent
security, but it is a big improvement over some of
the languages and systems that preceded Java.   Moreover, security is
an ongoing concern of the Java development process.   Flaws in the existing
security system are fixed very quickly.   Advanced methods are
constantly being developed and incorporated into the system.  One such
enhancement is the use of encryption to guarantee the integrity of
classes transferred over the network.

\section{Java Servlets and Java Server Pages}

In this chapter we have been discussing the client/server model of
computing.  Thus far we have learned how to implement client/server
applications using socket connections between programs running on
different computers. Because it requires the programmer to directly
create and manage the socket protocol, this socket-level approach is a
low-level approach.  It is important to know about sockets and how
they are used, but most client/server applications are programmed at a
much higher level by using Java library classes.

Our focus in this section will be to give you a brief sense of how
Java programs can be integrated into the Web pages. We will discuss
two approaches: {\em Java Server Pages (JSP)} and {\em Java servlets}.  As
Web-based approaches, both of these require the use of HTML (HyperText
Markup Language) code, the language that is used for coding Web
pages. This means that in order to write your own servlets and JSPs
you would really have to learn a little about HTML code.  Fortunately,
learning HTML code is not difficult and although it doesn't fit within
the scope of this book, you can easily find books or Web sites that
cover basic HTML coding.  Moreover, in order for servlets and JSPs to
work, they must be associated with a Web server that is specially
configured to understand Java code.  At the end of this section we
will provide links to a Web site where you can learn more about HTML
and about how to set up your own JSPs and servlets.

\subsection{Java Server Pages}

A Java Server Page (JSP) is a Web page that contains small snippets of
Java code.  The simple example discussed here was downloaded from on
online tutorial at 

\begin{jjjlisting}
\begin{lstlisting}[commentstyle=\color{black}]
 http://developer.apple.com/internet/java/tomcat1.html
\end{lstlisting}
\end{jjjlisting}

\noindent The Java code on a JSP embedded within {\tt <\% ... \%>}
brackets and interspersed among a page's HTML tags. The Java code can
extend over one or more lines. Figure~\ref{fig-jspcode} shows the
complete sourcecode for a simple JSP.

\begin{figure}[h!]
\jjjprogstart
\begin{jjjlisting}
\begin{lstlisting}
<html><head><title>Very Simple JSP Example</title></head>
<body bgcolor="white">

<h1>Very Basic JSP</h1>

Current time: <%= new java.util.Date() %>
<br><br>

Reload this page to watch the greeting change.
<br><br><b>

<!-- including lines of Java code in an HTML page -->
<%
   int um = (int)( Math.random() * 5 );
   switch ( um )
   {
      case 0: out.println("Welcome"); break;
      case 1: out.println("Bienvenidos"); break;
      case 2: out.println("Bienvenue"); break;
      case 3: out.println("Bienvenuti"); break;
      case 4: out.println("Willkommen"); break;
      default: out.println("Huh? " + um);
   }
   out.println("<br>");
%>
</b>
</body>
</html>
\end{lstlisting}
\end{jjjlisting}
\jjjprogstop{A simple JavaServer Page (JSP).
\label{fig-jspcode}\label{pg-fig-jspcode}}
\end{figure}


In this example we see two uses of Java
code.  In the first case, a JSP expression tag is used to display the
current date on the Web page:

\begin{jjjlisting}
\begin{lstlisting}
  Current time: <%= new java.util.Date() %>
\end{lstlisting}
\end{jjjlisting}

\noindent A JSP expression element begins with {\tt <\%=} and ends
with {\tt \%>}.  The expression contained within the tag is
evaluated, converted into a Java {\tt String} and inserted into the
Web page.  In this case the {\tt Date} object is evaluated and its
string value is displayed on the Web page (Fig.~\ref{fig-jspscreen}).

\begin{figure}[tb]
\figaright{chptr15/jspscreen.eps}{A screen shot of a JSP.
\label{fig-jspscreen}\label{pg-fig-jspscreen}}
\end{figure}

In the second case, a {\em scriptlet} of Java code uses the {\tt
Math.random()} method to display a random greeting on the Web page.  A
{\bf scriptlet} extends over several lines and is contained within the
{\tt <\%...\%>} tag (Fig.~\ref{fig-jspcode}).  Note the use of the
output statement, {\tt out.println()}. The {\tt out} object is a
built-in output stream.  Anything written to {\tt out} will be
transmitted as part of the HTML code that is sent to the Web page. In
this case, one of the greetings is displayed each time the page is
reloaded.

Obviously, this simple example only scratches the surface of what you
can do with JSP.  If you want to learn more about JSP, there are many
helpful online tutorials available, such as {\tt http://www.jsptut.com/}.
However, remember that in order to experiment with JSP, it will be
necessary to have access to a JSP-aware Web server either on your
own computer or on one provided by your service provider.

\subsection{Java Servlets}

A Java {\em servlet} is another high-level approach to developing
client/server applications.  A {\bf servlet} is a Java program that
runs on a Web server and processes Web pages using the HyperText
Transfer Protocol (HTTP).  In a Web application, the browser serves as
the client.

Many URLs that we access on the web are pure HTML files that are
simply transmitted back to the browser by the Web server.  For
example, the URL for a simple HTML document on the author's Web site
is:

\begin{jjjlisting}
\begin{lstlisting}[commentstyle=\color{black}]
http://www.cs.trincoll.edu/~ram/jjj/hello.html
\end{lstlisting}
\end{jjjlisting}

\noindent If you type that URL into a Web browser, the Web server at
{\tt www.cs.trincoll.edu} would transmit the following text file
to your browser, which would then render and display the document.

\begin{jjjlisting}
\begin{lstlisting}
<HTML>
<HEAD>  
   <TITLE>Very Simple HTML Document</TITLE>
</HEAD>

<BODY>
    <CENTER><H1>Hello</H1></CENTER>
</BODY>
</HTML>
\end{lstlisting}
\end{jjjlisting}

\noindent If we want the server to do some processing and submit the
results of that processing to the browser, we could use a Java
servlet.  A servlet can perform some processing task and return the
results of that task to the browser in the form of an HTML document.

The difference between a Java servlet and a Java applet is that
\marginnote{Servlets vs. Applets} an applet performs all of its processing
on the client side of the client/server connection. A servlet performs
its processing on the server side.  When you load a Java applet into a
browser, the Web server downloads the applet's bytecode into the
browser.  The browser then runs the byte code, assuming, of course, it
is equipped with a plugin for the Java Virtual Machine (JVM).  When
you access a Java servlet from a browser, the Web server performs some
computation and transmits just the results to the browser.

There are several advantages of servlets over applets. First, servlets
cut down significantly on the amount of data that has to be
transmitted to the browser. Second, because the
servlet returns an HTML-encoded page, there are many fewer
platform-related problems.  All browsers can interpret HTML code, but
not all browsers have the right plugins for interpreting Java applets.
Third, servlets are not subject to the same security and privacy
restrictions as Java applets, which, as we saw earlier in the chapter,
must be run as untrusted code.  Finally, Java servlets can directly
access large databases and other resources that are stored on the
server.  Access to such resources via an applet would be very
difficult and inefficient.

So, servlets have many advantages over applets. Because of these
advantages they have quickly become an industry standard for
developing client/server applications on the Web.

\subsection{A Simple Servlet Example}

To illustrate the difference between a Java servlet and a simple HTML
page, Figure~\ref{fig-hiservlet} shows a servlet that creates a
Web page that says ``Hello.''  As you can see, a servlet is a
Java program. In addition to libraries that you are already familiar
with, such as {\tt java.io}, it also imports names from two new
libraries: {\tt javax.servlet} and {\tt javax.servlet.http}.  The
program defines a single class, the {\tt HelloServlet} class, which is
a subclass of {\tt HttpServlet}, the standard superclass for all Java
servlets.  

\begin{figure}[tb]
\jjjprogstart
\begin{jjjlisting}
\begin{lstlisting}
import java.io.*;
import java.text.*;
import java.util.*;
import javax.servlet.*;
import javax.servlet.http.*;

public class HelloServlet extends HttpServlet {

    public void doGet(HttpServletRequest request,
                      HttpServletResponse response)
        throws IOException, ServletException
    {
        response.setContentType("text/html");
        PrintWriter out = response.getWriter();

        out.println("<HTML>");
        out.println("<HEAD>");
        out.println("<TITLE>Simple Servlet</TITLE>");
        out.println("</HEAD>");
        out.println("<BODY>");
        out.println("<H1> Hi, from a Java Servlet.</H1>");
        out.println("</BODY>");
        out.println("</HTML>");
    }
}

\end{lstlisting}
\end{jjjlisting}
\jjjprogstop{A simple Java servlet.
\label{fig-hiservlet}\label{pg-fig-hiservlet}}
\end{figure}

The servlet defines the {\tt doGet()} method. This is a method that is
defined in the {\tt HttpServlet} superclass. Our {\tt HelloServlet} is
overriding that method.  In general, Web browsers make two types of
requests when they request a Web page, a {\bf get} or a {\bf post}.  We
won't go into the differences between these requests. The result in
either case is that the Web server will respond to the request by
transmitting some text data to the browser.  When
a browser makes a get request, the server will automatically call the
servlet's {\tt doGet()} method. That's why we have to override it.
The {\tt HttpServlet} class also has a default {\tt doPost()} method,
which is called automatically to handle {\em post} requests.

Note the two parameters in the {\tt doGet()} method: the {\tt
HttpServlet\-Request} and the {\tt HttpServletResponse}.  The {\tt
doPost()} method has the same two parameters. These are the objects
that are used to hold the data that are communicated between the
client and the server.  When the client (browser) makes a get request,
the {\tt HttpServletRequest} objects hold the data contained in
the request. These data might include data that a user has typed into
an HTML form.  We will see an example of how to extract these data
in the next section.  

The {\tt HttpServletResponse} object is where the servlet will write
its response. As you can see from examining the code, the {\tt
HttpServlet\-Response} object has an associated output stream, a {\tt
PrintWriter}, and it is a simple matter to write text to that output
stream.  Note that the text we write is HTML code that is practically
identical to the code contained in the previous HTML example.

\subsection{The Nim Servlet}

The simple servlet in the preceding section illustrates how the
servlet communicates with the client---by writing HTML code to the
{\tt HttpServlet\-Response} object.  Let's now look at an example that
uses two-way communication between the client and server. To keep the
example simple, we will revisit once again on our One Row Nim game.  In
this application the servlet will manage the One Row Nim game and will
play against a human player, who will access the game through a Web
browser.  

\begin{figure}[h!]
\figa{chptr15/nimscreen.eps}{The interface for the Nim servlet.
\label{fig-nimscreen}\label{pg-fig-nimscreen}}
\end{figure}

The browser interface for this version of the game is shown in
Figure~\ref{fig-nimscreen}. As you can see, it is a simple Web
page. The sticks in this instance are replaced by pennies. In addition
to reporting the total number of pennies left, the page displays
images of pennies. This Web page itself is organized as a simple HTML
form, which contains one text field for the user's input.  Each time
the user hits the RETURN key in the text field, the user's input is
transmitted to the servlet where it is processed.  The servlet then
transmits a new page to the user's browser, which shows the updated
state of the game. 

Let's now look at the servlet program itself, whose code is shown in
Figures~\ref{fig-nimservlet} and~\ref{fig-nimservlet2}.  This servlet
program is quite a bit longer than the simple hello server, but it is
not really any more complex or difficult.  The {\tt NimServlet}
extends the {\tt HttpServlet} superclass and overrides the {\tt
doGet()} method.  Note that it also overrides the {\tt doPost()}
method, by simply having that method call the {\tt doPost()} method. So
this servlet will work for both {\em get} and {\em post} requests.

\begin{figure}[tb]
\jjjprogstart
\begin{jjjlisting}
\begin{lstlisting}
import java.io.*;
import java.util.*;
import javax.servlet.*;
import javax.servlet.http.*;

public class NimServlet extends HttpServlet {
    
    private OneRowNim nim = null;
    private NimPlayer nimPlayer = null;

    public void doPost(HttpServletRequest request, 
                       HttpServletResponse response)
      throws IOException, ServletException
    {
        doGet(request, response);
    }

    //  The doGet() method goes here.

} // NimServlet
\end{lstlisting}
\end{jjjlisting}
\jjjprogstop{Java code for the {\tt NimServlet}, minus the {\tt doGet()} method.
\label{fig-nimservlet}\label{pg-fig-nimservlet}}
\end{figure}

\begin{figure}[p]
\jjjprogstart
\begin{jjjlistingleft}[37pc]{-11pc}
\begin{lstlisting}[basicstyle=\scriptsize]
public void doGet(HttpServletRequest request, HttpServletResponse response)
    throws IOException, ServletException
{
  response.setContentType("text/html");
  PrintWriter out = response.getWriter();
  HttpSession session = request.getSession(true);

  out.println("<html>");
  out.println("<body>");
  out.println("<head>");
  out.println("<title>Simple Nim Game</title>");
  out.println("</head>");
  out.println("<body>");

  out.println("<center><h1>One Row Nim</h1></center>");
  out.println("<center><h3>The Rules</h3>");
  out.println("A random number of pennies is thrown on the table.<BR>");
  out.println("Two players alternate making moves,<BR>");
  out.println("picking up between 1 and 3 pennies on each move.<BR>");
  out.println("The player who picks up the last penny loses.<BR><BR>");

  if (nim == null) {
    nim = new OneRowNim(7 + (int)(Math.random() * 11));
    nimPlayer = new NimPlayer(nim);
    out.println("<h4>You go first!</h4></center>");
  } else {
    int userTakes = Integer.parseInt(request.getParameter("pickup"));
    if (!nim.takeSticks(userTakes)) {
       out.println("<h4><font color='red'>Woops. That's an illegal move!. Try again. </font><h4>");
    } else if (!nim.gameOver()) {
       nim.changePlayer();
       out.println("<h4>So, you took <font color='blue'>" + userTakes + "</font><BR>"); 
       out.println("That leaves me with <font color='blue'>" + nim.getSticks() + "</font><BR>");
       int iTake = nimPlayer.move();
       out.println("OK. I take <font color='red'>" + iTake + " </font>pennies.<BR></h4>");
       nim.takeSticks(iTake);
       nim.changePlayer();
    } // if not gameover
  } // else nim != null
  if (!nim.gameOver()) {
    if (nim.getPlayer() == 1)
      out.println("<center><h3>Who's Turn: <font color='magenta'> Your turn </font><h3></center>");
    else
      out.println("<center><h3>Who's Turn: <font color='magenta'> My turn </font><h3></center>");
    out.println("<center><h3>Pennies Left: <font color='red'>" + nim.getSticks() + "</font><h3></center>");
    out.println("<center>");
    for (int k=0; k < nim.getSticks(); k++)
      out.println("<img src='http://xroads.virginia.edu/~CAP/LINCOLN/cent.jpg' width='25' height='25'>");
    out.println("</center><br>");

    out.println("<center>");
    out.println("<form action='/jjj3e/NimServlet' method='POST'>");
    out.println("<table border='0'>");
    out.println("<tr><td>How many do you pickup?: </td>" +
        "<td><input type='text' name='pickup' value='0'></td></tr>");
    out.println("</table>");
    out.println("</center>");
    out.println("</form>");
  } else {
    out.println("<h3><font color='red'>Game over!</font><h3>");
    if (nim.getPlayer() == 1)
      out.println("<center><h4>And the winner is : <font color='magenta'> Me.</font><h4></center>");
    else
      out.println("<center><h4>And the winner is: <font color='magenta'> You.</font><h4></center>");
    out.println("<center><h4><font color='magenta'> Nice game!</font><h4></center>");
    out.println("<center><h4>To play again, just reload the page.<h4></center>");
    nim = null;
  } // else game over
  out.println("</body>");
  out.println("</html>");
}// doGet
\end{lstlisting}
\end{jjjlistingleft}
\jjjprogstop{Java code for the {\tt NimServlet}'s {\tt doGet()} method.
\label{fig-nimservlet2}\label{pg-fig-nimservlet2}}
\end{figure}

{\tt NimServlet} uses two other objects: a {\tt OneRowNim} object and
a {\tt NimPlayer} object.  You should be familiar with these from
Chapter~8, so we won't go over their internal details here. The
{\tt OneRowNim} object manages the playing of the game and the
{\tt NimPlayer} object acts as a computer-based player of the game.
Note that variable references for these two objects are declared in
the beginning of the class definition, but the objects themselves
are declared within the {\tt doGet()} method.

One of the tricky parts of {\tt NimServlet} is how we declare the {\tt
OneRowNim} object.  As you might already know, the HTTP protocol is said to be a
{\em stateless} protocol, which means that each time a browser submits
a request to a Web server, the Web server sees the request as a
completely independent communication. The server does not, in and of
itself, maintain an internal state that keeps track of a series of
transactions with a particular browser session.  For example, when you
are shopping for books on Amazon, each time you go to a new page, the
Amazon web server treats that request as a completely independent
action.  Web applications use various techniques to get around the
stateless nature of the HTTP protocol. One technique is to use {\em
cookies} to record the progress of a session. A {\bf cookie} is a
small text file containing data that the server uses to keep track of
a user's session. Data that identifies the user and the state of the
transaction---for example, buying a book---are passed back and forth
between the browser and the server each time the user visits the
Amazon Web site.

Java's servlet library contains methods and objects that support the
use of cookies. But rather than use cookies, we will use the {\tt
OneRowNim} object itself to keep track of the state of the Nim game.
The first time the user submits a request to Nim servlet---that is,
when the user first visits the servlet's URL---the servlet will create
an instance of the {\tt OneRowNim} object. Creating a {\tt OneRowNim}
object will have the effect of initializing the game, including the
creation of a {\tt NimPlayer} to play the server's moves.  The {\tt
OneRowNim} object will persist throughout the playing of the game and
will handle all subsequent user's move. When the game is over, the
{\tt NimServlet} will, in effect, dispose of the {\tt OneRowNim}
object by setting its reference to {\tt null}.  Thus, in outline
form, the code for creating and disposing of the {\tt OneRowNim}
object goes as follows:

\begin{jjjlisting}
\begin{lstlisting}
// First request: Start a new Nim game
if (nim == null) { 
   nim = new OneRowNim(7 + (int)(Math.random() * 11));
   nimPlayer = new NimPlayer(nim);
// Code deleted here.
else {
   // Code for playing the game goes here.
}
if (!nim.gameOver()) {
  // Code for playing the game goes here.
} else {
   // Code deleted here.
  nim = null;
}
\end{lstlisting}
\end{jjjlisting}

\noindent Those places where code has been deleted in this segment
would contain Java code for responding to the user's input and
deciding how many pennies to take. 

Unlike the {\tt HelloServlet}, the {\tt NimServlet} accepts input from
the client.  The code for handling user input is as follows:

\begin{jjjlisting}
\begin{lstlisting}
int userTakes = 
        Integer.parseInt(request.getParameter("pickup"));
\end{lstlisting}
\end{jjjlisting}

\noindent This statements reads the user's input from the text field
on the Web page by using the {\tt request.getParameter()} method.
This is one of the public methods of the {\tt HttpServletRequest}
object.  The name of the text field is 'pickup', which is provided as
an argument in this method call.  As we noted above, the text field
itself is on element of the HTML form contained in the servlet's
Web page. The HTML code for creating the form element is also generated
by the servlet:

\begin{jjjlistingleft}[29pc]{-3pc}
\begin{lstlisting}
out.println("<form action='/jjj3e/NimServlet' method='POST'>");
out.println("<table border='0'>");
out.println("<tr><td>How many do you pick up?: </td>" +
  "<td><input type='text' name='pickup' value='0'></td></tr>");
\end{lstlisting}
\end{jjjlistingleft}

\noindent Unless you already know something about HTML, you won't
completely understand this code. We will give a minimal
explanation. In HTML, a text field is known as a input element of type
'text'.  Note that this code segment names the element 'pickup', which
allows our program to refer to it by that name.

The remaining details in the servlet have to do with managing the game
and repeat concepts that were covered in Chapter~8. We won't repeat
them here, other than to note that any output sent to the client must
be in the form of HTML statements, hence the appearance throughout the
code of HTML tags, which are the elements in the angle brackets.

\subsection{Setting Up and Using Java Servlets}

Java servlets can only run on a Web server that is specially
configured to interpret them.  To experiment with
the servlets discussed in this chapter, just go to the
following URL:

\begin{jjjlisting}
\begin{lstlisting}[commentstyle=\color{black}]
http://www.cs.trincoll.edu/~ram/jjj/servlets
\end{lstlisting}
\end{jjjlisting}

\noindent That web page contains links to both the {\tt HelloServlet}
and {\tt NimServlet}.  It also contains links to Web sites where you
can learn more about creating servlets. In order to create and run
your own servlets, you will need access to a Web server that has been
specially configured to run servlets.  There are several very good
free servers that support Java servlets. You can download one of these
onto your own computer and follow the directions on how to set it up.
Links to Java servlet sites are also provided on our servlets page. 


\secSMHleft{Chapter Summary}
%\addcontentsline{toc}{section}{\S~~~ Chapter Summary}

\secKTH{Technical Terms}
\begin{KT}
busy waiting

callback method

client

client/server protocols

domain name

ethernet protocol

File Transfer Protocol (FTP)

get

HyperText Transfer Protocol (HTTP)

internet

Internet

Internetworking Protocol (IP)

Java Server Page (JSP)

packet

port

post

protocol

router

sandbox security model

scriptlet

server

servlet

Simple Mail Transfer Protocol (SMTP)

socket

trusted code

Uniform Resource Locator (URL)

World Wide Web (WWW)

\end{KT}


\secSMHtwo{Summary of Important Points}
\begin{SMBL}
\item  An {\it internet} is a collection of two or more distinct networks
joined by {\it routers}, which have the task of translating one
network's language to the other's.  The {\it Internet} is a network of networks
that uses the {\it Internet Protocol (IP)} as the translation medium.

\item  A {\it protocol} is a set of rules that controls the transfer
of information between two computers in a network.  The {\it HyperText
Transfer Protocol (HTTP)} governs information exchange on the World
Wide Web (WWW). The {\it Simple Mail Transfer Protocol} controls mail
service on the Internet.  The {\it File Transfer Protocol (FTP)}
controls the transfer of files between Internet computers.  The {\it
Domain Name System (DNS)} governs the use of names on the Internet.

\item  A {\it client/server} application is one that divides
its task between a client, which requests service, and a
server, which provides service.  Many Internet applications
and protocols are based on the client/server model.

\item  Lower-level protocols, such as the {\it ethernet protocol}
and {\it token ring protocol}, govern the transmission of data between
computers on a single network.  The {\it Internet Protocol (IP)}
translates between such protocols.

\item   A {\it Uniform Resource Locator (URL)} is a standard way of
specifying addresses on the Internet.  It consists of several parts
separated by slashes and colons: {\tt method://host:port/path/file}.
The {\tt java.net.URL} class is used to represent URLs.

\item  Files of text or data (images, audio files) on the Internet
or Web can be downloaded using the same {\tt InputStream}s and
{\tt OutputStream}s as files located on a disk.  To read or write
a resource located on a network, you need to connect its URL to
an input or output stream.

\item  The {\tt java.awt.Toolkit} class contains useful methods
for downloading {\tt Image}s into an application.

\item  A {\it socket} is a two-way communication channel between
two running programs on a network.  The {\tt java.net.Socket}
class can be used to set up communication channels for
client/server applications.  The {\it server} process
listens at a socket for requests from a client.  The {\it client}
process requests service from a server listening at a particular
socket.  Once a connection exists between client and server,
input and output streams are used to read and write data over
the socket.
\end{SMBL}

%\section*{}{Answers to Self-Study Exercises}
\secANSH
%\addcontentsline{toc}{section}{\S~~~ Answers to Self-Study Exercises}

\begin{ANS}
\item  The fully connected mesh topology requires the most cables.

\item  The fully connected mesh topology would have the most potential
to use alternate routes if one of the host computers crashed.

\item  The star topology would be rendered completely useless if
its central hub crashed.

\item  Prentice Hall's Web server is located at

\begin{jjjlisting}
\begin{lstlisting}[commentstyle=\color{black}]
http://www.prenhall.com 
\end{lstlisting}
\end{jjjlisting}

\noindent The protocol is {\tt http}. The host computer is named {\tt www}.
Prentice Hall's domain name is {\tt prenhall}, and it is part of the
{\tt com} (commercial) Internet domain.


\item  \mbox{ }
\begin{ANSBL}
\item  For buying a piece of software at a bookstore,
the server would be the sales clerk.  The protocol would
be to select the software from off the shelf, bring
it to the checkout counter, give the sales clerk 
money, and get a receipt.

\item  For buying a piece of software over the phone,
the server would be the telephone sales clerk.  The protocol would
be to select from a catalog, provide the sales clerk with
your credit card information, and say goodbye.

\item  For buying a piece of software over the Internet,
the server would be the computer that handles the transaction.  The
protocol would be to select the item from a Web-based form, provide
the form with personal and payment information, and click on the
Buy button.
\end{ANSBL}


\item  To play sounds along with slides in the {\tt SlideShowApplet},
you would make the following modifications to the code:

\begin{jjjlisting}
\begin{lstlisting}
private URL soundURL[] = new URL[NIMGS];
\end{lstlisting}
\end{jjjlisting}

\noindent Declare an array of {\tt URL}s to store the URLs of
the audio files you want to play.

\hspace*{1pc}Assign URLs to the array at the same time
you input the images:

\begin{jjjlisting}
\begin{lstlisting}
for (int k=0; k < NIMGS; k++) {
  url = 
   new URL( "http://www.cs.trincoll.edu/~ram/jjj/slide" + 
                                              k + ".gif");
  slide[k] = getImage( url );
  soundURL[k] = 
   new URL("http://www.cs.trincoll.edu/~ram/jjj/sound" +
                                               k + ".au");
}
\end{lstlisting}
\end{jjjlisting}

\noindent Each time an image is displayed in {\tt paint()}, play
the corresponding sound by using the URL from the array:

\begin{jjjlisting}
\begin{lstlisting}
public void paint(Graphics g) {
    if (currentImage != null) {
        g.drawImage(currentImage,10,10,this);
        play( soundURL[currentImage] );
    }
}
\end{lstlisting}
\end{jjjlisting}

\item  The scramble service would be implemented by defining
two new classes: The {\tt ScrambleServer} class is a
subclass of {\tt Server}, and the {\tt ScrambleClient} class
is a subclass of {\tt Client}.
The {\tt ScrambleClient} would implement the {\tt requestService()}
method and the {\tt ScrambleServer} would implement the
{\tt provideService()} method.

\item  \mbox{ }
\begin{ANSBL}
\item  If you specify the wrong host name or port, you will get the
following exception:
{\tt java.net.Connect\-Exception: Connection refused}.

\item  If you leave off the \verb|\n| in the {\tt writeToSocket()}
call, nothing will go wrong because the {\tt writeToSocket()} method
will catch this error and add the end-of-line character to the string
before sending it to the server.   The server reads lines from the
client, so every communication must end with \verb|\n| or the protocol
will break down.
\end{ANSBL}
\end{ANS}


\secEXRHtwo{Exercises}
%\addcontentsline{toc}{section}{\S~~~ Exercises}
\begin{EXRtwo}

\item  Explain the difference between each of the following pairs of terms:

\begin{EXRtwoLL}
\begin{multicols}{2}
\item  {\it Stream} and {\it socket}.
\item  {\it Internet} and {\it internet}.
\item  {\it Domain name} and {\it port}.
\item  {\it Client} and {\it server}.
\item  {\it Ethernet} and {\it Internet}.
\item  {\it URL} and {\it domain name}.
\end{multicols}
\end{EXRtwoLL}
\marginnote{\raggedright{\bf Note:} For programming exercises, {\bf first} draw 
a UML class diagram describing all classes and
their inheritance relationships and/or associations.}

\item  What is a {\it protocol}? Give one or two examples
of protocols that are used on the Internet.

\item  What service is managed by the HTTP protocol?

\item  Give examples of client applications that use
the HTTP \mbox{protocol.}

\item  Why is it important that applets be limited
in terms of their network and file system access?  Describe the various
networking restrictions that apply to Java applets.

\item  What does the {\tt Internet Protocol} do?  Describe
how it would be used to join an ethernet
and a token ring network.

\item  Describe one or two circumstances under which
a {\tt Connect\-Exception} would be thrown.

\item  Modify the {\tt SlideShowApplet} so that it plays
an audio file along with each slide.

\item  Design and implement a Java applet that downloads
a random substitution cryptogram and provides an interface that
helps the user try to solve the cryptogram.  The interface should
enable the user to substitute an arbitrary letter for the
letters in the cryptogram.  The cryptogram files should be
stored in the same directory as the applet itself.

\item  Design and implement a Java application that displays
a random message (or a random joke) each time the user clicks a
{\tt GetMessage} button.  The messages should be stored in a set of files in
the same directory as the applet itself.  Each time the button is
clicked, the applet should download one of the message files.

\item  Write a client/server application of the message or
joke service described in the previous exercise.  Your implementation
should extend the {\tt Server} and {\tt Client} classes.

\item  Write an implementation of the scramble service.  Given a word,
the scramble service will return a string containing all possible
permutations of the letter combinations in the word.  For example, given ``man,''
the scramble service will return ``amn, anm, man, mna, nam, nma.''
Use the {\tt Server} and {\tt Client} classes in your design.  (See the
Self-Study Exercises for a description of the design.)

\item  {\bf Challenge:} Modify the Nim server game in this chapter so that
the client and server can negotiate the rules of the game, including
how many sticks, how many pick ups per turn, and who goes first.

\end{EXRtwo}
%
