%%%  Chapter 7: Strings and String Processing
%%%  3rd Edition

\setcounter{SSTUDYcount}{1}
\setcounter{chapter}{6}
\chapter{Strings and String  Processing}
\label{chapter-strings}
\label{pg-chapter-strings}

\CObegin
\secCOBH{Objectives}

\noindent After studying this chapter, you will
\begin{COBL}
\item  Be more familiar with Java {\tt String}s.
\item  Know how to solve problems that involve manipulating strings.
\item  Be able to use loops in designing string-processing algorithms.
\end{COBL}

\secCOLH{Outline}
\begin{COL}
\item Introduction
\item {\tt String} Basics
\item Finding Things Within a {\tt String}
\item Example: Keyword Search
\item From the Java Library: {\tt StringBuffer}
\item Retrieving Parts of Strings
\item Example: Processing Names and Passwords
\item Processing Each Character in a {\tt String}
\item Comparing {\tt String}s
\item From the Java Library: {\tt StringTokenizer}
\item Handling Text in a Graphics Context (Optional)
\par\small\item[] Chapter Summary
\par\small\item[] Solutions to Self-Study Exercises
\par\small\item[] Exercises
\end{COL}
\COend

\section{Introduction}

You have already had an introduction to {\tt String}s in the early
chapters of this text.  In Chapter~2, we introduced the {\tt String}
data type and showed how to create {\tt String} objects and use {\tt
String} methods, such as {\tt length()}, {\tt concat()}, and {\tt
equals()}.

We have seen {\tt String}s used for GUI I/O operations when used {\tt
String}s as the contents of {\tt JTextField}s and other text
components, as the values of {\tt JLabel}s, as the labels for {\tt
JButton}s, and so on. {\tt String}s are also used extensively in
command-line interfaces.

Another important task that {\tt String}s are used for are as a
standard way of presenting or displaying information about objects.
As we saw in Chapter~2, one of the key conventions of the Java class
hierarchy is that every class inherits the {\tt Object.toString()}
method, which can be used to provide a string representation of any
object.  For example, \mbox{\tt Integer.toString()} converts an {\tt
int} to a {\tt String}, so that it can be used in {\tt JTextField}s or
{\tt JLabel}s.

Programmers often have to work with strings.   Think of some of the
tasks performed by a typical word processor, such as cut, paste, copy,
and insert.  When you cut and paste text from one part of the document
to another, the program has to move one string of text, the cut, from
one location in the document and insert it in another.

Strings are also important because they are our first look at a {\em
data structure}.  A {\bf data structure}\index{data structure} is a
collection of data that is organized (structured) in some way.  A {\bf
string} is a collection of character ({\tt char}) data.  Strings are
important data structures in a programming language, and they are used
to represent a wide variety of data.

The main purpose of this chapter is to provide a detailed discussion
of Java's string-related classes, including the {\tt String}, {\tt
StringBuffer}, and {\tt StringTokenizer} classes.  These are the
important classes for writing string-processing applications.  Our
goal is to introduce the
%\begin{figure}
%\begin{graphic}
%\begin{figure}[tb]
\marginfig{chptr07/stringclass.eps}{The {\tt java.lang.String} class.
\label{fig-p349f1}
\label{pg-fig-p349f1}}
%\end{figure}
%\end{graphic}
%\end{figure}
important {\tt String} methods and illustrate common string-processing
algorithms.  We will review how to create strings from scratch and from
other data types. We will learn how to find characters and substrings
inside bigger strings.  We will learn how to take strings apart and
how to rearrange their parts.   Finally, we will learn how to apply these
string-processing skills in a program that plays the game of Hang Man.

\section{String Basics}
%%%\RHright{{\tt String} Basics}
\label{sec-strconstr}
\label{pg-sec-strconstr}
\noindent Before we cover the new material on {\tt String}s, let's
first review what we know about this topic.  In Java, {\tt String}s
are considered full-fledged objects.  A {\tt String} object is a
sequence of the characters that make up the string, plus the methods
that are used to manipulate the string.  The {\tt java.lang.String}
class (Fig.~\ref{fig-p349f1}) \marginnote{Are strings objects?}  is a
direct subclass of {\tt Object}, and it contains many public methods
that can be used to perform useful operations on strings (such as
concatenation).  We will discuss a selection of the more commonly used
methods, but for a full listing and description of the {\tt String}
methods~see 

\begin{jjjlisting}
\begin{lstlisting}[commentstyle=\color{black}]
http://java.sun.com/j2se/1.5.0/docs/api/
\end{lstlisting}
\end{jjjlisting}
\WWW


\noindent Like other object variables, {\tt String} variables serve as
{\it references} to their respective objects.  However, unlike other
Java objects, {\tt String}s have certain characteristics in common
with the primitive data types.  For example, as we have already seen,
Java allows for literal strings.  A {\bf string literal}
is a sequence of zero or more characters contained in double quotes,
such as ``Socrates'' and ``'' (the {\bf empty string}).  Java allows
us to perform operations on literal strings, such as concatenation. As
we have already seen, the expression {\tt "Hello" + "world"} results
in the string {\tt "Helloworld"}.  Java also allows us to use string
literals to initialize {\tt String} variables with an assignment
statement.  These exceptional features greatly simplify the use of
{\tt String}s in our programs. Given how much we use {\tt String}s,
incorporating these features into Java seems like a good design
decision.

\subsection{Constructing Strings}
\noindent To create {\tt String} objects, the {\tt String} class provides
many constructors, including the following:

\begin{jjjlisting}
\begin{lstlisting}
public String();        // Creates an empty string
   // Copy constructor: Creates a copy of a string
public String(String initial_value); 
\end{lstlisting}
\end{jjjlisting}

\noindent When we create an object using the first constructor, as in

\begin{jjjlisting}
\begin{lstlisting}
String name = new String();
\end{lstlisting}
\end{jjjlisting}

\noindent Java will create a {\tt String} object and make {\tt name}
\marginfig{chptr07/p349f2.eps}{An empty string is a {\tt String} object
with value ``'' and count 0.\break
\label{fig-emptystring}\label{pg-fig-emptystring}}
the reference to it.  Figure~7.2 shows a hypothetical representation
of a {\tt String} object. In addition to storing the sequence of
characters that make up the string, Java also stores an integer value
representing the number of characters in the string.  We have chosen
to represent these two elements as the private instance variables,
{\tt value}, for the sequence of characters, and {\tt count} for the
number of characters. In fact, we don't know exactly how Java stores
the sequence of characters. That information is hidden. As Figure~7.2
illustrates, when we use the default constructor, the value of the is
the empty string and its count is 0.


The second constructor is the {\em copy constructor} for the {\tt
String} class.  A {\bf copy constructor} is a constructor that makes a
duplicate, sometimes called a clone, of an object. Many Java classes
have copy constructors.  Consider the following statements:

\begin{jjjlisting}
\begin{lstlisting}
String s1 = new String("Hello"););
String s2 = new String(s1);
\end{lstlisting}
\end{jjjlisting}

\noindent These two statements would result in two distinct {\tt
String} objects, both storing the word ``Hello''.

Note that in the first of the preceding statements, we used the
literal string ``Hello'' in the constructor.  When Java encounters a
\marginfig{chptr07/p349f3.eps}{The literal {\tt String} ``Socrates.''\break
\label{fig-literalstr}\label{pg-fig-literalstr}}
new literal string in a program, it constructs an object for it.  For
example, if your program contained the literal ``Socrates,'' Java
would create an object for it and treat the literal itself as a
reference to the object (Fig.~7.3).

We often use a string literal to assign a value to a {\tt String}
variable: 

\begin{jjjlisting}
\begin{lstlisting}
String s;      // The value of s is initially null
s = "Socrates";// s now refers to "Socrates" object
\end{lstlisting}
\end{jjjlisting}

\noindent In this case, the reference variable {\it s} is initially
{\tt null}---that is, it has no referent, no object, to refer to.
However, after the assignment statement, {\it s} would refer to the
literal object ``Socrates,'' which is depicted in Figure~7.3. Given
these two statements, we still have only one object, the {\tt String}
object containing the word ``Socrates.''.  But now we have two
references to it: the literal string ``Socrates,'' and the reference
variable {\it s}.

Assignment statements can also be used as initializers when declaring
a {\tt String} variable:
\marginfig{chptr07/p350f1.eps}{The variables {\tt name1}, {\tt name2},
and {\tt name3} serve as references to the literal {\tt String}
objects ``Socrates'' and ``''.
\label{fig-literalrefs}
\label{pg-fig-literalrefs}}

\begin{jjjlisting}
\begin{lstlisting}
String name1 = "";  // Reference to the empty string
String name2 = "Socrates";// References to "Socrates"
String name3 = "Socrates";
\end{lstlisting}
\end{jjjlisting}

\noindent In this example, Java does not construct new {\tt String}
objects.  Instead, as Figure~7.4 shows, it simply makes the variables
{\tt name1}, {\tt name2}, and {\tt name3} serve as references to the
same objects that are referred to by the literal strings ``'' and
``Socrates.'' This is a direct consequence of Java's policy of
creating only one object to serve as the referent of a literal string,
no matter how many occurrences there are of that literal in the
program.  Thus, these declarations result in no new objects, just new
references to existing objects.  The justification for this policy is
that it saves lots of memory in our programs. Instead of creating a
{\tt String} object for each occurrence of the literal ``Socrates,''
Java creates one object and lets all occurrences of ``Socrates'' refer
to that object.

Finally, consider the following declarations, which do invoke the
{\tt String} constructors:
\marginfig{chptr07/p351f1.eps}{Together with the objects in Figure~7.4, there
are now four different {\tt String} objects with
eight different references to them, including the literals ``Socrates''
\break and ``''.
\label{fig-strobjects}
\label{pg-fig-strobjects}}

\begin{jjjlisting}
\begin{lstlisting}
String name4 = new String();    // Creates an object
String name5 = new String("Socrates");
String name6 = name4;
\end{lstlisting}
\end{jjjlisting}

\noindent In this case, as shown in Figure~7.5,  Java creates 
two new objects and sets {\tt name4} to refer to the first and {\tt
name5} to refer to the second.  It gives {\tt name4} the empty string
as its value, and it gives {\tt name5} ``Socrates'' as its value.  But
these two objects must be distinguished from the objects corresponding
to the literals (``''~and ``Socrates'') themselves.  The
declaration of {\tt name6} just creates a second reference to the
object referred to by {\tt name4}.

\JavaRule[false]{Strings.}{Java {\tt String}s are
full-fledged objects, but they have some properties in common with
primitive types.  They can have literal values and they can be used in
assignment statements.}

\JavaRule[false]{String Declaration and Instantiation.}{Unless
a {\tt String()} constructor is called explicitly, no new {\tt String}
object is created when declaring a {\tt String} variable and
assigning it an initial value.}


\subsection{Concatenating Strings }
\noindent Another way to build a {\tt String} object is to concatenate
two other strings. Recall from Chapter~2 that there are two ways to
perform string concatenation in Java: We can use the {\tt concat()}
method or the concatenation operator, $+$.

\begin{jjjlisting}
\begin{lstlisting}
String lastName = "Onassis";
String jackie = 
   new String("Jacqueline " + "Kennedy " + lastName);
System.out.println("Jacqueline".concat(lastName));
\end{lstlisting}
\end{jjjlisting}

\noindent The second of these statements uses the {\it concatenation
%\begin{marginalnote}\it String concatenation\end{marginalnote}
\marginnote{String concatenation}
operator}, $+$, to create the {\tt String} ``Jacqueline Kennedy Onassis.''
The third statement uses the {\tt String} method, {\tt concat()},
to print ``JacquelineOnassis.''

\noindent Using the + symbol as the string concatenation operator is
\marginnote{Operator overloading} another example of {\it operator
overloading}---using the same operator for two or more different
operations---which we encountered in Chapter~5.

\JavaRule{String Concatenation.}{When surrounded on
either side by a {\tt String}, the + symbol is used as a binary {\bf
concatenation} operator.  It has the effect of joining two strings
together to form a single string.}

\noindent Note that primitive types are automatically promoted to {\tt
String}s when they are mixed with concatenation operators.  Thus, the
statement

\begin{jjjlisting}
\begin{lstlisting}
System.out.println("The sum of 5 and 5 = "+ (5 + 5));
\end{lstlisting}
\end{jjjlisting}

\noindent will print the string ``The sum of 5 and 5 = 10.''  Note
that the integer addition---(5 + 5)---is performed first, before the
integer result is converted into a String.  If we had left off the
parentheses around the addition operation, the second plus sign would
also be interpreted as a concatenation operator. Thus,

\begin{jjjlistingleft}[28pc]{-2pc}
\begin{lstlisting}
System.out.println("The concatenation of 5 and 5 = " + 5 + 5);
\end{lstlisting}
\end{jjjlistingleft}

\noindent would print ``The concatenation of 5 and 5 = 55.''

\secEXRHone{Self-Study Exercises}
\begin{SSTUDY}
\item  What will be printed by each of the following segments of code?
\begin{EXRLL}
\item  \verb|String s1 = "silly"; System.out.println(s1);|
\item  \verb|String s2 = s1; System.out.println(s2);|
\item  \verb|String s3 = new String (s1 + " stuff");|
       \\ \verb|    System.out.println(s3);|
\end{EXRLL}

\item  Write a {\tt String} declaration that satisfies each
of the following descriptions:
\begin{EXRLL}
\item  Initialize a {\tt String} variable, {\it str1}, to the empty
        string.
\item  Instantiate a {\tt String} object, {\it str2}, and initialize
        it to the word {\it stop}.
\item  Initialize a {\tt String} variable, {\it str}, to the concatenation of
{\it str1} and {\it str2}.
\end{EXRLL}

\item  Evaluate the following expressions:

\begin{jjjlisting}
\begin{lstlisting}
int M = 5, N = 10;
String s1 = "51", s2 = "75";
\end{lstlisting}
\end{jjjlisting}

\begin{EXRLL}
\begin{multicols}{3}
\item \verb|M + N|
\item \verb|M + s1|
\item \verb|s1 + s2|     
\end{multicols}
\end{EXRLL}


\item  Draw a picture, similar to Figure~7.5, showing the
objects and references that are created by the following declarations:
\begin{jjjlisting}
\begin{lstlisting}
String s1, s2 = "Hello", s3 = "Hello";
String s4 = "hello";
String s5 = new String("Hello");
String s6 = s5;
String s7 = s3;
\end{lstlisting}
\end{jjjlisting}

\end{SSTUDY}

\subsection{Indexing {\tt String}s }

\noindent Programmers often need to take strings apart or put them
together or rearrange them. Just think of the many word-processing
tasks, such as cut and paste, that involve such operations.  To help
simplify such operations, it is useful to know how many characters a
string contains and to number, or {\em index}, the characters that
make up the string.

The number of characters in a string is called its {\it length}.  The
%\begin{marginalnote}\it String length\end{marginalnote}
\marginnote{String length}
{\tt String} instance method, {\tt length()}, returns an integer that
gives the {\tt String}'s length.  For example, consider the following
{\tt String} declarations and the corresponding values of the
{\tt length()} method for each case:

\begin{jjjlisting}
\begin{lstlisting}
String string1 = "";           string1.length()  ==> 0
String string2 = "Hello";      string2.length()  ==> 5
String string3 = "World";      string3.length()  ==> 5
String string4 = string2 + " " 
    + string3;                 string4.length()  ==> 11
\end{lstlisting}
\end{jjjlisting}
%-3.5pc

The position of a particular character in a string is called its {\bf
\marginfig{chptr07/7f5.eps}{The string ``Socrates''
has eight characters, indexed from 0 to 7. This is an example
of {\it zero indexing}.
\label{fig-zeroindex}
\label{pg-fig-zeroindex}}
string index}. All {\tt String}s in Java are {\bf zero
indexed}\index{zero indexed}---that is, the index of the first
character is zero.  (Remember, zero indexing is contrasted with {\bf
unit indexing}, in which we start counting at 1.)  For example, in
``Socrates,'' the letter {\it S} occurs at index 0, the letter {\it o}
occurs at index 1, {\it r} occurs at index 3, and so on.  Thus, the
{\tt String} ``Socrates'' contains eight characters indexed from 0 to
7 (Fig.~7\mbox{.}6).  Zero indexing is customary in programming
languages.  We will see other examples of this when we talk about
arrays and vectors.

\JavaRule{String Indexing.}{Strings are indexed
starting at 0. The first character in a string is at position 0.}

\JavaTIP{DEBUGGING TIP}{Zero Versus Unit Indexing.}
{Syntax and semantic errors will result if you forget that strings are
zero indexed.  In a string of {\it N} characters, the first character
occurs at index 0 and the last at index $N-1$. This is
different from the {\tt String.length()} method, which gives the
number of characters in the string, counting from 1.}

\subsection{Converting Data to Strings}
\noindent The {\tt String.valueOf()} method is a {\it class method}
that is used to convert a value of some primitive type into a {\tt
String} object.  For example, the expression, {\tt
String.valueOf(128)} converts its {\tt int} argument to the~{\tt
String} ``128.''

There are different versions of  {\tt valueOf()}, each of which
has the following type of signature:

\begin{jjjlisting}
\begin{lstlisting}
static public String valueOf(Type);
\end{lstlisting}
\end{jjjlisting}

\noindent where {\tt Type} stands for any primitive data type,
including {\tt boolean}, {\tt char}, {\tt int}, {\tt double}, and so
on.

The {\tt valueOf()} method is most useful for initializing
{\tt String}s.  Because {\tt valueOf()} is a class method, it can be used as
follows to instantiate new {\tt String} objects:

\begin{jjjlisting}
\begin{lstlisting}
String number = String.valueOf(128); // Creates "128"
String truth = String.valueOf(true); // Creates "true"
String bee = String.valueOf('B');    // Creates "B"
String pi = String.valueOf(Math.PI);  // Creates "3.14159"
\end{lstlisting}
\end{jjjlisting}

\noindent We have already seen that Java automatically promotes
primitive type values to {\tt String} where necessary, so why do we
need the {\tt valueOf()} methods?  For example, we can initialize a
{\tt String} to ``3.14159'' as follows:

\begin{jjjlisting}
\begin{lstlisting}
String pi = new String(""+Math.PI);// Creates "3.14"
\end{lstlisting}
\end{jjjlisting}

\noindent In this case, because it is part of a concatenation
expression, the value of {\tt Math.PI} will automatically be promoted
to a {\tt String} value. The point of the {\tt valueOf()} method is
twofold. First, it may be the method that the Java compiler relies on
to perform string promotions such as this one. Second, using it in a
program---even when it is not completely necessary---makes the
promotion operation explicit rather than leaving it implicit. This
\marginnote{Readability}
helps to make the code more readable. (Also, see Exercise 7.9.)

\secEXRHone{Self-Study Exercises}
\begin{SSTUDY}
\item  Evaluate each of the following expressions:

\begin{EXRLL}
\begin{multicols}{2}
\item  \verb|String.valueOf (45)|     
\item  \verb|String.valueOf (128 - 7)|
\item  \verb|String.valueOf ('X')|    
\end{multicols}
\end{EXRLL}


\item  Write an expression to satisfy each of the following \mbox{descriptions:}

\begin{EXRLL}
\item  Convert the integer value 100 to the string "100". 
\item  Convert the character 'V' to the string "V".       
\item  Initialize a new String object to  X times Y.      
\end{EXRLL}

\end{SSTUDY}


%%%RAN\RHrightsec{7.3}{Finding Things Within a String}
\section{Finding Things Within a String}

\noindent Programmers often have to find the location of a particular
character or substring in a string. For example, user names and
passwords are sometimes stored in a single string in which the name
and password are separated from each other by a special character,
such as a colon ({\tt username:password}). In order to get the name or
password from such a string, it is convenient to have methods that
will search the string and report the index of the colon character.

The {\tt indexOf()} and {\tt lastIndexOf()}
methods\index{lastIndexOf() method} are instance methods that can be
used to find the index position of a character or a substring within a
{\tt String}. There are several versions of each:

\begin{jjjlisting}
\begin{lstlisting}
public int indexOf(int character);
public int indexOf(int character, int startingIndex);
public int indexOf(String string);
public int indexOf(String string, int startingIndex);

public int lastIndexOf(int character);
public int lastIndexOf(int character, int startingIndex);
public int lastIndexOf(String string);
public int lastIndexOf(String string, int startingIndex);
\end{lstlisting}
\end{jjjlisting}

\noindent The {\tt indexOf()} method searches from left to right within
a {\tt String} for either a character or a substring.  The
{\tt lastIndexOf()} method searches from right to left for a character or
substring.  To illustrate, suppose we have declared the following
{\tt String}s:

\begin{jjjlisting}
\begin{lstlisting}
String string1 = "";
String string2 = "Hello";
String string3 = "World";
String string4 = string2 + " " + string3;
\end{lstlisting}
\end{jjjlisting}



\noindent Recalling that {\tt String}s are indexed starting at 0,
searching for {\it o} in the various strings gives the following results:

\begin{jjjlisting}[27pc]
\begin{lstlisting}
string1.indexOf('o') ==> -1  string1.lastIndexOf('o') ==> -1
string2.indexOf('o') ==>  4  string2.lastIndexOf('o') ==>  4
string3.indexOf('o') ==>  1  string3.lastIndexOf('o') ==>  1
string4.indexOf('o') ==>  4  string4.lastIndexOf('o') ==>  7
\end{lstlisting}
\end{jjjlisting}


\noindent Because {\tt string1} is the empty string, ``'', it does not
%\begin{marginalnote}\it Sentinel return value\end{marginalnote}
\marginnote{Sentinel return value}
contain the letter {\it o}. Therefore, {\tt indexOf()} returns $-1$, a
value that cannot be a valid index for a {\tt String}. This convention
is followed in {\tt indexOf()} and {\tt lastIndexOf()}.  Because
{\tt string2} and {\tt string3} each contain only one occurrence of the
letter {\it o}, both {\tt indexOf()} and {\tt lastIndexOf()} return the
same value when used on these {\tt String}s.  Because {\tt string4}
contains two occurrences of {\it o}, {\tt indexOf()} and
{\tt lastIndexOf()} return different values in this case.
%\begin{figure}
%\begin{graphic}
%\begin{figure}[tb]
\marginfig{chptr07/7f6.eps}{The indexing of the ``Hello World'' string.
\label{fig-helloindex}
\label{pg-fig-helloindex}}
%\end{figure}
%\end{graphic}
%\end{figure}
As Figure~7.7 shows, the first {\it o} in ``Hello World''
occurs at index 4, the value returned by {\tt indexOf()}. The second
{\it o} occurs at index 7, which is the value returned by
{\tt lastIndexOf()}.

By default, the single-parameter versions of {\tt indexOf()} and
{\tt last\-IndexOf()} start their searches at their respective (left or
right) ends of the string.   The two-parameter versions of these
methods allow you to specify both the direction and starting point of
the search.   The second parameter specifies the starting
index.  Consider these examples:

\begin{jjjlisting}
\begin{lstlisting}
string4.indexOf('o', 5)     ==> 7
string4.lastIndexOf('o', 5) ==> 4
\end{lstlisting}
\end{jjjlisting}

\noindent If we start searching in both cases at index 5, then
{\tt indexOf()} will miss the {\it o} that occurs at index 4. The first
{\it o} it finds will be the one at index 7. Similarly,
{\tt lastIndexOf()} will miss the {\it o} that occurs at index 7 and will find
the {\it o} that occurs at index 4.

The {\tt indexOf()} and {\tt lastIndexOf()} methods can also be used
to find substrings:


\begin{jjjlistingleft}[28pc]{-2pc}
\begin{lstlisting}
string1.indexOf("or") ==> -1  string1.lastIndexOf("or") ==> -1
string2.indexOf("or") ==> -1  string2.lastIndexOf("or") ==> -1
string3.indexOf("or") ==>  1  string3.lastIndexOf("or") ==>  1
string4.indexOf("or") ==>  7  string4.lastIndexOf("or") ==>  7
\end{lstlisting}
\end{jjjlistingleft}

\noindent The substring ``or'' does not occur in either
{\tt string1} or {\tt string2}. It does occur beginning at location 1
in {\tt string3} and beginning at location 7 in {\tt string4}. For
this collection of examples, it doesn't matter whether we search from
left to right or right to left.

\secEXRHone{Self-Study Exercises}
\begin{SSTUDY}

\item  Suppose the {\tt String} variable {\tt s} has been
initialized to ``mom.'' Evaluate each of the following
expressions:

\begin{EXRLL}

\begin{multicols}{3}
\item  \verb|s.indexOf("m");|  
\item  \verb|s.indexOf("o");|  
\item  \verb|s.indexOf("M");|  
\end{multicols}

\end{EXRLL}

\item  Evaluate the  expressions given the {\tt String} declaration
\verb|String s1 = "Java, Java, Java";|

\begin{EXRLL}\columnsep=0pt
\scriptsize
\begin{multicols}{2}
\item  \verb|s1.length()|                      
\item  \verb|String.valueOf(s1.length())|      
\item  \verb|s1.indexOf('a')|                  
\item  \verb|s1.lastIndexOf('a')|              
\item  \verb|s1.indexOf("av")|                 
\item  \verb|s1.lastIndexOf("av")|           
\item  \verb|s1.indexOf('a', 5)|               
\item  \verb|s1.lastIndexOf('a', 5)|           
\item \verb|s1.indexOf("av", s1.length() - 10)|      
\item \verb|s1.lastIndexOf("av",| \\\verb|s1.length() - 4)|   
\item \verb|s1.indexOf("a", s1.indexOf("va"))|       
\end{multicols}
\normalsize
\end{EXRLL}

\item  Evaluate the following expression:

\begin{jjjlisting}
\begin{lstlisting}
  String tricky = "abcdefg01234567";
  tricky.indexOf(String.valueOf( tricky.indexOf("c")));
\end{lstlisting}
\end{jjjlisting}
\end{SSTUDY}

\section{Example: Keyword Search}
\noindent One of the most widely used Web browser functions is the search
utility.   You probably know how it works.  You type in a keyword and
click on a button, and it returns with a list of Web pages that
contain the keyword.

Suppose you were writing a browser in Java.  How would you implement
this function? Of course, we don't know yet how to read files or Web
pages, and we won't cover that until Chapter~11.
But, for now, we can write a method that will search a string for all
occurrences of a given keyword.  That's at least part of the task that
the browser's search engine would have to do.

So we want a method, {\tt keywordSearch()}, that takes two
%\begin{marginalnote}\it Method design\end{marginalnote}
\marginnote{Method design}
{\tt String} parameters, one for the string that's being searched, and the
other representing the keyword.  Let's have the method return a
{\tt String} that lists the number of keyword occurrences, followed
by the index of each occurrence.  For example, if we asked this method
to find all occurrences of {\it is} in ``This is a test,'' it should return
the string ``2: 2 5'' because there are two occurrences of
{\it is}, one starting at index 2 and the other at index 5 in the string.

The algorithm for this method will require a loop, because we want to
%\begin{marginalnote}\it Algorithm design\end{marginalnote}
know the location of every occurrence of the keyword in the string.
One way to do this would be to use the {\tt indexOf()} method to
search for the location of substrings in the string.   If it finds the
\marginnote{\vspace{-24pt}Algorithm design}
keyword at index {\it N}, it should record that location and then continue
searching for more occurrences starting at index $N+1$ in the string.  It
should continue in this way until there are no more occurrences.

\begin{jjjlisting}
\begin{lstlisting}
Suppose S is our string and K is the keyword.
Initialize a counter variable and result string.
Set Ptr to the indexOf() the first occurrence of K in S.
While (Ptr != -1)
    Increment the counter
    Insert Ptr into the result string
    Set Ptr to the next location of the keyword in S
Insert the count into the result string
Return the result string as a String
\end{lstlisting}
\end{jjjlisting}

\noindent As this pseudocode shows, the algorithm uses a while loop
%\begin{marginalnote}\it Implementation\end{marginalnote}
\marginnote{Implementation}
with a {\it sentinel bound}.  The algorithm terminates when the
{\tt indexOf()} method returns a $-1$, indicating that there are no more
occurrences of the keyword in the string.

Translating the pseudocode into Java gives us the method shown in
Figure~\ref{fig-keywordsearch}.  Note how string concatenation is used
to build the \mbox{\tt resultStr}. Each time an occurrence is found, its
location ({\tt ptr}) is concatenated to the right-hand side of the
{\tt resultStr}.  When the loop terminates, the number of occurrences
({\tt count}) is concatenated to the left-hand side of the
{\tt resultStr}.

\begin{figure}[h]
\jjjprogstart
\begin{jjjlisting}
\begin{lstlisting}
/**
 * Pre:  s and keyword are any Strings
 * Post: keywordSearch() returns a String containing the
 *  number of occurrences of keyword in s, followed
 *  by the starting location of each occurrence
 */
public String keywordSearch(String s, String keyword) {
  String resultStr = "";
  int count = 0;
  int ptr = s.indexOf(keyword);
  while (ptr != -1) {
    ++count;
    resultStr = resultStr + ptr + " ";
    ptr = s.indexOf(keyword, ptr + 1);// Next occurrence
  }
  resultStr = count + ": " + resultStr;// Insert the count
  return resultStr;                  // Return as a String
} // keywordSearch()
\end{lstlisting}
\end{jjjlisting}
\jjjprogstop{The {\tt key\-word\-Search()} method.
\label{fig-keywordsearch}
\label{pg-fig-keywordsearch}}
\end{figure}

\subsection*{Testing and Debugging}
\noindent What test data should we use for the {\tt keywordSearch()} method?
%\begin{marginalnote}\it What test data do we need?\end{marginalnote}
\marginnote{What test data do we need?}  One important consideration
in this case is to test that the method works for all possible
locations of the keyword within the string.  Thus, the method should
be tested on strings that contain keyword occurrences at the
beginning, middle, and end of the string.  We should also test the
method with a string that doesn't contain the keyword.  Such tests
will help verify that the loop will terminate properly in all cases.
Given these considerations, Table~7.1 shows the tests that were made.
As you can see from these results, the method did produce the expected
outcomes.  While these tests do not guarantee its correctness, they
provide considerable evidence that the algorithm works correctly.


\begin{table}[htb]
%\hphantom{\caption{Testing the {\tt keywordSearch()} method.}
%\label{tab-tests}
%\label{pg-tab-tests}}
\TBT{0pc}{Testing the {\tt keywordSearch()} method.}
\hspace*{-6pt}\begin{tabular}{ll} 
\multicolumn{2}{l}{\color{cyan}\rule{27pc}{1pt}}\\[2pt]
%%\TBCH{Test Performed}         & \TBCH{Expected Result}
{Test Performed}         & {Expected Result}
\\[-4pt]\multicolumn{2}{l}{\color{cyan}\rule{27pc}{0.5pt}}\\[2pt]
\verb|keywordSearch("this is a test","is")|           & 2: 2 5        \cr
\verb|keywordSearch("able was i ere i saw elba","a")|\hspace*{5pt} & 4: 0 6 18 24  \cr
\verb|keywordSearch("this is a test","taste")|        & 0:            
\\[-4pt]\multicolumn{2}{l}{\color{cyan}\rule{27pc}{1pt}}
\end{tabular}
\endTB
\end{table}

\JavaTIP[false]{EFFECTIVE DESIGN}{Test Data.}{In designing test data to
check the correctness of a string searching algorithm, it's
important to use data that test all possible outcomes.}


%%%RAM\begin{javalibrary}{{\fontBXsevenCDT java.lang.StringBuffer}}
\section{From the Java Library: java.lang.StringBuffer}
\WWWjava

\tBOXseven{One problem} with the {\tt keywordSearch()} method is that
it is not very efficient because a {\tt String} in Java is a {\bf
read-only} object.  This means that once it has been instantiated, a
{\tt String} cannot be changed.  You cannot insert new characters or
delete existing characters from it.

\JavaRule{Strings Are Immutable.}{Once instantiated,
a Java {\tt String} cannot be altered in any way.}

\noindent Given this fact, how is it possible that the {\tt
resultStr} in the {\tt keyword\-Search()} ends up with the
correct value?
he answer is that every time we assign a new value to {\tt
resultStr}, Java has to create a new {\tt String} object.  
Figure~7.9 illustrates the process. Thus, given the statement
\begin{jjjlisting}
\begin{lstlisting}
resultStr = resultStr + ptr + " ";
\end{lstlisting}
\end{jjjlisting}

\noindent Java will evaluate the right-hand side, which creates a new
{\tt String} object whose value would be the concatenation of the
right-hand-side elements, \verb|resultStr + ptr + " "|
(Fig.~7\mbox{.}9a). It would then assign the new object as the new
referent of {\tt resultStr} (Fig.~7.9b).  This turns the previous
referent of {\tt resultStr} into an {\bf orphan object}---that is,
into an object that no longer has any references to~it.  Java will
eventually dispose of these orphaned objects, removing them from
memory in a process known as {\bf garbage collection}.  However,
creating and disposing of objects is a task that consumes the
computer's time.

The fact that this assignment statement occurs within a loop means
that several new objects are created and later garbage collected.
Because object creation is a relatively time-consuming and
memory-consuming operation, this algorithm is somewhat wasteful of
Java's resources.  
\marginfig{chptr07/p359f1.eps}{\small Evaluating {\tt resultStr =
resultStr + ptr + " "} creates an orphan object that must be garbage
collected.
\label{fig-orphan}
\label{pg-fig-orphan}}

Of course, except for the inefficiency of doing it this way, no real
harm is done by this algorithm used in the {\tt keywordSearch()}
method.  Java's garbage collector will automatically reclaim the
memory used by the orphaned object. However, this algorithm does
consume more of Java's resources than other algorithms we might use.

\JavaRule{Automatic Garbage Collection.}{An object that
has no reference to it can no longer be used in a program.  Therefore,
Java will automatically get rid of it.  This is known as
{\it garbage collection}.}

\noindent A more efficient way to write the {\tt keywordSearch()}
method would make use of a {\tt StringBuffer} to store and construct
the {\tt resultStr}.  Like the {\tt String} class, the {\tt
java.lang.StringBuffer} class also represents a string of characters.
However, unlike the {\tt String} class, a {\tt StringBuffer} can be
modified, and it can grow and shrink in length as
%\begin{marginalnote}\it Choosing the appropriate data structure\end{marginalnote}
\marginnote{\vspace{36pt}\raggedright Choosing the appropriate data structure}
necessary.  As Figure~7.10 shows, the {\tt StringBuffer}
class contains several of the same kind of methods as the {\tt String}
class, for example, {\tt charAt()} and {\tt length()}. But it also
contains methods that allow characters and other types of data to be
inserted into a string, such as {\tt append()}, {\tt insert()}, and
{\tt setCharAt()}.  Most string-processing algorithms use {\tt
StringBuffer}s instead of {\tt String}s as their preferred data
structure.

\JavaTIP{PROGRAMMING TIP}{StringBuffer.}{A {\tt StringBuffer} should
be used instead of a {\tt String} for any task that involves modifying
a string.}

\noindent The {\tt StringBuffer} class provides several methods that are useful
for string processing. The constructor method, {\tt
StringBuffer(String)}, makes it easy to convert a {\tt String} into a
{\tt StringBuffer}. Similarly, once you are done processing the
buffer, the {\tt toString()} method makes it easy to convert a {\tt
StringBuffer} back into a {\tt String}.

The typical way to use a {\tt StringBuffer} is shown in the following
revised version of the {\tt keywordSearch()} method:

\begin{jjjlisting}
\begin{lstlisting}
public String keywordSearch(String s, String keyword) {
                                   // Create StringBuffer
  StringBuffer resultStr = new StringBuffer(); 
  int count = 0;
  int ptr = s.indexOf(keyword);
  while (ptr != -1) {
    ++count;
    resultStr.append(ptr + " ");  // Append to buffer
    ptr = s.indexOf(keyword, ptr + 1);
  }
  resultStr.insert(0, count + ": ");
  return resultStr.toString(); // Convert buffer to String
} // keywordSearch()
\end{lstlisting}
\end{jjjlisting}

\noindent We declare {\tt resultStr} as a {\tt StringBuffer}
\marginfig{chptr07/p360f1.eps}{The {\tt java.lang.StringBuffer} class.
\label{fig-p360f1}
\label{pg-fig-p360f1}}
instead of a {\tt String}. Then, instead of concatenating the {\tt
ptr} and reassigning the {\tt resultStr}, we {\tt append()} the {\tt
ptr} to the {\tt resultStr} for each occurrence of a
keyword. Similarly, after the loop exits, we {\tt insert()} the {\tt
count} at the front (index 0) of the {\tt resultStr}. Finally, we
convert {\tt resultStr} into a {\tt String} by using the {\tt
toString()} method before returning the method's result.

One advantage of the {\tt StringBuffer} class is that there are several
versions of its {\tt insert()} and {\tt append()} methods.  These make
it possible to insert any type of data---{\tt int}, {\tt double},
{\tt Object}, and so on---into a {\tt StringBuffer}. The method itself
takes care of converting the data into a string for~us.


To summarize, {\tt String} objects in Java are {\it immutable}. So
\marginnote{Strings are immutable}
when a {\tt String} is ``modified,'' this really means that a new {\tt String}
object is created and the old {\tt String} object must be garbage collected.
This is somewhat inefficient, especially if done repeatedly within a loop.
To avoid these inefficiencies, use a {\tt StringBuffer} instead of a {\tt String}
in such contexts.

\section{Retrieving Parts of Strings}
\noindent Programmers often need to retrieve an individual character or
a part of a string from a string, as, for example, in a word
processing program when a part of a string is copied or deleted. In
this section we look at methods that help us with these kinds of tasks.

The {\tt charAt(int index)} method is a {\tt String} instance method
that can be used to retrieve the character stored at a certain
index.  The several varieties of the {\tt substring()} method can be
used to retrieve a substring of characters from a {\tt String}. These
methods are defined as follows:

\begin{jjjlisting}
\begin{lstlisting}
public char charAt(int index)
public String substring(int startIndex)
public String substring(int startIndex, int endIndex)
\end{lstlisting}
\end{jjjlisting}

\noindent The {\tt charAt()} method\index{charAt() method}
returns the character located at the index supplied as its parameter.
Thus, {\tt str.charAt(0)} retrieves the first character in {\tt str},
while {\tt str.charAt(str.length()-1)} retrieves the last character.

The {\tt substring()} methods\index{substring() method} work in a
similar way, except that you need to specify both the starting and the
ending index of the substring you wish to retrieve.  The first version
of {\tt substring(int startIndex)} takes a single parameter and
returns a {\tt String} consisting of all the characters beginning with
{\tt startIndex} and continuing up to the end of the {\tt String}. For
example, if the {\tt str} is ``HelloWorld'', then {\tt str.substring(5)}
would return ``World'' and {\tt str.substring(3)} would return
``loWorld'':

\begin{jjjlisting}
\begin{lstlisting}
String str = "HelloWorld";
str.substring(5)            ==> "World"
str.substring(3)            ==> "loWorld"
\end{lstlisting}
\end{jjjlisting}

\noindent The {\tt substring(int, int)} version requires that you specify both
the starting and ending index of the substring.   The second
index always points to the character that is one beyond the last
character in the {\tt String} you want to retrieve.  For example,

\begin{jjjlisting}
\begin{lstlisting}[commentstyle=\small]
 //   INDEX:   0123456789
 String str = "HelloWorld";
 str.substring(5,7)               ==> "Wo"
 str.substring(0,5)               ==> "Hello"
 str.substring(5, str.length())   ==> "World"
\end{lstlisting}
\end{jjjlisting}


\noindent Note here that when we want to retrieve ``Wo''
from {\tt str}, we specify its substring as indexes 5 and 7; the 7
points to the character just beyond ``Wo.'' Similarly,
{\tt substring(0,5)}, picks out the first five characters (``Hello'').  
In the third example, the {\tt length()} method specifies
the substring beginning at index 5 and extending to the end of the
string.  This is equivalent to {\tt str.substring(5)}:

\begin{jjjlisting}
\begin{lstlisting}[commentstyle=\small]
 //   INDEX:   0123456789
 String str = "HelloWorld";
 str.substring(5, str.length())   ==> "World"
 str.substring(5)                 ==> "World"
\end{lstlisting}
\end{jjjlisting}


\noindent The fact that the second parameter in {\tt substring()}
refers to the character one beyond the desired substring may seem a
bit confusing at first, but it is actually a very useful way to
%\begin{marginalnote}\it Delimited strings\end{marginalnote}
designate a substring.  For example, many string-processing problems
have to do with retrieving substrings from a {\bf delimited string},
which is a string that contains special characters that separate
the string into certain substrings.
\marginnote{Delimited strings}
For example, consider the string ``substring1:substring2,''
in which the {\it delimiter} is the colon, {\tt ':'}. The following code
retrieves the substring preceding the delimiter:

\begin{jjjlisting}
\begin{lstlisting}
String str = "substring1:substring2";
int n = str.indexOf(':');
str.substring(0,n)               ==> "substring1"
\end{lstlisting}
\end{jjjlisting}

\noindent Thus, by making the second index of {\tt
substring()} refer to the character one beyond the last character in
the desired substring, we can use {\tt indexOf()} and {\tt
substring()} together to process delimited strings.  Note that it is
not necessary to use a temporary variable {\it n} to store the index
of the delimiter, because the two method calls can be nested:

\begin{jjjlisting}
\begin{lstlisting}
String str = "substring1:substring2";
str.substring(0,str.indexOf(':'))  ==> "substring1"
\end{lstlisting}
\end{jjjlisting}

\JavaTIP{DEBUGGING TIP}{{\tt substring(int p1, int
p2)}.}{Don't forget that the second parameter in the {\tt substring()}
methods refers to the character just past the last character in the
substring.}

\secEXRHone{Self-Study Exercises}
\begin{SSTUDY}
\item  Given the {\tt String} declaration 

\begin{jjjlisting}
\begin{lstlisting}
String s = "abcdefghijklmnopqrstuvwxyz";
\end{lstlisting}
\end{jjjlisting}

\noindent evaluate each of the following expressions:

\begin{EXRLL}
\begin{multicols}{2}
\item  \verb|s.substring(20)|       
\item  \verb|s.substring(1, 5)|     
\item  \verb|s.substring(23)|       
\item  \verb|s.substring(23, 25)|   
\item  \verb|s.substring(s.indexOf('x'))|
\end{multicols}
\end{EXRLL}


\item  Given the preceding declaration of {\tt s}, evaluate each
of the following expressions:

\begin{EXRLL}
\item  \verb|s.substring(20, s.length())|                  
\item  \verb|s.substring(s.indexOf('b'), s.indexOf('f'))|  
\item  \verb|s.substring(s.indexOf("xy"))|                 
\item  \verb|s.substring(s.indexOf(s.charAt(23)))|         
\item  \verb|s.substring(s.length() - 3)|                  
\end{EXRLL}

\end{SSTUDY}


\section{Example: Processing Names and Passwords}
\noindent Many computer systems store user names and passwords as delimited
strings, such as

\begin{jjjlisting}
\begin{lstlisting}
smith:bg1s5xxx
mccarthy:2ffo900ssi
cho:biff4534ddee4w
\end{lstlisting}
\end{jjjlisting}


\noindent Obviously, if the system is going to process passwords,
it needs some way to take apart these name-password pairs.

Let's write methods to help perform this task.  The first method will
%\begin{marginalnote}\it Algorithm design\end{marginalnote}
be passed a name-password pair and will return the name.  The second
method will be passed a name-password pair and will return the
password.  In both cases, the method takes a single {\tt String}
parameter and returns a {\tt String} result:
\marginnote{\vspace{-36pt}Algorithm design}

\begin{jjjlisting}
\begin{lstlisting}
String getName(String str);
String getPassword(String str);
\end{lstlisting}
\end{jjjlisting}

\noindent To solve this problem we can make use of two {\tt String} methods.  We
use the {\tt indexOf()} method to find the location of the {\it
delimiter}---which is the colon,  ``{\tt :}''---in the name-password pair
and then we use {\tt substring()} to take the substring occurring
before or after the delimiter.  It may be easier to see this if we
take a particular example:


\begin{jjjlisting}
\begin{lstlisting}
 INDEX:           1         2
 INDEX: 012345678901234567890
        jones:b34rdffg12    // (1)
        cho:rtf546          // (2)
\end{lstlisting}
\end{jjjlisting}

\noindent In the first case, the delimiter occurs at index position 5 in
the string.  Therefore, to take the name substring, we would use
{\tt substring(0,5)}. To take the password substring, we would
use {\tt substring(6)}.  Of course, in the general case, we would
use variables to indicate the position of the delimiter, as in the
following methods:

\begin{jjjlisting}[27pc]
\begin{lstlisting}
public static String getName(String str) {
  int posColon = str.indexOf(':');  // Find the delimiter
  String result = str.substring(0, posColon); // Get name
  return result;
}

public static String getPassword(String str) {
  int posColon = str.indexOf(':');  // Find the delimiter
  String result = str.substring(posColon + 1); // Get passwd
  return result;
}
\end{lstlisting}
\end{jjjlisting}


\noindent Note in both of these cases we have used
local variables, {\tt posColon} and \mbox{\tt result}, to store the
intermediate results of the computation---that is, the index of the
``{\tt :}'' and the name or password substring.

An alternative way to code these operations would be to use nested
method calls to reduce the code to a single line:

\begin{jjjlisting}
\begin{lstlisting}
return str.substring(0, str.indexOf(':'));
\end{lstlisting}
\end{jjjlisting}

\noindent In this line, the result of {\tt str.indexOf(\verb!'!:\verb!'!)} is
passed immediately as the second argument to {\tt str.substring()}.
This version dispenses with the need for additional variables.
And the result in this case is not unreasonably complicated.
But whenever you are faced with a trade-off of this sort---nesting
versus additional variables---you should opt for the style that
will be easier to read and understand.

\JavaTIP{EFFECTIVE DESIGN}{Nested Method Calls.}{Nested method calls are
fine as long as there are not too many levels of nesting.  The goal
should be to produce code that is easy to read and understand.}

%%%RAM\RHrightsec{7.7}{Processing Each Character in a String}

\section{Processing Each Character in a {\tt String}}
%%%RAM\RHright{Processing Each Character in a {\tt String}}
\noindent Many string-processing applications require you to process each
character in a string.  For example, to encrypt the string ``hello'' into
``jgnnq'', we have to go through each letter of the string and change
each character to its substitute.

These types of algorithms usually involve a counting loop bounded by
\marginnote{Counting loop algorithm} the length of the string. Recall
that the {\tt length()} method determines the number of characters in
a {\tt String} and that strings are zero indexed.  This means that the
first character is at index 0, and the last character is at index {\tt
length()-1}.  For example, to print each character in a string on a
separate line, we would step through the string from its first to its
last character and print each character::

\begin{jjjlisting}[26.5pc]
\begin{lstlisting}
// Precondition:  str is not null
// Postcondition: the letters in str will have been printed
public void printLetters(String str) {
  for (int k = 0; k < str.length(); k++)// For each char
    System.out.println(str.charAt(k));  //  Print it
}
\end{lstlisting}
\end{jjjlisting}

\noindent Note that our loop bound is \verb|k < str.length()|, since
%\begin{marginalnote}\it Counting bound\end{marginalnote}
\marginnote{Counting bound}
the index of the last character of any {\tt String} is
{\tt length()-1}. Note also the use of {\tt str.charAt(k)} to retrieve the
{\it k}th character in {\tt str} on each iteration of the loop.

Note the use of pre- and postconditions in the method's comment block.
The precondition states that {\tt str} has been properly
initialized---that is, it is not {\tt null}. The postcondition merely
states the expected behavior of the method.

\subsection{Off-by-One Error}
\noindent A frequent error in coding counter-controlled loops is known as the
{\bf off-by-one error}\index{off-by-one error}, which can occur in many different ways.  For
example, if we had coded the loop boundary condition as
\verb|k <= str.length()|, this would cause an
off-by-one error, because the last
%\begin{marginalnote}\it Off-by-one error\end{marginalnote}
\marginnote{Off-by-one error}
character in {\tt str} is at location {\tt length()-1}.  This would
lead to a Java {\tt IndexOutOfBoundsException}, which would be reported
as soon as the program executed this statement.

The only way to avoid off-by-one errors is to check your loop bounds
whenever you code a loop.  Always make sure you have the loop counter's
initial and final values correct.

\JavaTIP{DEBUGGING TIP}{Off-by-One Errors.}{Loops should be carefully
checked to make sure they don't commit an off-by-one error.  During
program testing, develop data that tests the loop variable's initial
and final values.}

\subsection{Example: Counting Characters}
\noindent As another example of an algorithm that processes
each character in a string, consider the problem of computing the
frequency of the letters in a given document.  Certain text analysis
programs, such as programs that analyze encrypted data and spam
filters, perform this type of function. 


The {\tt countChar()} method will count the number of occurrences of
%\begin{marginalnote}\it Method design\end{marginalnote}
\marginnote{Method design}
any particular character in a {\tt String}
(Fig.~\ref{fig-countchar}).  This method takes two parameters: a
{\tt String} parameter that stores the string being searched and a
{\tt char} parameter that stores the character being counted.

\begin{figure}[h!]
\jjjprogstart
\begin{jjjlisting}
\begin{lstlisting}
// Precondition: Neither str nor ch are null
// Postcondition: countchar() == the number of ch in str
public int countChar(String str, char ch) {
  int counter = 0;              // Initialize a counter
  for (int k = 0; k < str.length(); k++)// For each char
    if (str.charAt(k) == ch)           //  If it's a ch
      counter++;                          //   count it
  return counter;                  // Return the result
}
\end{lstlisting}
\end{jjjlisting}
\jjjprogstop{A method to count the occurrence of a particular
character in a string.
\label{fig-countchar}
\label{pg-fig-countchar}}
\end{figure}

Begin by initializing the local variable, {\tt counter}, to
\marginnote{Algorithm design}
0. As in the previous example, the {\tt for} loop here will iterate through
each character of the {\tt String}---from 0 to {\tt length()-1}.  On
each iteration a check is made to see if the character in the {\it k}th
position ({\tt str.charAt(k)}) is the character being counted.  If so,
{\tt counter} is incremented.  The method ends by returning
{\tt counter}, which, when the method completes, will store an integer
representing the number of {\it ch}'s in {\tt str}.


\subsection{Example: Reversing a String}
\noindent Another interesting method that processes each character in
a string is the {\tt reverse()} method. This is a method that reverses
the letters in a string. For example, the reverse of {\tt "java"} is
{\tt "avaj"}.  

The algorithm for the {\tt reverse()} method should use a simple
\marginnote{Algorithm design} counting loop to reverse the letters in
its {\tt String} parameter.  In this case, however, we can process the
string from right to left, beginning at its last character and ending
with its first character.  That way we can just append each character,
left to right, in the result string:   

\begin{jjjlisting}
\begin{lstlisting}
 /*
  * Pre:  s is any non null string
  * Post: s is returned in reverse order
  */
public String reverse(String s) {
    StringBuffer result = new StringBuffer();
    for (int k = s.length()-1; k >= 0; k--) {
        result.append(s.charAt(k));
    } //for
    return result.toString();
} // reverse()
\end{lstlisting}
\end{jjjlisting}

\noindent Note that as in the other string-manipulation
algorithms---for example, {\tt keywordSearch()}---we should us a {\tt
StringBuffer} to store the method's result.  Thus we declare the {\tt
result} {\tt StringBuffer} at the beginning of the method and convert
it back into a {\tt String} at the end of the method.

\JavaTIP{PROGRAMMING TIP}{Changing Each Character in a String.}{Algorithms that require 
you to alter a string should use a
{\tt StringBuffer} to store the result.}

\subsection{Example: Capitalizing the First Letter }
\noindent Another string manipulation method is
the {\tt capitalize()} method, which returns a {\tt String} whose
\marginnote{\it Algorithm design}
initial letter is capitalized but whose other letters are lowercase --
for example, ``Hello''.  We use the {\tt static} {\tt toUpperCase()}
and {\tt toLowerCase()} methods from the {\tt Character} class to
convert individual letters.  We could also have used the methods of
the same name that we wrote in Section~5.8.  The algorithm converts
the first letter to upper case and then loops through the remaining
letters converting each to lowercase:

\begin{jjjlisting}[27pc]
\begin{lstlisting}
/*
 * Pre:  s is any non null string
 * Post: s is returned with only its first letter capitalized
 */
public String capitalize(String s) {
  if (s.length() == 0)    // Special case: empty string
    return s;
  StringBuffer result = new StringBuffer();
  result.append(Character.toUpperCase(s.charAt(0))); 
                          // Convert the first letter
  for (int k = 1; k < s.length(); k++) { // And the rest 
    result.append(Character.toLowerCase(s.charAt(k)));
  } //for
  return result.toString();
} // capitalize()
\end{lstlisting}
\end{jjjlisting}

\secEXRHone{Self-Study Exercises}
\begin{SSTUDY}

\item Write a Java program to test the methods described in this
section.  Organize the methods themselves into a single class, named
{\tt StringProcessor}, and design a second class to serve as the user
interface. Because these methods are similar to the utility methods of
the {\tt Math} class, it would be useful to declare them static.  The
user interface should prompt the user to input a string and should
then print out the result of passing that string to each of the
methods we developed.

\item Add a method to the {\tt StringProcessor} class that will remove
all blanks from a string.  It should take a {\tt String} parameter and
should return a {\tt String} result.

\end{SSTUDY}


\subsection{Miscellaneous {\tt String} Methods}
\noindent In addition to the several {\tt String} class methods we have
discussed---{\tt valueOf()}, {\tt equals()}, {\tt indexOf()},
{\tt lastIndexOf()}, {\tt charAt()}, {\tt substring()}---Table~7.2 
shows some of the other useful methods
in the {\tt String} class.  Note that because of what we said about
the read-only nature of {\tt String}s, methods such as {\tt toUpperCase()},
{\tt toLowerCase()}, and {\tt trim()} do not  change their string.  Instead
they produce a new string.  If you want to use one of these methods to convert
a string, you must reassign its result back to the original string:

\begin{jjjlisting}
\begin{lstlisting}
String s = new String("hello world");
s = s.toUpperCase(); // s now equals "HELLO WORLD"
\end{lstlisting}
\end{jjjlisting}
%%%RAM\endcasestudy

\begin{table}[tb]
\TBT{-12.5pc}{Some useful {\tt String} methods applied to the literal string "Perfection."}
\hspace*{-6pt}\hspace*{-12.5pc}
\begin{tabular}{ll}
\multicolumn{2}{l}{\color{cyan}\rule{40.5pc}{1pt}}\\[2pt]
%%%RAM\TBCH{Method Signature}                                 &\TBCH{Example}
{Method Signature}                                 &{Example}
\\[-4pt]\multicolumn{2}{l}{\color{cyan}\rule{40.5pc}{0.5pt}}\\[2pt]
{\tt boolean endsWith(String suffix)}                &\verb|"Perfection".endsWith("tion")|     $\Rightarrow$ true \cr
{\tt boolean startsWith(String prefix)}              &\verb|"Perfection".startsWith("Per")|    $\Rightarrow$ true \cr
{\tt boolean startsWith(String prefix, int offset)}\hspace*{5pt}  &\verb|"Perfection".startsWith("fect",3)| $\Rightarrow$ true \cr
{\tt String toUpperCase()}                             &\verb|"Perfection".toUpperCase()|        $\Rightarrow$ \verb|"PERFECTION"|\cr
{\tt String toLowerCase()}                             &\verb|"Perfection".toLowerCase()|        $\Rightarrow$ \verb|"perfection"|\cr
{\tt String trim()}                                    &\verb|"Perfection".trim()|               $\Rightarrow$ \verb|"Perfection"|
\\[-4pt]\multicolumn{2}{l}{\color{cyan}\rule{40.5pc}{1pt}}
\end{tabular}
\endTB
\end{table}

\section{Comparing Strings}

\noindent Comparing strings is another important task. For example,
when a word processor performs a search and replace operation, it
needs to identify strings in the text that match the target string.

Strings are compared according to their {\em lexicographic
order}---that is, the order of their characters.  For the letters of
the alphabet, lexicographic order just means alphabetical order.
Thus, {\it a} comes before {\it b} and {\it d} comes after {\it
c}. The string ``hello'' comes before ``jello'' because {\it h} comes
before {\it j} in the alphabet.

For Java and other programming languages, the definition of
lexicographic order is extended to cover all the characters that make
up the character set.  We know, for example, that in Java's Unicode
character set the uppercase letters come before the lowercase letters
(Table~5.13).  So, the letter {\it H} comes before the letter
{\it h} and the letter {\it Z} comes before the letter {\it a}.


Lexicographic order\index{lexicographic order} can be extended to
%\begin{marginalnote}\it H precedes h\end{marginalnote}
\marginnote{H precedes h}
include strings of characters.  Thus, ``Hello'' precedes ``hello'' in
lexicographic order because its first letter, {\it H}, precedes the first
letter, {\it h}, in ``hello.'' Similarly, the string ``Zero'' comes before
``aardvark,'' because {\it Z} comes before {\it a}.  To determine
lexicographic order for strings, we must perform a
character-by-character comparison, starting at the first character and
proceeding left to right.  As an example, the following strings are
arranged in lexicographic order:

\begin{jjjlistingleft}[30.5pc]{-4.5pc}
\begin{lstlisting}[stringstyle=\color{black}]
"" "!" "0" "A" "Andy" "Z" "Zero" "a" "an" "and" "andy" "candy" "zero"
\end{lstlisting}
\end{jjjlistingleft}

\noindent We can define {\bf lexicographic order}\index{lexicographic order} for strings as follows:

\JavaRule[false]{Lexicographic Order.}{For strings {\it s1}
and {\it s2}, {\it s1} precedes {\it s2} in lexicographic order if its
first character precedes the first character of {\it s2}. If their
first characters are equal, then {\it s1} precedes {\it s2} if its
second character precedes the second character of {\it s2}; and so
on.  An empty string is handled as a special case, preceding
all other strings.}


\noindent Perhaps a more precise way to define lexicographic
order is to define a Java method:

\begin{jjjlisting}
\begin{lstlisting}
public boolean precedes(String s1, String s2) {
                           // Pick shorter length
  int minlen = Math.min(s1.length(), s2.length()); 
               // For each char in shorter string}
  for (int k=0; k < minlen; k++) { 
    if (s1.charAt(k) != s2.charAt(k))// If chars unequal
       //  return true if s1's char precedes s2's
      return s1.charAt(k) < s2.charAt(k); 
  }
            // If all characters so far are equal
        //  then s1 < s2 if it is shorter than s2
  return s1.length() < s2.length();  
} // precedes()}  
\end{lstlisting}
\end{jjjlisting}

\noindent This method does a character-by-character comparison of
%\begin{marginalnote}\it Algorithm: Loop bound\end{marginalnote}
\marginnote{Algorithm: Loop bound} the two strings, proceeding left to
right, starting at the first character in both strings.  Its {\tt for}
loop uses a counting bound, which starts at {\it k} equal to zero and
counts up to the length of the shorter string.  This is an important
point in designing this algorithm.  If you don't stop iterating when
you get past the last character in a string, your program will
generate a {\tt StringIndexOutOfBounds} exception.  To prevent this
error, we need to use the shorter length as the loop bound.

Note that the loop will terminate early if it finds that the
respective characters from {\it s1} and {\it s2} are unequal.  In that
case, {\it s1} precedes {\it s2} if {\it s1}'s {\it k}th character
precedes {\it s2}'s.  If the loop terminates normally, that means that
all the characters compared were equal.  In that case, the shorter
string precedes the longer.  For example, if the two strings were
``alpha'' and ``alphabet,'' then the method would return true, because
``alpha'' is shorter than ``alphabet.''

\secEXRHone{Self-Study Exercises}
\begin{SSTUDY}
\item  Arrange the following strings in lexicographic order:

\begin{jjjlisting}
\begin{lstlisting}
zero bath bin alpha Alpha Zero Zeroes a A z Z
\end{lstlisting}
\end{jjjlisting}


\item  Modify the {\tt precedes()} method so that it will also return true
when {\it s1} and {\it s2} are equal---for example, when
{\it s1} and {\it s2} are both ``hello''.

\end{SSTUDY}


\subsection{Object Identity Versus Object Equality}
\noindent Java provides several methods for comparing {\tt String}s:

\begin{jjjlisting}
\begin{lstlisting}
public boolean equals(Object anObject); // Overrides Object.equals()
public boolean equalsIgnoreCase(String  anotherString);
public int compareTo(String  anotherString);
\end{lstlisting}
\end{jjjlisting}

\noindent The first comparison method, {\tt
equals()}\index{String.equals()}, overrides the {\tt Object.equals()}
method.  Two {\tt String}s are equal if they have the exact same
letters in the exact same \marginnote{Equality vs.  identity} order.
Thus, for the following declarations,

\begin{jjjlisting}
\begin{lstlisting}
String s1 = "hello";
String s2 = "Hello";
\end{lstlisting}
\end{jjjlisting}

\noindent {\tt s1.equals(s2)} is {\tt false}, but \verb|s1.equals("hello")|
is {\tt true}.

You have to be careful when using Java's {\tt equals()}
%\begin{marginalnote}\it Equality vs.  identity\end{marginalnote}
method.  According to the default definition of {\tt equals()}, defined
in the {\tt Object} class, ``equals'' means ``identical.''  Two
{\tt Object}s are equal only if their names are references to the same
object.
\marginfig{chptr07/7f11.eps}{Venus is the morning star, so ``Venus''
and ``the morning star'' are two references to the same object.
\label{fig-venus}\label{pg-fig-venus}}

This is like the old story of the morning star and the evening star,
which were thought to be different objects before it was discovered
that both were just the planet Venus.  After the discovery, it was
clear that ``the morning star'' and ``the evening star'' and ``Venus'' were
just three different references to one and the same object
(Fig.~7.12).

We can create an analogous situation in Java by using the following
{\tt JButton} definitions:

\begin{jjjlisting}
\begin{lstlisting}
JButton b1 = new Button("a");
JButton b2 = new Button("a");
JButton b3 = b2;
\end{lstlisting}
\end{jjjlisting}

\noindent Given these three declarations, {\tt b1.equals(b2)}
and {\tt b1.equals(b3)} would be {\tt false}, but {\tt b2.equals(b3)}
would be {\tt true} because {\it b2} and {\it b3} are just two
names for the same object (Fig.~7.13). So, in this case,
\marginfig{chptr07/7f12.eps}{For most objects, equality means identity.
JButtons {\tt b2} and {\tt b3} are identical (and, hence, equal), but
JButtons {\tt b1} and {\tt b2} are not identical (and, hence, unequal).
\label{fig-buttons}
\label{pg-fig-buttons}}
``equals'' really means ``identical.''

Moreover, in Java, when it is used to compare two objects, the equality operator
(==) is interpreted in the same way as the default {\tt Object.equals()}
method.   So, it really means object identity.   Thus, {\tt b1 == b2} 
would be {\tt false}, because {\tt b1} and {\tt b2}
are different objects, but {\tt b2 == b3} would be {\tt true} because
{\tt b2} and {\tt b3} refer to the same object.


These points are illustrated in the program shown in
Figure~\ref{fig-equals}. This program uses methods {\tt isEquals()}
%% proglist ch7/equals/TestEquals.java
\begin{figure}[h!]
\jjjprogstart
\begin{jjjlisting}[32pc]
\begin{lstlisting}
import java.awt.*;

public class TestEquals {
  static Button b1 = new Button ("a");
  static Button b2 = new Button ("b");
  static Button b3 = b2;

  private static void isEqual(Object o1, Object o2) {
    if (o1.equals(o2))
      System.out.println(o1.toString() + " equals " + o2.toString());
    else
      System.out.println(o1.toString() + " does NOT equal " + 
                                                      o2.toString());
  } // isEqual()

  private static void isIdentical(Object o1, Object o2) {
    if (o1 == o2)
      System.out.println(o1.toString() + " is identical to " + 
                                                      o2.toString());
    else
      System.out.println(o1.toString() + " is NOT identical to " + 
                                                      o2.toString());
  } // isIdentical()

  public static void main(String argv[]) {
    isEqual(b1, b2);         // not equal
    isEqual(b1, b3);         // not equal
    isEqual(b2, b3);         // equal

    isIdentical(b1, b2);     // not identical
    isIdentical(b1, b3);     // not identical
    isIdentical(b2, b3);     // identical
  } // main()
} // TestEquals
\end{lstlisting}
\end{jjjlisting}
\jjjprogstop{The {\tt TestEquals} program tests Java's default
{\tt equals()} method, which is defined in the {\tt Object} class.
\label{fig-equals}
\label{pg-fig-equals}}
\end{figure}
and {\tt isIdentical()} to perform the comparisons and print the
results. This program will produce the following output:

\begin{jjjlisting}[32pc]
\begin{lstlisting}
java.awt.Button[button0,0,0,0x0,invalid,label=a]
   does NOT equal java.awt.Button[button1,0,0,0x0,invalid,label=b]
java.awt.Button[button0,0,0,0x0,invalid,label=a]
   does NOT equal java.awt.Button[button1,0,0,0x0,invalid,label=b]
java.awt.Button[button1,0,0,0x0,invalid,label=b]
   equals java.awt.Button[button1,0,0,0x0,invalid,label=b]
java.awt.Button[button0,0,0,0x0,invalid,label=a]
   is NOT identical to java.awt.Button[button1,0,0,0x0,invalid,label=b]
java.awt.Button[button0,0,0,0x0,invalid,label=a]
   is NOT identical to java.awt.Button[button1,0,0,0x0,invalid,label=b]
java.awt.Button[button1,0,0,0x0,invalid,label=b]
   is identical to java.awt.Button[button1,0,0,0x0,invalid,label=b]
\end{lstlisting}
\end{jjjlisting}



\subsection{String Identity Versus String Equality}


\noindent In comparing Java {\tt String}s, we must be careful to distinguish
between object identity and string equality.   Thus, consider the
following declarations, which create the situation shown 
in Figure~\ref{fig-strings}.

%\begin{figure}
\begin{figure}[h!]
\figaleft{chptr07/p375f1.eps}{For {\tt String} objects, equality\index{String equality}
and identity\index{String identity} are different.   Two distinct
(nonidentical) String objects are equal if they store the same string
value.  So {\tt s1, s2, s4, s5,} and {\tt s6} are equal.   Strings
{\tt s1} and {\tt s4} are identical, and so are strings {\tt s5} and
{\tt s6}.
\label{fig-strings}
\label{pg-fig-strings}}
\end{figure}
%\end{figure}

\begin{jjjlisting}
\begin{lstlisting}
String s1 = new String("hello");
String s2 = new String("hello");
String s3 = new String("Hello");
String s4 = s1;            // s1 and s4 are now identical
String s5 = "hello";
String s6 = "hello";
\end{lstlisting}
\end{jjjlisting}

\noindent Given these declarations, we would get the
\marginnote{\vspace{-24pt}Equality vs.  identity}
following results if we compare the equality of the {\tt String}s:

\begin{jjjlisting}
\begin{lstlisting}
s1.equals(s2) ==> true  s1.equalsIgnoreCase(s3)  ==> true
s1.equals(s3) ==> false s1.equals(s5)            ==> true
s1.equals(s4) ==> true  s1.equals(s6)            ==> true
\end{lstlisting}
\end{jjjlisting}

\noindent and the following results if we compare their identity:

\begin{jjjlisting}
\begin{lstlisting}
s1 == s2  ==> false            s1 == s3  ==> false
s1 == s4  ==> true             s1 == s5  ==> false
s5 == s6  ==> true
\end{lstlisting}
\end{jjjlisting}

\noindent The only true identities among these {\tt String}s are {\it s1} and {\it s4},
and {\it s5} and {\it s6}.  In the case of {\it s5} and {\it s6}, both
are just references to the literal string, ``hello'', as we described
in Section~\ref{sec-strconstr}.  The program in Figure~\ref{fig-equals2}
illustrates these points. 

%% proglist ch7/equals/TestStringEquals.java
\begin{figure}[p]
\jjjprogstart
\begin{jjjlisting}[35pc]
\begin{lstlisting}
import java.awt.*;

public class TestStringEquals {
  static String s1 = new String("hello"); // s1 and s2 are equal, not identical
  static String s2 = new String("hello");
  static String s3 = new String("Hello"); // s1 and s3 are not equal
  static String s4 = s1;                  // s1 and s4 are identical
  static String s5 = "hello";             // s1 and s5 are not identical
  static String s6 = "hello";             // s5 and s6 are identical

  private static void testEqual(String str1, String str2) {
    if (str1.equals(str2))
      System.out.println(str1 + " equals " + str2);
    else
      System.out.println(str1 + " does not equal " + str2);
  } // testEqual()

  private static void testIdentical(String str1, String str2) {
    if (str1 == str2)
      System.out.println(str1 + " is identical to " + str2);
    else
      System.out.println(str1 + " is not identical to " + str2);
  } // testIdentical()

  public static void main(String argv[]) {
    testEqual(s1, s2);        // equal
    testEqual(s1, s3);        // not equal
    testEqual(s1, s4);        // equal
    testEqual(s1, s5);        // equal
    testEqual(s5, s6);        // equal

    testIdentical(s1, s2);    // not identical
    testIdentical(s1, s3);    // not identical
    testIdentical(s1, s4);    // identical
    testIdentical(s1, s5);    // not identical
    testIdentical(s5, s6);    // identical
  } // main()
}// TestStringEquals
           ------Program Output-----
           hello equals hello
           hello does not equal Hello
           hello equals hello
           hello equals hello
           hello equals hello
           hello is not identical to hello
           hello is not identical to Hello
           hello is identical to hello
           hello is not identical to hello
           hello is identical to hello
\end{lstlisting}
\end{jjjlisting}
\jjjprogstop{Program illustrating the difference between
string equality and identity.
\label{fig-equals2}
\label{pg-fig-equals2}}
\end{figure}

\secEXRHone{Self-Study Exercises}
\begin{SSTUDY}

\item  Given the  {\tt String} declarations,

\begin{jjjlisting}
\begin{lstlisting}
String s1 = "java", s2 = "java", s3 = "Java";
String s4 = new String(s2);
String s5 = new String("java");
\end{lstlisting}
\end{jjjlisting}

\noindent evaluate the following expressions:

\begin{EXRLL}
\begin{multicols}{3}
\item  \verb|s1 == s2|       
\item  \verb|s1.equals(s2)|  
\item  \verb|s1 == s3|       
\item  \verb|s1.equals(s3)|  
\item  \verb|s2 == s3|       
\item  \verb|s2.equals(s4)|  
\item  \verb|s2 == s4|       
\item  \verb|s1 == s5|       
\item  \verb|s4 == s5|       
\end{multicols}
\end{EXRLL}


\item  Why are the variables in {\tt TestStringEquals} declared {\tt static}?

\item  Given the following declarations,

\begin{jjjlisting}
\begin{lstlisting}
String s1 = "abcdefghijklmnopqrstuvwxyz";
String s2 = "hello world";
\end{lstlisting}
\end{jjjlisting}

\noindent write Java expressions to carry out each of the following
operations:

\begin{EXRLL}
\item  Swap the front and back half of {\tt s1} giving a new string.
\item  Swap "world" and "hello" in {\tt s2} giving a new string.
\item  Combine parts of {\tt s1} and {\tt s2} to create a new string "hello abc".
\end{EXRLL}

\end{SSTUDY}


\section{From the Java Library: \\java.util.StringTokenizer}

\tBOXseven{One of the} most widespread string-processing tasks is that
of breaking \WWWjava up a string into its components, or {\bf tokens}.
For example, when processing a sentence, you may need to break the
sentence into its constituent words, which are considered the sentence
tokens.  When processing a name-password string, such as
``boyd:14irXp'', you may need to break it into a name and a password.
Tokens are separated from each other by one or more characters which
is known as {\bf delimiters}. Thus, for a sentence, {\em white space},
including blank spaces, tabs, and line feeds, serve as the delimiters.
For the password example, the colon character serves as a delimiter.

Java's {\tt java.util.StringTokenizer}\index{StringTokenizer class} class
\marginfig{chptr07/p377f1.eps}{The {\tt java.util.StringTokenizer} class.
\label{fig-p377f1}
\label{pg-fig-p377f1}}
is specially designed for breaking strings into their tokens
(Fig.~7.17). When instantiated with a {\tt String} parameter, a {\tt
StringTokenizer} breaks the string into tokens, using white space as
delimiters.  For example, if we instantiated a {\tt StringTokenizer}
as in the code

\begin{jjjlisting}
\begin{lstlisting}
StringTokenizer sTokenizer
  = new StringTokenizer("This is an English sentence.");
\end{lstlisting}
\end{jjjlisting}

\noindent it would break the string into the following
tokens, which would be stored internally in the {\tt StringTokenizer}
in the  order shown:

\begin{jjjlisting}
\begin{lstlisting}
This
is
an
English
sentence.
\end{lstlisting}
\end{jjjlisting}

\noindent Note that the period is part of the last token
(``sentence.''). This is because punctuation marks are not
considered delimiters by default.  

If you wanted to include punctuation symbols as delimiters, you could
use the second {\tt StringTokenizer()} constructor, which takes a
second {\tt String} parameter (Fig.~7.17).  The second
parameter specifies a string of those characters that should be used
as delimiters.  For example, in the  instantiation,

\begin{jjjlisting}
\begin{lstlisting}
StringTokenizer sTokenizer
    = new StringTokenizer("This is an English sentence.",
                          "\b\t\n,;.!");
\end{lstlisting}
\end{jjjlisting}

\noindent various punctuation symbols (periods, commas, and so on)
are included among the delimiters. Note that escape sequences 
(\verb|\b\t\n|) are used to specify blanks, tabs, and newlines. 

The {\tt hasMoreTokens()} and {\tt nextToken()} methods can be used to
process a delimited string, one token at a time.   The first method
returns {\tt true} as long as more tokens remain; the second gets the
next token in the list.  For example, here's a code segment that will
break a standard URL string into its constituent parts:

\begin{jjjlisting}
\begin{lstlisting}
String url = "http://java.trincoll.edu/~jjj/index.html";
StringTokenizer sTokenizer = new StringTokenizer(url,":/");
while (sTokenizer.hasMoreTokens()) {
    System.out.println(sTokenizer.nextToken());
}
\end{lstlisting}
\end{jjjlisting}

\noindent This code segment will produce the following output:

\begin{jjjlisting}
\begin{lstlisting}
http
java.trincoll.edu
~jjj
index.html
\end{lstlisting}
\end{jjjlisting}

\noindent The only delimiters used in this case were the ``{\tt :}'' and ``{\tt /}''
symbols.  And note that {\tt nextToken()} does not return the empty string
between ``{\tt :}'' and ``{\tt /}'' as a token.

\markright{{\color{cyan}SECTION\,\,\arabic{section}\,\,$\bullet$}\,\, Handling Text in a Graphics Context}

\section{Handling Text in a Graphics Context \\(Optional)}

\markright{{\color{cyan}SECTION\,\,\arabic{section}\,\,$\bullet$}\,\, Handling Text in a Graphics Context}

\noindent In order to create attractive GUIs, it is often necessary to
be able to select and control the font that is used.  Even a simple
drawing task, such as being able to center a message in a panel,
requires that we know the font's dimensions and be able to manipulate
\marginfig{chptr07/fontmethods.eps}{Methods to access the {\tt Font} and
{\tt FontMetrics} objects associated with each {\tt Graphics} context.
\label{fig-fontmethods}
\label{pg-fig-fontmethods}}
them.  In this section, we learn how to work with Java's fonts and
font control methods.

Each graphics context has an associated {\tt Font} and {\tt
FontMetrics} object, and the {\tt Graphics} class
(Fig.~7.18) provides several methods to access them.
A {\tt FontMetrics} is an object that encapsulates important data
about a font, such as its height and width.  Java assigns a default
font to each {\tt Graphics} object.  For example, this is the font
used by the {\tt drawString()} method, which we used in our very first
Java programs back in Chapter~1.  The particular font used is system
dependent, but to override the default one can simply invoke the {\tt
setFont()} method:

\begin{jjjlisting}
\begin{lstlisting}
g.setFont(new Font("TimesRoman", Font.ITALIC, 12));
\end{lstlisting}
\end{jjjlisting}

\noindent In this case, the {\tt Font()} constructor is used
to specify a 12-point, italicized, {\it TimesRoman} font.  Once the
font is set, it will be used in all subsequent drawings.

\subsection{The {\tt Font} and {\tt FontMetrics} Classes}
\noindent The {\tt Font} class (Fig.~7.19) provides a
platform-independent representation of an individual font.  A font is
distinguished by its name, size, and style, and the {\tt Font} class
includes {\tt protected} instance variables for these properties, as
well as a constructor method that allows these three characteristics
to be specified. 

%\begin{figure}
%\begin{graphic}
\marginfig{chptr07/font.eps}{The {\tt Font} class.
\label{fig-font}
\label{pg-fig-font}}
%\end{graphic}
%\end{figure}

In order to understand how fonts work, it is necessary to distinguish
between a {\em character}, which is a symbol that represents a certain
letter or digit, and a {\bf glyph}, which is a shape used to display
the character.  When you display a string, such as ``Hello'', Java
maps each individual character into a corresponding shape, as defined
by the particular font that is selected. 

Java distinguishes between {\em physical} and {\em logical} fonts.  A
{\bf physical font} is an actual font library that contains the data
and tables needed to associate the correct glyph with a given
character.  Generally speaking, a given platform (host computer plus
operating system) will have a collection of such fonts available on
it.

A {\bf logical font} is one of five font families that are supported
by the Java runtime environment. These include Serif, SansSerif,
Monospaced, Dialog, and DialogInput.  Java also supports the following
font styles: {\tt PLAIN}, {\tt BOLD}, {\tt ITALIC}, and {\tt
BOLD+ITALIC}.  Whereas the physical fonts are platform dependent, the
logical fonts and styles are platform independent. When used in a
program, they are mapped to real fonts available on the host system.
If the host system does not have an exact match for the specified
font, it will supply a substitute.  For example, if you specify a
48-point, italic, Monospaced font,

\begin{jjjlisting}
\begin{lstlisting}
Font myFont = new Font("Monospaced", Font.ITALIC, 48);
\end{lstlisting}
\end{jjjlisting}

\noindent the system may map this to a 24-point, italic
Courier font, if that is the largest fixed-spaced font available.

The {\tt Font()} constructor is designed to work with any set of
arguments.  Thus, if you supply the name of a font that is not
available, the system will supply a default font as a substitute.  For
example, on my system, specifying a nonexistent font named Random,

\begin{jjjlisting}
\begin{lstlisting}
g.setFont(new Font("Random", Font.ITALIC, 12) );
g.drawString("Hello World! (random, italic, 12)", 30, 45);
\end{lstlisting}
\end{jjjlisting}

\noindent produces the same font used as the mapping for a
font named {\em Dialog}.

\JavaTIP[false]{EFFECTIVE DESIGN}{Font Portability.}{The fact that {\tt Font()}
will produce a font for virtually any set of arguments is important in
ensuring that a Java program will run on any platform.  This is another
example of how Java has been designed for portability.}

The {\tt Component.setFont()} method can be used to assign a specific
font to a button or window or other graphics component.  All AWT and
JFC components have an associated font, which can be accessed using
the {\tt Component.setFont()} and {\tt Component.getFont()} methods.
For example, the following code could be used to override a {\tt
Button}'s font:

\begin{jjjlisting}
\begin{lstlisting}
Button b = new Button("Label");
b.setFont(new Font("Times", Font.ITALIC, 14));
\end{lstlisting}
\end{jjjlisting}

\noindent If 14-point, italic, Times font is not available on the
host system, a substitute will be supplied.

\subsection{Font Metrics}
\noindent To illustrate how to use the {\tt FontMetrics} class, let's write a
%\begin{marginalnote}\it Problem statement\end{marginalnote}
\marginnote{Problem statement}
``Hello World'' application that centers its message both horizontally and
vertically in its window.  The message should be centered
regardless of the size of the application window.  Thus, we will have to
position the text relative to the window size, which is something we
learned in positioning geometric shapes.   The message
should also be centered no matter what font is used.  This will require
us to know certain characteristics of the font itself, such as the
height and width of its characters, whether the characters have a
fixed or variable width, and so on.  In order to get access to
these properties, we will use the {\tt FontMetrics} class.

%\begin{figure}
%\begin{figure}[tb]
\marginfig{chptr07/fontmetrics.eps}{An illustration of the various font
measurements.
\label{fig-fontmetrics}\label{pg-fig-fontmetrics}}
%\end{figure}
%\end{figure}

Figure~7.20 illustrates the various properties that
are associated with a font.   The {\bf baseline}\index{baseline} of a font refers to
the line on which the bottom of most characters occurs.  When drawing a
string, the {\it x-} and {\it y-}coordinates determine the baseline of
the string's first character.  Thus, in

\begin{jjjlisting}
\begin{lstlisting}
g.drawString("Hello World", 10, 40);
\end{lstlisting}
\end{jjjlisting}

\noindent the bottom left of the {\it H} in ``Hello World'' would be
located at (10, 40).

All characters ascend some distance above the baseline.  This is known
as the character's {\bf ascent}\index{ascent}.  Some characters, such as {\it y},
may extend below the baseline, into what's known as the {\it
descent}. Each font has a {\it maximum descent}.  Similarly, some
characters, such as accent characters, may extend above the {\it
maximum ascent} into a space known as the {\it leading}.

The {\it height} of a font is defined as the sum (in pixels) of the
ascent, descent, and leading values.  The height is a property of the
font itself rather than of any individual character.   Except for
fixed-width fonts, in which the width of all characters is the same,
the characters that make up a font have varying widths.  The width of
an individual character is known as its {\it advance}.



The {\tt FontMetrics} class (Fig.~7.21) provides methods
\marginfig{chptr07/fontmetrics2.eps}{The {\tt FontMetrics} class.
\label{pg-fig-fontmetrics2}}
for accessing a font's properties.  These can be useful to control the
layout of text on a GUI.  For example, when drawing multiple lines of
text, the {\tt getHeight()} method is useful for determining how much
space should be left between lines.  When drawing character by
character, the {\tt charWidth()} method can be used to determine how
much space must be left between characters.  Alternatively, the {\tt
stringWidth()} method can be used to determine the number of pixels
required to draw the entire string.

\subsection{Example: Centering a Line of Text}
\noindent Given this background, let's take on the task of centering a message
in an application window.  In order for this application to work for any font, we
must take care not to base its design on characteristics of the
%\begin{marginalnote}\it Algorithm design: Generality\end{marginalnote}
\marginnote{Algorithm design: Generality}
particular font that we happen to be using.   To underscore this point,
let's design it to work for a font named Random, which, as we noted
earlier, will be mapped to some font by the system on which the application
is run.  In other words, we will let the system pick a font for this
application's message.  An interesting experiment would be to run the application
on different platforms to see what fonts are chosen.

The only method we need for this application is the {\tt paint()} method.
Let's begin by setting the font used by the graphics context to a random
font.  To get the characteristics of this font, we create a {\tt FontMetrics}
object and get the font metrics for the font we just created:

\begin{jjjlisting}
\begin{lstlisting}
g.setFont(new Font("Random", Font.BOLD, 24));
FontMetrics metrics = g.getFontMetrics();
\end{lstlisting}
\end{jjjlisting}

\noindent The next step is to determine the {\tt JFrame}'s dimensions
using the {\tt getSize()} method. This method returns an object of
type {\tt Dimension}.  The {\tt java.awt.Dimension} class (Fig.~7.22)
represents the size (width and height) of a GUI component.  A {\tt
Dimension} makes it possible to manipulate an object's width and
height as a single entity.  Note that the {\tt height} and {\tt width}
variables are defined as {\tt public}, which is an exception from the
usual convention of defining instances variables as {\tt private} or
{\tt protected}. The justification for this exception is probably
to simplify the syntax of referring to an object's width and height.
For example, the following syntax can be used to refer to a component's
dimensions:

\begin{jjjlisting}
\begin{lstlisting}
Dimension d = new Dimension(100, 50);
System.out.println("width = " + d.width + 
                   " height = " + d.height);
\end{lstlisting}
\end{jjjlisting}

\noindent Note the redundancy built into the {\tt Dimension} class.  
For example, in addition to being able to set a {\tt Dimension}'s
\marginfig{chptr07/dimension.eps}{The {\tt Dimension} class.
\label{fig-dimension}
\label{pg-fig-dimension}}
instance variables directly, public access methods are provided.
Also, by defining more than one version of some access methods, the
class achieves a higher level of flexibility.  The same can be said
for providing several different constructors, including a copy
constructor.  Finally, note how it overrides the {\tt equals()} and
{\tt toString()} methods.  These are all examples of good
object-oriented design.

\JavaTIP[false]{EFFECTIVE DESIGN}{Redundancy.}{Redundancy is often a desirable
characteristic of object design.  It makes the object easier to use and
more widely applicable.}

The {\tt Dimension} object is used to calculate the {\it x-} and {\it
y-}coordinates for the string. In order to center the string
horizontally, we need to know its width, which is supplied by the {\tt
metrics} object.  If the {\tt JFrame} is {\tt
%\begin{marginalnote}\it Centering text\end{marginalnote}
\marginnote{Centering text}
d.width} pixels wide, then the following expression subtracts the
width of the string from the width of the {\tt JFrame} and then divides the
leftover space in half:

\begin{jjjlisting}
\begin{lstlisting}
                            // Calculate coordinates
int x = (d.width - metrics.stringWidth(str)) / 2; 
\end{lstlisting}
\end{jjjlisting}

\noindent Similarly, the following expression adds the height
of the string to the height of the {\tt JFrame} and divides the leftover
space in half:

\begin{jjjlisting}
\begin{lstlisting}
int y = (d.height + metrics.getHeight()) / 2;
\end{lstlisting}
\end{jjjlisting}

\noindent Taken together, these calculations give the coordinates for the
lower left pixel of the first character in ``Hello World!''  The only
remaining task is to draw the string (Fig.~\ref{fig-centertext}).
Because the {\tt paint()} method is called automatically whenever
the {\tt JFrame} is resized, this application, whose output is shown in
Figure~\ref{fig-centertextscreen}, will re-center its message whenever
it is resized by the user.

\begin{figure}[h!]
\jjjprogstart
\begin{jjjlisting}[28pc]
\begin{lstlisting}
import java.awt.*;
import javax.swing.*;

public class CenterText extends JFrame {
                // Print hello world! in center of frame
public void paint(Graphics g) {
  String str = "Hello World!";
  g.setFont(new Font("Random", Font.PLAIN, 24)); // Random font
  FontMetrics metrics = g.getFontMetrics(); //  And its metrics

  Dimension d = getSize();       // Get the frame's size
             // Clear the frame
  g.setColor(getBackground());
  g.fillRect(0,0,d.width,d.height);
  g.setColor(Color.black);
                                // Calculate coordinates
  int x = (d.width - metrics.stringWidth(str)) / 2; 
  int y = (d.height + metrics.getHeight()) / 2;

  g.drawString( str, x, y );          // Draw the string
} // paint()

public static void main(String args[]) {
  CenterText ct = new CenterText();
  ct.setSize(400,400);
  ct.setVisible(true);
}
} // CenterText
\end{lstlisting}
\end{jjjlisting}
\jjjprogstop{The {\tt CenterText} application.
\label{fig-centertext}\label{pg-fig-centertext}}
\end{figure}

\JavaTIP{PROGRAMMING TIP}{Generality.}{By using a component's size and
font as the determining factors, you can center text on virtually any
component.  These values are available via the component's {\tt
getFont()} and {\tt getSize()} methods.}


\begin{figure}[bt]
\figaleft{chptr07/centertextscreen.eps}{The {\tt CenterText} application keeps its
message centered no matter how its window is resized.
\label{fig-centertextscreen}
\label{pg-fig-centertextscreen}}
\end{figure}

\secSMH{Chapter Summary}
\secKTH{Technical Terms}

\begin{KT}
ascent

baseline

concatenation

copy constructor 

data structure

delimited string

delimiter

empty string

garbage collection 

glyph

lexicographic order

logical font

off-by-one error

orphan object

physical font

read only

string

string index

string literal

token

unit indexed

zero indexed

\end{KT}


\pagebreak
\secKTH{Summary of Important Points}
\begin{SMBL}
\item  A {\tt String\index{String}} literal is a sequence of 0 or more characters
enclosed within double quotation marks.   A {\tt String} object is a
sequence of 0 or more characters, plus a variety of class and instance
methods and variables.

\item A {\tt String} object is created automatically by Java the first
time it encounters a {\it literal string}, such as ``Socrates,'' in a
program.  Subsequent occurrences of the literal do not cause
additional objects to be instantiated.  Instead, every occurrence of
the literal ``Socrates'' refers to the initial object.

\item A {\tt String} object is created whenever the {\tt new}
operator is used in conjunction with a {\tt String()}
constructor---for example, {\tt new String("hello")}.

\item  The {\tt String} concatenation operator is the overloaded
$+$ symbol; it is used to combine two {\tt String}s into a single
{\tt String}: ``hello'' + ``world'' \verb|==>| ``helloworld''.

\item {\tt String}s are indexed starting at 0.  The {\tt indexOf()}
and {\tt last\-Index\-Of()} methods\index{String methods} are used for
finding the first or last occurrence of a character or substring
within a {\tt String}. The {\tt valueOf()} methods convert a nonstring
into a {\tt String}. The {\tt length()} method determines the number
of characters in a {\tt String}.  The {\tt charAt()} method returns
the single character at a particular index position.  The various {\tt
substring()} methods return the substring at particular index
positions in a {\tt String}.

\item The overloaded {\tt equals()} method returns {\tt true} if two
{\tt String}s contain the same exact sequence of characters.  The {\tt
==} operator, when used on {\tt String}s, returns true if two
references designate the same {\tt String} object.

\item {\tt String} objects are {\it immutable}. They cannot be 
modified.

\item  A {\tt StringBuffer} is a string object that can be 
modified using methods such as {\tt insert()} and {\tt append()}.

\item  A {\tt StringTokenizer} is an object that can be used to
break a {\tt String} into a collection of {\it tokens} separated by
{\it delimiters}. The whitespace characters---tabs, blanks, and
newlines---are the default delimiters.

\item  The {\tt FontMetrics} class is used to obtain the
specific dimensions of the the various {\tt Font}s.  It
is useful when you wish to center text.  {\tt Font}s are
inherently platform dependent.  For maximum portability,
it is best to use default fonts.

\end{SMBL}

\secANSH
\begin{ANS}
%{Exercise 7.1}
\columnsep = 0pt\begin{multicols}{4}
\item \mbox{ }
\begin{enumerate}
\item[a.]  silly       
\item[b.]  silly       
\item[c.]  silly stuff 
\end{enumerate}
\end{multicols}

%{Exercise 7.2}
\item  \mbox{ }
\begin{enumerate}
\item[a.]  \verb|String str1 = "";|              
\item[b.]  \verb|String str2 = new String("stop");|
\item[c.]  \verb|String str3 = str1 + str2;|
\end{enumerate}



%{Exercise 7.3}
\columnsep = 0pt\begin{multicols}{4}
\item \mbox{ }
\begin{enumerate}
\item[a.]  15        
\item[b.]  \verb|"551"|
\item[c.]  \verb|"5175"| 
\end{enumerate}
\end{multicols}

%{Exercise 7.4} 
\item  See Figure~\ref{fig-strobjects3}.
\label{ans-strobjects2}
\label{pg-ans-strobjects2}

%\begin{figure}
\begin{figure}[h]
\figa{chptr07/p392f1.eps}{Answer to Exercise 7.4.
Note that {\it s1} is {\tt null} because it has not been instantiated
and has not been assigned a literal value.
\label{fig-strobjects3}
\label{pg-fig-strobjects3}}
\end{figure}
%\end{figure}


%{Exercise 7.5}
\begin{multicols}{4}
\item \mbox{ }
\begin{enumerate}
\item[a.]  \verb|"45"|
\item[b.]  \verb|"121"|
\item[c.]  \verb|"X"|
\end{enumerate}
\end{multicols}


%{Exercise 7.6}
\item  \mbox{ }
\begin{enumerate}
\item[a.]  \verb|String.valueOf(100)|   
\item[b.]  \verb|String.valueOf('V');|  
\item[c.]  \verb|String s = new String(String.valueOf(X * Y));|  
\end{enumerate}


%{Exercise 7.7}
\begin{multicols}{4}
\item \mbox{ }
\begin{enumerate}
\item[a.]  0      
\item[b.]  1      
\item[c.]  $-1$   
\end{enumerate}
\end{multicols}


%{Exercise 7.8}
\begin{multicols}{4}
\item\mbox{ }

\mbox{ }

\mbox{ }
\begin{enumerate}
\item[a.]  16      
\item[b.]  \verb|"16"|
\item[c.]  1       
\item[d.]  15      
\item[e.]  1       
\item[f.]  13      
\item[g.]  7       
\item[h.]  3       
\item[i.]  7       
\item[j.]  7       
\item[k.]  3       
\end{enumerate}
\end{multicols}



%{Exercise 7.9}
\item  Evaluate the following expression:

\begin{jjjlisting}
\begin{lstlisting}
String tricky = "abcdefg01234567";
tricky.indexOf(String.valueOf(tricky.indexOf("c")));
tricky.indexOf(String.valueOf(2));
tricky.indexOf("2");
Answer: 9
\end{lstlisting}
\end{jjjlisting}


%{Exercise 7.10}
\begin{multicols}{4}
\item\mbox{ }

\begin{enumerate}
\item[a.]  \verb|"uvwxyz"|
\item[b.]  \verb|"bcde"|    
\item[c.]  \verb|"xyz"| 
\item[d.]  \verb|"xy"|
\item[e.]  \verb|"xyz"|
\end{enumerate}
\end{multicols}



%{Exercise 7.11}
\begin{multicols}{4}
\item\mbox{ }

\begin{enumerate}
\item[a.]  \verb|"uvwxyz"|
\item[b.]  \verb|"bcde"|
\item[c.]  \verb|"xyz"|
\item[d.]  \verb|"xyz"|
\item[e.]  \verb|"xyz"|
\end{enumerate}
\end{multicols}


%{Exercise 7.12}

\item  A class to test the string methods.

\begin{jjjlisting}[27pc]
\begin{lstlisting}
public class StringProcessorTest {
  public static void main(String[] args) {
    KeyboardReader kb = new KeyboardReader();
    kb.prompt("Input a String or - stop - to quit: ");
    String str = kb.getKeyboardInput();
    while (!str.equals("stop")){
      kb.display("Testing printLetters()\n");
      StringProcessor.printLetters(str);
      kb.display("testing countChars()\n");
      kb.display("Total occurences of e = ");
      kb.display(StringProcessor.countChar(str,'e') + "\n");
      kb.display("Testing reverse()\n");
      kb.display(StringProcessor.reverse(str)+ "\n");
      kb.display("Testing capitalize()\n");
      kb.display(StringProcessor.capitalize(str) + "\n\n");
      kb.prompt("Input a String or - stop - to quit: ");
      str = kb.getKeyboardInput();
    } // while
  } //main()
} // StringProcessorTest class
\end{lstlisting}
\end{jjjlisting}

%{Exercise 7.13}

\item  Method to remove all blanks from a string:
\verb| |

\begin{jjjlisting}
\begin{lstlisting}
 // Pre: s is a non null string
 // Post: s is returned with all its blanks removed
 public String removeBlanks(String s) {
   StringBuffer result = new StringBuffer();
   for (int k = 0; k < s.length();  k++)
     if (s.charAt(k) != ' ')      // If this is not a blank
       result.append(s.charAt(k));  //  append it to result
   return result.toString();
 }
\end{lstlisting}
\end{jjjlisting}


%{Exercise 7.14}
\item  A Alpha Z Zero Zeroes a alpha bath bin z zero


%{Exercise 7.15}

\item  To modify {\tt precedes} so that it also
returns true when its two string arguments are equal, just
change the operator in the final return statement to \verb|<=|:

\begin{jjjlisting}
\begin{lstlisting}
    if (s1.charAt(k) <= s2.charAt(k) )
      return true;
\end{lstlisting}
\end{jjjlisting}


%{Exercise 7.16}
\begin{multicols}{4}
\item\mbox{ }

\mbox{ }

\mbox{ }

\begin{enumerate}
\item[a.]  true  
\item[b.]  true  
\item[c.]  false 
\item[d.]  false 
\item[e.]  false 
\item[f.]  true  
\item[g.]  false 
\item[h.]  false 
\item[i.]  false 
\end{enumerate}
\end{multicols}

%{Exercise 7.17}
\item  The variables in {\tt TestStringEquals} are declared {\tt static}
because they are used in {\tt static} methods.  Whenever you call a
method directly from {\tt main()}, it must be {\tt static} because
{\tt main()} is static.  Remember that {\tt static} elements are associated
with the class, not with its instances.  So {\tt main()} can only
use static elements because they don't depend on the existence
of instances.



%{Exercise 7.18}
\item\mbox{ }
\begin{enumerate}
\item[a.]  \verb|String s3 = s1.substring(s1.indexOf('n'))| \\             
       \verb|+ s1.substring(0,s1.indexOf('n'));|  
\item[b.]  \verb|String s4 = s2.substring(6) + " " + s2.substring(0,5);| 
\item[c.]  \verb|String s5 = s2.substring(0,6) + s1.substring(0,3);|     
\end{enumerate}

\end{ANS}

\secEXRHtwo{Exercises}
\marginnote{\vspace{12pt}\raggedright{\bf Note:} For programming exercises, {\bf first} draw 
a UML class diagram describing all classes and
their inheritance relationships and/or associations.}
\begin{EXRtwo}

\item  Explain the difference between the following pairs of terms:
\begin{EXRtwoLL}
\item  {\it Unit indexing} and {\it zero indexing}.
\item  {\it Data structure} and {\it data type}.
\item  {\tt StringBuffer} and {\tt String}.
\item  {\tt String} and {\tt StringTokenizer}.
\item  {\it Declaring a variable} and {\it instantiating a} {\tt String}.
\item  A {\tt Font} and a {\tt FontMetrics} object.
\end{EXRtwoLL}

\item  Fill in the blanks.

\begin{EXRtwoLL}\baselineskip=14pt
\item  When the first character in a string has index 0, this is known as \rule{40pt}{0.5pt}\,.
\item  A sequence of characters enclosed within quotes is known as a \rule{40pt}{0.5pt}\,.
\end{EXRtwoLL}\baselineskip=11pt

\item Given the {\tt String} {\it str} with the value ``to be or not
to be that is the question,'' write Java expressions to extract each
of the substrings shown below.  For each substring, provide two sets
of answers.  One that uses the actual index numbers of the
substrings---for example, the first ``to'' goes from 0 to 2---and a
second more general solution that will also retrieve the 
substring from the following string ``it is easy to become what you
want to become.''  ({\it Hint}: In the second case, use {\tt length()}
and {\tt indexOf()} along with {\tt substring()} in your expressions.
If necessary, you may use local variables to store intermediate
results.  The answer to (a) is provided as an example.)

\begin{EXRtwoLL}
\item  the first ``to'' in the string
\begin{jjjlisting}
\begin{lstlisting}
str.substring(0, 2)                            // Answer 1
str.substring(
     str.indexOf("to"), str.indexOf("to") + 2) // Answer 2
\end{lstlisting}
\end{jjjlisting}

\begin{multicols}{2}
\item  the last ``to'' in the string
\item  the first ``be'' in the string
\item  the last ``be'' in the string
\item  the first four characters in the string
\item  the last four characters in the string
\end{multicols}
\end{EXRtwoLL}


\begin{minipage}{27pc}
\item Identify the syntax errors in each of the following, assuming
that {\tt s} is the literal string ``exercise'':

\begin{EXRtwoLL}
\begin{multicols}{2}
\item  \verb|s.charAt("hello")|
\item  \verb|s.indexOf(10)|
\item  \verb|s.substring("er")|
\item  \verb|s.lastIndexOf(er)|
\item  \verb|s.length|
\end{multicols}

\end{EXRtwoLL}
\end{minipage}

\item  Evaluate each of the following expressions, assuming that
{\tt s} is the literal string ``exercise'':

\begin{EXRtwoLL}
\begin{multicols}{2}
\item  \verb|s.charAt(5)|
\item  \verb|s.indexOf("er")|
\item  \verb|s.substring(5)|
\item  \verb|s.lastIndexOf('e')|
\item  \verb|s.length()|
\end{multicols}
\end{EXRtwoLL}

\item  Write your own {\tt equalsIgnoreCase()} method using
only other {\tt String} methods.

\item  Write your own {\tt String} equality method without
using {\tt String.} \mbox{\tt equals()}. ({\it Hint}: Modify the {\tt precedes()}
method.)

\item Even though Java's {\tt String} class has a built-in {\tt
toLowerCase()} method, write your own implementation of this method.
It should take a {\tt String} parameter and return a {\tt String}
with all its letters written in lowercase.

\item Write a method that converts its {\tt String} parameter so that
letters are written in blocks five characters long.  For example,
consider the following two versions of the same sentence:

\begin{jjjlisting}
\begin{lstlisting}
Plain :   This is how we would ordinarily write a sentence.
Blocked : Thisi showw ewoul dordi naril ywrit easen tence.
\end{lstlisting}
\end{jjjlisting}

\item  Design and implement an applet that lets the user type a document
into a {\tt TextArea} and then provides the following analysis of the
document: the number of words in the document, the number of
characters in the document,~and the percentage of words that have more
than six letters.

\item  Design and write an applet that searches for single-digit numbers
in a text and changes them to their corresponding words.  For example,
the string ``4 score and 7 years ago'' would be converted into ``four
score and seven years ago''.

\item  A palindrome is a string that is spelled the same way backward
and forward.  For example, {\it mom, dad, radar, 727} and
{\it able was i ere i saw elba} are all examples of palindromes.
Write a Java applet that lets the user type in a word or phrase
and then determines whether the string is a palindrome.

\item Write a maze program that uses a string to store a
representation of the maze.  Write a method that accepts a {\tt
String} parameter and prints a two-dimensional representation of a
maze. For example, the maze shown here, where {\tt O} marks the
entrance and exit can be generated from the following string:

\begin{jjjlisting}
\begin{lstlisting}[basicstyle=\scriptsize]
String: XX_XXXXXXXX__XXX_XXXX_XX____XXX_XX_XX_XXX____X____XXXXXXXX_X
  O
XX XXXXXXX
X  XXX XXX
X XX    XX
X XX XX XX
X    X    O
XXXXXXXX X
\end{lstlisting}
\end{jjjlisting}


\item  Write a method that takes a delimited string to
store a name and address, from which you can print a mailing label.  For example, if
the string contains ``Sam Penn:14 Bridge St.:Hoboken, NJ 01881,''
the method should print the label shown in the margin.
\marginnote{\tt
\rule{8pc}{1pt}\\
Sam Penn\\
14 Bridge St.\\
Hoboken, NJ 01881\\[-4pt]
\rule{8pc}{1pt}\\
}

\item  Design and implement an applet that plays Time Bomb with the user.
Here's how the game works.  The computer picks a secret word and then
prints one asterisk for each letter in the word: * * * * *.  The user
guesses at the letters in the word.   For every correct guess, an
asterisk is replaced by a letter:\break * e * * *.  For every incorrect
guess, the time bomb's fuse grows shorter.  When the fuse disappears,
after say, six incorrect guesses, the bomb explodes.  Store the secret
words in a delimited string and invent your own representation for
the time bomb.

\item {\bf Challenge: } The global replace function is a
string-processing algorithm found in every word processor.  Write a
method that takes three {\tt String} arguments: a document, a target
string, and a replacement string.  The method should replace every
occurrence of the target string in the document with the replacement
string.  For example, if the document is ``To be or not to be, that is
the question,'' and the target string is ``be,'', and the replacement
string is ``see,'' the result should be, ``To see or not to see, that
is the question.''

\item  {\bf Challenge: } Design and implement an applet that plays
the following game with the user.  Let the user pick a letter between
{\it A} and {\it Z}. Then let the computer guess, the secret letter.  For
every guess the player has to tell the computer whether it's too
high or too low.  The computer should be able to guess the letter
within five guesses.  Do you see why?

\item  {\bf Challenge:} A {\it list} is a sequential
data structure.  Design a {\tt List} class that uses a comma-delimited
{\tt String}---such as, ``a,b,c,d,12,dog''---to implement a list.
Implement the following methods for this class:

\begin{jjjlisting}
\begin{lstlisting}
void addItem( Object o );      // Use Object.toString()
String getItem(int position);
String toString();
void deleteItem(int position);
void deleteItem(String item);
int getPosition(String item);
String getHead();              // First element
List getTail();                // All but the first element
int length();                  // Number of items
\end{lstlisting}
\end{jjjlisting}

\item {\bf Challenge:} Use a delimited string to create a {\tt
PhoneList} class with an instance method to insert names and phone
numbers, and a method to look up a phone number when a user provides a
person's name.  Since your class will take care of looking things up,
you don't have to worry about keeping the list in alphabetical order.
For example, the following string could be used as such a directory:

\begin{jjjlistingleft}[30pc]{-4pc}
\begin{lstlisting}
mom:860-192-9876::bill g:654-0987-1234::mary lancelot:123-842-1100
\end{lstlisting}
\end{jjjlistingleft}

\item  Design and implement an applet or application that
displays a multi-line message in various fonts and sizes input by the
user.   Let the user choose from among a fixed selection of fonts,
sizes, and styles.




\end{EXRtwo}

%\end{objectorienteddesign}
