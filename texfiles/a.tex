\setcounter{table}{0}
\setcounter{figure}{0}
\renewcommand{\thetable}{\mbox{A.\arabic{table}}}%
\renewcommand{\thefigure}{\mbox{A--\arabic{figure}}}%
%
%%\appendixleft{A}{Coding Conventions}


\chapter{Coding Conventions}
\markboth{{\color{cyan}APPENDIX\,A\,\,$\bullet$\,\,}Coding Conventions}
{{\color{cyan}APPENDIX\,A\,\,$\bullet$\,\,}Coding Conventions}

\label{appendix-codestyle}



\noindent This appendix covers various aspects of programming style and coding
conventions.  It follows the conventions suggested in the Java \mbox{Language}
Specification ({\tt http://java.sun.com/docs/books/jls/}), which is
summarized on Sun's Java Web site ({\tt http://java.sun.com/docs/}).
The conventions have been modified somewhat to fit the needs of an
academic programming course.   For further details see


\WWWleft
\begin{jjjlisting}
\begin{lstlisting}[commentstyle=\color{black}]
http://java.sun.com/docs/codeconv/index.html
\end{lstlisting}
\end{jjjlisting}

Coding conventions improve the readability and maintainability of the
code.  Because maintenance is often done by programmers who did not
have a hand in designing or writing the original code, it is important
that the code follow certain conventions.  For a typical piece of
commercial software, much more time and expense are invested in
maintaining the code than in creating the code.

\section*{Comments}
\noindent Java recognizes two types of comments: {\it C-style} comments use the
same syntax found in C and C++. They are delimited by
\verb!/* ... */! and \verb!//!. The first set of delimiters is used
to delimit a multiline comment.  The Java compiler
will ignore all text that occurs between \verb|/*| and \verb|*/|.  The
second set of delimiters is used for a single-line comment.  Java will
ignore all the code on the rest of the line following a double slash
(\verb|//|).  C-style comments are called {\it implementation comments}
and are mainly used to describe the implementation of your code.

{\it Documentation} comments are particular to Java.  They are
delimited by \verb|/** ... */|.  These are used mainly to describe the
specification or design of the code rather than its
implementation.  When a file containing documentation comments is
processed by the {\it javadoc} tool that comes with the Java
Development Kit (JDK), the documentation comments will be incorporated
into an HTML document.  This is how online documentation has been
created for the Java library classes.

\subsection*{Implementation Commenting Guidelines}
\noindent Implementation (C-style) comments should be used to provide an
overview of the code and to provide information that is not easily
discernible from the code itself.  They should not be used as a
substitute for poorly written or poorly designed code.


In general, comments should be used to improve the readability of the
code.  Of course, readability depends on the intended audience.   Code
that's easily readable by an expert programmer may be completely
indecipherable to a novice.  Our commenting guidelines are aimed at
someone who is just learning to program in Java.

\subsection*{Block Comments}
\noindent A {\it block comment} or {\it comment block} is a multiline
comment that is used to describe files, methods, data structures,
and algorithms:

\begin{jjjlisting}
\begin{lstlisting}
 /*
  * Multiline comment block
  */
\end{lstlisting}
\end{jjjlisting}

\subsection*{Single-Line Comments}
\noindent A single-line comment can be delimited either by \verb|//| or by
\verb|/* ... */|. The  \verb|//| is also used to {\it comment out}
a line of code that you want to skip during a particular run.  The
following example illustrates these uses:

\begin{jjjlisting}
\begin{lstlisting}
 /* Single line comment */
 System.out.println("Hello");    // End of line comment
 // System.out.println("Goodbye");
\end{lstlisting}
\end{jjjlisting}

\noindent Note that the third line is commented out and
would be ignored by the Java compiler.

In this text, we generally use slashes for single-line and end-of-line
comments.  And we frequently use end-of-line comments to serve as a
running commentary on the code itself.  These types of comments serve a
pedagogical purpose---to teach you how the code works.  In a
``production environment'' it would be unusual to find this kind of
running commentary.

\subsection*{Java Documentation Comments}
\noindent Java's online documentation has been generated by the {\tt javadoc}
tool that comes with the Java Development Kit (JDK). To conserve
space, we use documentation comments only sparingly in the programs
listed in this textbook itself.  However, {\tt javadoc} comments are
used more extensively to document the online source code that accompanies
the textbook.

Documentation comments are placed before classes, interfaces,
constructors, methods, and fields.  They generally take the
following form:

\begin{jjjlisting}
\begin{lstlisting}
 /**
  * The Example class blah blah
  * @author J. Programmer
  */
public class Example { ...
\end{lstlisting}
\end{jjjlisting}

\noindent Note how the class definition is aligned with the beginning of the
comment.  {\tt Javadoc} comments use special tags, such as {\it author} and
{\it param}, to identify certain elements of the documentation.  For details
on {\tt javadoc}, see

\begin{jjjlisting}
\begin{lstlisting}[commentstyle=\color{black}]
http://java.sun.com/j2se/1.5.0/docs/tooldocs/
\end{lstlisting}
\end{jjjlisting}

\section*{Indentation and White Space}
\noindent The use of indentation and white space helps to improve the
readability of the program.   {\it White space} refers to the use of
blank lines and blank space in a program.  It should be used to
separate one program element from another, with the goal being to draw
attention to the important elements of the program.

\begin{BL}
\item  Use a blank line to separate method definitions and to
separate a class's instance variables from its methods.
\item  Use blank spaces within expressions and statements to
enhance their readability.
\item  Be consistent in the way you use white space in your program.
\end{BL}

Code should be indented in a way that shows the logical structure of
the program.  You should use a consistent number of spaces as the size
of the indentation tab.   The Java Language Specification recommends four
spaces.

In general, indentation should represent the {\it contained in}
relationships within the program.  For example, a class definition
contains declarations for instance variables and definitions of
methods.  The declarations and definitions should be indented by the
same amount throughout the class definition.  The statements contained
in the body of a method definition should be indented:

\begin{jjjlisting}
\begin{lstlisting}
public void instanceMethod() {
    System.out.println("Hello");
    return;
}
\end{lstlisting}
\end{jjjlisting}

\noindent An if statement
contains an if clause and an else clause, which should be indented:

\begin{jjjlisting}
\begin{lstlisting}
if (condition)
    System.out.println("If part");   // If clause
else
    System.out.println("Else part"); // Else clause
\end{lstlisting}
\end{jjjlisting}

\noindent The statements contained in the body of a loop should be indented:

\begin{jjjlisting}
\begin{lstlisting}
for (int k = 0; k < 100; k++) {
    System.out.println("Hello " + 'k'); // Loop body
}
\end{lstlisting}
\end{jjjlisting}

Finally, indentation should be used whenever a statement or expression
is too long to fit on a single line.  Generally, lines should be no
longer than 80 characters.

\section*{Naming Conventions}
\noindent The choice of identifiers for various elements within a program
can help improve the readability of the program.  Identifiers should
be descriptive of the element's purpose.  The name of class should
be descriptive of the class's role or function.  The name of a
method should be descriptive of what the method does.

\spstrict The way names are spelled can also help improve a program's readability.
Table~A.1 summarizes the various conventions recommended
by the Java Language Specification and followed by professional Java
programmers.\spnormalstr


\begin{table}[htb]
%\hphantom{\caption{Naming rules for Java identifiers.}}
\TBT{0pc}{Naming rules for Java identifiers.}
\hspace*{-6pt}\begin{tabular}{lll}
\multicolumn{3}{l}{\color{cyan}\rule{30pc}{1pt}}\\[2pt]
%%%%\TBCH{{\bf Identifier Type}} & \TBCH{{\bf Naming Rule}} & \TBCH{{\bf Example}}
{\bf Identifier Type} & {\bf Naming Rule} & {\bf Example}
\\[-4pt]\multicolumn{3}{l}{\color{cyan}\rule{30pc}{0.5pt}}\\[2pt]
Class&Nouns in mixed case with the first&OneRowNim\cr
&letter of each internal word capitalized.&TextField\\[6pt]
Interfaces&Same as class names. Many interface names&Drawable\cr
&end with the suffix {\it able}.&ActionListener\\[6pt]
Method&Verbs in mixed case with the first letter in&actionPerformed()\cr
&lowercase and the first letter of internal&sleep()\cr
&words capitalized.&insertAtFront()\\[6pt]
Instance Variables&Same as method names. The name should&maxWidth\cr
&be descriptive of how the variable is used.&isVisible\\[6pt]
Constants&Constants should be written in uppercase with&MAX\_LENGTH\cr
&internal words separated by \_.&XREF\\[6pt]
Loop Variables&Temporary variables, such as loop variables,&int k;\cr
&may have single character names: i, j, k.&int i;
\\[-4pt]\multicolumn{3}{l}{\color{cyan}\rule{30pc}{1pt}}
\end{tabular}
\endTB
\end{table}


\section*{Use of Braces}
\noindent Curly braces \verb|{ }| are used to mark the beginning and end of a
block of code.  They are used to demarcate a class body, a method body,
or simply to combine a sequence of statements into a single code
block.  There are two conventional ways to align braces and we have
used both in the text.  The opening and closing brace may be aligned in
the same column with the enclosed statements indented:

\begin{jjjlisting}
\begin{lstlisting}
public void sayHello()
{
    System.out.println("Hello");
}
\end{lstlisting}
\end{jjjlisting}

\noindent This is the style that's used in the first part of the
book, because it's easier for someone just learning the syntax
to check that the braces match up.

Alternatively, the opening brace may be put at the end of the line
where the code block begins, with the closing brace aligned under the
beginning of the line where the code block begins:

\begin{jjjlisting}
\begin{lstlisting}
public void sayHello() {
   System.out.println("Hello");
}
\end{lstlisting}
\end{jjjlisting}

\noindent This is the style that's used in the last two parts of the
book, and it seems the style preferred by professional Java programmers.

Sometimes even with proper indentation, it it difficult
to tell which closing brace goes with which opening brace.  In those
cases, you should put an end-of-line comment to indicate what the brace
closes:

\begin{jjjlisting}
\begin{lstlisting}
public void sayHello() {
    for (int k=0; k < 10; k++) {
        System.out.println("Hello");
    } //for loop
} // sayHello()
\end{lstlisting}
\end{jjjlisting}

\section*{File Names and Layout}
\noindent Java source files should have the {\tt .java} suffix, and Java
bytecode files should have the {\tt .class} suffix.

A Java source file can only contain a single {\tt public}
class.  Private classes and interfaces associated with a public class
can be included in the same file.


\subsection*{Source File Organization Layout}
\noindent All source files should begin with a comment block that contains
important identifying information about the program, such as the name
of the file, author, date, copyright information, and a brief
description of the classes in the file.  In the professional software
world, the details of this ``boilerplate'' comment will vary from one
software house to another.  For the purposes of an academic computing
course, the following type of comment block would be appropriate:

\begin{jjjlisting}
\begin{lstlisting}
 /*
  * Filename: Example.java
  * Author: J. Programmer
  * Date:  April, 20 1999
  * Description: This program illustrates basic 
  *  coding conventions.
  */
\end{lstlisting}
\end{jjjlisting}

\noindent The beginning comment block should be followed by any package
and import statements used by the program:

\begin{jjjlisting}
\begin{lstlisting}
package java.mypackage;
import java.awt.*;
\end{lstlisting}
\end{jjjlisting}

\noindent The {\it package} statement should only be used if the code
in the file belongs to the package.  None of the examples in this book
use the package statement.  The {\it import} statement allows you to
use abbreviated names to refer to the library classes used in your
program.   For example, in a program that imports {\tt java.awt.*} we
can refer to the {\tt java.awt.Button} class as simply {\tt
Button}. If the import statement were omitted, we would have to use the
fully qualified name.

The {\tt import} statements should be followed by the class definitions
contained in the file.  Figure~\ref{fig-example1} illustrates how a
simple Java source file should be formatted and documented.

\begin{figure}[tbhp]
\jjjprogstart
\begin{jjjlisting}
\begin{lstlisting}
/*
 * Filename: Example.java
 * Author: J. Programmer
 * Date:  April, 20 1999
 * Description: This program illustrates basic 
 *     coding conventions.
 */

import java.awt.*;

/**
 * The Example class is an example of a simple 
 *   class definition.
 * @author J. Programmer
 */
public class Example {

  /** Doc comment for instance variable, var1 */
  public int var1;

  /**
   * Constructor method documentat comment describes
   *   what the constructor does.
   */
  public Example () {
   // ... method implementation goes here
  }

  /**
   *  An instanceMethod() documentation comment describes
   *   what the method does.
   *  @param N is a parameter than ....
   *  @return This method returns blah blah
   */
  public int instanceMethod( int N ) {
   // ... method implementation goes here}
  }
} // Example
\end{lstlisting}
\end{jjjlisting}
\jjjprogstop{A sample Java source file.}
{fig-example1}
\end{figure}


\section*{Statements}
\subsection*{Declarations}
\noindent There are two kinds of declaration statements: field declarations,
which include a class's instance variables, and local variable
declarations.

\begin{BL}
\item  Put one statement per line, possibly followed by an
end-of-line comment if the declaration needs explanation.

\item  Initialize local variables when they are declared.  Instance
variables are given default initializations by Java.

\item  Place variable declarations at the beginning of code blocks in which
they are used rather than interspersing them throughout the code
block.
\end{BL}




\noindent The following class definition illustrates these points:

\begin{jjjlisting}
\begin{lstlisting}
public class Example {
   private int size = 0;     // Window length and width
   private int area = 0;     // Window's current area

   public void myMethod() {
      int mouseX = 0;        // Beginning of method block

      if (condition) {
          int mouseY = 0;    // Beginning of if block
      ...
      } // if
   } // myMethod()
} // Example
\end{lstlisting}
\end{jjjlisting}


\section*{Executable Statements}
\noindent Simple statements, such as assignment statements, should be written
one per line and should be aligned with the other statements in the
block.   Compound statements are those that contain other
statements.  Examples would include if statements, for statements,
while statements, and do-while statements.  Compound statements should
use braces and appropriate indentation to highlight the statement's
structure.  Here are some examples of how to code several kinds of
compound statements:

\begin{jjjlisting}[27pc]
\begin{lstlisting}
 if (condition) {        // A simple if statement
     statement1;
     statement2;
 } // if
 if (condition1) {       // An if-else statement
     statement1;
 } else if (condition2) {
     statement2;
     statement3;
 } else {
     statement4;
     statement5;
 } // if/else
for (initializer; entry-condition; updater) { // For loop
    statement1;
    statement2;
} // for
while (condition) {          // While statement
    statement1;
    statement2;
} // while
do {                        // Do-while statement
    statement1;
    statement2;
} while (condition);
\end{lstlisting}
\end{jjjlisting}

\section*{Preconditions and Postconditions}
\noindent A good way to design and document loops and methods is to specify
their preconditions and postconditions.  A {\it precondition} is a
condition that must be true before the method (or loop) starts.  A {\it
postcondition} is a condition that must be true after the method (or
loop) completes.   Although the conditions can be represented 
formally---using boolean expressions---this is not necessary.  It suffices
to give a clear and concise statement of the essential facts before
and after the method (or loop).

Chapter 6 introduces the use of preconditions and postconditions and
Chapters 6 through 8 provide numerous examples of how to use them.  It
may be helpful to reread some of those examples and model your
documentation after the examples shown there.


\section*{Sample Programs}
\noindent For specific examples of well-documented programs used in
the text, see the online source code that is available on the
accompanying Web site at

\begin{jjjlisting}
\begin{lstlisting}[commentstyle=\color{black}]
http://www.prenhall.com/morelli
\end{lstlisting}
\end{jjjlisting}
%
