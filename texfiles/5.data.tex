%%%  Chapter 5: Java Data and Operators
%%%  3rd Edition

\setcounter{chapter}{4}
\setcounter{SSTUDYcount}{1}
\chapter{Java Data and Operators}
\label{chapter-models}


\CObegin
\secCOBH{Objectives}

\noindent After studying this chapter, you will
\begin{COBL}
\item  Understand the role that data play in effective program design.
\item  Be able to use all of Java's primitive types and their operators.
\item  Appreciate the importance of information hiding.
\item  Be able to use class constants and class methods.
\item  Know how to use Java's {\tt Math} and {\tt NumberFormat} classes.
\item  Be able to perform various kinds of data conversions.
\end{COBL}


\secCOLH{Outline}
\begin{COL}
\item Introduction
\item Boolean Data and Operators
\item[] {{\color{cyan}Special Topic:} Are We Computers?}
\item Numeric Data and Operators
\item From the Java Library: {\tt java.lang.Math}
\item Numeric Processing Examples
%%%%\item Object-Oriented Design: Information Hiding
\item From the Java Library: {\tt java.text.NumberFormat}
\item Character Data and Operators
\item Example: Character Conversions
\item Problem Solving  = Representation  + Action
\par\small\item[] Chapter Summary
\par\small\item[] Solutions to Self-Study Exercises
\par\small\item[] Exercises
\end{COL}
\COend

\section{Introduction}
\noindent This chapter has two primary goals.  One is to elaborate on Java's
{\it primitive data types}, which were first introduced in Chapter~1.
We will cover boolean, integer, character, and real number data types,
including the various operations that you can perform on these types.
We will provide examples, including several modifications of the {\tt
OneRowNim} class, to show typical uses of the various data types.

Our second goal is to illustrate the idea that programming is a matter
of choosing an appropriate way to represent a problem as well as
choosing an appropriate sequence of actions to solve the problem.
Programming is a form of problem solving that can be viewed as a
two-part process: {\it representation} and {\it action}.

{\bf Representation} means finding a way to look at the problem.  This
might involve seeing the problem as closely related to a known problem
or seeing that parts of the problem can be broken up into smaller
problems that you already know how to solve.  In terms of programming
problems, representation often means choosing the right kinds of
objects and structures.

{\bf Action} is the process of taking well-defined steps to solve a
problem.  Given a particular way of representing the problem, what
steps must we take to arrive at its solution?

Choosing an appropriate representation is often the key to solving a
problem.  For example, consider this problem: Can a chess board, with
its top-left and bottom-right squares removed, be completely tiled by
dominoes that cover two squares at a time?

One way to solve this problem might be to represent the chess board and
dominoes as shown in Figure~\ref{fig-chessboard}. If we represent the
board in this way, then the actions needed to arrive at a solution
involve searching for a tiling that completely covers the board.  In
other words, we can try one way of placing the dominoes on the board.
If that doesn't work, we try another way.  And so on.  This process
will be very time consuming, because there are millions of different
ways of trying to tile the board.

%\begin{figure}
\begin{figure}[tb]
\figaleft{chptr05/5f1.eps}{Can the chess board be tiled with dominoes?
} {fig-chessboard}
\end{figure}
%\end{figure}

An alternative way to represent this problem comes from seeing that
the top-left and bottom-right squares of the board are both white.  If
you remove them, you'll have a board with 62 squares, 32 black and 30
white.  Because each domino must cover one white and one black square,
it is impossible to tile a board with an unequal number of black and
white squares.

Thus, by representing the problem as the total number of black and
white squares, the actions required to solve it involve a very simple
reasoning process.  This representation makes it almost trivial to find
the solution.  On the other hand, the {\it brute force} representation
presented first---trying all possible combinations---made it almost
impossible to solve the problem.


\section{Boolean Data and Operators}
\noindent As we learned in Chapter~1, the {\tt boolean} type is one 
of Java's primitive types.  For this type, there are only
two possible values, {\tt true} and {\tt false}.  The {\tt boolean}
type is derived from the work of British mathematician George Boole\index{Boole, George}, 
\marginnote{George Boole}
who in the 1850s, developed an algebra to process
logical expressions such as {\it p and q}. Such {\it boolean
expressions} produce a value that is either {\it true} or {\it false}.
Every modern programming language provides some means of representing
boolean expressions.

The {\tt boolean} type has several important uses.  As we 
saw in Chapter~1, expressions of the form {\tt num == 7} and
{\tt 5 < 7}  have {\tt boolean} values.  Similarly, as we saw in Chapter~3,
\marginnote{Conditional statement}
the {\tt boolean} type is also used to represent the condition in the
if statement:

%\begin{showoff}
$$
\matrix{
\hbox{\tt if}
& ( \hbox{\it boolean expression} )\hfill\cr
& \hbox{\it statement} ;\hfill\cr
}
$$
%\end{showoff}

\noindent For this reason, boolean expressions are also
called {\it conditions}.  Along these same lines, a {\tt boolean}
variable can be used as a {\it flag} or a {\it signal} to ``remember''
\marginnote{Boolean flag}
whether or not a certain condition holds.  For example, in the
following code fragment, we use {\tt isDone} to mark when a particular
process is completed:

\begin{jjjlisting}
\begin{lstlisting}
boolean isDone = false; // Initialize the flag
...                     // Do some processing task
isDone = true;          // Set flag when the task done
...                     // Do some other stuff
if (isDone)             // Check if finished the task
...                     //  If so, do something
else
...                     //  Or, do something else
\end{lstlisting}
\end{jjjlisting}

\subsection{Boolean (or Logical) Operations}
\noindent Like all the other simple data types, the {\tt boolean} type consists
%\begin{marginalnote}\it Data and operations\end{marginalnote}
\marginnote{Data and operations}
of certain data---the values {\tt true} and {\tt false}---and certain
actions or operations that can be performed on those data.  For the
boolean type there are four basic operations: AND (signified by \&\&),
OR (signified by $\mid\mid$), EXCLUSIVE-OR (signified by $\wedge$),
and NOT (signified by !).  These are defined in the {\em truth table}
shown in Table~5.1. A truth tables defines boolean operators by giving
their values in all possible situations. The first two columns of the
table give possible boolean values for two {\it operands}, {\tt o1}
and {\tt o2}. An {\bf operand} is a value used in an operation. Note
that each row gives a different value assignment to the two operands,
so that all possible assignments are represented. The remaining
columns give the values that result for the various operators given
the assignment of values to {\tt o1} and {\tt o2}.

\begin{table}[htb]
%\hphantom{\caption{Definitions of the boolean operators:  AND (\&\&),
%OR ($\mid\mid$), EXCLUSIVE-OR ($\wedge$), and  NOT (!)}}
\TBT{3.0pc}{Truth-table definitions of the boolean operators: \hfill\break
\hspace*{3.0pc}AND (\&\&), OR ($\mid\mid$), EXCLUSIVE-OR ($\wedge$), and  NOT (!)}
\hspace*{2.5pc}\begin{tabular}{llllll} 
\multicolumn{6}{l}{\color{cyan}\rule{21pc}{1pt}}\\[2pt]
%%%RAM\TBCH{o1}&\TBCH{o2}&\TBCH{o1 \&\& o2}&\TBCH{o1 $\mid\mid$ o2}&
%%%RAM\TBCH{o1 $\wedge$ o2}&\TBCH{!o1}
{o1}&{o2}&{o1 \&\& o2}&{o1 $\mid\mid$ o2}&{o1 $\wedge$ o2}&{!o1}
\\[-4pt]\multicolumn{6}{l}{\color{cyan}\rule{21pc}{0.5pt}}\\[2pt]
{\tt true}&{\tt true}&{\tt true}&{\tt true}&{\tt false}&{\tt false}\cr
{\tt true}&{\tt false}&{\tt false}&{\tt true}&{\tt true}&{\tt false}\cr
{\tt false}&{\tt true}&{\tt false}&{\tt true}&{\tt true}&{\tt true}\cr
{\tt false}&{\tt false}&{\tt false}&{\tt false}&{\tt false}&{\tt true}
\\[-4pt]\multicolumn{6}{l}{\color{cyan}\rule{21pc}{1pt}}
\end{tabular}
\endTB
\end{table}


To see how to read this table, let's look at the AND operation, which
is defined in column 3. The AND operator is a {\bf binary
operator}---that is, it requires two operands,
%\begin{marginalnote}\it Binary operator\end{marginalnote}
\marginnote{Binary operator}
{\it o1} and {\it o2}. If both {\it o1} and {\it o2} are true, then
({\it o1} \&\& {\it o2}) is true  (row1).  If either {\it o1} or {\it o2} or
both {\it o1} and {\it o2} are false, then the expression ({\it o1}
\&\& {\it o2}) is false  (rows 2 and 3).  The only case in which ({\it o1} \&\& {\it
o2}) is true is when both {\it o1} and {\it o2} are true (row 4).

The boolean OR operation (column 4 of Table~5.1) is also a binary
operation.  If both {\it o1} and {\it o2} are false, then $ (o1
\mid\mid o2)$ is false (row 4).  If either {\it o1} or {\it o2} or
both {\it o1} and {\it o2} are true, then the expression $(o1 \mid\mid
o2)$ is true (rows 1-3).  Thus, the only case in which $(o1 \mid\mid
o2)$ is false is when both {\it o1} and {\it o2} are false.

The boolean EXCLUSIVE-OR operation (column 5 of Table~5.1) is a binary
operation, which differs from the OR operator in that it is true when
either {\it o1} or {\it o2} is true (rows 2 and 3), but it is false
when both {\it o1} and {\it o2} are true (row 1).

The NOT operation (the last column of Table~5.1) is a
%\begin{marginalnote}\it Unary operator\end{marginalnote}
\marginnote{Unary operator}
{\bf unary operator}---it takes only one operand---and it simply
reverses the truth value of its operand.  Thus, if {\it o1} is true,
!{\it o1} is false, and vice versa.


%% \secEXRHone{Self-Study Exercises}
%% \noindent Click on the links below to test your understanding of the
%% boolean operators:

%% \begin{itemize}
%% \item http://localhost:8900/jjj/boolean-ops.html
%% \end{itemize}



\subsection{Precedence and Associativity}
\noindent In order to evaluate complex boolean expressions, it is necessary 
to understand the order in which boolean operations are carried out by
the computer.  For example, what is the value of the following
expression?

\begin{jjjlisting}
\begin{lstlisting}
true || true && false
\end{lstlisting}
\end{jjjlisting}

\noindent The value of this expression depends on whether we evaluate
the $\mid\mid$ first or the \&\& first.  If we evaluate the $\mid\mid$
first, the expression's value will be false; if we evaluate the \&\&
first, the expression's value will be true.  In the following example,
we use parentheses to force one operation to be done before the
other:

\begin{jjjlisting}
\begin{lstlisting}
EXPRESSION                EVALUATION
----------                ----------
( true || true ) && false ==> true && false ==> false
true || ( true && false ) ==> true || false ==> true
\end{lstlisting}
\end{jjjlisting}

\noindent As these evaluations show, we can use parentheses to
force one operator or the other to be evaluated first.  However, in
\marginnote{Parentheses supersede}
Java, the \&\& \mbox{operator} has higher precedence\index{precedence}
than the $\mid\mid$ operator.  Therefore, the second alternative
corresponds to the default interpretation that Java would apply to the
expression that has no parentheses.  In other words, given the
expression $true \; \mid\mid \; true \; \&\& \; false$, the AND
operation would be evaluated before the OR operation even though the
OR operator occurs first (i.e., to the left) in the unparenthesized
expression.

\begin{table}[htb]
%\hphantom{\caption{Precedence order of the boolean operators}}
\TBT{4pc}{Precedence order of the boolean operators}
\hspace*{3.5pc}\begin{tabular}{cll} 
\multicolumn{3}{l}{\color{cyan}\rule{19pc}{1pt}}\\[2pt]
%%%RAM\TBCH{Precedence Order}\hspace*{12pt}&\TBCH{Operator}\hspace*{12pt}&\TBCH{Operation}
{Precedence Order}\hspace*{12pt}&{Operator}\hspace*{12pt}&{Operation}
\\[-4pt]\multicolumn{3}{l}{\color{cyan}\rule{19pc}{0.5pt}}\\[2pt]
1&{\tt (  )}&Parentheses\\
2&{\tt !}&NOT\\
3&{\tt $\wedge$}&EXCLUSIVE-OR\\
4&{\tt \&\&}&AND\\
5&{\tt ||}&OR
\\[-4pt]\multicolumn{3}{l}{\color{cyan}\rule{19pc}{1pt}}
\end{tabular}
\endTB
\end{table}

As this example illustrates, the boolean operators have a built-in
{\bf precedence order} which is used to determine how boolean
expressions are to be evaluated (Table~5.2).  A simple method for
evaluating an expression is to parenthesize the expression and then
evaluate it.  For example, to evaluate the complex expression

\begin{jjjlisting}
\begin{lstlisting}
true || !false ^ false && true
\end{lstlisting}
\end{jjjlisting}

\noindent we would first parenthesize it according to the
precedence rules set out in Table~5.2,
which gives the following expression:

\begin{jjjlisting}
\begin{lstlisting}
true || (((!false) ^ false) && true)
\end{lstlisting}
\end{jjjlisting}

\noindent We can then evaluate this fully parenthesized
expression, step by step, starting at the innermost
parentheses:

\begin{jjjlisting}
\begin{lstlisting}
Step 1. true || ((true ^ false) && true)
Step 2. true || (true && true)
Step 3. true || true
Step 4. true
\end{lstlisting}
\end{jjjlisting}

\JavaTIP[false]{PROGRAMMING TIP}{Parentheses.}{Parentheses can (and should) 
be used to clarify any expression that appears ambiguous or to
override Java's default precedence rules.}

In addition to operator precedence, it is necessary to know about
an\break operator's {\it associativity} in order to evaluate boolean
expressions of the form ${\tt (op1 \mid\mid op2
\mid\mid op3)}$. Should this expression be evaluated as ${\tt ((op1
\mid\mid op2) \mid\mid op3)}$ or as ${\tt (op1 \mid\mid (op2 \mid\mid
op3))}$?  The binary boolean operators all associate from left to
right.  Thus, the  expressions

\begin{jjjlisting}
\begin{lstlisting}
true ^ true ^ true   // Same as: (true ^ true) ^ true
true && true && true // Same as: (true && true) && true
true || true || true // Same as: (true || true) || true
\end{lstlisting}
\end{jjjlisting}

\noindent would be evaluated as follows:

\begin{jjjlisting}
\begin{lstlisting}
EXPRESSION               EVALUATION
----------------         -----------------
(true ^ true)  ^ true    ==> false ^ true  ==> true
(true && true)  && true  ==> true  && true ==> true
(true || true)  || true  ==> true  || true ==> true
\end{lstlisting}
\end{jjjlisting}

\subsection{Short-Circuit Evaluation}

\noindent Another important feature of the boolean operators
is that they utilize a form of evaluation known as {\em short-circuit
evaluation}. In {\bf short-circuit evaluation}, a boolean expression is
evaluated from left to right, and the evaluation is discontinued as
soon as the expression's value can be determined, regardless of
whether it contains additional operators and operands. For example,
in the expression

\begin{jjjlisting}
\begin{lstlisting}
  expr1 && expr2
\end{lstlisting}
\end{jjjlisting}

\noindent if {\tt expr1} is false, then the AND expression must be
false, so {\tt expr2} need not evaluated.  Similarly, in the expression

\begin{jjjlisting}
\begin{lstlisting}
 expr1 || expr2
\end{lstlisting}
\end{jjjlisting}

\noindent if {\tt expr1} is true, then the OR expression must be true, so
{\tt expr2} need not evaluated.

In addition to being a more efficient form of evaluating boolean
expressions, short-circuit evaluation has some practical uses. For
example, we can use short-circuit evaluation to guard against null
pointer exceptions.  Recall from Chapter~2 that a null pointer
exception results when you try to use an uninstantiated reference
variable---that is, a reference variable that has not been assigned an
object. For example, if we declare a {\tt OneRowNim} variable without
instantiating it and then try to use it, a null pointer exception will
result:

\begin{jjjlisting}
\begin{lstlisting}
OneRowNim game;        // Uninstantiated Reference
if (!game.gameOver())  // Null pointer exception
    game.takeSticks(num);
\end{lstlisting}
\end{jjjlisting}

\noindent In this code, a null pointer exception results when we
use {\tt game} in the method call {\tt game.gameOver()}.  We can use
short-circuit evaluation to prevent the exception from occurring:

\begin{jjjlisting}
\begin{lstlisting}
if ((game != null) && (!game.gameOver())
    game.takeSticks(num);
\end{lstlisting}
\end{jjjlisting}

\noindent In this case, because {\tt game != null} is false,
neither method call involving {\tt game} is made, thus avoiding the
exception.

%% \secEXRHone{Self-Study Exercise}
%% %%\begin{SSTUDY}
%% \begin{SSTUDY}
%% \item  Suppose the following variable declarations are made:

%% \begin{jjjlisting}
%% \begin{lstlisting}
%% boolean A = true, B = true, C = true;
%% boolean X = false, Y = false, Z = false;
%% \end{lstlisting}
%% \end{jjjlisting}

%% \noindent Given these declarations, evaluate each of the following
%% expressions  (don't forget to take operator precedence and
%% associativity into account):

%% \begin{EXRLL}
%% %%%\begin{jjjlisting}
%% %%%\begin{lstlisting}
%% \begin{multicols}{2}
%% \item  \verb!A || B && Z!           
%% \item  \verb!A ^ X && Z!            
%% \item  \verb!A ^ X || C!            
%% \item  \verb|!A && !B|             
%% \item  \verb!A && B && X && Y!     
%% \item  \verb|(!X ^ A) && (B ^ !Y)| 
%% \end{multicols}
%% %%%\end{lstlisting}\
%% %%%\end{jjjlisting}
%% \end{EXRLL}
%% \end{SSTUDY}

\section*{{\color{cyan}Special Topic:} Are We Computers?}

{George Boole} published his seminal work, {\it An Investigation of the
Laws of Thought}, in 1854. His achievement was in developing an
algebra for logic---that is, a purely abstract and symbolic system
for representing the laws of logic.  Boole's was not the first attempt
to explore the relationship between the human mind and an abstract
system of computation.  Back in 1655, Thomas Hobbes had already
claimed that all thought was computation.\\
\hspace*{1pc}It is estimated that the human brain contains ($10^{12}$ $=$
$10{,}000{,}000{,}000{,}000$) {\it neurons}, and each neuron contains something
like 10,000 {\it dendrites}, the fibers that connect one neuron to
another.  Together, the neurons and dendrites make up a web of enormous
complexity.  Since the 1840s it has been known that the brain is
primarily electrical, and by the 1940s scientists had developed a
pretty good model of the electrical interactions among
neurons.  According to this model, neurons emit short bursts of
electricity along their {\it axons}, which function like output
wires.  The bursts leap over the gap separating axons and dendrites,
which function like the neurons' input wires.\\
\hspace*{1pc}In 1943, just before the first digital computers were developed,
Warren McCulloch, a neurophysiologist, and Walter Pitts, a
mathematician, published a paper titled, ``A Logical Calculus of the
Ideas Imminent in Nervous Activity.'' In this paper, they showed that
all of the boolean operators---AND, OR, NOT, and EXCLUSIVE-OR---could be represented 
by the behavior of small sets of neurons.   For
example, they showed that three neurons could be connected together in
such a way that the third neuron fired if and only if both of the
other two neurons fired.   This is exactly analogous to the definition
of the boolean AND operator.\\
\hspace*{1pc}A few years later, when the first computers were built, many
scientists and philosophers were struck by the similarity between the
logic elements that made up the computer's circuits and
the neuronal models that McCulloch and Pitts had developed.\\
\hspace*{1pc}The area of neural networks is a branch of artificial intelligence (one
of the applied areas of computer science) and is based on this insight
by McCulloch and Pitts.  Researchers in this exciting and rapidly
advancing field develop neural network models of various kinds of
human thinking and perception.
%% Footnote: McCulloch, W.S. and Pitts, W. (1943) ``A logical calculus
%% of the ideas immanent in nervous activity,'' Bulletin of Mathematical
%% Biophysics, 5: 115-133.

%\end{BOXfive}%

\subsection{Using Booleans in OneRowNim}

\noindent Now that we have introduced the {\tt boolean}
data type, let's use it to improve the {\tt OneRowNim} class, the
latest version of which, from Chapter~3, is given in
Figure~\ref{fig-ornclass}.  Previously we used an {\tt int} variable,
{\tt player}, to represent who's turn it is.  For a two-person game,
such as One Row Nim, a {\tt boolean} variable is well suited for this
purpose, because it can toggle between true and false. For example,
let's declare a variable, {\tt onePlaysNext}, and initialize it to
true, to represent the fact that player one will play first:

\begin{jjjlisting}
\begin{lstlisting}
private boolean onePlaysNext = true;
\end{lstlisting}
\end{jjjlisting}

\noindent When {\tt onePlaysNext} is true, it will be player one's
turn. When it is false, it will be player two's turn. Note that we are
deliberately remaining uncommitted as to whether one or the other
player is the computer.  

Given this new variable, it is necessary to redefine the methods that
had previously used the {\tt player} variable. The first method that
needs revision is the constructor:

\begin{jjjlisting}
\begin{lstlisting}
public OneRowNim(int sticks, int starter)
{   nSticks = sticks;
    onePlaysNext = (starter == 1);
}  // OneRowNim() constructor3
\end{lstlisting}
\end{jjjlisting}

\noindent In the constructor, the {\tt starter} parameter is used 
with a value of 1 or 2 to set which player goes first. Note how we use
an assignment statement to set {\tt onePlaysNext} to true if {\tt
starter} equals 1; otherwise it is set to false.  The assignment
statement first evaluates the expression on its right hand side ({\tt
starter == 1}). Because this is a boolean expression, it will have a
value of true or false, which will be assigned to {\tt
onePlaysNext}. Thus, the assignment statement is equivalent to the
following if/else statement:

\begin{jjjlisting}
\begin{lstlisting}
if (player == 1)
   onePlaysNext = true;
else
   onePlaysNext = false;
,\end{lstlisting}
\end{jjjlisting}

\noindent The remaining changes are shown in Figure~\ref{fig-newonerow}.
There are only two instance methods that need revision to accommodate
\begin{figure}[h!]
\jjjprogstart
\begin{jjjlisting}
\begin{lstlisting}
public class OneRowNim
{    private int nSticks = 7;
     private boolean onePlaysNext = true;

     public OneRowNim()
     {
     } //OneRowNim() constructor1
     public OneRowNim(int sticks)
     {   nSticks = sticks;
     }  // OneRowNim() constructor2
     public OneRowNim(int sticks, int starter)
     {   nSticks = sticks;
         onePlaysNext = (starter == 1);
     }  // OneRowNim() constructor3
     public boolean takeSticks(int num)
     {   if (num < 1 || num > 3 || num > nSticks) 
             return false;                // Error
         else                             // Valid move
         {   nSticks = nSticks - num;
             onePlaysNext = !onePlaysNext;
             return true;
         } //else
     } // takeSticks()
     public int getSticks()
     {   return nSticks;
     } {\color{cyan} // getSticks()}
     public int getPlayer()
     {   if (onePlaysNext) return 1;
         else return 2;
     } // getPlayer()
     public boolean gameOver()
     {   return (nSticks <= 0);
     } // gameOver()
     public int getWinner()
     {   if (nSticks < 1) return getPlayer();
         else return 0;  // game is not over
     } // getWinner()
     public void report()
     {   System.out.println("Number of sticks left: " 
                             + getSticks());
         System.out.println("Next turn by player " 
                             + getPlayer());
     }   // report()
} // OneRowNim class
\end{lstlisting}
\end{jjjlisting}
\jjjprogstop{The revised {\tt One\-Row\-Nim} uses a {\tt boolean}
variable to keep track of who's turn it is.}
{fig-newonerow}
\end{figure}
the use of {\tt boolean} variables. The {\tt takeSticks()} method
contains two revisions. The first uses the boolean OR operator to test
whether a move is valid:

\begin{jjjlisting}
\begin{lstlisting}
public boolean takeSticks(int num)
{   if (num < 1 || num > 3 || num > nSticks) 
        return false;                // Error
    else                             // Valid move
    {   nSticks = nSticks - num;
        onePlaysNext = !onePlaysNext;
        return true;
    } //else
} // takeSticks()
\end{lstlisting}
\end{jjjlisting}

\noindent It also uses the boolean NOT operator to toggle 
the value of {\tt onePlaysNext}, to switch to the other player's turn:

\begin{jjjlisting}
\begin{lstlisting}
onePlaysNext = !onePlaysNext;
\end{lstlisting}
\end{jjjlisting}

\noindent Finally, the {\tt getPlayer()} method now uses a
if/else statement to return either 1 or 2 depending on who's
turn it is:

\begin{jjjlisting}
\begin{lstlisting}
public int getPlayer()
{   if (onePlaysNext) 
        return 1;
    else return 2;
} // getPlayer()
\end{lstlisting}
\end{jjjlisting}




\section{Numeric Data and Operators}
\noindent Java has two kinds of numeric data: integers, which have no fractional
part, and real numbers or {\em floating-point numbers}, which contain
a fractional component.  Java recognizes four different kinds of
integers: {\tt byte}, {\tt short}, {\tt int}, and {\tt long}, which
are distinguished by the number of {\it bits} used to represent
them. A {\bf binary digit}, or {\bf bit}, is a 0 or a 1. (Recall that
computers read instructions as series of 0s and 1s.) Java has two
different kinds of real numbers, {\tt float} and {\tt double}, which
are also distinguished by the number of bits used to represent them.
See Table~5.3.

\begin{table}[h!]
%\hphantom{\caption{Java's numeric types}}
\TBT{0pc}{Java's numeric types}
\hspace*{-6pt}\begin{tabular}{lcc}
\multicolumn{3}{l}{\color{cyan}\rule{27pc}{1pt}}\\[2pt]
%%%RAM\TBCH{{\bf  Type}&{\bf Bits}&{\bf Range of Values}} 
{\bf  Type}&{\bf Bits}&{\bf Range of Values}
\\[-4pt]\multicolumn{3}{l}{\color{cyan}\rule{27pc}{0.5pt}}\\[2pt]
{\it byte}&~8&$-128  \hbox{ to }  +127$\cr
{\it short}&16&$-32768  \hbox{ to }   32767$\cr
{\it int}&32&$-2147483648   \hbox{ to }   2147483647$\cr
{\it long}&64&$-2^{63}  \hbox{ to }  2^{63}-1$
\\[-4pt]\multicolumn{3}{l}{\color{cyan}\rule{27pc}{0.5pt}}\\[2pt]
{\it float}&32&$-3.40292347E+38 \hbox{ to } +3.40292347E+38$\cr
{\it double}&64&$-1.79769313486231570E+308  \hbox{ to } +1.79769313486231570E+308$
\\[-4pt]\multicolumn{3}{l}{\color{cyan}\rule{27pc}{1pt}}
\end{tabular}
\endTB
\end{table}

The more bits a data type has, the more values it can represent.  One
bit can represent two possible values, 1 and 0, which can be used to
stand for true and false, respectively.  Two bits can represent four
possible values: 00, 01, 10, and 11; three bits can represent eight
possible values: 000, 001, 010, 100, 101, 110, 011, 111.  And, in
general, an {\it n}-bit quantity can represent $2^n$ different values.

As illustrated in Table~5.3, the various integer types represent positive or
negative whole numbers.  Perhaps the most commonly used integer type
in Java is the {\tt int} type, which is represented in 32 bits.  This
means that Java can represent $2^{32}$ different {\tt int} values,
which range from $-2{,}147{,}483{,}648$ to
$2{,}147{,}483{,}647$, that is, from $-2^{31}$ to $(2^{31} -1)$.
Similarly, an 8-bit integer, a {\tt byte}, can represent $2^8$ or 256
\marginnote{Integer data types}
different values, ranging from $-128$ to $+127$.  A 16-bit integer, a
{\tt short}, can represent $2^{16}$ different values, which range from
$-32768$ to $32767$. And a 64-bit integer, a {\tt long}, can represent
whole number values ranging from $-2^{63}$ to $2^{63}-1$.

For floating-point numbers, a 32-bit {\tt float} type can represent
$2^{32}$ different real numbers and a 64-bit {\tt double} value can
represent $2^{64}$ different real numbers.

\JavaTIP{EFFECTIVE DESIGN}{Platform Independence.}{In Java, a data type's
size (number of bits) is part of its definition and, therefore, remains
consistent across all platforms.   In C and C++, the size of a data type
is dependent on the compiler.}

It is worth noting that just as model airplanes are representations of
real airplanes, Java's numeric types are representations or
\marginnote{Data types are abstractions}
models of the numbers we deal with in mathematics.  In designing
Java's data types, various trade-offs have been made in order to come
up with practical implementations.

One trade-off is that the set of integers is infinite, but Java's
{\tt int} type can only represent a finite number of values.   Similarly,
%\begin{marginalnote}\it Representation trade-offs\end{marginalnote}
\marginnote{Representation trade-offs}
Java cannot \mbox{represent} the infinite number of values that occur
between, say, 1.111 and 1.112. So, certain real numbers cannot be
represented at all.  For example, because Java uses binary numbers to
represent its numeric types, one number that cannot be represented
exactly is $\frac{1}{10}$. This inability to exactly represent a value
is known as {\bf round-off error}.  Being unable to represent certain
values can cause problems in a program. For example, it might be
\marginnote{Round-off error}
difficult to represent dollars and cents accurately in a program.

Another source of problems in dealing with numeric data is due to
limits in their precision. For example, a decimal number represented
as a \mbox{\tt double} value can have a maximum of 17 {\it significant
digits}, and a {\tt float} can have a maximum 8. A significant digit
is one that contributes to the number's value.  If you tried to store
values such as 12345.6789 or 0.123456789 in a {\tt float} variable,
they would be rounded off to 12345.679 and 0.12345679, respectively,
causing a possible error.

\JavaTIP{DEBUGGING TIP}{Significant Digits.}{In using numeric data, 
be sure the data type you choose has enough precision to represent the
values your program needs.}

\secEXRHone{Self-Study Exercises}
%% \begin{SSTUDY}
%% \item  List all of the binary values 
%% that can be represented in 4 bits---that is, all values in the range
%% 0000 to 1111.

%% \item  If a 6-bit representation were used for an integer type, how
%% many different integers could be represented?

%% \item  If you were writing a program to process scientific data that
%% had to be accurate to at least 12 significant (decimal) digits,
%% what type of data would you use?

%% \end{SSTUDY}

\subsection{Numeric Operations}
\noindent The operations that can be done on numeric data include the standard
%\begin{marginalnote}\it Numeric operators\end{marginalnote}
algebraic operations: {\it addition} (+), {\it subtraction} ($-$), {\it
multiplication} (*), {\it division} (/), as well as the {\it modulus}
(\%) operator.  Note that in Java, the multiplication symbol is {\tt *}
and not the $\times$.  The arithmetic operators are binary operators,
\marginnote{Numeric operators}
meaning that they each take two operands.  Table~5.4
\begin{table}[htb]
%\hphantom{\caption{The standard arithmetic operators in Java}}
\TBT{2pc}{The standard arithmetic operators\index{arithmetic operators} in Java}
\hspace*{1.5pc}\begin{tabular}{lcll}
\multicolumn{4}{l}{\color{cyan}\rule{23pc}{1pt}}\\[2pt]
%%%RAM\TBCH{{\bf Operation} &  {\bf Operator} & {\bf Java} & {\bf Algebra}}
{\bf Operation} &  {\bf Operator} & {\bf Java} & {\bf Algebra}
\\[-4pt]\multicolumn{4}{l}{\color{cyan}\rule{23pc}{0.5pt}}\\[2pt]
{\it Addition}&$+$&$x  +  2$&$     x  +  2$\cr
{\it Subtraction}&$-$&$m  -  2$&$     m  -  2$\cr
{\it Multiplication}&*&$m$   *   $2$&$     2m  \hbox{ \it or }  2 \times m$\cr
{\it Division}&/&$x  /  y$&$     x  \div  y  \hbox{ \it or }  \frac{x}{y}$\cr
{\it Modulus}&\%&$x  \%  y$&$   x  \hbox{ \it modulo }  y \hbox{ \it (for  integers  $x$ and $y$)}$
\\[-4pt]\multicolumn{4}{l}{\color{cyan}\rule{23pc}{1pt}}
\end{tabular}
\endTB
\end{table}
compares expressions involving the Java operators with their standard
algebraic counterparts.

Although these operations should seem familiar, there are some important
differences between their use in algebra and their use in a Java
program.   Consider the following list of expressions:

\begin{jjjlisting}
\begin{lstlisting}
3 / 2     ==>  value 1     An integer result
3.0 / 2.0 ==>  value 1.5   A floating-point result
3 / 2.0   ==>  value 1.5   A floating-point result
3.0 / 2   ==>  value 1.5   A floating-point result
\end{lstlisting}
\end{jjjlisting}

\noindent In each of these cases we are dividing the quantity 3 by
the quantity~2. However, different results are obtained depending on
the {\it type} of the operands involved.  When both operands are
integers, as in (3/2), the result must also be an
integer\index{integer division}. Hence, (3/2) has the value 1, an
integer.   Because integers cannot have a fractional part, the 0.5 is
simply discarded.   Integer division (/) always gives an integer
%\begin{marginalnote}\it Integer division gives integer result\end{marginalnote}
\marginnote{Integer division gives an integer\hfill\break result}
result.  Thus, the value of (6/2) is 3 and the value of (7/2) is also
3. Because 3.5 is not an integer, the result of dividing 7 by 2 cannot
be 3.5.  

\JavaTIP{DEBUGGING TIP}{Integer Division.}{A common source of error among
beginning programmers is forgetting that integer division always gives
an integer result.}

\noindent On the other hand, when either operand is a real number, as in the
last three cases, the result is a real number.  Thus, while the same
symbol (/) is used for dividing integers and real numbers, there are
really two different operations involved here: {\it integer division}
and {\it floating-point division\index{floating-point
division}}. Using the same symbol (/) for different operations
(integer division and real division) is known as {\bf operator
overloading}. It is similar to method overloading, which was discussed
in Chapter~3.

What if you want to keep the remainder of an integer division?  Java
%\begin{marginalnote}\it Modular arithmetic\end{marginalnote}
\marginnote{Modular arithmetic}
provides the modulus operator (\%), which takes two 
operands.  The expression (7 \% 5) gives the remainder after dividing 7
by 5---2 in this case.   In general, the expression $(m \; \% \; n)$ (read {\it m}
mod {\it n}) gives the remainder after {\it m} is divided by {\it n}. Here are
several examples:

\begin{jjjlisting}
\begin{lstlisting}
7 % 5   ==> 7 mod 5 equals 2
5 % 7   ==> 5 mod 7 equals 5
-7 % 5  ==> -7 mod 5 equals -2
7 % -5  ==> 7 mod -5 equals 2
\end{lstlisting}
\end{jjjlisting}

\noindent The best way to interpret these examples is to perform
long division on the operands keeping both the quotient and the
remainder.  For example, when you do long division on $-7 \div 5$, you
get a quotient of -1 and a remainder of -2. The quotient is the value
of $-7/5$ and the remainder is the value of $-7 \% 5$.  When you
do long division on $7 \div -5$, you get a quotient of -1 and a
remainder of 2. The quotient is the value of $7/-5$ and the
remainder is the value of $7\%-5$.  

We will encounter many practical uses for the modulus operator in our
programs. For a simple example, we use it when we want to determine
whether an integer is even or odd. Numbers that leave a 0 remainder
when divided by 2 are even:

\begin{jjjlisting}
\begin{lstlisting}
if (N % 2 == 0)
    System.out.println(N + " is even");
\end{lstlisting}
\end{jjjlisting}

\noindent More generally, we could use the mod operator to define
divisibility by 3, 4, 10, or by any number.

\subsubsection*{Numeric Promotion Rules}

\noindent Java is considered a {\it strongly typed} language because all expressions 
in Java, such as
%\begin{marginalnote}\it Expressions have a type\end{marginalnote}
\marginnote{Expressions have a type}
(3/2), have a type associated with them.  In cases where one
arithmetic operand is an integer and one is a floating-point number,
Java {\it promotes} the integer into a floating-point value and
performs a floating-point operation.

{\bf Promotion} is a matter of converting one type to another type.
For example, in the expression (5 + 4.0), the value 5 must be promoted
to 5.0 before floating-point addition can be performed on (5.0 + 4.0).
Generally speaking, automatic promotions such as these are allowed in
Java whenever it is possible to perform the promotion {\it without
loss of information}.  Because an integer (5) does not have a
fractional component, no information will be lost in promoting it to a
real number (5.0).  On the other hand, you cannot automatically
convert a real number (5.4) to an integer (5) because that might lead
to loss of information.  This leads to the following rule:

\JavaRule{Integer Promotion.}{In an operation that contains an 
integer and a floating-point operand, the integer is {\it promoted} to
a floating-point value {\it before} the operation is performed.}

\noindent This rule is actually an instance of a more general rule, for whenever
an expression involves operands of different types, some operands must
be converted before the expression can be evaluated.  Consider the
following example:

\begin{jjjlisting}
\begin{lstlisting}
byte n = 125;
short m = 32000;
n * m;
\end{lstlisting}
\end{jjjlisting}

\noindent In this case, {\it (n * m)} involves two
different integer types, {\tt byte} and {\tt short}. Before evaluating
this expression Java must first promote the {\tt byte} to a
{\tt short} and carry out the operation as the multiplication of two
{\tt short}s.  Conversion of {\tt short} to {\tt byte} would not be possible
because there's no way to represent the value 32000 as a {\tt byte}.

It is important to note that this conversion rule applies regardless
of the actual values of the operands.   In applying the rule, Java looks
at the operand's type, not its value.  So even if {\it m} were assigned a
value that could be represented as a byte (for example, 100), the
%\begin{marginalnote}\it Promotion is automatic\end{marginalnote}
\marginnote{Promotion is automatic}
promotion would still go from smaller to larger type.  This leads to following
the general rule:


\JavaRule{Type Promotion\index{promotion rule}.}{In
general, when two different types are involved in an operation, the
smaller type---the one with fewer bits---is converted to the larger
type before the operation is performed.  To do otherwise would risk
losing information.}

\noindent Table~5.5 summarizes the actual promotion rules used by
Java in evaluating expressions involving mixed operands.  Note that the
last rule implies that integer expressions involving {\tt byte} or
{\tt short} or {\tt int} are performed as {\tt int}. This explains why
integer literals---such as $56$ or $-108$---are represented as
{\tt int} types in Java.
\begin{table}[h]
%\hphantom{\caption{Java promotion rules for mixed arithmetic operators. If two
%rules apply, choose the one that occurs first in this table.}}
\TBT{2.5pc}{Java promotion rules for mixed arithmetic operators.\hfill\break
\hspace*{2.5pc}If two rules apply, choose the one that occurs first in this table.}
\hspace*{2pc}\begin{tabular}{lll}
\multicolumn{3}{l}{\color{cyan}\rule{22pc}{1pt}}\\[2pt]
%%%RAM\TBCH{If either operand is}&&\TBCH{The other is promoted to}
{If either operand is}&&{The other is promoted to}
\\[-4pt]\multicolumn{3}{l}{\color{cyan}\rule{22pc}{0.5pt}}\\[2pt]
{\tt double}&&{\tt double}\cr
{\tt float}&\hspace*{36pt}&{\tt float}\cr
{\tt long}&&{\tt long}\cr
{\tt byte} {\it or\/} {\tt short}&&{\tt int}
\\[-4pt]\multicolumn{3}{l}{\color{cyan}\rule{22pc}{1pt}}
\end{tabular}
\endTB
\end{table}


%% \secEXRHone{Self-Study Exercises}
%% \begin{SSTUDY}
%% \item  Evaluate each of the following integer expressions:

%% \begin{EXRLL}
%% \begin{multicols}{3}
%% \item  \verb!8 / 2!  
%% \item  \verb!9 / 2!  
%% \item  \verb!6 / 8!  
%% \item  \verb!9 % 2!  
%% \item  \verb!8 % 6!  
%% \item  \verb!6 % 8!  
%% \item  \verb!8 % 4!  
%% \item  \verb!4 % 8!  
%% \end{multicols}
%% \end{EXRLL}


%% \item  Evaluate each of the following expressions, paying special
%% attention to the {\it type} of the result in each case:

%% \begin{EXRLL}
%% \begin{multicols}{3}
%% \item  \verb!8 / 2.0!  
%% \item  \verb!9 / 2.0!  
%% \item  \verb!6 / 8!  
%% \item  \verb!0.0 / 2!  
%% \item  \verb!8.0 / 6.0! 
%% \end{multicols}
%% \end{EXRLL}


%% \item  Suppose that the following variable declarations are made:

%% \begin{jjjlisting}
%% \begin{lstlisting}
%% byte m = 3; short n = 4; int p = 5; 
%% long q = 6; double r = 7.0;
%% \end{lstlisting}
%% \end{jjjlisting}

%% \noindent Use type promotion rules to determine
%% the type of  expression and then evaluate each of the following
%% expressions:

%% \begin{EXRLL}
%% \begin{multicols}{3}
%% \item  \verb!m + n!      
%% \item  \verb!p * q!      
%% \item  \verb!m + n + r!  
%% \item  \verb!p * q * m!  
%% \item  \verb!r - m!      
%% \end{multicols}
%% \end{EXRLL}
%% \end{SSTUDY}


\subsection{Operator Precedence}
\noindent The built-in precedence order for arithmetic\index{arithmetic
%\begin{marginalnote}\it Arithmetic precedence order\end{marginalnote}
\marginnote{Arithmetic precedence order}
precedence} operators is shown in Table~5.6.  Parenthesized
\begin{table}[h!]
%\hphantom{\caption{Precedence order of the arithmetic operators}}
%%%\vspace{-6pt}
\TBT{3.5pc}{Precedence order of the arithmetic operators\index{arithmetic operators}}
\hspace*{3pc}\begin{tabular}{cll}
\multicolumn{3}{l}{\color{cyan}\rule{20pc}{1pt}}\\[2pt]
%%%RAM\TBCH{Precedence} &                &                  \cr
%%%RAM\TBCH{Order}      & \TBCH{Operator} & \TBCH{Operation}  
{Precedence} &                &                  \cr
{Order}      & {Operator} & {Operation}  
\\[-4pt]\multicolumn{3}{l}{\color{cyan}\rule{20pc}{0.5pt}}\\[2pt]
1&$( \; )$&{\it Parentheses}\cr
2&$\ast \;\; / \;\; \%$&{\it Multiplication, Division, Modulus}\cr
3&$+ \;\; -$&{\it Addition, Subtraction}
\\[-4pt]\multicolumn{3}{l}{\color{cyan}\rule{20pc}{1pt}}
\end{tabular}
\endTB
\end{table}
expressions have highest precedence\index{precedence} and are
evaluated first.  Next come the multiplication, division, and modulus
operators, followed by addition and subtraction.  When we have an
unparenthesized expression that involves both multiplication and
addition, the multiplication would be done first, even if it occurs to
the right of the plus sign.  Operators at the same level in the
precedence hierarchy are evaluated from left to right.  For example,
consider the following expression:

\begin{jjjlisting}
\begin{lstlisting}
9 + 6 - 3 * 6 / 2
\end{lstlisting}
\end{jjjlisting}


\noindent In this case, the first operation to be applied will be the
multiplication (*), followed by division (/), followed by addition (+),
and then finally the subtraction ($-$).  We can use parentheses to clarify
the order of evaluation.  A parenthesized expression is evaluated
outward from the innermost set of parentheses:

\begin{jjjlisting}
\begin{lstlisting}
Step 1.  ( (9 + 6) - ((3 * 6) / 2 ) )
Step 2.  ( (9 + 6) - (18 / 2 ) )
Step 3.  ( (9 + 6) - 9 )
Step 4.  ( 15 - 9 )
Step 5.  6
\end{lstlisting}
\end{jjjlisting}


\noindent Parentheses can (and should) always be used to clarify
the order of operations in an expression.  For example, addition will
be performed before multiplication in the following expression:

\begin{jjjlisting}
\begin{lstlisting}
(a + b) * c
\end{lstlisting}
\end{jjjlisting}

\noindent Another reason to use parentheses is that Java's precedence 
and promotion rules will
sometimes lead to expressions that look fine but contain subtle 
errors.  For example, consider the following expressions:

\begin{jjjlisting}
\begin{lstlisting}
System.out.println(5/3/2.0);   // 0.5
System.out.println(5/(3/2.0)); // 3.33
\end{lstlisting}
\end{jjjlisting}

\noindent The first gives a result of 0.5, but the use of parentheses
in the second gives a result of 3.33. If the second is the
expected interpretation, then the parentheses here helped
avoid a subtle semantic error.

\JavaTIP{PROGRAMMING TIP}{Parenthesize!}{To avoid subtle bugs caused by
Java's precedence and promotion rules, use parentheses to specify the order of
evaluation in an expression.}

\secEXRHone{Self-Study Exercise}
%% \begin{SSTUDY}
%% \item  Parenthesize and then evaluate each of the  expressions
%% that follow, taking care to observe operator precedence rules.  Watch for subtle
%% syntax errors.

%% \begin{EXRLL}
%% \begin{multicols}{3}
%% \item  \verb!4 + 5.0 * 6!    
%% \item  \verb!(4 + 5) * 6!    
%% \item  \verb!4 + 5 / 6!      
%% \item  \verb!(4 + 5) / 6!    
%% \item  \verb!4 + 5 % 3!      
%% \item  \verb!(4 + 5) % 3!    
%% \item  \verb!9 % 2 * 7 / 3!  
%% \item  \verb!5.0 / 2 * 3!    
%% \end{multicols}
%% \end{EXRLL}

%% \end{SSTUDY}


\subsection{Increment and Decrement Operators}
\noindent Java provides a number of unary operators that are used to increment
or decrement an integer variable.   For example, the expression
{\it k++} uses the {\it increment operator}\index{increment operator ++}
{\it ++} to increment the value of the integer variable {\it k}.  The
expression {\it k++} is equivalent to the following Java statements:

\begin{jjjlisting}
\begin{lstlisting}
int k;
k = k + 1;// Add 1 to k and assign the result back to k
\end{lstlisting}
\end{jjjlisting}

\noindent The unary {\it ++} operator applies to
a single integer operand, in this case to the variable {\it k}. It
increments {\it k}'s value by 1 and assigns the result back to {\it k}. It may
be used either as a {\it preincrement}\index{preincrement} or a 
%\begin{marginalnote}\it Preincrement and postincrement\end{marginalnote}
\marginnote{Preincrement and postincrement}
{\it postincrement\index{postincrement}} operator.  In the expression
{\it k++}, the operator {\it follows} the operand, indicating that it is
being used as a {\it postincrement} operator.  This means that the
increment operation is done {\it after} the operand's value is used.

Contrast that with the expression {\it ++k} in which the {\it ++}
operator {\it precedes} its operand.  In this case, it is used as a {\it
preincrement} operator, which means that the increment operation is
done {\it before} the operand's value is used.

When used in isolation, there is no practical difference between
{\it k++} and {\it ++k}.  Both are equivalent to {\it k = k + 1}. However,
when used in conjunction with other operators, there is a significant
difference between preincrement and postincrement.  For example, in the
following code segment,


\begin{jjjlisting}
\begin{lstlisting}
int j = 0, k = 0;   // Initially both j and k are 0
j = ++k;            // Final values of both j and k are 1
\end{lstlisting}
\end{jjjlisting}

\noindent the variable {\it k} is incremented {\it before}
%\begin{marginalnote}\it Precedence order\end{marginalnote}
\marginnote{Precedence order}
its value is assigned to {\it j}. After execution of the assignment
statement, {\it j} will equal 1 and {\it k} will equal 1. The 
sequence is equivalent to

\begin{jjjlisting}
\begin{lstlisting}
int j = 0, k = 0;   // Initially both j and k are 0
k = k + 1;
j = k;              // Final values of both j and k are 1
\end{lstlisting}
\end{jjjlisting}

\noindent However, in the following example,

\begin{jjjlisting}
\begin{lstlisting}
int i = 0, k = 0; // Initially both i and k are 0
i = k++;          // Final value of i is 0 and k is 1
\end{lstlisting}
\end{jjjlisting}

\noindent the variable {\it k} is incremented {\it after}
its value is assigned to {\it i}. After execution of the assignment
statement, {\it i} will have the value 0 and {\it k} will have the
value 1. The preceding sequence is equivalent to

\begin{jjjlisting}
\begin{lstlisting}
int i = 0, k = 0;   // Initially both i and k are 0
i = k;
k = k + 1;          // Final value of i is 0 and k is 1
\end{lstlisting}
\end{jjjlisting}

\noindent In addition to the increment operator, Java also supplies the {\it
%\begin{marginalnote}\it Predecrement and postdecrement\end{marginalnote}
\marginnote{Predecrement and postdecrement}
decrement\index{decrement operator --}} operator {\it $--$}, which can
also be used in the predecrement\index{predecrement} and
postdecrement\index{postdecrement} forms.   The expression {\it $--$ k}
will first decrement {\it k}'s value by 1 and then use {\it k} in
any expression in which it is embedded.  The expression {\it k$--$} will
use the current value of {\it k} in the expression in which {\it k} is
contained and then it will decrement {\it k}'s value by
1. Table~5.7 summarizes the increment and decrement
operators.  The unary increment and decrement operators have higher
precedence than any of the binary arithmetic operators.

\begin{table}[htb]
%\hphantom{\caption{Java's increment and decrement operators}}
\TBT{4pc}{Java's increment and decrement operators}
\hspace*{3.5pc}\begin{tabular}{lll}
\multicolumn{3}{l}{\color{cyan}\rule{19pc}{1pt}}\\[2pt]
%%%RAM\TBCH{Expression}\hspace*{18pt}   & \TBCH{Operation}\hspace*{30pt} &  \TBCH{Interpretation} 
{Expression}\hspace*{18pt}   & {Operation}\hspace*{30pt} &  {Interpretation} 
\\[-4pt]\multicolumn{3}{l}{\color{cyan}\rule{19pc}{0.5pt}}\\[2pt]
$j = ++k$&{\it Preincrement}&$k=k+1;j=k;$\cr
$j = k++$&{\it Postincrement}&$j=k;k=k+1;$\cr
$j = --k$&{\it Predecrement}&$k=k-1;j=k;$\cr
$j = k--$&{\it Postdecrement}&$j=k;k=k-1;$
\\[-4pt]\multicolumn{3}{l}{\color{cyan}\rule{19pc}{1pt}}
\end{tabular}
\endTB
\end{table}

\JavaRule{Pre- and Postincrement/Decrement.}{If an
expression like {\it ++k} or {\it $--$k} occurs in an expression, {\it
k} is incremented or decremented {\it before} its value is used in the
rest of the expression.   If an expression like {\it k++} or {\it k$--$}
occurs in an expression, {\it k} is incremented or decremented {\it
after} its value is used in the rest of the expression.}

\JavaTIP{PROGRAMMING TIP}{Increment and Decrement Operators.}{Because of
their subtle behavior, be careful in how you use the unary increment
and decrement operators.  They are most appropriate and useful for
incrementing and decrementing loop variables, as we'll see later.}

\secEXRHone{Self-Study Exercise}
%% \begin{SSTUDY}
%% \item  What value will {\it j} and {\it k} have after each of the
%%  calculations that follow? Assume that {\it k} has the value 0 and {\it
%% j} has the value 5 before each operation is performed.

%% \begin{EXRLL}
%% \begin{multicols}{3}
%% \item  \verb!k = j;!    
%% \item  \verb!k = j++;!  
%% \item  \verb!k = ++j;!  
%% \item  \verb!k = j--;!  
%% \item  \verb!k = --j;!   
%% \end{multicols}
%% \end{EXRLL}

%% \end{SSTUDY}

\subsection{Assignment Operators}
\noindent In addition to the simple assignment\index{assignment operator =}
operator (=), Java supplies a number of shortcut
assignment operators that allow you to combine an arithmetic operation
and an assignment in one operation.  These operations can be used with
either integer or floating-point operands.  For example, the {\tt +=}
operator allows you to combine addition and assignment into one
expression.  The statement

\begin{jjjlisting}
\begin{lstlisting}
k += 3;
\end{lstlisting}
\end{jjjlisting}

\noindent is equivalent to the statement

\begin{jjjlisting}
\begin{lstlisting}
k = k + 3;
\end{lstlisting}
\end{jjjlisting}

\noindent Similarly, the statement

\begin{jjjlisting}
\begin{lstlisting}
r += 3.5 + 2.0 * 9.3 ;
\end{lstlisting}
\end{jjjlisting}

\noindent is equivalent to

\begin{jjjlisting}
\begin{lstlisting}
r = r + (3.5 + 2.0 * 9.3);  // i.e., r = r + 22.1;
\end{lstlisting}
\end{jjjlisting}

\noindent As these examples illustrate, when using the {\tt +=}
operator, the expression on its right-hand side is first evaluated
and then {\it added} to the current value of the variable on its left-hand side.

Table~5.8 lists the other assignment operators that
can be used in combination with the arithmetic operators.   For each of
these operations, the interpretation is the same: Evaluate the
\begin{table}[h]
%\hphantom{\caption{Java's assignment operators}}
\TBT{0pc}{Java's assignment operators\index{assignment operators}}
\hspace*{-6pt}\begin{tabular}{llll}
\multicolumn{4}{l}{\color{cyan}\rule{27pc}{1pt}}\\[2pt]
%%%RAM\TBCH{Operator}\hspace*{6pt}&\TBCH{Operation}&\TBCH{Example}\hspace*{6pt}&\TBCH{Interpretation}
{Operator}\hspace*{6pt}&{Operation}&{Example}\hspace*{6pt}&{Interpretation}
\\[-4pt]\multicolumn{4}{l}{\color{cyan}\rule{27pc}{0.5pt}}\\[2pt]
$=$&{\it Simple assignment}&$m = n;$&$m = n;$\cr
$+\!=$&{\it Addition then assignment}&$m\ +\!=\ 3;$&$m = m + 3;$\cr
$-\!=$&{\it Subtraction then assignment}&$m\ -\!=\ 3;$&$m = m - 3;$\cr
$\ast \!=$&{\it Multiplication then assignment}\hspace*{12pt}&$m\ \ast \!=\ 3;$&$m = m\,\ast \,3;$\cr
$/\!=$&{\it Division then assignment}&$m\ /\!=\ 3;$&$m = m / 3;$\cr
$\%\!=$&{\it Remainder then assignment}&$m\ \%\!=\ 3;$\hspace*{6pt}&$m = m \% 3;$
\\[-4pt]\multicolumn{4}{l}{\color{cyan}\rule{27pc}{1pt}}
\end{tabular}
\endTB
\end{table}
expression on the right-hand side of the operator and then perform
the arithmetic operation (such as addition or multiplication) to the
current value of the variable on the left of the operator.


\secEXRHone{Self-Study Exercises}
%% \begin{SSTUDY}
%% \item  What value will {\it j} and {\it k} have after each of the 
%% calculations that follow? Assume that {\it k} has the value 10 and that {\it j}
%% has the value 5 before each operation is performed.

%% \begin{EXRLL}
%% \begin{multicols}{3}
%% \item  \verb!k += j;!          
%% \item  \verb!k -= j++;!        
%% \item  \verb!k *= ++j * 2;!    
%% \item  \verb!k /= 25 * j--;!   
%% \item  \verb!k %= j - 3;!      
%% \end{multicols}
%% \end{EXRLL}


%% \item  Write four different statements that add 1 to the {\tt int} {\it k}.

%% \end{SSTUDY}

\subsection{Relational Operators}
\noindent There are several {\it relational} operations that can be performed on
integers:  $<$, $>$, $< =$, $> =$, $= =$, and $!\!=$.  These correspond to
the algebraic operators $<$, $>$, $\leq$, $\geq$, $=$, and
$\neq$. Each of these operators takes two operands (integer or
real) and returns a boolean result.  They are defined in
Table~5.9.

\begin{table}[htb]
%\hphantom{\caption{Relational operators}}
\TBT{4pc}{Relational operators\index{relational operators}}
\hspace*{3.5pc}\begin{tabular}{lll}
\multicolumn{3}{l}{\color{cyan}\rule{19pc}{1pt}}\\[2pt]
%%%RAM\TBCH{Operator}\hspace*{6pt} & \TBCH{Operation}  &  \TBCH{Java Expression}
{Operator}\hspace*{6pt} & {Operation}  &  {Java Expression}
\\[-4pt]\multicolumn{3}{l}{\color{cyan}\rule{19pc}{0.5pt}}\\[2pt]
$< $&{\it Less than}&$5 <  10$\cr
$> $&{\it Greater than}&$10  >  5$\cr
$< =$&{\it Less than or equal to}&$5 < = 10$\cr
$> =$&{\it Greater than or equal to}\hspace*{6pt}&$10 > =  5$\cr
$= =$&{\it Equal  to}&$5 = = 5$\cr
$!\!=$&{\it Not equal to}&$5\ !\!=\ 4$
\\[-4pt]\multicolumn{3}{l}{\color{cyan}\rule{19pc}{1pt}}
\end{tabular}
\endTB
\end{table}

Note that several of these relational operators require two symbols in
Java.  Thus, the familiar equals sign (=) is replaced in Java by ==.
%\begin{marginalnote}\it Equals vs.  assigns\end{marginalnote}
\marginnote{Equals vs.  assigns}
This is so  the equality operator can be distinguished from the
assignment operator.   Also, less than or equal to ($<$=), greater than
or equal to ($>$=), and not equal to (!=) require two symbols,
instead of the familiar $\leq$, $\geq$, and $\neq$ from algebra.  In
each case, the two symbols should be consecutive.  It is an error in
Java for a space to appear between the $<$ and = in $< =$.


\JavaTIP{DEBUGGING TIP}{Equality and Assignment.}{A common semantic error
among beginning programmers is to use the assignment operator (=) when
the equality operator (==) is intended.}

Among the relational operators, the inequalities ($<$, $>$, $< =$, and $> =$)
have higher precedence than the equality operators ($= =$ and $!\!=$). In an
expression that involves both kinds of operators, the inequalities
would be evaluated first.   Otherwise, the expression is evaluated from
left to right.

Taken as a group the relational operators have lower
precedence\index{precedence} than the arithmetic operators.
Therefore, in evaluating an expression that involves both arithmetic
and relational operators, the arithmetic operations are done first.
Table~5.10 includes all of the numeric operators
introduced so~far.

\begin{table}[htb]
%\hphantom{\caption{Numeric operator precedence\index{precedence table} including relations}}
\TBT{3pc}{Numeric operator precedence including relations}
\hspace*{2.5pc}\begin{tabular}{cll}
\multicolumn{3}{l}{\color{cyan}\rule{21pc}{1pt}}\\[2pt]
%%%RAM\TBCH{Precedence} &                &                 \cr
%%%RAM\TBCH{Order}      & \TBCH{Operator} & \TBCH{Operation}
{Precedence} &                &                 \cr
{Order}      & {Operator} & {Operation}
\\[-4pt]\multicolumn{3}{l}{\color{cyan}\rule{21pc}{0.5pt}}\\[2pt]
1 &$  (\,)$&{\it Parentheses}\cr
2 &$  +\!+\ \ --$&{\it Increment,  decrement}\cr
3 &$  \ast \ \ /\ \ \%$&{\it Multiplication,  division,  modulus}\cr
4 &$  +\ \ -$&{\it Addition,  subtraction}\cr
5 &$  <\,\,\,\,>\ \ < =\ \ > =$&{\it Relational  operators}\cr
6 &$  = =\ \ !\!=$&{\it Equality  operators}
\\[-4pt]\multicolumn{3}{l}{\color{cyan}\rule{21pc}{1pt}}
\end{tabular}
\endTB
\end{table}

To take an example, let us evaluate the following complex 
\mbox{expression:}

\begin{jjjlisting}
\begin{lstlisting}
9 + 6 <= 25 * 4 + 2
\end{lstlisting}
\end{jjjlisting}

\noindent To clarify the implicit operator precedence, we
first parenthesize the 
\mbox{expression}

\begin{jjjlisting}
\begin{lstlisting}
( 9 + 6 ) <= ( (25 * 4 ) + 2 )
\end{lstlisting}
\end{jjjlisting}

\noindent and then evaluate it step by step:

\begin{jjjlisting}
\begin{lstlisting}
Step 1. ( 9 + 6 ) <= ( (25 * 4 ) + 2 )
Step 2. ( 9 + 6 ) <= ( 100 + 2 )
Step 3. 15 <= 102
Step 4. true
\end{lstlisting}
\end{jjjlisting}

\noindent The following expression is an example of an ill-formed expression:

\begin{jjjlisting}
\begin{lstlisting}
9 + 6 <= 25 * 4 == 2
\end{lstlisting}
\end{jjjlisting}

\noindent That the expression is ill formed
 becomes obvious if we parenthesize it and then attempt to evaluate
it:

\begin{jjjlisting}
\begin{lstlisting}
Step 1. ( ( 9 + 6 )  <= ( 25 * 4 ) ) == 2
Step 2. ( 15  <= 100 ) == 2
Step 3. true == 2      // Syntax error results here
\end{lstlisting}
\end{jjjlisting}

\noindent The problem here is that the expression
{\tt true == 2} is an attempt to compare an {\tt int} and a
{\tt boolean} value, which can't be done.  As with any other binary
operator, the == operator requires that both
%\begin{marginalnote}Strong typing\end{marginalnote}
\marginnote{Strong typing}
of its operands be of the
same type.  This is another example of Java's strong type checking.


\secEXRHone{Self-Study Exercises}
%% \begin{SSTUDY}
%% \item  For each of these questions, what is 
%% the value of {\it m}? Assume that {\it k} equals 2, {\it j} equals 3,
%% and {\it m} equals 10, before each question.  Don't forget about
%% precedence.

%% \begin{EXRLL}
%% \begin{multicols}{3}
%% \item  \verb!m = j++ + k ;!     
%% \item  \verb!m = ++j + k;!      
%% \item  \verb!m += j + k;!       
%% \item  \verb!m *= j++ + k++;!   
%% \item  \verb!m *= ++j + ++k;!   
%% \end{multicols}
%% \end{EXRLL}


%% \item  Evaluate each of the following expressions, taking care to observe
%% operator precedence rules.  If an expression is illegal, mark it
%% illegal and explain why.

%% \begin{EXRLL}
%% \begin{multicols}{2}
%% \item  \verb!4 + 5 ==  6 * 2!                  
%% \item  \verb!(4 + 5)! $< =$ \verb!6 / 3!       
%% \item  \verb!4 + 5 / 6! $> =$ \verb!10 % 2!    
%% \item  \verb!(4 = 5) / 6!                      
%% \item  \verb|4 + 5 % 3 != 7 - 2|               
%% \item  \verb!(4 + 5) % 3 = 10 -4!              
%% \item  \verb!9 % 2 * 7 / 3 > 17!               
%% \end{multicols}
%% \end{EXRLL}

%% \end{SSTUDY}

\section{{\bf From the Java Library} {\tt java.lang.Math}}

%\addcontentsline{toc}{section}{\S~~~ From the Java Library: The {\tt Math} Class}
%<sec_mathclass:\thiseqno><pg_sec_mathclass:\thispageno>
\tBOXseven{The} {\tt java.lang.Math} class provides many common
\WWWjava 
mathematical functions that will prove useful in performing numerical
computations.  As an element of the {\tt java.lang} package, it is
included implicitly in all Java programs. Table~5.11 lists some of the
most commonly used {\tt Math} class \mbox{methods.}

\begin{table}[htb]
%\hphantom{\caption{A selection of {\tt Math} class methods}}
\TBT{0pc}{A selection of {\tt Math} class methods}
\hspace*{-6pt}\begin{tabular}{lll}
\multicolumn{3}{l}{\color{cyan}\rule{27pc}{1pt}}\\[2pt]
%%%RAM\TBCH{Method}&\TBCH{Description}&\TBCH{Examples}
{Method}&{Description}&{Examples}
\\[-4pt]\multicolumn{3}{l}{\color{cyan}\rule{27pc}{0.5pt}}\\[2pt]
{\it int abs(int  x)}&
{\it Absolute  value  of x}&
{\it if  x  $>$=  0  abs(x)  is  x}\cr
{\it long abs(long  x)}&&
{\it if  x  $<$  0  abs(x)  is  $-$x}\cr
{\it float abs(float x)}&
%{\it double abs(double x)}&
&\cr
&&\cr
{\it int ceil(double  x)}&
{\it Rounds  x  to  the  smallest}&
{\it ceil(8.3)  is  9}\cr
&{\it integer  not  less  than  x}&
{\it ceil($-$8.3)  is  $-$8}\cr
&&\cr
{\it int floor(double  x)}&
{\it Rounds  x  to  the  largest}&
{\it floor (8.9)  is  8}\cr
&{\it integer  not  greater  than  x}&
{\it floor($-$8.9)  is  $-$9}\cr
&&\cr
{\it double log(double x)}&
{\it Natural  logarithm  of x}&
{\it log (2.718282)  is  1.0}\cr
&&\cr
{\it double pow(double x, double  y)}&
{\it  x  raised  to  the  y power  ($x^y$)}&{\it pow(3, 4 )  is  81.0}\cr
&&{\it pow(16.0, 0.5)  is  4.0}\cr
&&\cr
{\it double random()}&
{\it Generates  a  random}&
{\it random()  is  0.5551}\cr
&{\it number in  the  interval   [0,1)}&
{\it random() is 0.8712}\cr
&&\cr
{\it long round(double  x)}&
{\it Rounds  x  to  an  integer}&
{\it round(26.51)  is  27}\cr
&&{\it round (26.499)  is  26}\cr
&&\cr
{\it double sqrt(double  x)}&
{\it Square root of x}&
{\it sqrt(4.0) is  2.0}
\\[-4pt]\multicolumn{3}{l}{\color{cyan}\rule{27pc}{1pt}}
\end{tabular}
\endTB
\end{table}

All {\tt Math} methods are {\tt static} {\it class methods} and are,
therefore, invoked through the class name. For example, we would
calculate $2^4$ as \mbox{\tt Math.pow(2,4)}, which evaluates to
16. Similarly, we compute the square root of 225.0 as
{\tt Math.sqrt(225.0)}, which evaluates to 15.0.

Indeed, Java's {\tt Math} class cannot be instantiated and cannot be
subclassed. Its basic definition is

\begin{jjjlisting}
\begin{lstlisting}
public final class Math    // Final, can't subclass
{   private Math() {}      // Private, can't invoke
    ...
    public static native double sqrt(double a)
           throws ArithmeticException;
}
\end{lstlisting}
\end{jjjlisting}

\noindent By declaring the {\tt Math} class {\tt public final},
we indicate that it can be accessed ({\tt public}) but it cannot be
extended or subclassed ({\tt final}). By declaring its default
constructor to be {\tt private}, we prevent this class from being
instantiated.  The idea of a class that cannot be subclassed and
cannot be instantiated may seem a little strange at first. The
justification for it here is that it provides a convenient and
efficient way to introduce helpful math functions into the Java
language.

%%%RAM\begin{textblock}{1}(-0.75,4)
%%%RAM\noindent\epsfig{file=e2kl-D1:PH-e2kl:Morelli:0333700MOREL:CommonArt:BX7_Texture/cyan.eps,%   height=51pc,width=3.25pc,clip=}
%%%RAM\end{textblock}

Defining the {\tt Math} class in this way makes it easy to use its
methods, because you don't have to create an instance of it. It is
also a very efficient design because its methods are static elements
of the {\tt java.lang} package.  This means they are loaded into memory
at the beginning of your program's execution, and they persist in
memory throughout your program's lifetime.  Because {\tt Math} class
methods do not have to be loaded into memory each time they are
invoked, their execution time will improve dramatically.

\JavaTIP[false]{EFFECTIVE DESIGN}{Static Methods.}{A method should be declared
{\tt static} if it is intended to be used whether or not there is an
instance of its class.}

\section{Numeric Processing Examples}

In this section we consider several numeric programming examples.
They are carefully chosen to illustrate different issues and concepts
associated with processing numeric data. 

\subsection{Example: Rounding to Two Decimal Places}
\noindent As an example of how to use {\tt Math} class methods,
let's consider the problem of rounding numbers. When dealing with
applications that involve monetary values---dollars and cents---it is
often necessary to round a calculated result to two decimal
places. For example, suppose a program computes the value of a
certificate of deposit (CD) to be 75.19999. Before we output this
result, we would want to round it to two decimal places---to 75.20.
\marginnote{Algorithm design}
The following algorithm can be used to accomplish this:

\begin{jjjlisting}
\begin{lstlisting}
1. Multiply the number by 100, giving 7519.9999.
2. Add 0.5 to the number giving 7520.4999.
3. Drop the fractional part giving 7520
4. Divide the result by 100, giving 75.20
\end{lstlisting}
\end{jjjlisting}

\noindent Step 3 of this algorithm can be done using the
{\tt Math.floor(R)} method, which rounds its real argument, {\em R},
to the largest integer not less than {\em R} (from Table~5.11). If the
number to be rounded is stored in the {\tt double} variable {\it R},
then the following expression will round {\tt R} to two decimal
places:

\begin{jjjlisting}
\begin{lstlisting}
R = Math.floor(R * 100.0 + 0.5) / 100.0;
\end{lstlisting}
\end{jjjlisting}

\noindent Alternatively, we could use the {\tt Math.round()} method
(Table~5.11). This method rounds a floating-point
value to the nearest integer. For example, {\tt Math.round(65.3333)}
rounds to 65 and {\tt Math.round(65.6666)} rounds to 66. The
following expression uses it to round to two decimal places:

\begin{jjjlisting}
\begin{lstlisting}
R = Math.round(100.0 * R) / 100.0;
\end{lstlisting}
\end{jjjlisting}

\noindent Note that it is important here to divide by $100.0$ and
not by $100$. Otherwise, the division will give an integer result and
we'll lose the two decimal places.

\JavaTIP[false]{DEBUGGING TIP}{Division.}{Using the correct type of literal 
in division operations is necessary to ensure that you get the correct
type of result.}

\subsection{Example: Converting Fahrenheit to Celsius}
\noindent To illustrate some of the issues that arise in using numeric
data, let's design a program that performs temperature conversions
from Fahrenheit to Celsius and vice versa.

\subsubsection*{Problem Decomposition}
\noindent This problem requires two classes, a {\tt Temperature} class and a
\marginnote{What objects do we need?}
{\tt TemperatureUI} class.  The {\tt Temperature} class will
perform the temperature conversions, and {\tt TemperatureUI} will
serve as the user interface (Fig.~\ref{fig-convapplobjs}).

%\begin{figure}
\begin{figure}[h]
\figaleft{chptr05/f5-2.eps}{Interacting objects: The user interacts with the user 
interface ({\tt Tem\-per\-a\-tureUI}), which interacts with the {\tt
Temperature} object.
} {fig-convapplobjs}

\end{figure}
%\end{figure}

%%%RAM\subsection{Class Design: {\ttHtwo Temperature} }
\subsubsection*{Class Design: {\tt Temperature} }
\noindent The purpose of the {\tt Temperature} class is to perform the
%\begin{marginalnote}\it What data do we need?\end{marginalnote}
temperature conversions.  To convert a Celsius temperature to
\marginnote{What data do we need?}
Fahrenheit or vice versa, it is not necessary to store the temperature
value.  Rather, a conversion method could take the Celsius (or
Fahrenheit) temperature as a parameter, perform the conversion, and
return the result. Therefore, the {\tt Temperature} class does not
need any instance variables. Note that in this respect the {\tt
Temperature} class resembles the {\tt Math} class. Unlike {\tt
OneRowNim}, which stores the game's state---the number of sticks
remaining and whose turn it is---the {\tt Math} and {\tt Temperature}
classes are stateless.

Thus, following the design of the {\tt Math} class, the {\tt
\marginnote{What methods do we need?}
Temperature} class will have two public static methods:
%\begin{marginalnote}\it What methods do we need?\end{marginalnote}
one to convert from Fahrenheit to Celsius and one to convert from
Celsius to Fahrenheit.  Recall that static methods are associated with
the class rather than with its instances. Therefore, we needn't
instantiate a {\tt Temperature} object to use these methods. Instead,
we can invoke the methods through the class itself.

The methods will use the standard conversion formulas: $F =
\frac{9}{5}C + 32$ and $C = \frac{5}{9}(F - 32)$. Each of these
methods should have a single parameter to store the temperature value
that is being converted.

Because we want to be able to handle temperatures such as 98.6, 
we should use real-number data for the methods' parameters.  Generally
speaking, because Java represents real literals such as 98.6 as {\tt
double}s, the {\tt double} type is more widely used than {\tt float}.
Because {\tt double}s are more widely used in Java, using {\tt double}
wherever a floating point value is needed will cut down on the number
of implicit data conversions that a program would have to perform.
Therefore, each of our conversion methods should take a {\tt double}
parameter and return a {\tt double} result.  These considerations lead

\JavaTIP{PROGRAMMING TIP}{Numeric Types.}{Java uses the {\tt int} type for
integer literals and {\tt double} for real-number literals.  When
possible, using {\tt int} and {\tt double} for numeric variables and
parameters reduces the number of implicit conversions a program would
have to perform.}

\subsubsection*{Implementation: {\tt Temperature}}

\noindent The implementation of the {\tt Temperature} class is shown in
Figure~\ref{fig-temperature}.  Note that because {\tt celsToFahr()}
uses the {\tt double} value {\tt temp} in its calculation, it uses
floating-point literals (9.0, 5.0, and 32.0) in its conversion
expression. This helps to reduce the reliance on Java's built-in
promotion rules, which can lead to subtle errors.  For example,
\marginfig{chptr05/p242f2.eps}{The {\tt Temperature} class. Note that static 
elements are underlined in UML.  (NEEDS REVISION)}
{fig-p242f2}
to the design shown in Figure~\ref{fig-p242f2},
suppose we had written what looks like an equivalent expression using
integer literals:

\begin{jjjlisting}
\begin{lstlisting}
return(9 / 5 * temp + 32); // Error: equals (temp + 32)
\end{lstlisting}
\end{jjjlisting}

%% proglist ch5/temperature/Temperature.java
\begin{figure}[b!]
\jjjprogstart
\begin{jjjlisting}
\begin{lstlisting}
public class Temperature
{   
    public Temperature() {}

    public static double fahrToCels(double temp)
    {   
        return (5.0 * (temp - 32.0) / 9.0);
    }
    public static double celsToFahr(double temp)
    {   
        return (9.0 * temp / 5.0 + 32.0);
    }
} // Temperature
\end{lstlisting}
\end{jjjlisting}
\jjjprogstop{The {\tt Temperature} class.}
{fig-temperature}
\end{figure}

\noindent Because 9 divided by 5 gives the integer result 1, this
expression is always equivalent to {\tt temp + 32}, which is not the
correct conversion formula.  This kind of subtle semantic error
%\begin{marginalnote}\it Semantic error\end{marginalnote}
\marginnote{Semantic error}
can be avoided if you avoid mixing types wherever possible.

\JavaTIP[false]{PROGRAMMING TIP}{Don't Mix Types.}{You can reduce 
the incidence of semantic errors caused by implicit type conversions
if, whenever possible, you explicitly change all the literals in an
expression to the same type.}

\subsubsection*{Testing and Debugging}

\noindent The next question to be addressed is how should 
this program be tested? As always, you should test
%\begin{marginalnote}\it Testing strategy\end{marginalnote}
\marginnote{Testing strategy}
the program in a stepwise fashion.  As each method is coded, you
should test it both in isolation and in combination with the other
methods, if \mbox{possible.}

Also, you should develop appropriate {\it test data}.
%\begin{marginalnote}\it Designing test data\end{marginalnote}
\marginnote{Designing test data}
It is not enough to just plug in any values.  The values you use should
test for certain potential problems.  For this program, the following
tests are appropriate:

\begin{BL}
\item  Test converting 0 degrees C to 32 degrees F.
\item  Test converting 100 degrees C to 212 degrees F.
\item  Test converting 212 degrees F to 100 degrees C.
\item  Test converting 32 degrees F to 0 degrees C.
\end{BL}

\noindent The first two tests use the {\tt celsToFahr()} method
to test the freezing point and boiling point temperatures, two
boundary values for this problem. A {\bf boundary value} is a value at
the beginning or end of the range of values that a variable or
calculation is meant to represent.  The second pair of tests performs
similar checks with the {\tt fahrToCels()} method.  One advantage of
using these particular values is that we know what results the methods
should return.

\JavaTIP{EFFECTIVE DESIGN}{Test Data.}{Developing appropriate test data is
an important part of program design.  One type of test data should
check the boundaries of the particular calculations you are making.}

\JavaTIP{DEBUGGING TIP}{Test, Test, Test!}{The fact that your 
program runs correctly on some data is no guarantee of its
correctness.  The more testing, and the more careful the testing you
do, the better.}

\subsubsection*{The {\tt TemperatureUI} Class}
\noindent The purpose of the {\tt TemperatureUI} class is to serve as a 
user interface---that is, as an interface between the user and a {\tt
Temperature} object.  It will accept a Fahrenheit or Celsius
temperature from the user, pass it to one of the public methods of the
{\tt Temperature} object for conversion, and display the result that
is returned. 

As we discussed in Chapter~4, the user interface can take various
forms, ranging from a command-line interface to a graphical
interface. Figure~\ref{fig-tempcmdline} shows a design for the user
interface based on the command-line interface developed in
Chapter~4. The {\tt TemperatureUI} uses a {\tt KeyboardReader} to
handle interaction with the user and uses {\tt static} methods in the
{\tt Temperature} class to perform the temperature conversions.

\begin{figure}[h!]
\figa{chptr05/tempui.eps}{A command-line user interface.}
{fig-tempcmdline}
\end{figure}


\secEXRHone{Self-Study Exercises}
\begin{SSTUDY}
\item  Following the design in Figure~\ref{fig-tempcmdline},
implement the {\tt TemperatureUI} class and use it to test the methods
in {\tt Temperature} class. The {\tt run()} method should use an {\bf
input-process-output} algorithm: Prompt the user for input, perform
the necessary processing, and output the result. Note that because
{\tt Temperature}'s conversion methods are class methods, you do not
need to instantiate a {\tt Temperature} object in this project. You
can invoke the conversion methods directly through the {\tt
Temperature} class:

\begin{jjjlisting}
\begin{lstlisting}
double fahr = Temperature.celsToFahr(98.6);
\end{lstlisting}
\end{jjjlisting}

\item  Following the design for the GUI developed in Chapter~4,
\marginfig{chptr05/5f5.eps}{Layout design of a GUI that
performs temperature conversions.}
{fig-convapplgui}
implement a GUI to use for testing the {\tt Temperature} class.
The GUI should have the layout shown in Figure~\ref{fig-convapplgui}.
\end{SSTUDY}

\subsection{Example: Using Class Constants}

\noindent As we noted in Chapter~\ref{chapter-intro}, in
addition to instance variables, which are associated with instances
(objects) of a class, Java also allows class variables, which are
associated with the class itself.  One of the most common uses of such
variables is to define named constants to replace literal values. A
{\bf named constant} is a variable that cannot be changed once it has
been given an initial value.  In this section, we use our running
example, {\tt OneRowNim}, to illustrate using class constants.

Recall that methods and variables that are associated with a class must
be declared with the {\tt static} modifier.  If a variable is declared
{\tt static}, there is exactly one copy of that variable created no
matter how many times its class is instantiated.  To turn a variable
into a constant, it must be declared with the {\tt final} modifier.
Thus, the following would be examples of a {\bf class constants},
constant values that are associated with the class rather than with
its instances:

\begin{jjjlisting}
\begin{lstlisting}
public static final int PLAYER_ONE = 1;
public static final int PLAYER_TWO = 2;
public static final int MAX_PICKUP = 3;
public static final int MAX_STICKS = 7;
\end{lstlisting}
\end{jjjlisting}

\noindent The {\tt final} modifier indicates that the value of
a variable cannot be changed. When {\tt final} is used in a variable
declaration, the variable must be assigned an initial value. After a
{\tt final} variable is properly declared, it is a syntax error to
attempt to try to change its value. For example, given the
preceding declarations, the following assignment statement would
cause a compiler error:

\begin{jjjlisting}
\begin{lstlisting}
PLAYER_ONE = 5;// Syntax error; PLAYER_ONE is a constant
\end{lstlisting}
\end{jjjlisting}

\noindent Note how we use uppercase letters and underscore
characters (\_) in the names of constants. This is a convention
that professional Java programmers follow, and its purpose is
to make it easy to distinguish the constants from the variables
in a program. This makes the program easier to read and understand.

\JavaTIP{PROGRAMMING TIP}{Readability.}{To make your programs more
readable, use uppercase font for constant identifiers.}

Another way that named constants improve the readability of a program
is by replacing the reliance on literal values.  For example, for the
{\tt OneRowNim} class, compare the following two {\tt if} conditions:

\begin{jjjlisting}
\begin{lstlisting}
if (num < 1 || num > 3 || num > nSticks) ...
if (num < 1 || num > MAX_PICKUP || num > nSticks) ...
\end{lstlisting}
\end{jjjlisting}

\noindent Clearly, the second condition is easier to read
and understand. In the first condition, we have no good idea what the
literal value 3 represents. In the second, we know that MAX\_PICKUP
represents the most sticks a player can pick up. 

Thus, to make {\tt OneRowNim} more readable, we should replace all
occurrences of the literal value 3 with the constant MAX\_PICKUP.
This same principle would apply to some of the other literal values in
the program.  Thus,
\marginnote{Readability}
instead of using 1 and 2 to represent the two players, we could use
PLAYER\_ONE and PLAYER\_TWO to make methods such as the following
easier to read and understand:

\begin{jjjlisting}
\begin{lstlisting}
public int getPlayer()
{   if (onePlaysNext) 
         return PLAYER_ONE;
    else return PLAYER_TWO;
} // getPlayer()
\end{lstlisting}
\end{jjjlisting}

\JavaTIP[false]{PROGRAMMING TIP}{Readability.}{To make your programs more
readable, use named constants instead of literal values.}

\noindent Another advantage of named constants (over literals) is that 
their use makes the program easier to modify and maintain. For
example, suppose that we decide to change {\tt OneRowNim} so that the
\marginnote{Maintainability}
maximum number of sticks that can be picked up is 4 instead of 3.  If
we used literal values, we would have to change all occurrences of 4
that were used to represent the maximum pick up. If we used a named
constant, we need only change its declaration to:

\begin{jjjlisting}
\begin{lstlisting}
public static final int MAX_PICKUP = 4;
\end{lstlisting}
\end{jjjlisting}

\JavaTIP{EFFECTIVE DESIGN}{Maintainability.}{Constants should be
used instead of literal values in a program.  This will make the
program easier to modify and maintain\index{Maintainability
Principle}.}

So far, all of the examples we have presented show why named constants
(but not necessarily class constants) are useful. Not all constants
are class constants. That is, not all constants are declared {\tt
static}.  However, the idea of associating constants with a class
makes good sense.  In addition to saving memory resources, by creating
just a single copy of the constant, constants such as MAX\_STICKS and
PLAYER\_ONE make more conceptual sense to associate with the class
itself rather than with any particular {\tt OneRowNim} instance.

Class constants are used extensively in the Java class library.  For
example, as we saw in Chapter~2, Java's various built-in colors are
represented as constants of the {\tt java.awt.Color} class---{\tt
Color.blue} and {\tt Color.red}. Similarly, {\tt java.awt.Label}
uses {\tt int} constants to specify how a label's text should be
aligned: {\tt Label.CENTER}.

Another advantage of class constants is that they can be used {\em
before} instances of the class exist.  For example, a class constant
(as opposed to an instance constant) may be used during object
instantiation:

\begin{jjjlisting}
\begin{lstlisting}
OneRowNim game = new OneRowNim(OneRowNim.MAX_STICKS);
\end{lstlisting}
\end{jjjlisting}

\noindent Note how we use the name of the class to refer to
the class constant. Of course, MAX\_STICKS has to be a public variable
in order to be accessible outside the class. To use MAX\_STICKS as a
constructor argument it has to be a class constant because at this
point in the program there are no instances of {\tt OneRowNim}.
A new version of {\tt OneRowNim} that uses class constants is
shown in Figure~\ref{fig-constnim}.

It is important to note that Java also allows class constants to be
referenced through an instance of the class. Thus, once we
have instantiated {\tt game}, we can refer to MAX\_STICKS with
either {\tt OneRowNim.MAX\_STICKS} or {\tt game.MAX\_STICKS}. 

\begin{figure}[p]
\jjjprogstart
\begin{jjjlisting}
\begin{lstlisting}
public class OneRowNim
{   public static final int PLAYER_ONE = 1;
    public static final int PLAYER_TWO = 2;
    public static final int MAX_PICKUP = 3;
    public static final int MAX_STICKS = 11;
    public static final boolean GAME_OVER = false;

     private int nSticks = MAX_STICKS;
     private boolean onePlaysNext = true;

     public OneRowNim()
     {
     } //OneRowNim() constructor1
     public OneRowNim(int sticks)
     {   nSticks = sticks;
     }  // OneRowNim() constructor2
     public OneRowNim(int sticks, int starter)
     {   nSticks = sticks;
         onePlaysNext = (starter == PLAYER_ONE);
     }  // OneRowNim() constructor3
     public boolean takeSticks(int num)
     {   if (num < 1 || num > MAX_PICKUP || num > nSticks) 
             return false;                // Error
         else                             // Valid move
         {   nSticks = nSticks - num;
             onePlaysNext = !onePlaysNext;
             return true;
         } //else
     }//takeSticks()
     public int getSticks()
     {   return nSticks;
     } //getSticks()
     public int getPlayer()
     {   if (onePlaysNext) 
             return PLAYER_ONE;
         else return PLAYER_TWO;
     } //getPlayer()
     public boolean gameOver()
     {   return (nSticks <= 0);
     } // gameOver()
     public int getWinner()
     {   if (nSticks < 1) 
             return getPlayer();
         else return 0;         // Game is not over
     } // getWinner()
     public String report()
     {   return ("Number of sticks left: " + getSticks()
          + "\nNext turn by player " + getPlayer() + "\n");
     }   // report()
} // OneRowNim class
\end{lstlisting}
\end{jjjlisting}
\jjjprogstop{This version of {\tt One\-Row\-Nim} uses named constants.}
{fig-constnim}
\end{figure}

\secEXRHone{Self-Study Exercise}
\begin{SSTUDY}

\item  Implement a command-line interface class named
{\tt KBTestOneRowNim}, that uses our new version of {\tt OneRowNim}.
Make use of the {\tt MAX\_STICKS} and {\tt MAX\_PICKUP} in the user
interface.  

\end{SSTUDY}

\subsection{OBJECT-ORIENTED DESIGN: \\ Information Hiding}
%%\secHfourleft{\mbox{OBJECT-ORIENTED} DESIGN: \\Information Hiding}

\noindent The fact that our new versions of {\tt OneRowNim}---we've developed
two new versions in this chapter---are {\it backward compatible} with
the previous version is due in large part to
\marginnote{Preserving the public interface}
the way we have divided up its public and private elements.  Because
the new versions still present the same public interface, programs
that use the {\tt OneRowNim} class, such as the {\tt OneRowNimApp}
from Chapter~4 (Fig.~4.24), can continue to use the class without
changing a single line of their own code. To confirm this, see the
Self-Study Exercise at the end of this section.

Although we have made significant changes to the underlying
\marginnote{Information hiding}
representation of {\tt OneRowNim}, the implementation details---its
data and algorithms---are hidden from other objects.  As long as {\tt
OneRowNim}'s public interface remains compatible with the old version,
changes to its private elements won't cause any inconvenience to those
objects that were dependent on the old version.  This ability to
change the underlying implementation without affecting the outward
functionality of a class is one of the great benefits of the
information hiding principle.

\JavaTIP{EFFECTIVE DESIGN}{Information Hiding.}{In designing a class, other
objects should be given access just to the information they need and
nothing more.}

The lesson to be learned here is that the public parts of a class
should be restricted to just those parts that must be accessible to
other objects.  Everything else should be private.  Things work better,
in Java programming and in the real world, when objects are designed
with the principle of information hiding in mind.

\secEXRHone{Self-Study Exercise}
\begin{SSTUDY}
\item  To confirm that our new version of {\tt OneRowNim} still works
correctly with the user interfaces we developed in Chapter~4, compile
and run it with {\tt OneRowNimApp} (Fig.~4.24).

\end{SSTUDY}

%\end{objectorienteddesign}


\subsection{Example: A Winning Algorithm for One Row Nim}

Now that we have access to numeric data types and operators, lets
develop an algorithm that can win the One Row Nim game. Recall that in
Chapter~4 we left things such that when the computer moves, it always
takes 1 stick. Let's replace that strategy with a more sophisticated
approach.

If you have played One Row Nim, you have probably noticed that in a
game with 21 sticks, you can always win the game if you leave your
opponent with 1, 5, 9, 13, 17, or 21 sticks.  This is obvious for the
case of 1 stick. For the case where you leave your opponent 5 sticks,
no matter what the opponent does, you can make a move that leaves the
other player with 1 stick. For example, if your opponent takes 1
stick, you can take 3; if your opponent takes 2, you can take 2; and,
if your opponent takes 3, you can take 1. In any case, you can win the
game by making the right move, if you have left your opponent with 5
sticks. The same arguments apply for the other values: 9, 13, 17, and
21.

What relationship is common to the numbers in this set? Notice
that if you take the remainder after dividing each of these numbers
by 4 you always get 1:

\begin{jjjlisting}
\begin{lstlisting}
   1 % 4  == 1
   5 % 4  == 1
   9 % 4  == 1
  13 % 4  == 1
  17 % 4  == 1
  21 % 4  == 1
\end{lstlisting}
\end{jjjlisting}

\noindent Thus, we can base our winning strategy on the goal of
leaving the opponent with a number of sticks, {\it N}, such that {\it
N \% 4} equals 1.  

To determine how many sticks to take in order to leave the opponent
with {\it N}, we need to use a little algebra.  Let's suppose that
{\tt sticksLeft} represents the number of sticks left before our
turn. The first thing we have to acknowledge is that if {\it sticksLeft \%
4 == 1}, then we have been left with 1, 5, 9, 13, and so on, sticks,
so we cannot force a win.  In that case, it doesn't matter how many
sticks we pick up. Our opponent should win the game.

So, let's suppose that {\it sticksLeft \% 4 != 1}, and let {\tt M} be
the number of sticks to pickup in order to leave our opponent with
{\tt N}, such that {\it N \% 4 == 1}. Then we have the following two
equations:

\begin{jjjlisting}
\begin{lstlisting}
sticksLeft - M == N    
N % 4 == 1
\end{lstlisting}
\end{jjjlisting}

\noindent We can combine these into a single equation, which can
be simplified as follows:

\begin{jjjlisting}
\begin{lstlisting}
(sticksLeft - M)  % 4 == 1
\end{lstlisting}
\end{jjjlisting}

\noindent If {\it sticksLeft - M} leaves a remainder of 1 when divided by
4, that means that {\it sticksLeft - M} is equal some integer quotient, {\it Q}
times 4 plus 1:

\begin{jjjlisting}
\begin{lstlisting}
(sticksLeft - M)   ==  (Q * 4) + 1
\end{lstlisting}
\end{jjjlisting}

\noindent By adding {\tt M} to both sides and subtracting 1 from both sides of
this equation, we get:

\begin{jjjlisting}
\begin{lstlisting}
  (sticksLeft - 1)   ==  (Q * 4) + M
\end{lstlisting}
\end{jjjlisting}

\noindent This equation is saying that {\it (sticksLeft - 1) \% 4 == M}. That is,
that when you divide {\it sticksLeft-1} by 4, you will get a remainder of {\tt M},
which is the number of sticks you should pick up.  Thus, to decide how  many 
sticks to take, we want to compute:

\begin{jjjlisting}
\begin{lstlisting}
M == (sticksLeft -1) % 4
\end{lstlisting}
\end{jjjlisting}

\noindent To verify this, let's look at some examples:

\begin{jjjlisting}
\begin{lstlisting}
  sticksLeft     (sticksLeft -1) % 4     sticksLeft 
   Before                                  After
  ----------------------------------------------------
    9            (9-1) % 4 == 0         Illegal Move
    8            (8-1) % 4 == 3         5
    7            (7-1) % 4 == 2         5
    6            (6-1) % 4 == 1         5
    5            (5-1) % 4 == 0         Illegal Move
\end{lstlisting}
\end{jjjlisting}

\noindent The examples in this table show that when we use {\it (sticksLeft-1 \% 4)}
to calculate our move, we always leave our opponent with a losing
situation.  Note that when {\tt sticksLeft} equals 9 or 5, we can't
apply this strategy because it would lead to an illegal move.

Let's now convert this algorithm into Java code. In addition to incorporating
our winning strategy, this {\tt move()} method makes use of two important
{\tt Math} class methods:

\begin{jjjlisting}
\begin{lstlisting}
public int move() 
{ int sticksLeft = nim.getSticks();// Get number of sticks
  if (sticksLeft % (nim.MAX_PICKUP + 1) != 1)// If winnable
    return (sticksLeft - 1) % (nim.MAX_PICKUP +1);
  else {                         // Else pick random
    int maxPickup = Math.min(nim.MAX_PICKUP, sticksLeft);
    return 1 + (int)(Math.random() * maxPickup);
  }
}
\end{lstlisting}
\end{jjjlisting}

\noindent The {\tt move()} method will return an {\tt int} representing the
best move possible. It begins by getting the number of sticks left
from the {\tt OneRowNim} object, which is referred to as {\tt nim} in
this case.  It then checks whether it can win by computing {\it
(sticksLeft-1) \% 4}. However, note that rather than use the literal
value 4, we use the named constant {\tt MAX\_PICKUP}, which is
accessible through the {\tt nim} object.  This is an especially good
use for the class constant because it makes our algorithm completely
general -- that is, our winning strategy will continue to work even if
the game is changed so that the maximum pickup is 5 or 6. The {\tt
then} clause computes and returns {\it (sticksLeft-1) \%
nim.MAX\_PICKUP+1}, but here again it uses the class constant.

The else clause would be used when it is not possible to make a winning
move.  In this case we want to choose a random number of sticks between 1
and some maximum number. The maximum number depends on how many sticks
are left. If there are more than 3 sticks left, then the most we can
pick up is 3, so we want a random number between 1 and 3. However, if
there are 2 sticks left, then the most we can pick up is 2 and we
want a random number between 1 and 2. Note how we use the {\tt Math.min()}
method to decide the maximum number of sticks that can be picked up:

\begin{jjjlisting}
\begin{lstlisting}
int maxPickup = Math.min(nim.MAX_PICKUP, sticksLeft);
\end{lstlisting}
\end{jjjlisting}

\noindent The {\tt min()} method returns the minimum value between its
two arguments. 

Finally, note how we use the {\tt Math.random()} method to calculate a
random number between 1 and the maximum:

\begin{jjjlisting}
\begin{lstlisting}
 1 + (int)(Math.random() * maxPickup);
\end{lstlisting}
\end{jjjlisting}

\noindent The {\tt random()} method returns a real number between 0 and 0.999999 --
that is, a real number between 0 and 1 but not including 1:

\begin{jjjlisting}
\begin{lstlisting}
 0 <= Math.random() < 1.0
\end{lstlisting}
\end{jjjlisting}

\noindent If we multiply {\tt Math.random()} times 2, the result would be a value
between 0 and 1.9999999.  Similarly, if we multiplied it by 3, the result would
be a value between 0 and 2.9999999.  In order to use the random value, we have
to convert it into an integer, which is done by using the {\tt (int)} cast operator:

\begin{jjjlisting}
\begin{lstlisting}
  (int)(Math.random() * maxPickup);
\end{lstlisting}
\end{jjjlisting}

\noindent Recall that when a {\tt double} is cast into an {\tt int}, Java just
throws away the fractional part. Therefore, this expression will give
us a value between 0 and {\tt maxPickup-1}. If {\tt maxPickup} is 3, this will
give a value between 0 and 2, whereas we want a random value between 1 and 3.
To achieve this desired value, we merely add 1 to the result. Thus, using the
expression

\begin{jjjlisting}
\begin{lstlisting}
  1 + (int)(Math.random() * maxPickup)
\end{lstlisting}
\end{jjjlisting}

\noindent gives us a random number between 1 and {\tt maxPickup}, where 
{\tt maxPickup} is either 1, 2, or 3, depending on the situation of
the game at that point.

\secEXRHone{Self-Study Exercise}
\marginfig{chptr05/nimplayer.eps}{The {\tt NimPlayer} class.}
{fig-nimplayer}
\begin{SSTUDY}

\item  Implement a class named {\tt NimPlayer} that
incorporates the {\tt move()} method designed in this section.  The
class should implement the design shown in Figure~5.9.
That is, in addition to the {\tt move()} method, it should have an
instance variable, {\tt nim}, which will serve as a reference to 
the {\tt OneRowNim} game. Its constructor method should take
a {\tt OneRowNim} parameter, allowing the {\tt NimPlayer} to be 
given a reference when it is instantiated.

\item Modify {\tt OneRowNim}'s command-line interface
to play One Row Nim between the user and the computer, where the {\tt
NimPlayer} implemented in the previous exercise represents the
computer.

\end{SSTUDY}


\WWWjava 
\section{{\bf From the Java Library} \\{\tt java.text.NumberFormat}}
%\addcontentsline{toc}{section}{\S~~~ From the Java Library: The {\tt NumberFormat} Class}
\tBOXseven{Although} the {\tt Math.round()} method is useful for rounding
%\begin{figure}
%\begin{graphic}
\marginfig{chptr05/p267f1.eps}{The {\tt java.text.NumberFormat} class.}
{fig-pg267f1}
numbers, it is not suitable for business applications. Even
for rounded values, Java will drop trailing zeroes. So a value
such as \$10,000.00 would be output as \$10000.0. This wouldn't
be acceptable for a business report.

Fortunately, Java supplies the {\tt java.text.NumberFormat}
class precisely for the task of representing numbers as dollar
amounts, percentages, and other formats (Fig.~\ref{fig-pg267f1}).

The {\tt NumberFormat} class is an {\tt abstract} class, which means
that it cannot be directly instantiated. Instead, you would use its
static {\tt getInstance()} methods to create an instance that can then
be used for the desired formatting tasks.

Once a {\tt NumberFormat} instance has been created, its {\tt
format()} method can be used to put a number into a particular
format. The {\tt setMaximumFractionDigits()} and {\tt
setMaximumIntegerDigits()} methods can be used to control the number
of digits before and after the decimal point.

For example, the following statements can be used to format a
decimal number as a currency string in dollars and cents:
%\end{graphic}
%\end{figure}

\begin{jjjlisting}
\begin{lstlisting}
NumberFormat dollars = NumberFormat.getCurrencyInstance();
System.out.println(dollars.format(10962.555));
\end{lstlisting}
\end{jjjlisting}

\noindent These statements would cause the value 10962.555 to
be shown as \$10,962.56. Similarly, the  statements,

\begin{jjjlisting}
\begin{lstlisting}
NumberFormat percent = NumberFormat.getPercentInstance();
percent.setMaximumFractionDigits(2);
System.out.println(percent.format(6.55));
\end{lstlisting}
\end{jjjlisting}

\noindent would display the value 6.55 as 6.55\%. The 
utility of the {\tt Math} and {\tt NumberFormat} classes illustrates
the following principle:

\JavaTIP[false]{EFFECTIVE DESIGN}{Code Reuse.}{Often the best way to
solve a programming task is to find the appropriate methods in the
Java class library.}

%%%RAM\end{javalibrary}

\secEXRHone{Self-Study Exercise}
\begin{SSTUDY}

\item A Certificate of Deposit (CD) is an investment instrument
that accumulates interest at a given rate for an initial principal
\marginfig{chptr05/bankcd.eps}{The {\tt BankCD} class.}
{fig-bankcd}
over a fixed number of years.  The formula for compounding interest is
shown in Table~\ref{fig-bankcd}. It assumes that interest is compounded
annually. For daily compounding, the annual rate must be divided by
365, and the compounding period must be multiplied by 365, giving: $a
= p(1 + r/365)^{365n}$.  Implement a {\tt BankCD} class that
calculates the maturity value of a CD. Figure~5.11 gives the design
of the class. It should have three instance variables for the CD's
principal, rate, and years. Its constructor method sets the initial
values of these variables when a {\tt BankCD} is instantiated. Its two
public methods calculate the maturity value using yearly and daily
compounding interest, respectively.  Use the {\tt Math.pow()} method
to calculate maturity values. For example, the following expression
calculates maturity value with annual compounding:

\begin{jjjlisting}
\begin{lstlisting}
principal * Math.pow(1 + rate, years)
\end{lstlisting}
\end{jjjlisting}

\begin{table}[htb]
%\hphantom{\caption{Formula for calculating a CD's maturity value.}}
\TBT{3.5pc}{Formula for calculating compound interest}
\hspace*{3pc}\begin{tabular}{l}
\multicolumn{1}{l}{\color{cyan}\rule{20pc}{1pt}}\\[2pt]
           \multicolumn{1}{l}{$a = p(1 + r)^n$ where}\\[6pt]
$\bullet$\hspace*{5pt}{\it a} is the CD's value at the end of the {\it n}th year\\
$\bullet$\hspace*{5pt}{\it p} is the principal or original investment amount\\
$\bullet$\hspace*{5pt}{\it r} is the annual interest rate\\
$\bullet$\hspace*{5pt}{\it n} is the number of years or the compounding period
\\[-4pt]\multicolumn{1}{l}{\color{cyan}\rule{20pc}{1pt}}
\end{tabular}
\endTB
\end{table}

\item Design a command-line user interface to the {\tt BankCD} class
that lets the user input principal, interest rate, and years, and reports
the CD's maturity value with both yearly and daily compounding. Use 
{\tt NumberFormat} objects to display percentages and dollar figures in
an appropriate format.  The program's output should look something like
the following (user's inputs are in {\color{cyan}cyan}):

\vspace*{10pt plus5pt minus2pt}
\small
\begin{alltt}
************************ OUTPUT ********************
Compare daily and annual compounding for a Bank CD.
 Input CD initial principal, e.g.  1000.55 > {\color{cyan}2500}
 Input CD interest rate, e.g.  6.5 > {\color{cyan}7.8}
 Input the number of years to maturity, e.g., 10.5 > {\color{cyan}5 }
For Principal = $2,500.00 Rate= 7.8% Years= 5.0
 The maturity value compounded yearly is $3,639.43
 The maturity value compounded daily is: $3,692.30
************************ OUTPUT ********************
\end{alltt}
\normalsize

\end{SSTUDY}


\section{Character Data and Operators}

\noindent Another primitive data type in Java is the character type,
{\tt char}\index{char}. A character in Java is represented by a 16-bit
{\em unsigned} integer.  This means that a total of $2^{16}$ or 65536
%\begin{marginalnote}\it Unicode\end{marginalnote}
\marginnote{Unicode}
different Unicode characters can be represented, corresponding to the
integer values 0 to 65535. The {\bf Unicode} character set is an
international standard that has been developed to enable computer
languages to represent characters in a wide variety of languages, not
just English.  Detailed information about this encoding can be
obtained at

\begin{jjjlisting}
\begin{lstlisting}[commentstyle=\color{black}]
http://www.unicode.org/
\end{lstlisting}
\end{jjjlisting}

It is customary in programming languages to use unsigned integers to
represent characters.  This means that all the digits ($0, \dots,9$),
alphabetic letters ($a,\dots,z, A,\dots, Z$), punctuation symbols
(such as . ; , `` `' ! \_ -), and nonprinting control characters
(LINE\_FEED, ESCAPE, CARRIAGE\_RETURN, $\dots$) that make up the
computer's character set are represented in the computer's memory by
integers.  A more traditional set of characters is the {\it ASCII
(American Standard
%\begin{marginalnote}\it ASCII code\end{marginalnote}
\marginnote{ASCII code}
Code for Information Interchange)} character set.  ASCII is based on a
7-bit code and, therefore, defines $2^7$ or 128 different characters,
corresponding to the integer values 0 to 127. In order to make Unicode
backward compatible with ASCII\index{ASCII character set} systems, the
first 128 Unicode characters are identical to the ASCII
characters.  Thus, in both the ASCII and Unicode encoding, the printable
characters have the integer values shown in Table~5.13.

\begin{table}[htb]
%\hphantom{\caption{ASCII codes for selected characters}}
\TBT{0pc}{ASCII\index{ASCII} codes for selected characters}
%%%\vspace{-6pt}
\begin{jjjlisting}
\begin{lstlisting}[stringstyle=\color{black}]
 Code   32 33 34 35 36 37 38 39 40 41 42 43 44 45 46 47
 Char   SP !  "  #  $  %  &  '  (  )   *  +  ,  -  .  /

 Code   48 49 50 51 52 53 54 55 56 57
 Char   0  1  2  3  4  5  6  7  8  9

 Code   58 59 60 61 62 63 64
 Char   :  ;  <  =  >  ?  @

 Code   65 66 67 68 69 70 71 72 73 74 75 76 77
 Char   A  B  C  D  E  F  G  H  I  J  K  L  M

 Code   78 79 80 81 82 83 84 85 86 87 88 89 90
 Char   N  O  P  Q  R  S  T  U  V  W  X  Y  Z

 Code   91 92 93 94 95 96
 Char   [  \  ]  ^  _  `

 Code   97 98 99 100 101 102 103 104 105 106 107 108 109
 Char   a  b  c  d   e   f   g   h   i   j   k   l   m

 Code   110 111 112 113 114 115 116 117 118 119 120 121 122
 Char   n   o   p   q   r   s   t   u   v   w   x   y   z

 Code   123 124 125 126
 Char   {   |   }   ~
\end{lstlisting}
\end{jjjlisting}
\end{table}

\subsection{Character to Integer Conversions}

\noindent Is `A' a character or an integer?  The fact that character data are
stored as integers in the computer's memory can cause some confusion
about whether a given piece of data is a character or an integer.  In
other words, when is a character, for example `A', treated as the integer (65)
instead of as the character `A'?  The rule in Java is that a character
literal---`a' or `A' or `0' or `?'---is always treated as a
character, unless we explicitly tell Java to treat it as an
integer. So if we display a literal's value

\begin{jjjlisting}
\begin{lstlisting}
System.out.println('a');
\end{lstlisting}
\end{jjjlisting}

\noindent the letter `a' will be displayed. Similarly, if we assign `a' to a
{\tt char} variable and then display the variable's value,


\begin{jjjlisting}
\begin{lstlisting}
char ch = 'a';
System.out.println(ch);         // Displays 'a'
\end{lstlisting}
\end{jjjlisting}

\noindent the letter `a' will be shown. If, on the other hand, we wish to output a
character's integer value, we must use an explicit cast operator
as follows:

\begin{jjjlisting}
\begin{lstlisting}
System.out.println((int)'a') ;   // Displays 97
\end{lstlisting}
\end{jjjlisting}

\noindent A {\bf cast operation}, such as {\tt (int)}, 
converts one type of data ('a') into another (97). This is known as a
{\bf type conversion}. Similarly, if we wish to store a character's
integer value in a variable, we can {\it cast} the {\tt char} into an
{\tt int} as follows:

\begin{jjjlisting}
\begin{lstlisting}
int k = (int)'a';       // Converts 'a' to 97
System.out.println(k);  // Displays 97
\end{lstlisting}
\end{jjjlisting}

\noindent As these examples show, a {\it cast} is a type conversion operator.
%\begin{marginalnote}\it The cast operator\end{marginalnote}
\marginnote{The cast operator}
Java allows a wide variety of both explicit and implicit type
conversions.  Certain conversions (for example, promotions) take place
when methods are invoked, when assignment statements are executed,
when expressions are evaluated, and so on.

Type conversion in Java is governed by several rules
and exceptions. In some cases Java allows the programmer to make
implicit cast conversions. For example, in the following assignment
a {\tt char} is converted to an {\tt int} even though no
explicit cast operator is used:

\begin{jjjlisting}
\begin{lstlisting}
char ch;
int k;
k = ch; // convert a char into an int
\end{lstlisting}
\end{jjjlisting}

\noindent Java permits this conversion because no information will be lost.  A
%\begin{marginalnote}\it Implicit type conversion\end{marginalnote}
\marginnote{Implicit type conversion}
character {\tt char} is represented in 16 bits whereas an {\tt int} is
represented in 32 bits.  This is like trying to put a small object
into a large box. Space will be left over, but the object will fit
inside without being damaged. Similarly, storing a 16-bit {\tt char}
in a 32-bit {\tt int} will leave the extra 16 bits unused. This {\it
widening primitive conversion} changes one primitive type ({\tt char})
into a wider one ({\tt int}), where a type's {\it
%\begin{marginalnote}\it Widening conversion\end{marginalnote}
\marginnote{Widening conversion}
width} is the number of bits used in its representation.

On the other hand, trying to assign an {\tt int} value
to a {\tt char} variable leads to a syntax error:

\begin{jjjlisting}
\begin{lstlisting}
char ch;
int k;
ch = k;    // Syntax error: can't assign int to char
\end{lstlisting}
\end{jjjlisting}

\noindent Trying to assign a 32-bit {\tt int} to
16-bit {\tt char} is like trying to fit a big object into an
undersized box.  The object won't fit unless we shrink it in some way.
Java will allow us to assign an {\tt int} value to a {\tt char}
variable, but only if we perform an explicit cast on it:


\begin{jjjlisting}
\begin{lstlisting}
ch = (char)k; // Explicit cast of int k into char ch
\end{lstlisting}
\end{jjjlisting}

\noindent The {\tt (char)} cast operation performs a careful ``shrinking''
of the {\tt int} by lopping off the last 16 bits of the {\tt int}.
This can be done without loss of information provided that {\it k}'s value
is in the range 0 to 65535---that is, in the range of values that fit into
%\begin{marginalnote}\it Narrowing conversion\end{marginalnote}
\marginnote{Narrowing conversion}
a {\tt char} variable.  This {\it narrowing primitive conversion}
changes a wider type (32-bit {\tt int}) to a narrower type (16\mbox{-}bit
{\tt char}).  Because of the potential here for information loss, it is up
to the programmer to determine that the cast can be performed safely.

\JavaRule{Type Conversion.}{Java permits {\it implicit}
type conversions\index{type conversion rule} from a narrower type to a
wider type. A {\it cast} operator must be used when converting a wider
type into a narrower type.}

\noindent The cast operator can be used with any primitive type. It applies to
the variable or expression that immediately follows it. Thus,
parentheses must be used to cast the expression {\it m + n} into a
{\tt char}:

\begin{jjjlisting}
\begin{lstlisting}
char ch = (char)(m + n);
\end{lstlisting}
\end{jjjlisting}

\noindent The following statement would cause a syntax error because
the cast operator would only be applied to {\tt m}:

\begin{jjjlisting}
\begin{lstlisting}
char ch = (char)m + n; // Error: right side is an int
\end{lstlisting}
\end{jjjlisting}

\noindent In the expression on the right-hand side, the
character produced by {\tt (char)m} will be promoted to an {\tt int}
because it is part of an integer operation whose result will still be
an {\tt int}.  Therefore, it cannot be assigned to a {\tt char} without
an explicit cast.

\secEXRHone{Self-Study Exercise}
%% \begin{SSTUDY}
%% \item  Suppose that {\it m} and {\it n} are integer variables of type
%% {\tt int} and that {\it ch1} and {\it ch2} are character variables of type
%% {\tt char}.  Determine in each of the  cases that follow whether the
%% assignment statements are valid. If not, modify the statement to make
%% it valid.

%% \begin{EXRLL}
%% \begin{multicols}{3}
%% \item  \verb!m = n;!         
%% \item  \verb!m = ch1;!       
%% \item  \verb!ch2 = n;!       
%% \item  \verb!ch1 = ch2;!     
%% \item  \verb!ch1 = m - n;!   
%% \end{multicols}
%% \end{EXRLL}

%% \end{SSTUDY}


\subsection{Lexical Ordering}
\noindent The order in which the characters of a character set 
are arranged, their {\bf lexical order}, is an important feature
of the character set. It especially comes into play for such
tasks as arranging strings in alphabetical order. 

Although the actual integer values assigned to the individual
characters by ASCII and UNICODE encoding seem somewhat arbitrary, the
characters are, in fact, arranged in a particular order.  For example,
note that various sequences of digits, {\tt '0'$\ldots$'9'}, and
letters, {\tt 'a'$\ldots$'z'} and {\tt 'A'$\ldots$'Z'}, are
represented by sequences of integers (Table 5.11). This makes it
possible to represent the lexical order of the characters in terms of
the {\it less than} relationship among integers.  The fact that `a'
comes before `f' in alphabetical order is represented by the fact that
97 (the integer code for `a') is less than 102 (the integer code for
`f'). Similarly, the digit `5' comes before the digit `9' in an
alphabetical sequence because 53 (the integer code for `5') is less
than 57 (the integer code for `9').

This ordering relationship extends throughout the character set. Thus,
it is also the case that `A' comes before `a' in the lexical ordering
because 65 (the integer code for `A') is less than 97 (the integer
code for `a').  Similarly, the character `[' comes before `\verb|}|' because
its integer code (91) is less than 125, the integer code for `\verb|}|'.

\subsection{Relational Operators}
\noindent Given the lexical ordering of the {\tt char} type, the following
relational operators can be defined: $<$, $>$, $<$=, $>$=, ==, !=.
Given any two characters, {\it ch1} and {\it ch2}, the expression
{\it ch1 $<$ ch2} is true if and only if the integer value of {\it ch1} is
%\begin{marginalnote}{\tt char} {\it relations}\end{marginalnote}
\marginnote{{\tt\color{cyan}char} {\it\color{cyan}relations}}
less than the integer value of {\it ch2}.  In this case we say that
{\it ch1} {\it precedes} {\it ch2} in lexical order. Similarly, the
expression {\it ch1 $>$ ch2} is true if and only if the integer value of
{\it ch1} is greater than the integer value of {\it ch2}.  In this
case we say that {\it ch1} {\it follows} {\it ch2}.  And so on for the
other relational operators.  This means that we can perform comparison
operations on any two character operands (Table~5.14).

\begin{table}[htb]
%\hphantom{\caption{Relational operations on characters}}
%%%\vspace{-6pt}
\TBT{2.5pc}{Relational operations\index{relational operators} on characters}
\hspace*{2pc}\begin{tabular}{llll}
\multicolumn{4}{l}{\color{cyan}\rule{22pc}{1pt}}\\[2pt]
%%%RAM\TBCH{Operation}&\TBCH{Operator}&\TBCH{Java}&\TBCH{True Expression}
{Operation}&{Operator}&{Java}&{True Expression}
\\[-4pt]\multicolumn{4}{l}{\color{cyan}\rule{22pc}{0.5pt}}\\[2pt]
{\it Precedes}&$<$ &$ch1\ <\ ch2$&$'a'\ <\ 'b'$\cr
{\it Follows}&$>$ &$ch1\ >\ ch2$&$'c'\ >\ 'a'$\cr
{\it Precedes or equals}&$<=$&$ch1\ <=\ ch2$&$'a'\ <=\ 'a'$\cr
{\it Follows or equals}&$>=$&$ch2\ >=\ ch1$&$'a'\ >=\ 'a'$\cr
{\it Equal to}&$= =$&$ch1\ ==\ ch2$&$'a'\ ==\ 'a'$\cr
{\it Not equal to}&$!\!=$&$ch1\ !\!=\ ch2$&$'a'\ !\!=\ 'b'$
\\[-4pt]\multicolumn{4}{l}{\color{cyan}\rule{22pc}{1pt}}
\end{tabular}
\endTB
\end{table}


\section{Example: Character Conversions }
\noindent Another interesting implication of representing the
characters\index{character conversion} as integers is that we can
represent various character operations in terms of integer
operations. For example, suppose we want to capitalize a lowercase
%\begin{marginalnote}\it Lowercase to uppercase\end{marginalnote}
\marginnote{Lowercase to uppercase}
letter. Table~5.13 shows that the entire sequence of lowercase
letters {\tt ('a'~\dots~'z')} is displaced by 32 from the sequence of
uppercase letters {\tt ('A'~\dots~'Z')}, so we can convert any lowercase
letter into its corresponding uppercase letter by subtracting 32 from
its integer value, provided we perform an explicit cast on the
result. When we perform the cast {\tt (char) ('a'~-~32 )}, the
resulting value is {\tt 'A'}, as the following example shows:

\begin{jjjlisting}
\begin{lstlisting}
(char)('a' - 32)              ==>  'A'
\end{lstlisting}
\end{jjjlisting}

\noindent Recall that in evaluating  {\tt 'a' - 32}
Java will promote `a' to an {\tt int} and then perform the
subtraction.  Thus, a step-by-step evaluation of the expression
would go as follows:

\begin{jjjlisting}
\begin{lstlisting}
Step 1. (char)((int)'a' - 32)// Promote 'a' to int
Step 2. (char)(97 - 32)      // Subtract
Step 3. (char) (65)          // Cast result to a char
Step 4. 'A'                  // Results in 'A'
\end{lstlisting}
\end{jjjlisting}

\noindent Similarly, we can convert an uppercase letter into the corresponding
%\begin{marginalnote}\it Uppercase to lowercase\end{marginalnote}
\marginnote{Uppercase to lowercase}
lowercase letter by simply adding 32 to its integer code and casting
the result back to a char:

\begin{jjjlisting}
\begin{lstlisting}
(char)('J' + 32)              ==>  'j'
\end{lstlisting}
\end{jjjlisting}

\noindent We can group these ideas into a method that performs
conversion from lowercase to uppercase:

\begin{jjjlisting}
\begin{lstlisting}
char toUpperCase(char ch) {
  if ((ch >= 'a') && (ch <= 'z'))
    return ch - 32 ;  // Error: can't return an int
  return ch;
}
\end{lstlisting}
\end{jjjlisting}

\noindent This method\index{toUpperCase() method}
takes a single {\tt char} parameter and returns a {\tt char} value. It
begins by checking if {\it ch} is a lowercase letter---that is, if
{\it ch} falls between `a' and `z' inclusive.  If so, it returns the result
of subtracting 32 from {\it ch}.  If not, it returns {\it ch}
unchanged. However, the method contains a syntax error that becomes
%\begin{marginalnote}\it Type error\end{marginalnote}
\marginnote{Type error}
apparent if we trace through its steps. If we invoke it with the
expression {\tt toUpperCase('b')}, then since {\it `b'} is between {\it `a'}
and {\it `z'}, the method will return {\it `b' $-$ 32}.  Because the integer
value of {\it `b'} is 98, it will return {\it 98 $-$ 32} or 66, which is the integer
code for the character {\it `B'}.  However, the method is supposed to return
a {\tt char}, so this last statement will generate the following syntax
error:


\begin{jjjlisting}
\begin{lstlisting}
Incompatible type for return. An explicit cast needed 
to convert int to char.
>>    return ch - 32 ;
>>    ^
\end{lstlisting}
\end{jjjlisting}

\noindent In order to avoid this error, the result must be
converted back to {\tt char} before it can be returned:

\begin{jjjlisting}
\begin{lstlisting}
char toUpperCase (char ch) {
  if ((ch >= 'a') && (ch <= 'z'))
    return (char)(ch - 32);  // Explicit cast
  return ch;
}
\end{lstlisting}
\end{jjjlisting}

\noindent Another common type of conversion is to convert a
digit\index{digitToInteger() method} to its corresponding integer
value.  For example, we convert the character `9' to the integer 9 by
%\begin{marginalnote}\it Digit to integer\end{marginalnote}
\marginnote{Digit to integer}
making use of the fact that the digit `9' is 9 characters beyond the
digit `0' in the lexical order.  Therefore, subtracting `0' from `9'
gives integer 9 as a result:

\begin{jjjlisting}
\begin{lstlisting}
('9' - '0')  ==> (57 - 48) ==>   9
\end{lstlisting}
\end{jjjlisting}

\noindent More generally, the expression {\it $ch -$ `0'} will convert
any digit, {\it ch}, to its integer value. We can encapsulate these
ideas into a method that converts any digit into its corresponding
integer value:

\begin{jjjlisting}
\begin{lstlisting}
int digitToInteger(char ch) {
  if ((ch >= '0') && (ch <= '9'))
    return ch - '0';
  return -1 ;
}
\end{lstlisting}
\end{jjjlisting}

\noindent This method takes a single {\tt char} parameter and returns an
{\tt int}. It first checks that {\it ch} is a valid digit, and if
so, it subtracts the character `0' from it. If not, the method just
returns $-1$, which indicates that the method received an invalid input
parameter.  Obviously, when an object invokes this method, it should
first make sure that the value it passes is in fact a digit.

The Java application program shown in Figure~\ref{fig-testapplet}
illustrates several of the ideas discussed in this section. Note that
both the {\tt digitToInteger()} and {\tt toUpperCase()} are declared
static. This allows us to call them directly from the (static) {\tt
main()} method, as useful and justifiable shortcut if, as in this case,
we are just testing the methods. 

%% proglist ch5/charconvert/Test.java
\begin{figure}[h]
\jjjprogstart
\begin{jjjlisting}
\begin{lstlisting}
public class Test {
    public static void main(String argv[]) {
        char ch = 'a';          // Local variables
        int k = (int)'b';
        System.out.println(ch);
        System.out.println(k);  
        ch = (char)k;           // Cast needed here
        System.out.println(ch);
        System.out.println(toUpperCase('a'));
        System.out.println(toUpperCase(ch));
        System.out.println(digitToInteger('7'));
    }
    public static char toUpperCase(char ch) {
        if ((ch >= 'a') && (ch <= 'z'))
            return (char)(ch - 32);
        return ch;
    }
    public static int digitToInteger(char ch) {
        if ((ch >= '0') && (ch <= '9'))
            return ch - '0';
        return -1 ;
    }
} // Test
\end{lstlisting}
\end{jjjlisting}
\jjjprogstop{A Java program illustrating character\index{character
conversion example} conversions. When run, the program will generate
the following outputs, one per line: a, 98, b, A, B, 7.}
{fig-testapplet}
\end{figure}

\section{Problem Solving = Representation + Action}
\noindent As you have seen in this chapter, designing classes involves 
a careful interplay between representation (data) and action
(methods).  Our several modifications to the {\tt OneRowNim} class
illustrate that the data used to represent an object's state can
either complicate or simplify the design of the methods needed to
solve a problem.

We hope that it is becoming clear to you that in writing
object-oriented programs, choosing an appropriate data representation
is just as important as choosing the correct algorithm. The concept of
an object allows us to encapsulate representation and action into a
single entity. It is a very natural way to approach problem solving.

If you look closely enough at any problem, you will find this close
relationship between representation and action.  For example, compare
the task of performing multiplication using Arabic numerals---65 *
%\epage
12 = 380---and the same task using Roman numerals---LXV * XII =
DCCLXXX.~It's doubtful that our science and technology would be where
they are today if our civilization had to rely forever on the Roman way
of representing numbers!

\JavaTIP{EFFECTIVE DESIGN}{Representation and Action.}{Representation
(data) and action (methods) are equally important parts of the
problem-solving process.}

\secSMHleft{Chapter Summary}
%\addcontentsline{toc}{section}{\S~~~ Chapter Summary}\
\secKTH{Technical Terms}

\begin{KT}
action

binary operator

binary digit (bit) 

boundary value 

cast operation

class constant

input-process-output

named constant

operand

operator overloading

precedence order

promotion

round off error

short-circuit evaluation

type conversion

unary operator

Unicode

representation

\end{KT}

\secSMHtwo{Summary of Important Points}
\begin{SMBL}
\item  The way we approach a problem can often help
or hinder us in our ability to solve it. Choosing an appropriate {\it
representation} for a problem is often the key to solving it.

\item  In order to evaluate complex expressions, it is necessary to
understand the {\it precedence order} and {\it associativity} of the
operators involved. Parentheses can always be used to override an
operator's built-in precedence.

\item  Java provides several types of integer data, including the
8-bit {\tt byte}, 16-bit {\tt short}, 32-bit {\tt int}, and 64-bit
{\tt long} types. Unless otherwise specified, integer literals are
represented as {\tt int} data in a Java program.

\item  Java provides two types of floating-point data, the 32-bit
{\tt float} type and the 64-bit {\tt double} type. Unless otherwise
specified, floating-point literals are represented as {\tt double}
data.

\item  In general, if a data type uses {\it n} bits in its representation,
then it can represent $2^n$ different values.

\item  The fact that Java's primitive types are defined in terms
of a specific number of bits is one way that Java promotes {\it
platform independence}.

\item  It is necessary to distinguish integer operations from
floating-point operations even though the same symbols are used.
For example, (7/2) is 3, while (7.0/2) is 3.0.

\item  In revising a class
that is used by other classes it is important to preserve as much of
the class's {\it interface} as possible.

\item  In Java, character data are based on the Unicode character set,
which provides $2^{16}$ = 65,536 different character codes. To provide
backward compatibility with the ASCII code, the first 128 characters
are the ASCII coded characters.

\item Java operators are evaluated according to the precedence hierarchy
shown in Table~5.15. The lower the precedence
number, the earlier an operator is evaluated. So the operators at the
top of the table are evaluated before operators that occur below them
in the table.  Operators at the same precedence level are evaluated
according to their {\it association}, either left to right (L to R)
or right to left (R to L).

\begin{table}[htb]
%\hphantom{\caption{Java operator precedence and associativity table}}
\TBT{0pc}{Java operator precedence and associativity table\index{precedence table}}
\hspace*{-6pt}\begin{tabular}{clll}
\multicolumn{4}{r}{\color{cyan}\rule{27pc}{1pt}}\\[2pt]
%%%RAM\TBCH{Order}&\TBCH{Operator}&\TBCH{Operation}&\TBCH{Association}
{Order}&{Operator}&{Operation}&{Association}
\\[-4pt]\multicolumn{4}{r}{\color{cyan}\rule{27pc}{0.5pt}}\\[2pt]
0&{\tt (  )}&{\it Parentheses}&\cr
1&{\tt ++   --  . }&{\it Postincrement, postdecrement,} \\
&&\quad {\it dotOperator}&{\it L to R}\cr
2&{\tt ++   --  +  -  !}&{\it Preincrement, predecrement}&{\it R to L}\cr
&{\tt }&\quad {\it Unary plus, unary minus,} \\
&&\quad {\it boolean NOT}&{\it }\cr
3&{\tt $(type)\; new$}&{\it Type cast, object instantiation}&{\it R to L}\cr
4&{\tt *  /  \%}&{\it Multiplication, division, modulus}&{\it L to R}\cr
5&{\tt + -  +}&{\it Addition, subtraction, string} \\
&&\quad {\it concatenation}&{\it L to R}\cr
6&\verb!<  >  <=  >=!&{\it Relational operators}&{\it L to R}\cr
7&{\tt ==   !=}&{\it Equality operators}&{\it L to R}\cr
8&{\tt $\wedge$}&{\it Boolean XOR}&{\it L to R}\cr
9&{\tt \&\&}&{\it Boolean AND}&{\it L to R}\cr
10&{\tt $\mid\mid$}&{\it Boolean OR}&{\it L to R}\cr
11&{\tt = += -= *= /= \%=}&{\it Assignment operators}&{\it R to L}
\\[-4pt]\multicolumn{4}{r}{\color{cyan}\rule{27pc}{1pt}}
\end{tabular}
\endTB
\end{table}

\end{SMBL}

%\section*{Answers to Self-Study Exercises}
%% \secANSH
%% \begin{ANS}
%% \begin{multicols}{4}
%% \item \mbox{ }


%% \begin{enumerate}
%% \item[a.]  true 
%% \item[b.]  false
%% \item[c.]  true
%% \item[d.]  false
%% \item[e.]  false
%% \item[f.]  false
%% \end{enumerate}
%% \end{multicols}

%% \item  \mbox{ }

%% \begin{jjjlisting}
%% \begin{lstlisting}
%% 0000, 0001, 0010, 0011, 0100, 0101, 0110, 0111,
%% 1000, 1001, 1010, 1011, 1100, 1101, 1110, 1111
%% \end{lstlisting}
%% \end{jjjlisting}

%% \item  In 6 bits, you can represent $2^6 = 64$ different values.

%% \item  If you have to represent up to 12 significant digits, you should use
%% {\tt double}, which goes up to 17 digits.

%% \begin{multicols}{4}
%% \item  \mbox{ }

%% \mbox{ }


%% \begin{enumerate}
%% \item[a.]  4
%% \item[b.]  4
%% \item[c.]  0
%% \item[d.]  1
%% \item[e.]  2
%% \item[f.]  6
%% \item[g.]  0
%% \item[h.]  4
%% \end{enumerate}
%% \end{multicols}

%% \begin{multicols}{4}
%% \item  \mbox{ }

%% \begin{enumerate}
%% \item[a.]  4.0
%% \item[b.]  4.5
%% \item[c.]  0
%% \item[d.]  0.0
%% \item[e.]  1.33
%% \end{enumerate}
%% \end{multicols}



%% \begin{multicols}{4}
%% \item  \mbox{ }

%% \begin{enumerate}
%% \item[a.]  7 {\tt int}
%% \item[b.]  30 {\tt long}
%% \item[c.]  14.0 {\tt double}
%% \item[d.]  90 {\tt long}
%% \item[e.]  4.0 {\tt double}
%% \end{enumerate}
%% \end{multicols}


%% \begin{multicols}{4}
%% \item  \mbox{ }

%% \mbox{ }
%% \begin{enumerate}
%% \item[a.]  34.0   
%% \item[b.]  54     
%% \item[c.]  4      
%% \item[d.]  1      
%% \item[e.]  6      
%% \item[f.]  0      
%% \item[g.]  2      
%% \item[h.]  7.5    
%% \end{enumerate}
%% \end{multicols}



%% \begin{multicols}{4}
%% \item  \mbox{ }

%% %\mbox{ }

%% \begin{enumerate}
%% \item[a.]  k==5, j==5 
%% \item[b.]  k==5, j==6 
%% \item[c.]  k==6, j==6 
%% \item[d.]  k==5, j==4 
%% \item[e.]  k==4, j==4 
%% \end{enumerate}
%% \end{multicols}


%% \begin{multicols}{4}
%% \item  \mbox{ }

%% \begin{enumerate}
%% \item[a.]  k==15, j==5   
%% \item[b.]  k==5, j==6    
%% \item[c.]  k==120, j==6  
%% \item[d.]  k==0, j==4    
%% \item[e.]  k==0, j==5    
%% \end{enumerate}
%% \end{multicols}



%% \item  \quad{\tt k = k + 1;}
%% \qquad{\tt k += 1;}
%% \qquad{\tt k++; }
%% \qquad{\tt ++k; }

%% \begin{multicols}{4}
%% \item  \mbox{ }
%% \begin{enumerate}
%% \item[a.]  m = 5
%% \item[b.]  m = 6
%% \item[c.]  m = 15
%% \item[d.]  m = 50
%% \item[e.]  m = 70
%% \end{enumerate}
%% \end{multicols}

%% \begin{multicols}{4}
%% \item \mbox{ }

%% \mbox{ }
%% \begin{enumerate}
%% \item[a.]  false
%% \item[b.]  false
%% \item[c.]  true
%% \item[d.]  illegal
%% \item[e.]  true
%% \item[f.]  illegal
%% \item[g.]  false
%% \end{enumerate}
%% \end{multicols}

%% \item \verb| |

%% \begin{jjjlisting}[27pc]
%% \begin{lstlisting}
%% public class TemperatureUI
%% { private KeyboardReader reader; // Handles command line I/O
%%   public TemperatureUI() 
%%   { reader = new KeyboardReader(); // Create reader object
%%   }
%%   // Input-process-output algorithm to convert temperatures.
%%   public void run() 
%%   { reader.prompt("Converts Fahrenheit and Celsius.\n");
%%     reader.prompt("Input a temperature in Fahrenheit > ");  
%%     double tempIn = reader.getKeyboardDouble();
%%     double tempResult = Temperature.fahrToCels(tempIn);
%%     reader.display(tempIn + " F = " + tempResult + " C\n"); 
%%     reader.prompt("Input a temperature in Celsius > ");  
%%     tempIn = reader.getKeyboardDouble();
%%     tempResult = Temperature.celsToFahr(tempIn);
%%     reader.display(tempIn + " C = " + tempResult + " F\n "); 
%%  } // run()
%%  public static void main(String args[])
%%  { TemperatureUI ui = new TemperatureUI();  // Create and
%%    ui.run();                  //  run the user interface.
%%  } // main()
%% } // TemperatureUI
%% \end{lstlisting}
%% \end{jjjlisting}

%% \item \verb| |

%% \begin{jjjlisting}[35pc]
%% \begin{lstlisting}
%% import javax.swing.*;
%% import java.awt.*;
%% import java.awt.event.*;
%%  // Use this panel with a JApplet top-level window (as per Chapter 4)

%% public class TemperatureJPanel extends JPanel implements ActionListener
%% { private JTextField inField = new JTextField(15);   // GUI components
%%   private JTextField resultField = new JTextField(15);
%%   private JLabel prompt1 = new JLabel("Input Temperature >>");
%%   private JLabel prompt2 = new JLabel("Conversion Result:");
%%   private JButton celsToFahr = new JButton("C to F");
%%   private JButton fahrToCels = new JButton("F to C");
%%   private JPanel panelN = new JPanel(); // Panels 
%%   private JPanel panelC = new JPanel();  
%%   private JPanel panelS = new JPanel();
%%   private Temperature temperature = new Temperature(); // Temperature object

%%   public TemperatureJPanel()       // Set up  user interface
%%   { setLayout(new BorderLayout());  // Use BorderLayout
%%     panelN.setLayout(new BorderLayout());
%%     panelC.setLayout(new BorderLayout());
%%     panelS.setLayout(new BorderLayout());
%%     panelN.add("North", prompt1);     // Input elements
%%     panelN.add("South", inField);
%%     panelC.add("West", celsToFahr);   // Control buttons
%%     panelC.add("East", fahrToCels);
%%     panelS.add("North", prompt2);     // Output elements
%%     panelS.add("South", resultField);
%%     add("North", panelN);   // Input at the top
%%     add("Center", panelC);  // Buttons in the center
%%     add("South", panelS);   // Result at the bottom
%%     celsToFahr.addActionListener(this); // Register with listeners
%%     fahrToCels.addActionListener(this);
%%     setSize(175,200);
%%   } // TemperatureJPanel()

%%   public void actionPerformed(ActionEvent e)
%%   { String inputStr = inField.getText();             // User's input
%%     double userInput = Double.parseDouble(inputStr); // Convert to double
%%     double result = 0;
%%     if (e.getSource() == celsToFahr) {          // Process and report 
%%       result = temperature.celsToFahr(userInput);       
%%       resultField.setText(inputStr + " C = " + result  + " F");
%%     } else {
%%       result = temperature.fahrToCels(userInput);
%%       resultField.setText(inputStr + " F = " +  result  + " C");
%%     }
%%   } // actionPerformed
%% } // TemperatureJPanel
%% \end{lstlisting}
%% \end{jjjlisting}

%% \item \verb| |

%% \begin{jjjlisting}
%% \begin{lstlisting}
%% public class KBTestOneRowNim
%% { public static void main(String argv[])
%%   { KeyboardReader kb = new KeyboardReader();
%%     OneRowNim game = new OneRowNim(11);
%%     while(game.gameOver() == false)
%%     {   game.report();  // Prompt the user
%%       System.out.print("Input 1, 2, or 3: ");
%%       int sticks = kb.getKeyboardInteger(); // Get move
%%       game.takeSticks(sticks);  {\color{cyan} // Do move}
%%       System.out.println();
%%     } // while
%%     game.report();  // The game is now over
%%     System.out.print("Game won by player ");
%%     System.out.println(game.getWinner());
%%   } // main()
%% } // TestOneRowNim
%% \end{lstlisting}
%% \end{jjjlisting}


%% \item The new version of {\tt OneRowNim} should run properly
%% with the user interface from Chapter 4. This shows that our new
%% definition for {\tt OneRowNim} is backwards compatible with 
%% the old user interface. This is a good thing.

%% \item \verb| |

%% \begin{jjjlisting}
%% \begin{lstlisting}
%% public class NimPlayer
%% { private OneRowNim nim;
%%   public NimPlayer (OneRowNim game)
%%   { nim = game;
%%   } // NimPlayer()
%%   public int move() 
%%   { int sticksLeft = nim.getSticks();
%%     if (sticksLeft % (nim.MAX_PICKUP + 1) != 1)
%%       return (sticksLeft - 1) % (nim.MAX_PICKUP +1);
%%     else 
%%     { int maxPickup = Math.min(nim.MAX_PICKUP, sticksLeft);
%%       return 1 + (int)(Math.random() * maxPickup);
%%     } // else
%%   } // move()
%% } // NimPlayer
%% \end{lstlisting}
%% \end{jjjlisting}

%% \item \verb| |
%% \begin{jjjlisting}[27pc]
%% \begin{lstlisting}
%% public class KBComputerNim
%% { public static void main(String argv[])
%%   {   KeyboardReader kb = new KeyboardReader();
%%     OneRowNim game = new OneRowNim(OneRowNim.MAX_STICKS);
%%     NimPlayer computer = new NimPlayer(game);
%%     System.out.println("Let's play One Row Nim");
%%     while(game.gameOver() == false)  {   
%%       if (game.getPlayer() == game.PLAYER_ONE) 
%%       { kb.prompt("Sticks left = " + game.getSticks() + 
%%                                  " Your move. "); //Prompt
%%         kb.prompt("You can pick up between 1 and " + 
%%           Math.min(game.MAX_PICKUP,game.getSticks()) +" :");
%%         int sticks = kb.getKeyboardInteger(); // Get move
%%         game.takeSticks(sticks);              // Do move
%%       } else 
%%       { kb.prompt("Sticks left = " + game.getSticks() + 
%%                                  " My move. "); 
%%         int sticks = computer.move();
%%         game.takeSticks(sticks);
%%         System.out.println("I take " + sticks);
%%       } // else
%%     } // while
%%                                  // The game is now over
%%     kb.display("Sticks left = " + game.getSticks());  
%%     if (game.getWinner() == game.PLAYER_ONE)
%%       System.out.println(" You win. Nice game!");
%%     else
%%       System.out.println(" I win. Nice game!");
%%   } // main()
%% } // KBComputerNim
%% \end{lstlisting}
%% \end{jjjlisting}

%% \item \verb| |
%% \begin{jjjlisting}[27pc]
%% \begin{lstlisting}
%% public class BankCD 
%% { private double principal;   // The CD's initial principal
%%   private double rate;        // CD's interest rate
%%   private double years;       // Number of years to maturity
%%   public BankCD(double p, double r, double y) 
%%   {   principal = p;
%%     rate = r;
%%     years = y;
%%   } // BandCD()
%%   public double calcYearly() 
%%   {   return principal * Math.pow(1 + rate, years);
%%   } // calcYearly()
%%   public double calcDaily() 
%%   {   return principal * Math.pow(1 + rate/365, years*365);
%%   } // calcDaily()
%% } // BankCD
%% \end{lstlisting}
%% \end{jjjlisting}

%% \item \verb| |

%% \begin{jjjlistingleft}[34pc]{-8pc}
%% \begin{lstlisting}
%% import java.text.NumberFormat;   // For formatting \$nn.dd or n\%

%% public class TestBankCD 
%% { private KeyboardReader reader = new KeyboardReader();   
%%   private NumberFormat dollars = NumberFormat.getCurrencyInstance(); 
%%   private NumberFormat percent = NumberFormat.getPercentInstance();
%%   private BankCD cd;

%%   public void run()
%%   { reader.display("Compares daily and annual compounding for a CD.\n");
%%     reader.prompt("  Input the CD's initial principal, e.g.  1000.55 > ");
%%     double principal = reader.getKeyboardDouble();
%%     reader.prompt("  Input the CD's interest rate, e.g.  6.5 > ");
%%     double rate = reader.getKeyboardDouble() / 100.0;
%%     reader.prompt("  Input the number of years to maturity, e.g., 10.5 > ");
%%     double years = reader.getKeyboardDouble();
%%     cd = new BankCD(principal, rate, years);
%%     percent.setMaximumFractionDigits(2);
%%     System.out.println("For Principal = " + dollars.format(principal) +
%%                        " Rate= " + percent.format(rate) +
%%                        " Years= " + years);
%%     double cdAnnual = cd.calcYearly();  // Compounded yearly
%%     double cdDaily =  cd.calcDaily();   // Compounded annually
%%     System.out.println(" The maturity value compounded yearly is " +
%%                    dollars.format(cdAnnual));
%%     System.out.println(" The maturity value compounded daily is: " +
%%                    dollars.format(cdDaily));
%%   } // run()

%%   public static void main( String args[] ) 
%%   { TestBankCD cd = new TestBankCD();
%%     cd.run();
%%   } // main()
%% }// TestBankCD
%% \end{lstlisting}
%% \end{jjjlistingleft}
%% \begin{multicols}{3}
%% \item \mbox{ }

%% \mbox{ }

%% \begin{enumerate}
%% \item[a.]  valid
%% \item[b.]  valid
%% \item[c.]  ch2 = (char)n;
%% \item[d.]  valid
%% \item[e.]  ch1 = (char)(m-n);
%% \end{enumerate}
%% \end{multicols}
%% \end{ANS}


\secEXRHtwoleft{Exercises}
\marginnote{\vspace{12pt}\raggedright{\bf Note:} For programming exercises, {\bf first} draw 
a UML class diagram describing all classes and
their inheritance relationships and/or associations.}
%\addcontentsline{toc}{section}{\S~~~ Exercises}

%% \begin{EXRtwo}{}{}

%% \item  Explain the difference between the following pairs of terms:
%% %\begin{list}{}{}
%% \begin{EXRtwoLL}
%% \item  {\it Representation} and {\it action}.
%% \item  {\it Binary operator} and {\it unary operation}.
%% \item  {\it Class constant} and {\it class variable}.
%% \item  {\it Helper method} and {\it class method}.
%% \item  {\it Operator overloading} and {\it method overloading}.
%% \item  {\it Method call} and {\it method composition}.
%% \item  {\it Type conversion} and {\it type promotion}.
%% %\end{list}
%% \end{EXRtwoLL}

%% \item  For each of the following data types, list
%% how many bits are used in its representation and how many values can
%% be represented:

%% \begin{EXRtwoLL}
%% \begin{multicols}{5}

%% \item  \verb!int!
%% \item  \verb!char!
%% \item  \verb!byte!

%% \item  \verb!long!
%% \item  \verb!double!
%% \end{multicols}
%% \end{EXRtwoLL}

%% \item  Fill in the blanks.
%% %\begin{list}{}{}
%% \begin{EXRtwoLL}\baselineskip=12pt
%% \item  Methods and variables that are associated with
%% a class rather than with its instances must be declared \rule{40pt}{0.5pt}\,.
%% \item  When an operation involves values of two different types,
%% one value must be \rule{40pt}{0.5pt} before the expression can be evaluated.
%% \item  Constants should be declared \rule{40pt}{0.5pt}\,.
%% \item  Variables that take {\tt true} and {\tt false} as their
%% possible values are known as \rule{40pt}{0.5pt}\,.
%% %\end{list}
%% \end{EXRtwoLL}\baselineskip=11pt

%% \item  Arrange the following data types into a {\it promotion}
%% hierarchy: {\tt double}, {\tt float}, {\tt int}, {\tt short}, {\tt long}.

%% \item  Assuming that {\it o1} is true, {\it o2} is false,
%% and {\it o3} is false, evaluate each of the following expressions:

%% \begin{EXRtwoLL}
%% \begin{multicols}{3}

%% \item  \verb!o1 || o2 && o3!

%% \item  \verb!o1 ^ o2!

%% \item  \verb|!o1 && !o2|
%% \end{multicols}

%% \end{EXRtwoLL}

%% \item  Arrange the following operators in precedence order:

%% \begin{jjjlisting}
%% \begin{lstlisting}
%% + - () * / % < ==
%% \end{lstlisting}
%% \end{jjjlisting}

%% \item  Arrange the following operators into a precedence hierarchy:

%% \begin{jjjlisting}
%% \begin{lstlisting}
%% *,++, %, ==
%% \end{lstlisting}
%% \end{jjjlisting}

%% \item  Parenthesize and evaluate each of the following expressions
%% (if an expression is invalid, mark it as such):

%% \begin{EXRtwoLL}
%% \begin{multicols}{3}

%% \item  \verb!11 / 3 % 2 == 1!
%% \item  \verb!11 / 2 % 2 > 0!
%% \item  \verb!15 % 3 >= 21 %!

%% \item  \verb!12.0 / 4.0 >= 12 / 3!
%% \item  \verb!15 / 3 == true!
%% \end{multicols}

%% \end{EXRtwoLL}


%% \item  What value would {\it m} have after each of the 
%% statements that follow is executed?  Assume that {\it m, k, j} are
%% reinitialized before each statement.

%% \begin{jjjlisting}
%% \begin{lstlisting}
%%    int m = 5, k = 0, j = 1;
%% \end{lstlisting}
%% \end{jjjlisting}

%% \begin{EXRtwoLL}
%% \begin{multicols}{3}

%% \item  \verb!m = ++k + j;!
%% \item  \verb!m += ++k * j;!
%% \item  \verb!m %= ++k + ++j;!

%% \item  \verb!m = m - k - j;!
%% \item  \verb!m = ++m;!
%% \end{multicols}

%% \end{EXRtwoLL}

%% \item  What value would {\it b} have after each of the 
%% statements that follow is executed? Assume that {\it m, k, j} are reinitialized
%% before each statement. It may help to parenthesize the right-hand side
%% of the statements before evaluating them.

%% \begin{jjjlisting}
%% \begin{lstlisting}
%%    boolean b;
%%    int m = 5, k = 0, j = 1;
%% \end{lstlisting}
%% \end{jjjlisting}
%% \begin{EXRtwoLL}
%% \begin{multicols}{2}

%% \item  \verb!b = m > k + j;!
%% \item  \verb�b = m * m != m * j;�
%% \item  \verb!b = m <= 5 && m % 2 == 1;!
%% \item  \verb!b = m < k  || k < j;!
%% \item  \verb!b = --m == 2 * ++j;!
%% \end{multicols}

%% \end{EXRtwoLL}

%% \item  For each of the following expressions, if it is valid,
%% determine the value of the variable on the left-hand side
%% (if not, change it to a valid \mbox{expression):}

%% \begin{jjjlisting}
%% \begin{lstlisting}
%%     char c = 'a' ;
%%     int  m = 95;
%% \end{lstlisting}
%% \end{jjjlisting}
%% \begin{EXRtwoLL}
%% \begin{multicols}{3}

%% \item  \verb!c = c + 5;!
%% \item  \verb!c = 'A' + 'B';!
%% \item  \verb!m = c + 5;!

%% \item  \verb!c = (char) m + 1;!
%% \item  \verb!m = 'a' - 32;!
%% \end{multicols}

%% \end{EXRtwoLL}

%% \item  Translate each of the following expressions into Java:

%% \begin{EXRtwoLL}
%% \item  Area equals {\it pi} times the radius squared.
%% \item  Area is assigned {\it pi} times the radius squared.
%% \item  Volume is assigned {\it pi} times radius cubed divide by {\it h.}
%% \item  If {\it m} and {\it n} are equal, then {\it m} is incremented by one; otherwise {\it n} 
%% is incremented.
%% \item  If {\it m} is greater than {\it n}
%% times 5, then square {\it m} and double {\it n};
%% otherwise square {\it n} and double {\it m.}
%% \end{EXRtwoLL}

%% \item  What would be output by the following code segment?

%% \begin{jjjlisting}
%% \begin{lstlisting}
%% int m = 0, n = 0, j = 0, k = 0;
%% m = 2 * n++;
%% System.out.println("m= " + m + " n= " + n);
%% j += ( --k * 2 );
%% System.out.println("j= " + j + " k= " + k);
%% \end{lstlisting}
%% \end{jjjlisting}

%% Each of the  problems that follow asks you to write a method.
%% Of course, as you are developing the method in a stepwise
%% fashion, you should test it. Here's a simple application
%% program that you can use for this purpose:

%% \begin{jjjlisting}
%% \begin{lstlisting}
%% public class MethodTester {
%%     public static int square(int n) {
%%         return n * n;
%%     }

%%     public static void main(String args[]) {
%%         System.out.println("5 squared = " + square(5));
%%     }
%% }
%% \end{lstlisting}
%% \end{jjjlisting}

%% \noindent Just replace
%% the {\tt square()} method with your method. Note that
%% you must declare your method {\tt static} if you want
%% to call it directly from {\tt main()} as we do here.

%% \item  Write a method to calculate the sales tax for
%% a sale item. The method should take two {\tt double} parameters, one
%% for the sales price and the other for the tax rate. It should return a
%% {\tt double}.  For example, {\tt calcTax(20.0, 0.05)} should return
%% 1.0.


%% \item  {\bf Challenge:} Suppose you're writing a program that tells what
%% day of the week someone's birthday falls on this year. Write a method
%% that takes an {\tt int} parameter, representing what day of the year
%% it is, and returns a {\tt String} like ``Monday.'' For example, for
%% 2004, a leap year, the first day of the year was on Thursday. The
%% thirty-second day of the year (February 1, 2004) was a Sunday, so {\tt
%% getDayOfWeek(1)} should return ``Thursday'' and {\tt getDayOfWeek(32)}
%% should return ``Sunday.'' ({\it Hint}: If you divide the day of the
%% year by 7, the remainder will always be a number between 0 and 6,
%% which can be made to correspond to days of the week.)


%% \item  {\bf Challenge:} As part of the birthday program, you'll want a method that
%% takes the month and the day as parameters and returns what day of the
%% year it is. For example, {\tt getDay(1,1)} should return 1;
%% {\tt getDay(2,1)} should return 32; and {\tt getDay(12,31)} should return
%% 365. ({\it Hint}: If the month is 3, and the day is 5, you have to add the
%% number of days in January plus the number of days in February to 5 to
%% get the result: 31 + 28 + 5 = 64.)

%% \item  Write a Java method that converts a {\tt char}
%% to lowercase. For example, {\tt toLowerCase('A')}
%% should return `a'. Make sure you guard against
%% method calls like {\tt toLowerCase('a')}.

%% \item  {\bf Challenge:} Write a Java method that shifts a
%% {\tt char} by {\it n} places in the alphabet, wrapping around to the
%% start of the alphabet, if necessary. For example, {\tt shift('a',2)}
%% should return `c'; {\tt shift('y',2)} should return `a'. This method
%% can be used to create a Caesar cipher, in which every letter in a
%% message is shifted by {\it n} places---hfu ju? (Refer to Chapter~1
%% exercises for a refresher on Caesar cipher.)

%% \item  Write a method that converts its {\tt boolean} parameter
%% to a {\tt String}. For example, {\tt boolToString(true)}
%% should return ``true.''

%% \item  Write a Java application that first prompts the user for
%% three numbers, which represent the sides of a rectangular cube, and
%% then computes and outputs the volume and the surface area of the cube.

%% \item  Write a Java application that prompts the user for three
%% numbers and then outputs the three numbers in increasing order.

%% \item  Write a Java application that inputs two integers and
%% then determines whether the first is divisible by the second.
%% ({\it Hint}: Use the modulus operator.)

%% \item  Write a Java application that prints the following table:

%% \begin{jjjlisting}
%% \begin{lstlisting}
%% N   SQUARE   CUBE
%% 1   1        1
%% 2   4        8
%% 3   9        27
%% 4   16       64
%% 5   25       125
%% \end{lstlisting}
%% \end{jjjlisting}

%% \item  Design and write a Java GUI that converts kilometers to miles
%% and vice versa. Use a {\tt JTextField} for I/O and {\tt JButton}s for
%% the various conversion actions.

%% \item  Design and write a GUI that allows a user to calculate
%% the maturity value of a CD. The user should enter the principal,
%% interest rate, and years, and the applet should then display the
%% maturity value. Make use of the {\tt BankCD} class covered in this
%% chapter.  Use separate {\tt JTextField}s for the user's inputs and a
%% separate {\tt JTextField} for the result.

%% \item  Design and write a GUI that lets the user
%% input a birth date (month and day) and reports what day of the week
%% it falls on. Use the {\tt getDayOfWeek()} and {\tt getDay()} methods
%% that you developed in previous exercises.

%% \item  Design and write a GUI that allows the users
%% to input their exam grades for a course and computes
%% their average and probable letter grade. The applet should
%% contain a single {\tt JTextField} for inputting a grade
%% and a single {\tt JTextField} for displaying the average
%% and letter grade. The program should keep track internally
%% of how many grades the student has entered. Each time a
%% new grade is entered, it should display the current average
%% and probable letter grade.

%% \item One of the reviewers of this text has
%% suggested an alternative design for the {\tt Temperature} class
%% (Fig.~\ref{fig-temperature}). According to this design, the class
%% would contain an instance variable, say, {\tt temperature}, and access
%% methods that operate on it. The access methods would be: {\tt
%% setFahrenheit(double)}, \break{\tt getFahrenheit():double}, {\tt
%% setCelsius(double)}, and {\tt getCelsius():} {\tt double}. One way to
%% implement this design is to store the temperature in the Kelvin scale
%% and then convert from and to Kelvin in the access methods.  The
%% formula for converting Kelvin to Celsius is

%% \begin{jjjlisting}
%% \begin{lstlisting}
%% K = C + 273.15
%% \end{lstlisting}
%% \end{jjjlisting}

%% \noindent Draw a UML class diagram representing this design
%% of the {\tt Temperature} class. Which design is more object oriented,
%% this one or the one used in Figure~\ref{fig-temperature}?

%% \item Write an implementation of the {\tt Temperature}
%% class using the design described in the previous exercise. 

%% \end{EXRtwo}

